%!TEX root = ../tesis.tex
\chapter{Estado del Arte}
\label{sec:estado-arte}

En el cap\'itulo anterior introdujo el proceso b\'asico de reconocimiento, as\'i como m\'etodos
alternativos que se han venido desarrollando. Este cap\'itulo presenta algunas medidas del desempe\~no 
de los sistemas de reconocimiento del habla de modo a describir brevemente el estado del arte del área.

El rendimiento de los sistemas de reconocimiento del habla se mide, por lo general, en t\'erminos
de precisi\'on y velocidad. La precisi\'on se mide usualmente utilizando la tasa de error por palabras
(o \gls{wer} por sus siglas en ingl\'es), mientras que la velocidad se mide utilizando el 
\mbox{\gls{rtf} \cite{GaikwadAReview2010}.}


\section{Estado del Arte}
\label{sec:estado}
