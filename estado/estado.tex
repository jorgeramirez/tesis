\section{Estado del Arte}
\label{sec:estado}

El rendimiento de los sistemas de reconocimiento del habla se mide, por lo general, en t\'erminos
de precisi\'on y velocidad. La precisi\'on se mide usualmente utilizando la Tasa de Error por Palabras
(o WER por sus siglas en ingl\'es), mientras que la velocidad se mide utilizando el Factor de 
Tiempo \mbox{Real (RTF) \cite{GaikwadAReview2010}}.

\subsection{Tasa de Error por Palabras}
\label{sec:wer}

La Tasa de error por palabras (WER) es una m\'etrica de evaluaci\'on est\'andar del desempe\~no de un sistema de
reconocimiento del habla. Su c\'alculo se basa en qu\'e tanto la cadena retornada por el reconocedor,
denominada hip\'otesis, difiere con respecto a la transcripci\'on correcta \cite{Jurafsky}. 
Dada la transcripci\'on de referencia, para computar la tasa se debe calcular la distancia de edici\'on m\'inima
entre la hip\'otesis y la \mbox{transcripci\'on \cite{GaikwadAReview2010}}. Es decir, la m\'inima cantidad de palabras insertadas, eliminadas y
sustituidas. Formalmente, la tasa se calcula de acuerdo a la ecuaci\'on \eqref{eq:wer}

\begin{equation}
\label{eq:wer}
    WER = \frac{I + E + S}{N} 
\end{equation}

Donde

\begin{itemize}
    \item $I$ es el n\'umero de inserciones,
    \item $E$ es el n\'umero de eliminaciones,
    \item $S$ es el n\'umero de sustituciones, y
    \item $N$ es el n\'umero de palabras en la transcripci\'on de referencia.
\end{itemize}

En \cite{VimalaReview2012} se muestra una comparaci\'on de varias aplicaciones de reconocimiento del habla basada
en el conjunto de datos, extracci\'on de caracter\'isticas y el enfoque utilizado. Se puede observar que la precisi\'on
en el caso de reconocimiento de d\'igitos en ingl\'es es de 98\%, para el reconocimiento cont\'inuo del habla
de grandes vocabularios (en Rumano) es de 90,41\%.

\subsection{Factor de Tiempo Real}
\label{sec:rtf}

La velocidad de los sistemas de reconocimiento se miden en t\'erminos del Factor de Tiempo Real (RTF). Toma un tiempo $P$
procesar una entrada de duraci\'on $I$ \cite{VimalaReview2012}. Formalmente, RTF se calcula de acuerdo a la ecuaci\'on \eqref{eq:rtf}

\begin{equation}
\label{eq:rtf}
    RTF = \frac{P}{I}
\end{equation}

Cuando el factor de tiempo real es menor o igual que 1, se dice que el procesamiento se realiza en tiempo real.

La llegada de sistemas de reconocimiento del habla en tiempo real a los dispositivos m\'oviles y sistemas embebidos dio
oportunidad a uno nuevo rango de investigaciones relacionadas a la interacci\'on humano-computadora. PocketSphinx, es
un sistema de reconocimiento para sistemas embebidos con RTF igual a 0.87 y WER igual a 13,95\% para un vocabulario
de 1000 palabras \cite{HugginsDainesPocketSphinx2006}.

\subsection{Perplejidad}
\label{sec:perplexity}

Las probabilidades de los modelos estad\'isticos, como el modelo N-grama, provienen del \foreign{Corpus} sobre
sobre el cual fueron entrenados. El \foreign{Corpus} se divide en un conjunto de entrenamiento y otro de prueba.
Los par\'ametros del modelo estad\'istico se determinan a partir del conjunto de entrenamiento, y se utilizan
para calcular las probabilidades sobre el conjunto de prueba. Adem\'as, este paradigma de entrenamiento y prueba
puede utilizarse para evaluar diferentes arquitecturas N-grama \cite{Jurafsky}.

La perplejidad es una m\'etrica que indica que tan bien el Modelo de Lenguaje utilizado representa al \foreign{Corpus} 
de prueba \cite{RosenfeldStatistical1997}. Formalmente, la perplejidad se define como

\begin{equation*}
    2^{-l}
\end{equation*}

donde

\begin{equation*}
    l = \frac{1}{M}\sum_{i=1}^{M}\log_2p(x^{(i)})
\end{equation*}

siendo $M$ el n\'umero total de palabras en el \foreign{Corpus} de prueba y $p(x^{(i)})$ la probabilidad de una 
sentencia del conjunto. Cuanto menor sea el valor de perplejidad, mejor ser\'a el modelo de lenguaje modelando
nuevas sentencias. En \cite{MikolovEmpirical2011}, los autores presentan un avance en el estado del arte
utilizando una combinaci\'on de t\'ecnicas avanzadas para el modelado del lenguaje, logrando una perplejidad
de 79,4 para la porci\'on del \foreign{Wall Street Journal Corpus} conocida como \foreign{Peen Treebank}.
