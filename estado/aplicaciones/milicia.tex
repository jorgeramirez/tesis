\subsection{Milicia}
\label{sec:milicia}

El \foreign{NATO Research Study Group} ha llevado a cabo una serie de experimentos e investigaciones
enfocados a la aplicaci\'on de sistemas de reconocimiento del habla en aplicaciones militares.
Un punto de refencia hist\'orico  es el
trabajo de Beek y otros \cite{BeekAn1977}, en donde se identifican potenciales aplicaciones
de tecnolog\'ias del habla agrupadas en cuatro categor\'ias: seguridad, mando y control, transmisi\'on de datos y comunicaci\'on, y
procesamiento de voz distorsionada. Posteriormente, Weinstein \cite{WeinsteinOpportunities1991} incorpora
a las aplicaciones para entrenamiento como otra categor\'ia. Espec\'ificamente, \cite{PigeonUse2006} indica
las aplicaciones del reconocimiento del habla en la milicia:

\begin{itemize}
    \item Mando y Control, consiste en la interacci\'on humana con computadoras, sistemas
	de armas y sensores, por voz (en aviones de combate o helic\'opteros, por ejemplo). Pero esto
	requiere un alto desempe\~no, en tiempo real, de las tecnolog\'ias de reconocimiento.
    \item Acceso a computadoras e informaci\'on, son una parte crucial para las operaciones militares modernas. El
	reconocimiento del habla puede utilizarse para operar computadoras y consultar informaci\'on utilizando la voz. Esto
	es importante para personal que trabaja bajo mucha presi\'on y tienen la vista y las manos ocupadas.
    \item Inteligencia, implica el procesamiento de una gran variedad de tipos de informaci\'on (texto y audio). El inter\'es militar
	radica en la utilizaci\'on de tecnolog\'as del habla y lenguaje para el an\'alisis y procesamiento
	de la excesiva cantidad de informaci\'on disponible actualmente.
    \item Entrenamiento, consiste en utilizar el reconocimiento del habla en el tareas de entrenamiento de las fuerzas
	militares. Permitiendo que el personal pueda interactuar, mendiante la voz, con sistemas avanzados de simulaci\'on.
\end{itemize}
