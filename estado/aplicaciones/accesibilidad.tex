\subsection{Accesibilidad}
\label{sec:accesibilidad}

Las aplicaciones de reconocimiento del habla toman como entrada la voz del usuario y la transforman en
comandos que pueden ser entendidos por las computadoras. Estos comandos pueden controlar las aplicaciones
y, por lo tanto, sirven como alternativa de interacci\'on.

Las personas con discapacidad pueden beneficiarse de las aplicaciones de reconocimiento del habla que
generan autom\'aticamente los subt\'itulos de conversaciones como las que se dan en conferencias o
sal\'on de \mbox{clases \cite{LeitchHow2002}.}

El reconocimiento del habla tambi\'en puede ser \'util para personas con dificultades para usar las manos, ya
que los medios convencionales de interacci\'on son de poca o nula utilidad para estas
\mbox{personas \cite{AnanthiSurvey2013}.}

En \cite{AnanthiSurvey2013} los autores resaltan el rol que desempe\~na el reconocimiento del habla para personas
con discapacidad y adem\'as mencionan el tipo de aplicaci\'on que se utiliza, en cada caso, para mejorar la accesibilidad:

\begin{itemize}
    \item Para personas con sordera, existen los tel\'efonos subtitulados. Es un tel\'efono que despliega en
	una pantalla los subt\'itulos de la conversaci\'on actual \cite{PerfettiReading2000}. Estas personas
	podr\'ian utilizar el software de reconocimiento para convertir palabras a texto que puede ser le\'ido o
	convertido en lenguaje de se\~nas/Braille \cite{SchilperoordNonverbatim2005}.
    \item Las personas ciegas o con dificultades para ver usan varios productos con capacidad de reconocimiento del habla,
	por ejemplo: relojes, calculadoras y computadoras ``parlantes''. Estas computadoras parlantes utilizan software
	de lectura de pantalla.
    \item Las personas con dificultades para mover los brazos y manos pueden utilizar software de reconocimiento para
	navegar interfaces usando la voz\cite{AnanthiSurvey2013}.
\end{itemize}

La compa\~n\'ia Nuance provee soluciones de reconocimiento del habla, una parte de \'estas se encuentra en las Soluciones de Accesibilidad
para el Negocio. Estas tecnolog\'ias de asistencia permiten que personas con discapcidad puedan ejercer sus funciones en el mundo laboral,
proporcionando: operaci\'on de interfaces mediante voz, transcripci\'on autom\'atica de texto,
y \mbox{m\'as \cite{NuanceAccessibility}.}
