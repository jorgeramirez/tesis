%!TEX root = ../../tesis.tex
\subsection{Medicina y Derecho}
\label{sec:medicina}

El uso de aplicaciones de reconocimiento del habla para la transcripci\'on de registros m\'edicos, proceso que forma parte central
de todos los aspectos del sistema de salud \cite{DavidListening2009}, es un tema de investigaci\'on activo en el \'ambito 
\mbox{acad\'emico \cite{LaiMedSpeak1997, HappeCombining2002}}.

Existen productos comerciales ya implementados en hospitales, ofrecidas por compa\~n{\'\i}as como Nuance \cite{NuanceOptimizing, NuanceSpeech} 
o M*Modal \cite{MmodalSpeech}, las cuales promocionan su oferta como una soluci\'on innovadora y beneficiosa en cuanto a costo 
y est\'andares de calidad del cuidado de la salud.

Se ha documentado una precisi\'on de 98\% aproximadamente de los sistemas de reconocimiento del habla para transcripci\'on de registros
m\'edicos, tasa que resulta inferior a la precisi\'on de un profesional tomando notas \cite{DavidListening2009}. 
Caben destacar, sin embargo, los beneficios que ofrece la tecnolog{\'\i}a frente al registro manual de 
la \mbox{informaci\'on \cite{ZickVoice2001}}:

\begin{itemize}
	\item Si la responsabilidad de transcribir recae en el m\'edico, el principal beneficio es la mejora en la calidad de la atenci\'on.
	Mediante el uso de herramientas basadas en reconocimiento del habla, el profesional m\'edico no pierde tiempo ni concentraci\'on en
	una tarea tan estresante y tediosa, especialmente para los m\'edicos de emergencias.
	\item En el caso de utilizarse un servicio de transcripci\'on, donde un transcriptor profesional toma las notas, el beneficio que
	puede obtenerse en cuanto a costo es significativo. Se estima que para un hospital en EEUU que atiende 45000 pacientes al a\~no,
	el uso de herramientas basadas en reconocimiento del habla puede significar un ahorro de 334500\$ al a\~no.
\end{itemize}

Aunque la aceptaci\'on y adopci\'on de estas herramientas presentan un panorama alentador \cite{GrassoLong2003}, 
existen a\'un problemas por solucionar. Dificultades para insertar signos de puntuaci\'on y para ordenar el contenido dictado
de acuerdo al formato establecido para el reporte, entre otras, pueden llevar a la oposici\'on a una automatizaci\'on 
completa del \mbox{proceso \cite{DavidListening2009}}.

Una situaci\'on similar se presenta con respecto a la transcripci\'on reportes legales, tema sobre el cual existen publicaciones
cient{\'\i}ficas \cite{van-leeuwen2008improving, FalavignaAutomatic2009} y para el cual se dispone de productos comerciales
en la \mbox{actualidad \cite{NuanceLegal}}.
