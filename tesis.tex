\documentclass[final,fmstyle,a4paper]{./util/ucathesis}
% La opcion 'final' muestra los graficos, para generar una version sin los graficos utiliza la opcion 'draft'

% paquetes recomendados
%\usepackage[chapter]{theorems}
%\usepackage{symbols}
%\usepackage{url}
\usepackage{amsmath,amsthm}

\usepackage[T1]{fontenc}
\usepackage[spanish]{babel}
\usepackage[utf8]{inputenc}
\usepackage{csquotes}
\usepackage[style=numeric,sorting=none,backend=biber]{biblatex}
\usepackage{tabularx}
\usepackage{pdflscape}
\usepackage{listings}
\usepackage{subfigure}
\usepackage{array}
\usepackage{perpage}
\usepackage{backnaur}
\usepackage{enumerate}
\MakePerPage{footnote}
\addbibresource{referencias.bib}


% custom commands
\newcommand{\foreign}[1]{{\it #1}}
\DeclareMathOperator*{\argmax}{arg\,max}
\algsetup{indent=2em}
\renewcommand{\algorithmicrequire}{\textbf{Entrada:}}
\renewcommand{\algorithmicensure}{\textbf{Salida:}}
\renewcommand{\algorithmicif}{\textbf{si}}
\renewcommand{\algorithmicend}{\textbf{fin}}
\renewcommand{\algorithmicfor}{\textbf{por}}
\renewcommand{\algorithmicdo}{\textbf{hacer}}
\renewcommand{\algorithmicthen}{\textbf{entonces}}
\renewcommand{\algorithmicreturn}{\textbf{retorna}}



% \setcounter{tocdepth}{3}

% datos de la tesis
\title{Dise\~{n}o de Interfaces de Usuario basado en Reconocimiento del Habla}
\author{Rodrigo Manuel Parra Zacar\'{i}as y Jorge Daniel Ram\'{i}rez Medina}
\degree{Inform\'{a}tica}

\advisor{Ing.}{Mart\'{i}n Abente Lahaye, M.Sc.}

%\newtheorem{definicion}{Definicin}
%\numberwithin{algorithm}{chapter}

\logosource{./graphics/logo.jpg}
\institution{Universidad Nacional de Asunci\'{o}n}
\faculty{Facultad Polit\'{e}cnica}
\address{San Lorenzo - Paraguay}

\usepackage{glossaries}
\usepackage{etoolbox}
\usepackage[xindy]{imakeidx}

\newtoks\customtok

\renewcommand*{\newacronymhook}{%
 \edef\dosetkeys{\noexpand\setkeys{glossentry}{user1={},\the\glskeylisttok}}%
 \dosetkeys
 \ifcsempty{@glo@useri}%
 {%
   \expandafter\customtok\expandafter{\the\glsshorttok}%
 }%
 {%
   \edef\custom{\the\glsshorttok, \csexpandonce{@glo@useri}}%
   \expandafter\customtok\expandafter{\custom}%
 }%
}

\newcommand*{\custompostdesc}[1]{%
  \ifcsempty{glo@#1@useri}{}{ (\glsentryuseri{#1})}%
}

\renewcommand*{\CustomAcronymFields}{%
  user1={},%
  name={\the\glsshorttok},%
  description={\the\glslongtok\noexpand\custompostdesc{\the\glslabeltok}},%
  first={\the\glslongtok\space(\the\customtok)},%
  firstplural={\the\glslongtok\noexpand\acrpluralsuffix
    \space (\the\customtok)}%
  text={\the\glsshorttok},%
  plural={\the\glsshorttok\noexpand\acrpluralsuffix}%
}

\SetCustomStyle

\makeglossaries
\makeindex


\begin{document}

\maketitle     % esto hace las portadas

% Agradecimientos
%!TEX root = ../tesis.tex
\chapter*{Agradecimientos}

A \textbf{Martín Abente Lahaye}, por la orientación y el acompañamiento constantes 
proporcionados durante el desarrollo de este trabajo.

A la \textbf{Facultad Politécnica de la UNA}, por los años de formación académica recibidos, los cuales 
nos enorgullecen y nos comprometen a representar siempre de la mejor manera a tan distinguida casa de 
estudios.

A los \textbf{profesores y alumnos} de nuestra carrera que colaboraron con la realización 
de este trabajo mediante el presente más valioso que puede entregarse: el tiempo. 

A nuestros \textbf{amigos}, los que hicimos durante los años de carrera universitaria y los que vienen
de antes, cuyo apoyo incondicional hizo más pequeños los obstáculos y más cercana la meta.

A nuestras \textbf{familias}, que inspiran, acompañan y dan sentido a cada uno de nuestros pasos. La vida probablemente no alcance para devolver todo lo que hemos recibido de ellas. Este logro es tan suyo como nuestro.

A \textbf{Dios}, sin cuya bendición nada de esto hubiese sido posible.


% los siguientes comandos producen 'indices.

% Tabla de contenidos
\tableofcontents
% Lista de figuras
\listoffigures
% Lista de tablas
\listoftables
% Lista de algoritmos
\listofalgorithms
\printglossary[type=\acronymtype,title=Lista de Siglas]
\addcontentsline{toc}{chapter}{Lista de Siglas}
%\include{simbolos}

\newacronym{api}{API}{Interfaz de Programaci\'on de  Aplicaciones}
\newacronym{lpc}{LPC}{Codificaci\'on Predictiva Lineal}


\mainmatter  % inician los capitulos de la tesis

% incluye aqui los capitulos (un archivo .tex por capitulo)
%!TEX root = ../tesis.tex
\chapter{Introducci\'on}
\label{sec:intro}

El habla, y m\'as concretamente el lenguaje como medio de comunicaci\'on, es una
de las caracter{\'\i}sticas fundamentales que diferencian al ser humano de los dem\'as
animales y representa un factor clave de su evoluci\'on \cite{SchepartzLanguage1993}. La misma es considerada
el principal modo de comunicaci\'on y la forma m\'as eficiente y natural de intercambio 
de informaci\'on entre seres humanos \cite{GaikwadAReview2010}.

Atendiendo la importancia de la voz y el habla en la comunicaci\'on entre personas,
resulta l\'ogico el inter\'es por desarrollar tecnolog{\'\i}as que permitan una interacci\'on
similar entre una persona y una computadora. Es decir, resulta interesante la idea 
de ``hablar'' con una computadora.

El reconocimiento del habla, tambi\'en conocido como reconocimiento autom\'atico del habla,
es el proceso de convertir una se\~nal de voz en una secuencia de
palabras, mediante un algoritmo implementado program\'aticamente \cite{JaisalAReview2012}. 
Su integraci\'on con interfaces de usuario busca una interacci\'on humano-computadora m\'as
natural, de manera a superar las limitaciones existentes en el modelo convencional
de interacci\'on.

Los asistentes virtuales de Apple \cite{AppleSiri} y Google \cite{GoogleNow}, 
los centros de mando en los autom\'oviles de Ford \cite{FordSync} y Toyota \cite{ToyotaEntune},
y los SmartTV de Samsung \cite{SamsungVoiceControl}; todos ellos capaces de reaccionar a 
comandos por voz del usuario, son ejemplos de la incorporaci\'on de interfaces basadas 
en reconocimiento del habla en actividades de la vida diaria.

Los resultados de los estudios realizados por la respetada consultora tecnol\'ogica
Gartner confirman esta tendencia. Estos ubican al reconocimiento del habla entre
las tecnolog{\'\i}as que se estabilizar\'an, y cuyos beneficios estar\'an ampliamente
demostrados, en los pr\'oximos 2 a 5 a\~nos \cite{Gartner2013}. 

El escenario descrito resulta propicio para la investigaci\'on en el \'area.
Teniendo esto en cuenta, resulta importante contar con un estudio te\'orico
y pr\'actico del reconocimiento del habla, desde sus fundamentos hasta su
aplicaci\'on. El estudio ser{\'\i}a de gran utilidad como material introductorio
para quien quisiese estudiar y aplicar el reconocimiento del habla, pudiendo
servir de base para numerosos trabajos futuros.

Este es el eje central de este trabajo final de grado, cuyos objetivos y organizaci\'on se
exponen a continuaci\'on.

\section{Objetivo General}
\label{sec:objgral}

Realizar un estudio del trasfondo hist\'{o}rico, los fundamentos te\'{o}ricos 
y el estado del arte del reconocimiento del habla de modo a comprender, describir 
e introducir esta \'{a}rea de investigaci\'{o}n.  


\section{Objetivos Espec\'{i}ficos}
\label{sec:objspec}

\begin{itemize}
	\item Presentar y describir los antecedentes hist\'oricos y el estado del arte del reconocimiento
	del habla, teniendo en cuenta las diversas \'areas de aplicaci\'on del mismo.

    \item Analizar, ordenar y caracterizar el proceso t\'{i}pico de un sistema de reconocimiento del habla, 
        incluyendo los aspectos te\'{o}ricos involucrados en cada paso del mismo.

    \item Evaluar y seleccionar las herramientas disponibles que permiten la implementaci\'{o}n de soluciones 
        relacionadas al reconocimiento del habla.
    
    \item Dise\~{n}ar e implementar una interfaz mediante voz del usuario de manera a aplicar y 
    contrastar en la pr\'{a}ctica los conocimientos te\'{o}ricos adquiridos.
    
    \item Evaluar la soluci\'{o}n implementada de modo a obtener datos cuantitativos y cualitativos que 
        permitan extraer conclusiones sobre la aplicabilidad del reconocimiento del habla a las interfaces 
        de usuario.
\end{itemize}

\section{Organizaci\'on del Trabajo}
\label{sec:organizacion}


El presente trabajo est\'a organizado como se describe a continuaci\'on:

\begin{itemize}
    \item En el cap{\'\i}tulo~\ref{sec:antecedentes} se presenta un breve resumen de los antecedentes hist\'oricos del reconocimiento
	del habla, desde sus inicios hasta la actualidad.
    \item En el cap{\'\i}tulo~\ref{sec:areas-aplicacion} se presentan varias \'areas de aplicaci\'on del reconocimiento del habla, 
	mencionando avances y algunos trabajos de referencia en cada una.
    \item En el cap{\'\i}tulo~\ref{sec:proceso} se presenta el proceso t\'{i}pico de un sistema de reconocimiento del habla,
	describiendo los fundamentos matem\'aticos y algoritmos relacionados a cada paso. Adem\'as, se presentan algunas 
    medidas de desempe\~no de sistemas de reconocimiento del
	habla, incluy\'endose resultados obtenidos en trabajos de investigaci\'on con respecto a cada una.
    \item En el cap{\'\i}tulo~\ref{sec:tecnologias} se presentan los resultados de un estudio comparativo de las tecnolog{\'\i}as
	y herramientas disponibles para la implementaci\'on de sistemas de reconocimiento del habla.
    \item En el cap{\'\i}tulo~\ref{sec:problema} se presenta el problema planteado a modo de llevar a la pr\'actica los
	conocimiento te\'oricos aprendidos.
    \item En el cap{\'\i}tulo~\ref{sec:solucion} se presenta la soluci\'on propuesta para el problema planteado, incluyendo
	las herramientas a utilizar y otros detalles de implementaci\'on.
    \item En el cap{\'\i}tulo~\ref{sec:evaluacion} se presentan los aspectos relacionados a la evaluaci\'on realizada una vez
	implementada la soluci\'on. Se mencionan los objectivos y se describen la metodolog{\'\i}a utilizada
	y las variables que se tuvieron en cuenta.
    \item En el cap{\'\i}tulo~\ref{sec:resultados} se presentan los resultados obtenidos una vez realizada la evaluaci\'on
	del trabajo.
    \item En el cap{\'\i}tulo~\ref{sec:conclusiones} se presentan las conclusiones obtenidas a trav\'es de las distintas
	etapas de realizaci\'on del trabajo, desde las investigaciones iniciales hasta la evaluaci\'on.
    \item Finalmente, en el cap{\'\i}tulo~\ref{sec:trabajos-futuros} se presentan algunas oportunidades
    posibles para la realizaci\'on de trabajos futuros en el \'area.
\end{itemize}
\chapter{Antecedentes Hist\'{o}ricos}
\label{sec:antecedentes}

\chapter{\'Areas de Aplicaci\'on}
\label{sec:aplicaciones}

Este cap{\'\i}tulo busca presentar las aplicaciones del reconocimiento del habla, organizadas en diferentes
\'areas de acuerdo al \'ambito de utilizaci\'on de las mismas, haciendo \'enfasis en el impacto y beneficio
en cada caso.

As{\'\i}, a trav\'es de ejemplos concretos y ventajas reales, se pretende justificar y motivar el trabajo
en esta \'area de investigaci\'on, as{\'\i} como resaltar las numerosas oportunidades existentes en la misma.
%!TEX root = ../../tesis.tex
\section{Medicina y Derecho}
\label{sec:medicina}

El uso de aplicaciones de reconocimiento del habla para la transcripci\'on de registros m\'edicos, proceso 
que forma parte central de todos los aspectos del sistema de salud \cite{DavidListening2009}, 
es un tema de investigaci\'on activo en el \'ambito 
\mbox{acad\'emico \cite{LaiMedSpeak1997, HappeCombining2002}}.

Existen productos comerciales ya implementados en hospitales \cite{USATodayHospitals},
ofrecidas por compa\~n{\'\i}as como \foreign{Nuance} \cite{NuanceOptimizing, NuanceSpeech} 
o \foreign{M*Modal} \cite{MmodalSpeech}, las cuales promocionan su oferta 
como una soluci\'on innovadora y beneficiosa en cuanto a costo y est\'andares de calidad del cuidado de la salud.

Se ha documentado una precisi\'on de 98\% aproximadamente de los sistemas de reconocimiento del habla para 
transcripci\'on de registros m\'edicos, tasa que resulta ligeramente inferior a la precisi\'on de un 
profesional tomando notas \cite{ZickVoice2001}. Caben destacar, sin embargo, los beneficios que ofrece 
la tecnolog{\'\i}a frente al registro manual de la \mbox{informaci\'on}:

\begin{itemize}
	\item Si la responsabilidad de transcribir recae en el m\'edico, el principal beneficio es la mejora en 
	la calidad de la atenci\'on.
	Mediante el uso de herramientas basadas en reconocimiento del habla, el profesional m\'edico no pierde 
	tiempo ni concentraci\'on en esta tarea.
	\item En el caso de utilizarse un servicio de transcripci\'on, donde un transcriptor profesional toma las notas,
	el beneficio que puede obtenerse en cuanto a costo es significativo. Se estima que para un 
	hospital en EEUU que atiende 45.000 pacientes al a\~no, el uso de herramientas basadas en reconocimiento 
    del habla puede significar un ahorro de 334.500 USD al a\~no \cite{ZickVoice2001}.
\end{itemize}

Aunque la aceptaci\'on y adopci\'on de estas herramientas presentan un panorama alentador \cite{GrassoLong2003}, 
existen a\'un problemas por solucionar. Ciertos inconvenientes pueden pueden llevar a 
la oposici\'on a una automatizaci\'on completa del proceso, por ejemplo: dificultades para insertar 
signos de puntuaci\'on, para ordenar el contenido dictado de acuerdo al formato establecido para el reporte,
entre otros \cite{DavidListening2009}.

Una situaci\'on similar se presenta con respecto a la transcripci\'on reportes legales, 
tema sobre el cual existen publicaciones cient{\'\i}ficas \cite{van-leeuwen2008improving, FalavignaAutomatic2009} 
y para el cual se dispone de productos comerciales en la \mbox{actualidad \cite{NuanceLegal}}.

\section{Milicia}
\label{sec:milicia}

El \foreign{NATO Research Study Group} ha llevado a cabo una serie de experimentos e investigaciones
enfocados a la aplicaci\'on de sistemas de reconocimiento del habla en aplicaciones militares.
Un punto de refencia hist\'orico  es el
trabajo de Beek y otros \cite{BeekAn1977}, en donde se identifican potenciales aplicaciones
de tecnolog\'ias del habla agrupadas en cuatro categor\'ias: seguridad, mando y control, transmisi\'on de datos y comunicaci\'on, y
procesamiento de voz distorsionada. Posteriormente, Weinstein \cite{WeinsteinOpportunities1991} incorpora
a las aplicaciones para entrenamiento como otra categor\'ia. Espec\'ificamente, \cite{PigeonUse2006} indica
las aplicaciones del reconocimiento del habla en la milicia:

\begin{itemize}
    \item Mando y Control, consiste en la interacci\'on humana con computadoras, sistemas
	de armas y sensores, por voz (en aviones de combate o helic\'opteros, por ejemplo). Pero esto
	requiere un alto desempe\~no, en tiempo real, de las tecnolog\'ias de reconocimiento.
    \item Acceso a computadoras e informaci\'on, son una parte crucial para las operaciones militares modernas. El
	reconocimiento del habla puede utilizarse para operar computadoras y consultar informaci\'on utilizando la voz. Esto
	es importante para personal que trabaja bajo mucha presi\'on y tienen la vista y las manos ocupadas.
    \item Inteligencia, implica el procesamiento de una gran variedad de tipos de informaci\'on (texto y audio). El inter\'es militar
	radica en la utilizaci\'on de tecnolog\'as del habla y lenguaje para el an\'alisis y procesamiento
	de la excesiva cantidad de informaci\'on disponible actualmente.
    \item Entrenamiento, consiste en utilizar el reconocimiento del habla en el tareas de entrenamiento de las fuerzas
	militares. Permitiendo que el personal pueda interactuar, mendiante la voz, con sistemas avanzados de simulaci\'on.
\end{itemize}

%!TEX root = ../../tesis.tex
\subsection{Telefon\'ia}
\label{sec:telefonia}

El reconocimiento del habla fue introducido al \'area de las telecomunicaciones a inicios de la d\'ecada
de los 90 por dos motivos: para reducir costos y para generar ganancias en base a nuevos servicios \cite{RabinerApplications1997}.
A continuaci\'on se mencionan aplicaciones relacionadas a esta \'area, sistemas utilizados o sujetos de investigaci\'on, 
organizadas en estas dos categor{\'\i}as. 

Entre las aplicaciones que buscan la reducci\'on de costos pueden citarse:

\begin{itemize}
	\item  \emph{Automatizaci\'on de servicios de Operador}: sistemas como el VRCP de AT\&T 
	y el AABS de Nortel utilizan el reconocimiento del habla para manejar tareas tradicionalmente delegadas
	a un operador como consultas sobre facturaci\'on y llamadas \mbox{asistidas \cite{RabinerApplications1997}}.

	Para 1995, esta aplicaci\'on hac{\'\i}a posible un ahorro de m\'as de 100 millones de d\'olares 
	en \mbox{AT\&T \cite{WilponApplications1994}}.

	\item \emph{Automatizaci\'on de Consulta de Directorio}: sistemas que asisten a los operadores
	para responder una consulta al directorio telef\'onico, reduciendo el espacio de b\'usqueda mediante
	reconocimiento de ciudades o incluso nombres o \mbox{apellidos \cite{RabinerApplications1997}}.

	\item \emph{Enrutamiento de llamadas}: sistemas que permiten dirigir una llamada a la persona adecuada,
	por ejemplo, en el contexto de atenci\'on al \mbox{cliente \cite{Sachs97howmay}}.
\end{itemize}

Entre las aplicaciones que buscan producir ganancias en base a nuevos modelos de servicio se encuentran:

\begin{itemize}
	\item Servicios bancarios: sistemas que ofrecen consulta de balances, transacciones, etc. mediante
	reconocimiento del \mbox{habla \cite{PreeEnhancing1999}}.

	\item Pron\'osticos del clima: sistemas que proveen informaci\'on relacionada al clima a nivel regional o mundial
	en base a una consulta realizada a trav\'es del tel\'efono. Como ejemplo puede citarse el proyecto \foreign{Jupiter}
	de AT\&T \cite{ZueJupiter2000}, entre \mbox{otros \cite{ZibertBiliengual2003}}.

	\item Servicios de reserva: sistemas que permiten realizar reserva de pasajes de avi\'on o tren mediante
	reconocimiento del habla. Como ejemplos pueden citarse el proyecto \foreign{Mercury} de AT\&T \cite{Seneff2000Dialogue} 
    y el proyecto \foreign{Talk`n'Travel} de \mbox{Verizon \cite{StallardEvaluation2001}}.

	\item Traducci\'on Autom\'atica del habla: sistemas de traducci\'on que permiten a dos personas que hablan
	diferentes idiomas mantener una conversaci\'on a trav\'es del tel\'efono. En este punto cabe destacar el
	proyecto Spectra de \mbox{AT\&T \cite{Rangarajan2012}}.
\end{itemize}

Menci\'on especial merece el proyecto Watson de AT\&T \cite{AttWatson}, el cual consiste en un motor de reconocimiento de 
prop\'osito general y una colecci\'on de \emph{plugins}, que incluye funcionalidades de reconocimiento y s{\'\i}ntesis
del habla, reconocimiento de expresiones faciales, entre otras. 

Este proyecto sirve como base para muchos otros en AT\&T. Adem\'as, todas estas funcionalidades est\'an
disponibles para desarrolladores externos mediane una API. 
%!TEX root = ../../tesis.tex
\section{Accesibilidad}
\label{sec:accesibilidad}

Las aplicaciones de reconocimiento del habla toman como entrada la voz del usuario y la transforman en
comandos que pueden ser entendidos por las computadoras. Estos comandos pueden controlar las aplicaciones
y, por lo tanto, sirven como un medio de interacci\'on alternativo.

Las personas con discapacidad auditiva pueden beneficiarse de las aplicaciones de reconocimiento 
del habla que generan autom\'aticamente los subt\'itulos de conversaciones como las que se 
dan en conferencias o en un sal\'on de \mbox{clases \cite{LeitchHow2002}.}

El reconocimiento del habla tambi\'en puede ser \'util para personas con dificultades para usar las 
manos, ya que los medios convencionales de interacci\'on son de poca o nula utilidad para estas
\mbox{personas \cite{AnanthiSurvey2013}.}

En \cite{AnanthiSurvey2013} los autores resaltan el rol que desempe\~na el reconocimiento del habla 
para personas con discapacidad y adem\'as mencionan el tipo de aplicaci\'on que se utiliza, en cada caso, para mejorar la accesibilidad:

\begin{itemize}
    \item Para personas con sordera, existen los tel\'efonos subtitulados. Es un tel\'efono que despliega
    en una pantalla los subt\'itulos de la conversaci\'on actual \cite{PerfettiReading2000}. Estas personas
    podr\'ian utilizar el software de reconocimiento para convertir palabras a texto que puede ser le\'ido 
    o convertido en lenguaje de se\~nas/Braille \cite{SchilperoordNonverbatim2005}.
    \item Las personas ciegas o con dificultades para ver usan varios productos con capacidad de 
    s{\'\i}ntesis del habla, por ejemplo: relojes, calculadoras y computadoras ``parlantes''. 
    Estas computadoras parlantes utilizan software de lectura de pantalla.
    \item Las personas con dificultades para mover los brazos y manos pueden utilizar software de 
    reconocimiento para navegar interfaces usando la voz\cite{AnanthiSurvey2013}.
\end{itemize}

La compa\~n\'ia \foreign{Nuance} provee soluciones de reconocimiento del habla, una parte de \'estas 
en la categor{\'\i}a de Soluciones de Accesibilidad para el Negocio. Estas tecnolog\'ias de asistencia 
permiten que personas con discapacidad puedan ejercer sus funciones en el mundo laboral,
proporcionando: operaci\'on de interfaces mediante voz, transcripci\'on autom\'atica de texto,
y \mbox{m\'as \cite{NuanceAccessibility}.}

%!TEX root = ../../tesis.tex
\section{Industria Automotriz}
\label{sec:automotriz}

La implementaci\'on de sistemas de reconocimiento del habla para su utilizaci\'on en los autom\'oviles representa
tambi\'en un \'area de investigaci\'on activa desde hace varios \mbox{a\~nos \cite{HanaiS94, HuaSpeech2010}}. 
El desarrollo se divide en las siguientes \'areas \cite{Kirriemuir2003Speech}:

\begin{itemize}
	\item Dispositivos manos libres para la utilizaci\'on de tel\'efonos celulares en el interior del veh{\'\i}culo.
	\item Instrucciones mediante voz a los dispositivos de navegaci\'on. Por ejemplo: ``{?`}Qu\'e tan lejos est\'a el siguiente desv{\'\i}o?''.
	\item Interacci\'on por voz con sistema de control del veh{\'\i}culo. Por ejemplo: ``Enciende la radio y sintoniza la emisora X''.
	\item Sistemas de manejo por voz.
\end{itemize}

Como es de esperarse, el \'ultimo punto es el menos desarrollado por cuestiones de seguridad. A\'un as{\'\i}, grandes compa\~n{\'\i}as
de la industria automotriz, como Toyota, financian investigaciones relacionadas al reconocimiento del \mbox{habla \cite{HoshinoSpeech2004}}.

Prueba de la factibilidad y los avances logrados en el \'area son los sistemas de control por voz ya incluidos en los
veh{\'\i}culos de importantes marcas como el \foreign{Blue\&Me} de Fiat \cite{FiatBlue}, el \foreign{Sync} de Ford \cite{FordSync}
y \foreign{Entune} \cite{ToyotaEntune} de Toyota.

\section{Tel\'efonos Inteligentes e Interfaces Web}
\label{sec:smartphones}

Los significativos avances en la computaci\'on y redes inal\'ambricas ha ocasionado
que los esfuerzos de los investigadores se enfoquen en la aplicaci\'on de tecnolog\'ias de
reconocimiento del habla en dispositivos m\'oviles como, por ejemplo, los tel\'efonos 
\mbox{inteligentes \cite{TanAutomatic2008}}.

La arquitectura de los sistemas de reconocimiento del habla de acuerdo al proceso visto en el cap\'itulo \ref{sec:proceso},
puede descomponerse en dos partes: el \foreign{front-end} ac\'ustico, donde se lleva a cabo
la extracci\'on de caracter\'isticas  y el \foreign{back-end} donde se lleva a cabo la decodificaci\'on. 
Zaykovskiy en \cite{ZaykovskiySurvey2006} analiza las tres arquitecturas que se han desarrollado
para aplicar ASR en dispositivos m\'oviles y las clasifica de acuerdo a la ubicaci\'on de las dos partes
mencionadas anteriormente:

\begin{itemize}
    \item Sistemas embebidos de Reconocimiento del habla: este sistema opera completamente en el cliente.
    \item Red de reconocimiento del habla (NSR): este sistema opera completamente en el servidor. 
    \item Reconocimiento del habla distribuido (DSR): en este sistema la extracci\'on de caracter\'isticas se realiza
        en el cliente y el proceso de decodificaci\'on en el servidor.
\end{itemize}

Independientemente de la arquitectura, el reconocimiento del habla en dispositivos m\'oviles dio lugar a una 
serie de aplicaciones. Siri \cite{AppleSiri, OneAccordSiri} y Google Now \cite{GoogleNow} 
son aplicaciones de asistente personal inteligente, permiten responder preguntas,
hacer recomendaciones, realizar acciones, y otras tareas a trav\'es de servicios web. Verbio ASR \cite{VerbioASR} y 
Pocketsphinx \cite{HugginsDainesPocketSphinx2006, PocketSphinxHomePage} son motores de ASR para dispositivos
m\'oviles, que permiten desarrollar distintos tipos de aplicaciones (interfaces multimodales, etc)

En lo que respecta a aplicaciones web, mejorar la accesibilidad es el principal objetivo para los sistemas
de reconocimiento. Una tecnolog\'ia que se encuentra en desarrollo es la \foreign{Web Speech API} \cite{GoogleWebSpeechAPI}, 
que permite a los desarrolladores integrar reconocimiento y s\'intesis del habla a sus aplicaciones web utilizando
una API que forma parte del navegador\footnote{La \foreign{Web Speech API} es una propuesta de Google para integrar
una API de reconocimiento del habla al navegador. Actualmente solo es implementada por el navegador Google Chrome utilizando
los servidores de Google para el procesamiento.}. WAMI \cite{GruensteinWami2008, WamiHome} es otra API JavaScript
para desarrollar interfaces web accesibles mediante la voz. WAMI no proporciona un reconocedor, pero provee
las herramientas necesarias para dar una interfaz web a un motor de reconocimiento (Sphinx, por ejemplo).
%!TEX root = ../../tesis.tex
\section{Videojuegos}
\label{sec:videojuegos}

La integraci\'on del reconocimiento del habla al \'area de los videojuegos es otra l{\'\i}nea de
investigaci\'on existente \cite{SporkaNonSpeech2006, JanickiAutomatic2011}. 
Una aplicaci\'on habitual se da en los juegos educativos, en especial los orientados a ni\~nos 
con dificultades de lenguaje, concentraci\'on o aprendizaje en general; un ejemplo es el juego
comercial Say-N-Play \cite{SayNPlay}.

Gracias a la inclusi\'on de tecnolog{\'\i}as de reconocimiento del habla en los tel\'efonos modernos,
estas pueden ser utilizadas en diferentes aspectos relacionados a los juegos: como mecanismo de entrada 
de comandos, para \foreign{chat} multijugador, o como medio acceso a contenido 
\mbox{exclusivo \cite{JoselliMobile2009}}.

Cabe destacar tambi\'en el rumor reciente de una funcionalidad de control por voz en
la consola de juegos XBOX 720, sucesora de la \mbox{XBOX 360 \cite{IgnXbox}}.


\section{Dom\'otica}
\label{sec:domotica}

La industria de la dom\'otica esta creciendo r\'apidamente dentro del \'area de las tecnolog\'ias de la
comunicaci\'on e informaci\'on, motivada por la necesidad de ayudar a mejorar la autonom\'ia y bienestar 
de las personas discapacitadas o de edad avanzada, especialmente aquellas que viven solas 
\cite{AlshuVoice2011}.

Este crecimiento desarroll\'o el concepto de casas inteligentes, equipadas con sensores, actuadores, 
dispositivos automatizados y software centralizado que se encarga de controlar distintos elementos del 
hogar: persianas, iluminaci\'on, sistemas de alarmas, computadoras, y 
m\'as \cite{LecouteuxSpeech2011, UshaWireless2012}.

El reconocimiento del habla ha recibido limitada atenci\'on en el dominio de las casas inteligentes. 
Lecouteux y sus colaborades presentan en \cite{LecouteuxSpeech2011} el proyecto \foreign{Sweet-Home} que 
tiene como objetivo permitir que personas con ciertos impedimentos f\'isicos controlen su ambiente 
dom\'estico, utilizando la voz como medio de interacci\'on. En este trabajo los investigadores evaluan 
distintas t\'ecnicas para el reconocimientos
de comandos de voz en un ambiente equipado con micr\'ofonos ubicados a la distancia. El mejor desempe\~no 
obtenido presenta una \gls{wer} igual a 11,4\%.

HAL \cite{HAL} de la compa\~n\'ia \foreign{Home Automated Living} es un producto comercial que permite
controlar diversas funciones del hogar mediante la voz o a trav\'es de una interfaz web: 
iluminaci\'on, reproducci\'on de audio y v\'ideo, configuraci\'on de la temperatura, entre otras.
Adem\'as posibilita que los usuarios pregunten por el estado de cotizaciones en la 
bolsa, reporte del clima, resultados deportivos, etc.

% agregar las demas

%!TEX root = ../tesis.tex
\chapter{Proceso B\'{a}sico del Reconocimiento del Habla}
\label{sec:proceso}

A fin de comprender el funcionamiento de las aplicaciones de reconocimiento del habla, puede resultar de utilidad
representar el mismo como un proceso donde se recibe una se\~nal de voz y se obtiene una secuencia de palabras como
resultado. Este proceso est\'a compuesto a su vez por pasos o fases, cada uno con sus respectivas entradas y salidas.

A lo largo de la historia se han propuesto diferentes modelos para el reconocimiento del habla, cada uno
con un enfoque diferente para la soluci\'on de este \mbox{problema \cite{VimalaReview2012}}. 

Este cap{\'\i}tulo presenta y describe el proceso b\'asico del reconocimiento del habla desde un 
enfoque estad{\'\i}stico, es decir, basado en Modelos Ocultos de Markov (\gls{hmm}s por sus siglas en ingl\'es). 
A m\'as de 30 a\~nos de su presentaci\'on, esta metodolog{\'\i}a a\'un predomina y representa el estado 
del arte en el \'area, a tal punto de constituir la base de los sistemas modernos de reconocimiento del habla 
de prop\'osito \mbox{general \cite{BakerResearch2009, VimalaReview2012}}.

Adem\'as, se mencionan y describen brevemente modelos alternativos de reconocimiento del habla de modo a contrastar
las diferencias con el est\'andar actual.

%!TEX root = ../tesis.tex

\section[El Reconocimiento del Habla como Problema Estad{\'\i}stico]
{El Reconocimiento del Habla como Problema \\ Estad{\'\i}stico}
En t\'erminos simples, el reconocimiento del habla pretende determinar las palabras pronunciadas por una persona
a partir de su voz. De modo a comprender los detalles del proceso mediante el cual se obtiene este
resultado es importante, en primer lugar, definir formalmente el problema que se busca resolver.

Para un lenguaje $L$ y una entrada ac\'ustica $X$, el problema del reconocimiento del habla puede 
definirse como \cite{Jurafsky}:

\begin{quote}
\emph{La b\'usqueda de la oraci\'on m\'as probable perteneciente al lenguaje L, dada la entrada ac\'ustica X.}
\end{quote}

La secuencia de observaciones ac\'usticas $O$ puede representarse como:

\begin{align}
O = o_1,o_2,o_3,\ldots,o_T\label{eq:asrO}
\end{align}

donde la se\~nal de voz fue dividida en $T$ muestras de igual duraci\'on.

La oraci\'on de salida, a su vez, puede representarse como:

\begin{align}
W  = w_1,w_2,w_3,\ldots,w_M\label{eq:asrW}
\end{align}

donde la cadena est\'a compuesta por $M$ palabras.

De esta manera, siendo $\hat{W}$ una aproximaci\'on probabil{\'\i}stica de $W$, la definici\'on del reconocimiento del habla puede expresarse matem\'aticamente como:

\begin{align}
\hat{W} = \argmax_{W \in L} P(W|O)
\end{align}

Usando la Regla de Bayes la expresi\'on anterior puede reescribirse como:

\begin{align}
\hat{W} = \argmax_{W \in L} \frac{P(O|W)P(W)}{P(O)}
\end{align}

Se busca la oraci\'on con mayor probabilidad dada una entrada ac\'ustica. Esta entrada es
la misma para todas las oraciones evaluadas, por lo que su probabilidad de ocurrencia $P(O)$ se mantiene constante.
En otras palabras, el t\'ermino $P(O)$ es independiente de $W$, por lo cual puede despreciarse. 

Por tanto:

\begin{align}
\hat{W} = \argmax_{W \in L} P(O|W)P(W)
\end{align}

El t\'ermino $P(O|W)$ representa la probabilidad de una entrada ac\'ustica dada una secuencia de palabras, 
tambi\'en conocida como verosimilitud de observaci\'on o modelo ac\'ustico. El t\'ermino $P(W)$ es la
probabilidad \foreign{a priori} de ocurrencia de una secuencia de palabras, tambi\'en conocida como 
probabilidad previa o modelo de lenguaje. Esto es:

\begin{align}
\hat{W} = \argmax_{W \in L} \overbrace{P(O|W)}^\text{M. ac\'ustico}\overbrace{P(W)}^\text{M. de lenguaje}
\label{eq:asrFundamental}
\end{align}

Esta ecuaci\'on es el fundamento del enfoque estad{\'\i}stico al problema del reconocimiento del habla, base de los
sistemas \mbox{modernos \cite{RabinerStatistical2006}}.

La representaci\'on, y posterior soluci\'on, del problema que se plantea en la 
ecuaci\'on \ref{eq:asrFundamental} se lleva a cabo mediante la aplicaci\'on de la teor{\'\i}a de
los modelos ocultos de Markov. Este concepto, fundamental
para el enfoque estad{\'\i}stico del reconocimiento del habla, se presenta a continuaci\'on.

%!TEX root = ../tesis.tex
\section{Modelos Ocultos de Markov}
\label{sec:hmm}

Para introducir el concepto de modelo oculto de Markov (\gls{hmm} por sus siglas en ingl\'es), 
en el cual se basa el proceso de reconocimiento del habla que se busca explicar, 
se utilizar\'a un ejemplo cl\'asico de la literatura relacionada con el tema.
El ejemplo se denomina \emph{El modelo de la urna y la pelota} \cite{Rabiner89atutorial}:

\begin{figure}[H] 
\centering
\includegraphics[width=0.8\textwidth]{./graphics/urnas.png}
\caption{Representaci\'on gr\'afica del modelo de la urna y la pelota \cite{LleidaModelado}.}
\label{figure:urnas}
\end{figure}

\begin{quote}
	Se asume que hay N urnas (grandes) de vidrio en una habitaci\'on. Dentro de cada urna hay un gran 
	n\'umero de pelotas de colores. Un genio est\'a en la habitaci\'on y, de acuerdo a un proceso aleatorio, elige una urna inicial.

	De esta urna extrae una pelota de manera aleatoria y su color se anota como la observaci\'on, 
	pues el observador desconoce la urna de donde sali\'o la pelota.
	La pelota se coloca de nuevo en la urna y se repiten la selecci\'on de la urna y la pelota respectivamente.
	
	Este proceso genera una secuencia aleatoria de colores.
\end{quote}
\vspace*{1\baselineskip}

En el caso mencionado, pueden distinguirse dos procesos estoc\'asticos:
\begin{itemize}
	\item Del primer proceso se obtiene como salida una secuencia de urnas. Sin embargo, el observador
	no puede visualizar esta secuencia, es decir, la misma permanece oculta.
	\item Del segundo proceso se obtiene como salida una secuencia de colores. La probabilidad de observar
	un color depende de la urna seleccionada previamente, debido a que la cantidad de pelotas de un determinado
	color var{\'\i}a en cada urna.
\end{itemize}

Este sencillo sistema puede modelarse como un modelo oculto de Markov definiendo los siguientes par\'ametros:
\begin{enumerate}
	\item El n\'umero de urnas, a las cuales se denomina estados del modelo.
	\item La cantidad de colores posibles de las pelotas en las urnas, a los cuales se 
	denomina s{\'\i}mbolos observables.
	\item La funci\'on que determina la transici\'on entre urnas.
	\item La funci\'on que determina la elecci\'on de una pelota de determinado color, dada una urna.
	\item La funci\'on que determina la elecci\'on de la urna inicial.
\end{enumerate}

Con la ayuda de este ejemplo, podemos definir formalmente un \gls{hmm}. Un modelo oculto de Markov es un 
aut\'omata finito estoc\'astico entrenable \cite{KouemouHistory2011}, que implica un doble proceso estoc\'astico:

\begin{itemize}
\item El primer proceso, que produce una secuencia de estados, no observable. 
La secuencia de estados que produce permanece oculta.
\item El segundo proceso produce una secuencia de observaciones, donde la probabilidad de 
una observaci\'on est\'a dada por una funci\'on definida para cada estado correspondiente al proceso anterior.
\end{itemize}

Un modelo oculto de Markov puede ser caracterizado mediante los siguientes elementos \cite{Rabiner89atutorial}:

\begin{enumerate}
	\item $N$, el n\'umero de estados del modelo. Se representan los estados individuales como: 
		\begin{align}
			S=\{S_1,S_2,\ldots,S_N\}\label{eq:hmmS}
	\end{align}

	\item $M$, el n\'umero de s{\'\i}mbolos observables por estado. Se representan los s{\'\i}mbolos individuales 
		como: 
		\begin{align}
			V=\{v_1,v_2,\ldots,v_M\}\label{eq:hmmV}
	\end{align}

	\item La distribuci\'on de probabilidad de transici\'on de estados $A = \left\{a_{ij}\right\}$.
		Siendo $q_t$ el estado del modelo generado en el tiempo $t$, puede definirse $a_{ij}$ como:

		\begin{align}
			a_{ij} = P[q_{t+1} = S_j \mid q_t = S_i], & & 1 \leq i,j & \leq N\label{eq:hmmA}
		\end{align}

	\item La distribuci\'on de probabilidad de los s{\'\i}mbolos observables en el estado $j$, $b_j(v_k)$. 
		Esta distribuci\'on es una funci\'on de la observaci\'on $v_k$, definida en cada estado.
		Siendo $q_t$ el estado del modelo generado en el tiempo $t$:

		\begin{align}
			b_j(v_k) = P[v_k \text{ en el momento } t \mid q_t = S_j], & & 1 \leq j & \leq N \label{eq:hmmB}
			\\& & 1 \leq k & \leq M \nonumber
		\end{align}

	\item La distribuci\'on de probabilidad del estado inicial $\pi=\left\{\pi_i\right\}$, donde:
		\begin{align}
			\pi_i = P[q_1=S_i], & & 1 \leq i \leq N \label{eq:hmmPI}
		\end{align}
\end{enumerate}

\begin{figure}[H] 
\centering
\includegraphics[width=0.7\textwidth]{./graphics/hmm.png}
\caption{Representaci\'on gr\'afica de un modelo oculto de Markov.}
\label{figure:hmm}
\end{figure}
\section[Proceso Estad{\'\i}stico del Reconocimiento del Habla]
{Proceso Estad{\'\i}stico del Reconocimiento del \\Habla}


\begin{figure}[H] 
\centering
\includegraphics[width=0.8\textwidth]{./graphics/proceso.png}
\caption{Proceso b\'asico del reconocimiento del habla. Traducido a partir de \cite{VerenichASR}.}
\label{figure:proceso}
\end{figure}

La figura \ref{figure:proceso} ilustra de modo general la arquitectura de un sistema de reconocimiento del habla.
El proceso b\'asico del reconocimiento del habla, puede descomponerse en dos etapas o fases, cada una de las cuales
recibe una entrada (o varias) y produce una salida determinada.

\begin{itemize}
\item La primera fase o \emph{extracci\'on de caracter{\'\i}sticas} tiene como objetivo caracterizar la se\~nal
de voz para obtener una representaci\'on adecuada para el decodificador. Para tal efecto, produce vectores de
caracter{\'\i}sticas espectrales a partir del sonido que recibe como entrada.
\item La segunda fase o \emph{decodificaci\'on} tiene como objectivo producir la secuencia de palabras m\'as probable
dados los vectores de caracter{\'\i}sticas resultantes de la fase anterior. Para ello se sirve un modelo ac\'ustico, un
modelo de lenguaje y un algoritmo de decodificaci\'on.
\end{itemize}

Las siguientes secciones explican de manera m\'as detallada los conceptos y algoritmos relacionados a cada fase.
Tanto el modelo ac\'ustico como el modelo de lenguaje pueden requerir una fase de entrenamiento, 
previa a la utilizaci\'on del sistema de reconocimiento del habla.
Por motivos de claridad, los detalles de esta fase se presentan en \'ultimo lugar.

\subsection{Fase 1: Feature Extraction}
\label{sec:featureExtraction}

%!TEX root = ../tesis.tex
\section{Fase 2: Decodificaci\'on}
\label{sec:decoding}

En la fase de decodificaci\'on se utiliza un algoritmo decodificador, que depende de un modelo de lenguaje 
y el modelo ac\'ustico, para obtener la secuencia de palabras m\'as probable en base a los vectores 
de caracter{\'\i}sticas que se calcularon en la etapa anterior. 
Esta secci\'on describe los elementos involucrados en esta transformaci\'on, la cual constituye el paso
final del proceso simplificado del reconocimiento del habla que se pretende presentar.

\subsection{Modelo de Lenguaje}
Un modelo de lenguaje busca predecir la probabilidad de una secuencia de palabras pertenecientes a un lenguaje.

Formalmente, sea $V$ el conjunto finito de palabras que componen un lenguaje, tambi\'en conocido como 
vocabulario, y $V^\dag$ el conjunto infinito de oraciones que pueden formarse con palabras pertenecientes 
al vocabulario.

Un modelo de lenguaje \cite{CollinsLanguage} consiste en un conjunto finito $V$ y una funci\'on 
de probabilidad $P(x_1,x_2,\ldots,x_n)$ tal que:
\begin{enumerate}

\item $\forall (x_1,x_2,\ldots,x_n) \in V^\dag, P(x_1,x_2,\ldots,x_n) \ge 0$

\item $\displaystyle \sum_{(x_1,x_2,\ldots,x_n) \in V^\dag} P(x_1,x_2,\ldots,x_n) = 1$
\end{enumerate}


En resumen, un modelo de lenguaje define la probabilidad de ocurrencia de una secuencia de palabras
$x_1,x_2,\ldots,x_n$ para un lenguaje dado.

La t\'ecnica basada en n-gramas es la predominante para construir modelos de lenguaje, 
debido a su simplicidad y efectividad \cite{GaoComparative2010}.
Sea $W^L_1$ una cadena de $L $ palabras pertenecientes a un vocabulario $V$ dado. 
Un modelo de lenguaje basado en n-gramas asigna la probabilidad a $W^L_1$ de acuerdo a:

\begin{align}
p(W^L_1) = \displaystyle \prod^L_{i = 1} w_i \mid w^{i - 1}_1 \approx \displaystyle \prod^L_{i = 1} w_i \mid w^{i - 1}_{i - n + 1}
\end{align}

Esta aproximaci\'on se basa en la suposici\'on de Markov de que cada palabra depende solo de las $n - 1$ palabras precedentes, lo cual disminuye significativamente la complejidad del c\'alculo de la probabilidad que se busca para
secuencias de gran longitud.

A modo de ejemplo, se propone la tarea de calcular la probabilidad de la frase \mbox{``El hombre corre''}
utilizando un modelo basado en bigramas ($n=2$).

Sean:
\begin{itemize}
 	\item $\text{\textless} s\text{\textgreater}$ un s{\'\i}mbolo especial utilizado para indicar el inicio 
 	de la secuencia de palabras.
  	\item $\text{\textless} /s\text{\textgreater}$ un s{\'\i}mbolo especial utilizado para indicar el fin 
  	de la secuencia de palabras.
\end{itemize} 

Aplicando la f\'ormula presentada anteriormente, se tiene que:
\begin{equation*}
p(\text{el, hombre, corre}) = p(el \mid \text{\textless} s\text{\textgreater}) \, 
p(\emph{\text{hombre}} \mid el) \, p(corre \mid \emph{\text{hombre}}) \, 
p(\text{\textless} /s\text{\textgreater} \mid corre)
\end{equation*}


Otro tipo de modelo de lenguaje es el basado en una gram\'atica. Las gram\'aticas tienen como ventaja que pueden
utilizarse sin entrenamiento previo. Su principal desventaja est\'a en la dificultad de definir gram\'aticas formales
para lenguajes complejos \cite{Wang2000}.

Un modelo de lenguaje basado en una gram\'atica tradicional asigna una probabilidad de 0 o 1 a cualquier secuencia,
dependiendo de si esta puede o no derivarse a partir de las reglas de la gram\'atica. Sin embargo, si se cuenta con
datos de entrenamiento, es posible asignar una probabilidad a cada regla de modo a mejorar 
la estimaci\'on \cite{huang-handbook10}.

A modo de ejemplo, la siguiente gram\'atica podr{\'\i}a utilizarse como modelo de lenguaje para un sistema simple
de reconocimiento del habla para el acceso a informaci\'on:

\begin{bnf*}
\bnfprod{pregunta}
{\bnfts{Cu\'al} \bnfsp \bnfts{es} \bnfsp \bnfts{la} \bnfpn{info} \bnfsp  \bnfts{en} \bnfsp \bnfpn{ciudad}} \\
\bnfprod{info}
{\bnfts{temperatura} \bnfor \bnfts{presi\'on atmosf\'erica} \bnfor \bnfts{hora}} \\
\bnfprod{ciudad}
{\bnfts{Par{\'\i}s} \bnfor \bnfts{Nueva York} \bnfor \bnfts{Roma}}
\end{bnf*}

\subsection{Modelo Ac\'ustico}
El modelo ac\'ustico permite estimar el t\'ermino $P(O|W)$ de la ecuaci\'on \ref{eq:asrFundamental}, es decir,
la probabilidad de una entrada ac\'ustica dada una secuencia de palabras.
El mismo representa un factor cr{\'\i}tico para la precisi\'on de los resultados obtenidos, y puede decirse que
se trata del componente central de cualquier sistema de reconocimiento del habla \cite{huang-handbook10}.

Es en este elemento del reconocedor en el cual se aplica el concepto de modelos ocultos de Markov que se present\'o
anteriormente. Un modelo ac\'ustico est\'a compuesto por varios modelos ocultos de Markov, cada uno con
los siguientes par\'ametros:


\begin{enumerate}[A)]
	\item \textbf{Estados}


	Un fonema es la unidad b\'asica de sonido capaz de alterar el significado de una palabra \cite{Armbruster2003}.
	Los fonemas a su vez pueden dividirse en unidades subfon\'eticas, las cuales resultan m\'as adecuadas para
	caracterizar las transiciones entre fonemas propias del habla.

	A menudo se elige dividir los fonemas en tres partes para los sistemas de reconocimiento del habla.
	De esta manera, la primera parte depende del fonema anterior, la segunda representa al fonema en cuesti\'on
	y la tercera depende del fonema siguiente \cite{CMUConcepts}.

	Estas unidades fon\'eticas, fonemas o subfonemas, corresponden a los estados del modelo oculto de Markov.
	El decodificador permite obtener la secuencia de estados m\'as probable dada una secuencia de observaciones.
	Esta secuencia de estados $Q$ puede representarse como:

	\begin{align}
		Q = q_1,q_2,q_3,\ldots,o_T\label{eq:hmmQ}
	\end{align}

	Donde:
	\begin{itemize}
		\item $T$ es el n\'umero de observaciones.
		\item $q_1,q_2,q_3,\ldots,q_T \in S$ es la secuencia de estados.
	\end{itemize}

	La salida del decodificador supone un problema, teniendo en cuenta que el resultado deseado es la secuencia 
	de palabras $W$ de la ecuaci\'on \ref{eq:asrW}, no la de fonemas $Q$.

	El inconveniente anterior se soluciona mediante el diccionario fon\'etico, el cual contiene correspondencias entre palabras y secuencias de fonemas. El mismo forma parte del modelo ac\'ustico \cite{huang-handbook10}.

	\item \textbf{Observaciones}


	Las caracter{\'\i}sticas espectrales de la onda sonora representan los s{\'\i}mbolos observables del habla.
	Los vectores de caracter{\'\i}sticas que se calcularon en la etapa anterior corresponden a las observaciones
	del modelo oculto de Markov.

	Esto es, siendo $O$ la secuencia de observaciones de la ecuaci\'on \ref{eq:asrO} 
	y $V$ el conjunto de s{\'\i}mbolos observables de la ecuaci\'on \ref{eq:hmmV}, $o_1,o_2,o_3,\ldots,o_T \in V$.

	\item \textbf{Distribuciones de Probabilidad}

	Sea $o_t$ el vector de caracter{\'\i}sticas que se observa en el tiempo $t$, es decir, uno de los elementos
	de la ecuaci\'on \ref{eq:asrO}.
	La probabilidad de una observaci\'on como consecuencia de la pronunciaci\'on de un determinado fonema 
	corresponde a la funci\'on $b_j(v_k)$ del modelo oculto de Markov, definida en la ecuaci\'on \ref{eq:hmmB}.
	Por tanto, la probabilidad de observaci\'on de $o_t$ est\'a dada por $b_j(o_t)$ 

	Como un vector de caracter{\'\i}sticas puede tomar un gran n\'umero de valores, los m\'etodos utilizados con
	mayor frecuencia en la actualidad coinciden en tratarlo como una variable continua.

	Una \gls{fdp} de una variable aleatoria continua describe la probabilidad relativa seg\'un la cual 
	dicha variable aleatoria tomar\'a determinado valor \cite{Evans2011}.

	El m\'etodo m\'as extendido para calcular $b_j(o_t)$ se basa en funciones Gaussianas para definir 
	una \gls{fdp} \cite{Jurafsky}.
	La versi\'{o}n simple del m\'etodo Gaussiano asume que los valores del vector de observaciones $o_t$ presentan una distribuci\'on normal. A continuaci\'on, se presenta la definici\'on formal 
	de $b_j(o_t)$ para este m\'etodo.

	Sean:

	\begin{itemize}
		\item $\mu_j$ el vector media de la curva Gaussiana.
		\item $\Sigma_j$ la matriz covarianza de la curva Gaussiana.
	\end{itemize}

	\begin{align}
    	b_j(o_t) = \frac{1}{\sqrt{(2\pi)|\Sigma_j|}}e^{[(o_t-\mu_j)'\Sigma_j^{-1}(o_t-\mu_j)]}\label{eq:hmmGaussian}
	\end{align}

	En la pr\'actica, la mayor{\'\i}a de los reconocedores utiliza m\'as de una funci\'on Gaussiana, combinando
	los valores mediante una t\'ecnica conocida como mezclas Gaussianas \cite{huang-handbook10}.

	Existen adem\'as otros m\'etodos para definir la probabilidad de observaci\'on, como la cuantizaci\'on
	vectorial \cite{Burton1983} y las redes neuronales \cite{KristineApplying1995}.

	En general, los par\'ametros probabil{\'\i}sticos del \gls{hmm} se calculan durante la fase de entrenamiento,
	antes de la utilizaci\'on del sistema de reconocimiento del habla. Estos valores incluyen:

	\begin{itemize}
		\item La probabilidad de observar un determinado fonema al inicio de la oraci\'on: $\pi_i$.
		\item La probabilidad de transici\'on entre fonemas: $a_{ij}$.
		\item Los par\'ametros relacionados a $b_j(v_k)$. 

		En el caso del m\'etodo Gaussiano simple: los par\'ametros $\mu_j$ y $\Sigma_j$ de la 
		ecuaci\'on \ref{eq:hmmGaussian}.
	\end{itemize}

\end{enumerate}

\begin{figure}[H] 
\centering
\includegraphics[width=0.5\textwidth]{./graphics/hmm_palabra.png}
\caption{Posible representaci\'on ac\'ustica de la palabra ``mes''. Basado en \cite{Jurafsky}.}
\label{figure:hmm-palabra}
\end{figure}

La cantidad de modelos ocultos de Markov depende del enfoque del modelo ac\'ustico, pudiendo darse 
que \cite{Livescu2012}:

\begin{itemize}
 	\item El modelo define un \gls{hmm} por palabra:


 		Este tipo de modelo requiere de ejemplos de pronunciaci\'on de cada palabra del lenguaje para 
 		su entrenamiento, por lo cual es raramente utilizado en la pr\'actica.
 	\item El modelo define un \gls{hmm} por fonema:


 		Este tipo de modelo permite el reconocimiento
 		de palabras para las cuales no se cuenta con ejemplos de pronunciaci\'on.
 	\item El modelo define un \gls{hmm} por fonema, dependiendo del contexto:


 		El sonido particular de un fonema puede variar de acuerdo al fonema anterior y al siguiente.
 		Este tipo de modelo define un \gls{hmm} para cada una de esas variaciones, y es utilizado
 		en la mayor{\'\i}a de los sistemas de reconocimiento del habla \cite{Odell95theuse}.
 \end{itemize}  

\subsection{Espacio de B\'usqueda}
El objetivo de la fase de decodificaci\'on puede ser considerado esencialmente como un problema de 
b\'usqueda \cite{huang-handbook10}.
El algoritmo decodificador intenta encontrar la secuencia de estados m\'as probable dada la secuencia de
vectores de caracter{\'\i}sticas que recibe como entrada, en un \'unico \gls{hmm} que constituye su 
espacio de b\'usqueda.

Tomando como ejemplo el modelo de la urna y la pelota de la secci\'on \ref{sec:hmm}, el algoritmo decodificador
busca la secuencia de urnas m\'as probable dada la secuencia de colores que se observ\'o.

Teniendo en cuenta que el modelo ac\'ustico est\'a constituido por varios modelos ocultos de Markov, se plantea
la necesidad de definir un \'unico \gls{hmm} para el algoritmo decodificador. Suponiendo que el modelo ac\'ustico
define un \gls{hmm} por fonema, esto puede hacerse de la siguiente manera:

\begin{enumerate}
	\item Se concatenan los modelos ocultos de Markov de los fonemas de manera a obtener un \gls{hmm} 
	por cada palabra. 
	La probabilidad de transici\'on entre fonemas est\'a dada por el diccionario fon\'etico.
	\item Se concatenan los modelos ocultos de Markov de las palabras de manera a obtener un \'unico \gls{hmm} que
	constituye el espacio de b\'usqueda. La probabilidad de transici\'on entre palabras est\'a dada por el modelo
	de lenguaje.
\end{enumerate}

\begin{figure}[H] 
\centering
\includegraphics[width=0.5\textwidth]{./graphics/espacio.png}
\caption{Espacio de b\'usqueda para un lenguaje simple de cuatro palabras. Traducido a partir de \cite{RenalsSearch}.}
\label{figure:espacio-busqueda}
\end{figure}



\subsection{Algoritmo de Viterbi}
Una vez calculados los vectores de caracter{\'\i}sticas, y habiendo construido el espacio de b\'usqueda, 
se cuenta con todos los elementos necesarios para el algoritmo decodificador. Un algoritmo frecuentemente 
utilizado para la decodificaci\'on se denomina algoritmo de Viterbi.

El algoritmo de Viterbi, tomando como entrada una secuencia de observaciones y un \'unico aut\'omata, 
encuentra el camino \'optimo a trav\'es del aut\'omata, esto es, la mejor secuencia de estados. 
La descripci\'on y el pseudoc\'odigo que se presentan en esta secci\'on
est\'an basados principalmente en \cite{Jurafsky, Rabiner89atutorial}.

M\'as formalmente, se busca la mejor secuencia de estados $Q = (q_1,q_2,\ldots,q_T)$ 
dados una secuencia de observaciones $O = (o_1,o_2,\ldots,o_T)$ y un modelo oculto de Markov, $\lambda$.

El algoritmo utiliza una matriz de probabilidades $viterbi$, donde cada celda $viterbi[i,t]$ 
contiene la probabilidad del mejor camino teniendo en cuenta las $t$ primeras observaciones y 
terminando en el estado $i$ del modelo. Esto es:

\begin{align}
	viterbi[i,t] = \displaystyle \max_{q_1,q_2,\ldots,q_{t - 1}} P(q1,q2,\ldots,q_{t - 1},
		q_t = i,o_1,o_2,\ldots,o_t \mid \lambda) 	
\end{align} 

Para calcular los valores de $viterbi[i,t]$, el algoritmo de Viterbi asume la invariante de la 
programaci\'on din\'amica. Esto es, se asume que si el mejor camino para una secuencia de observaciones 
pasa por un estado $q_i$, entonces este camino incluye el mejor camino hasta $q_i$ inclusive. 
Este supuesto, aunque no siempre sea correcto, permite descomponer el problema y simplificar su soluci\'on,
mediante la siguiente relaci\'on de recurrencia:

\begin{align}
	viterbi[i,t] = \displaystyle \max_i (viterbi[i,t-1]a_{i,j})b_j(o_t)
\end{align}

\begin{algorithm}[H]
\caption{Algoritmo de Viterbi} \label{viterbi}
\begin{algorithmic}[1]
\REQUIRE $observaciones$ de longitud $T$, $grafo\mbox{-}estados$.
\ENSURE $estados$, el mejor camino.
\STATE $num\mbox{-}estados \leftarrow$ CANTIDAD-DE-ESTADOS($grafo\mbox{-}estados$) 
\STATE Crear una matriz de probabilidades $viterbi[num\mbox{-}estados, T]$
\FOR{cada estado $s$ desde $0$ hasta $num\mbox{-}estados$}
	\STATE $viterbi[s,0] = \pi_s$
\ENDFOR
\FOR{cada paso $t$ desde $0$ hasta $T - 1$}
	\FOR{cada estado $s$ desde $0$ hasta $num\mbox{-}estados$}
		\FOR{cada transici\'on $s$ desde s especificada por el $grafo\mbox{-}estados$}
		\STATE $nuevo\mbox{-}puntaje \leftarrow viterbi[s,t] * a[s,s'] * b_{s'}[o_t]$
		\IF{$viterbi[s',t+1] = 0 \parallel nuevo\mbox{-}puntaje > viterbi[s',t+1]$}
			\STATE $viterbi[s',t+1] \leftarrow nuevo\mbox{-}puntaje$
			\STATE $puntero\mbox{-}retroceso[s',t+1] \leftarrow s$
		\ENDIF  
		\ENDFOR
	\ENDFOR
\ENDFOR
\STATE $estados \leftarrow$ retroceso desde la celda con mayor valor en la \'ultima columna de $viterbi[]$
\\ \COMMENT{Usando $puntero-retroceso$}.
\RETURN $estados$
\end{algorithmic}
\end{algorithm}

El algoritmo de Viterbi procesa por completo el tiempo $t$ antes de continuar con el tiempo $t + 1$, por
lo cual se dice que es un algoritmo s{\'\i}ncrono en el tiempo \cite{huang-handbook10}.

En la pr\'actica, para vocabularios extensos, el espacio de b\'usqueda resulta demasiado grande.
Como soluci\'on a este problema suelen podarse los caminos poco probables en cada tiempo $t$,
utilizando una t\'ecnica conocida como b\'usqueda en haz \cite{Jurafsky}.

\subsection{Algoritmo A*}

Otro algoritmo que puede utilizarse para la decodificaci\'on es el algoritmo A*. Este algoritmo
utiliza una funci\'on heur{\'\i}stica, que debe definirse, para elegir los caminos a expandir
durante el proceso de b\'usqueda \cite{Russell2003Solving}. 
El problema de definir una heur{\'\i}stica adecuada a\'un no ha sido resuelto, aunque existen soluciones propuestas.

El algoritmo A* puede calcular probabilidades correspondientes al tiempo $t + 1$ sin haber completado 
las del tiempo $t$, por lo cual es considerado un algoritmo as{\'\i}ncrono en el tiempo.

Con una heur{\'\i}stica adecuada, el algoritmo A* puede utilizarse para espacios de b\'usqueda 
muy grandes \cite{huang-handbook10}.
A\'un as{\'\i}, el algoritmo de Viterbi con b\'usqueda en haz es el m\'etodo utilizado con mayor frecuencia
en los sistemas de reconocimiento del habla \cite{huang-handbook10}. 
Esto se debe principalmente a la ventaja en cuanto a eficiencia en t\'erminos de tiempo, 
relacionada a la b\'usqueda s{\'\i}ncrona en el tiempo \cite{huang-handbook10}.


\subsection{Resumen}

\begin{figure}[H] 
\centering
\includegraphics[width=0.8\textwidth]{./graphics/decodificacion.png}
\caption{Fase de decodificaci\'on. Gr\'afico basado en \cite{VerenichASR}.}
\label{figure:hmm}
\end{figure}

La entrada de la fase de decodificaci\'on es el conjunto de vectores de caracter{\'\i}sticas espectrales,
los cuales se calcularon en base a la entrada ac\'ustica durante la fase de extracci\'on de caracter{\'\i}sticas.

El modelo de lenguaje busca predecir la probabilidad de una secuencia de palabras pertenecientes a un lenguaje.
Dos tipos de modelo de lenguaje com\'unmente utilizados en el reconocimiento del habla son los basados en
n-gramas y los basados en gram\'aticas.

El modelo ac\'ustico permite estimar la probabilidad de una entrada ac\'ustica dada una secuencia de palabras.
El modelo ac\'ustico est\'a formado por varios modelos ocultos de Markov y el diccionario fon\'etico.

El algoritmo decodificador busca la secuencia de fonemas o subfonemas m\'as probable dados una 
secuencia de observaciones y un modelo oculto de Markov. Este \'unico modelo oculto de Markov se construye
mediante los componentes del modelo ac\'ustico y el modelo de lenguaje. El algoritmo de Viterbi y el 
algoritmo A* son dos algoritmos decodificadores utilizados frecuentemente en el reconocimiento del habla.

La secuencia de fonemas o subfonemas puede convertirse a una secuencia de palabras utilizando el diccionario
fon\'etico que forma parte del modelo ac\'ustico.

La salida de la fase de decodificaci\'on es la secuencia de palabras m\'as probable dados los vectores
de caracter{\'\i}sticas espectrales, los cuales se calcularon en base a la entrada ac\'ustica durante
la fase de extracci\'on de caracter{\'\i}sticas. Con esto llega a su culminaci\'on el proceso b\'asico
del reconocimiento del habla.
%!TEX root = ../tesis.tex
\subsection{Fase Previa: Entrenamiento}
\label{sec:training}

Para que el proceso que se ha presentado en las secciones anteriores funcione, es necesaria una fase
previa de entrenamiento de los modelos probabil{\'\i}sticos que componen el reconocedor del habla. 

Los cuatro modelos probabil{\'\i}sticos que deben entrenarse son \cite{Jurafsky}:
\begin{itemize}
	\item Modelo oculto de Markov L\'exico: la estructura del grafo de estados
	\item Probabilidades del modelo de lenguaje: $P(W)$
	\item Verosimilitudes de observaci\'on: $b_j(o_t)$
	\item Probabilidades de transici\'on: $a_{ij}$
\end{itemize}

Para ello se cuenta con \cite{Jurafsky}:
\begin{itemize}
	\item  Un corpus de voz, compuesto por una colecci\'on de grabaciones de voz junto
	con sus transcripciones de texto.
	\item  Un corpus mucho mayor de texto para entrenar el modelo de lenguaje, compuesto por las transcripciones
	del corpus de voz junto con otros textos similares.
	\item A menudo, un corpus menor de voz etiquetado fon\'eticamente. Esto es, donde fragmentos
	de la se\~nal est\'an asociados a su fonema correspondiente.
\end{itemize}

\emph{Estructura del grafo de estados} 

Se construye en base a diccionarios de pronunciaci\'on como el diccionario \mbox{CMUDdict \cite{CMUdict}}. 
Los estados se definen de acuerdo a los fonemas, o en muchos casos de acuerdo a subdivisiones de estos, denominadas subfonemas.

\emph{Modelo de lenguaje} 

Un modelo basado en n-gramas se entrena contando las ocurrencias de cada n-grama en el corpus de texto, para luego 
suavizar y normalizar cada conteo. As{\'\i}, el contador suavizado y normalizado de un n-grama constituye 
su \mbox{probabilidad \cite{CollinsLanguage}}.

\emph{Par\'ametros del modelo oculto de Markov} 

Las probabilidades de estimaci\'on se estiman inicialmente asumiendo que, para cada estado, cualquier transici\'on posible 
a otro estado es igualmente probable. Las verosimilitudes de observaci\'on se estiman inicialmente mediante el peque\~no corpus 
de voz etiquetado fon\'eticamente.

Una vez realizada la estimaci\'on inicial, el siguiente paso para entrenar estos par\'ametros difiere dependiendo de la utilizaci\'on
de redes neuronales o funciones Gaussianas \cite{Jurafsky}. A continuaci\'on se presenta brevemente el enfoque para cada caso, 
incluyendo referencias a materiales que desarrollan los conceptos de manera m\'as extendida:

\begin{itemize}
	\item \emph{Redes Neuronales}: el entrenamiento de la red neuronal se realiza utilizando el algoritmo de Propagaci\'on hacia 
	Atr\'as \cite{Russell2003Learning}. Este algoritmo requiere conocer el fonema correspondiente a cada observación. 

	Para ello se utiliza el alineamiento Viterbi forzado \cite{JelinekStatistical1998}, el cual recibe los vectores de características y la secuencia de palabras correctas y produce la mejor secuencia de estados, donde cada estado puede alinearse a un vector. 

	Se le denomina \emph{forzado} debido a que puede verse como el algoritmo de Viterbi con una restricción más: debe encontrar el mejor camino 
	\emph{que pase por una secuencia de palabras dada}.

	\item \emph{Distribuci\'on Gaussiana}: para el caso de las funciones de distribución de probabilidad Gaussianas, se utiliza
	el algoritmo de \mbox{Avance-Retroceso \cite{Jurafsky}}.

	Este algoritmo estima dos probabilidades, llamadas \emph{de avance} y \emph{de retroceso} dadas las estimaciones iniciales de
	$a$ y $b$. Luego, en base a estas probabilidades se mejora la estimación. Este proceso se repite hasta que los valores convergen.
\end{itemize}






\section{Otros Modelos de Procesos}
\label{sec:otrosModelos}

Otros enfoques han sido desarrollados para el problema de reconocimiento del habla. En las secciones que siguen
se presentan algunos de estos modelos.

\subsection{Distorsi\'on Din\'amica Temporal}
\label{sec:dtw}

La \gls{dtw} es una t\'ecnica t\'ipica del enfoque basado en comparaci\'on de patrones \cite{GaikwadAReview2010}. 
Este m\'etodo permite encontrar una alineaci\'on \'optima 
entre dos secuencias que pueden variar en tiempo y velocidad, bajo ciertas restricciones \cite{MullerInformation2007}.
Las secuencias se distorsionan de manera no lineal con respecto a la dimensi\'on tiempo para determinar una medida
de similitud independiente a ciertas variaciones no lineales que ocurran sobre la dimensi\'on tiempo \cite{AnusuyaSpeech2009}.

Este algoritmo forma un espacio de bidimensional $C$, denominado matriz de costo, para dos secuencias $X$ e $Y$ que varian en 
el tiempo. El objetivo es encontrar
una alineaci\'on entre $X$ e $Y$ de costo total (suma de distancias locales entre elementos) m\'inimo. 
Formalmente a esta alineaci\'on se la conoce como trayectoria de
deformaci\'on, consiste en una secuencia que satisface tres condiciones: restricci\'on de punto final, 
monoton\'ia y tama\~no de paso \cite{MullerInformation2007}. En la figura~\ref{figure:dtw} se puede observar
una representaci\'on gr\'afica del problema.

\begin{figure}[H]
\centering
\includegraphics[width=0.9\linewidth]{./graphics/dtw.png}
\caption[Alineaci\'on de secuencias mediante distorsi\'on din\'amica temporal.]{A la izquierda se pueden observar 
	dos secuencias similiares pero desfasadas. Para alinear estas secuencias se construye la matriz $C$ y
	se busca la trayectoria de deformaci\'on. Gr\'afico modificado de \cite{ThanawinDTW}.}
\label{figure:dtw}
\end{figure}

\gls{dtw} ha probado ser un m\'etodo eficiente para el reconocimiento de palabras pronunciadas 
pausadamente \cite{MyersALevel1981} y ha sido adaptado para el reconocimiento de palabras enlazadas,
con cierto nivel de \mbox{\'exito \cite{MyersALevel1981, SakoeTwoLevel1979, RabinerApplication1980}}.

\subsection{Redes Neuronales}
\label{sec:otrosModelosANN}

Como se mencion\'o anteriormente el concepto de Redes Neuronales Artificiales (RNA) aplicado a el reconocimiento
del habla se volvi\'o a introcudir en los a\~nos 80, 
y viene desarroll\'andose como m\'etodo alternativo desde entonces. Las RNA son capaces de resolver tareas de reconocimiento
m\'as complicadas, pero no escalan tan bien como los HMM cuando se trata de grandes vocabularios \cite{VimalaReview2012}.

Existen dos enfoques b\'asicos para la clasificaci\'on del habla utilizando redes neuronales: est\'atico y din\'amico. En la
clasificaci\'on est\'atica la red ve toda la entrada de voz a la vez, y toma una sola decisi\'on. Por otro lado, en el enfoque
din\'amico la red ve solo una peque\~na ventana de la entrada, y esta ventana se desliza sobre la entrada mientras la red
realiza una serie de decisiones locales, que deben ser integradas a una decisi\'on global posteriormente \cite{TebelskisSpeech1995}.

La clasificaci\'on de fonemas, el reconocimiento de palabras pueden llevarse a cabo con un alto
grado de precisi\'on utilizando m\'etodos est\'aticos o din\'amicos. Aunque el enfoque din\'amico, aplicado al reconocimiento de
palabras, se adapta mejor a la variabilidad de duraci\'on de una palabra \cite{TebelskisSpeech1995}.

Existen tambi\'en sistemas h\'ibridos RNA-HMM que utilizan las Redes Neuronales para el reconocimiento de fonemas y los Modelos
Ocultos de Markov para el modelo de lenguaje \cite{VimalaReview2012}.


%!TEX root = ../tesis.tex
 \chapter{Tecnolog\'ias y Herramientas}
\label{sec:tecnologias}

En la actualidad existe un gran n\'umero de herramientas de software que pueden utilizarse para la implementaci\'on de un
sistema de reconocimiento del habla. Si bien esta diversidad representa una ventaja, esto requiere
la selecci\'on de la tecnolog{\'\i}a adecuada. Es selecci\'on conforma la primera tarea a llevar a
cabo por las personas que deseen trabajar en esta \'area.

Este cap{\'\i}tulo presenta, categoriza y eval\'ua varias alternativas tecnolog{\'\i}cas disponibles, relacionadas
al reconocimiento del habla, con el fin de facilitar la elecci\'on de la herramienta adecuada
de acuerdo a las necesidades de cada caso. La siguiente clasificaci\'on y criterios que ser\'an presentados son
una propuesta que forma parte de este trabajo de grado.

Inicialmente, se clasifican las herramientas en las siguientes categor{\'\i}as:
\begin{itemize}
	\item Aplicaciones: herramientas software que permiten al usuario realizar una o
	varias tareas \mbox{espec{\'\i}ficas \cite{GoodwillComputer}}.
    \item \gls{api}: interfaces entre una base de c\'odigo y
	el programador \cite{DoucetteOnApi}. Esta abstracci\'on puede verse como una caja negra
	(los detalles de la implementaci\'on se mantienen ocultos) que provee una determinada funcionalidad.
	\item Librer{\'\i}as/Frameworks: con junto de m\'etodos o funciones a los que se puede invocar.
	En el caso de un framework, incluye tambi\'en patrones de dise\~no
	\mbox{predefinidos \cite{FowlerInversion}}.
\end{itemize}

Las categor{\'\i}as citadas anteriormente representan, diferentes enfoques para realizar el trabajo
de reconocimiento del habla. A modo de discernir entre los mismos, se establecen
los siguientes criterios generales de evaluaci\'on:
\begin{itemize}
	\item Conocimiento t\'ecnico necesario: nivel de conocimiento relativo al \'area necesario para la
	utilizaci\'on de la herramienta. 
	\item Productividad: raz\'on entre las funcionalidades que pueden implementarse,o tareas que pueden
	realizarse, utilizando la herramienta y recursos que toma la implementaci\'on.
	\item Flexibilidad: facilidad de adaptaci\'on de la herramienta para la soluci\'on de diferentes
	tipos de problemas.
\end{itemize}

La siguiente figura presenta de forma esquematizada el contenido del cap{\'\i}tulo, y puede tomarse como
gu\'ia para la lectura de las secciones del mismo.

\begin{figure}[H]
\centering
\includegraphics[width=1\linewidth]{./graphics/esquema-herramientas.png}
\caption{Esquema que presenta el contenido del cap\'itulo}
\label{figure:esquema-herramientas}
\end{figure}

\section{Aplicaciones}
\label{sec:aplicaciones}

% introduccion

Las soluciones de reconocimiento del habla, catalogadas como aplicaciones, propocionan
a los usuarios finales los medios necesarios para realizar determinadas tareas mediante
la voz: dictado autom\'atico, navegaci\'on de interfaces, etc\'etera; sin
la necesidad de tener un conocimiento previo de los conceptos relacionados al
reconocimiento del habla. \'Estas aplicaciones presentan las siguientes 
caracter\'isticas en cuanto a los criterios generales de evaluaci\'on:

\begin{itemize}
    \item Conocimiento t\'ecnico: requieren de poco conocimiento t\'ecnico debido a que \'estas
        soluciones se ofrecen como un producto a los usuarios finales, por lo tanto, sus funcionalidades
        pueden utilizarse directamente sin necesidad de entender los componentes que intervienen
         en el sistema.
    \item Productividad: teniendo en cuenta el criterio anterior, cabe resaltar que con
        las aplicaciones se puede lograr un buen grado de productividad, porque los esfuerzos
        de los usuarios se enfocan en realizar determinadas tareas utilizando las funcionalidades
         disponibles.
    \item Flexibilidad: la flexibilidad para esta categor\'ia de soluciones es reducida, porque las
        funcionalidades que ofrece cada aplicaci\'on se encuentran orientadas a resolver tareas
        espec\'ificas: dictado autom\'atico, para la transcripci\'on de documentos; reconocimiento
        de comandos, para navegar interfaces; reconocimiento del habla, para 
        automatizaci\'on de servicios de operador. Cada aplicaci\'on tiene un \'ambito de aplicabilidad
        lo cual impone un l\'imite en su flexibilidad, en comparaci\'on a otras categor\'ias que
         se ver\'an m\'as adelante.
\end{itemize}

\subsection{Criterios Espec\'ificos de Evaluaci\'on}

Los criterios espec\'ificos nos permiten evaluar a las aplicaciones en cuanto a factores
particulares y relevantes para esta categor\'ia. A continuaci\'on se presentan los criterios 
espec\'ificos de evaluaci\'on para las aplicaciones:

\begin{itemize}
    \item Precio
    \item Soporte para m\'ultiples idiomas
\end{itemize}

\subsection{Ejemplos de Aplicaciones}

\subsection{Simon}
\label{sec:simon}

\subsubsection{Palaver}
\label{sec:palaver}

\subsection{Nuance}
\label{sec:nuance}


%!TEX root = ../../tesis.tex
\section{Interfaces de Programaci\'on de Aplicaciones}
\label{apis}

% introduccion

Una interfaz de programaci\'on de aplicaciones (tambi\'en conocida como \gls{api} por sus siglas 
en ingl\'es) proporciona a los desarrolladores los medios necesarios para integrar funcionalidades de
reconocimiento del habla al \foreign{software} que implementan, sin necesidad de poseer
conocimiento en detalle del \'area. De acuerdo a los criterios de evaluaci\'on generales, 
pueden mencionarse las siguientes caracter{\'\i}sticas:

\begin{itemize}
 	\item Conocimiento t\'ecnico necesario: aunque no requieren conocimiento sobre detalles de implementaci\'on
 	del \'area de reconocimiento del habla, si resultan necesarios conocimientos b\'asicos de
 	programaci\'on, por lo cual resulta inadecuada su utilizaci\'on por parte de usuarios finales.
 	\item Productividad: la utilizaci\'on de una interfaz de programaci\'on de aplicaciones permite
 	al desarrollador abstraerse de la complejidad subyacente del problema, lo cual resulta en
 	un alto grado de productividad. Sin embargo, el ocultamiento de los detalles del proceso
 	puede ser un factor negativo en algunos \'ambitos, como el \mbox{acad\'emico}.
 	\item Flexibilidad: estas herramientas presentan un buen grado de flexibilidad, debido a
 	que no est\'an orientadas a una tarea en particular. La selecci\'on del problema espec{\'\i}fico
 	que se soluciona utilizando reconocimiento del habla es responsabilidad del desarrollador
 	que utiliza la interfaz. Cabe destacar, sin embargo, que los componentes propios de la
 	implementaci\'on de la interfaz de programaci\'on, al estar ocultos, no
 	pueden modificarse.
 \end{itemize}


\subsection{Criterios Espec\'ificos de Evaluaci\'on}

Los criterios espec{\'\i}ficos seg\'un los cuales se evaluar\'an las opciones disponibles en esta
categor{\'\i}a son:

\begin{itemize}
	\item Empresa o Instituci\'on Responsable
	\item Precio o Inversi\'on econ\'omica
	\item Soporte para m\'ultiples idiomas
	\item Soporte Offline
	\item Dependencia de Plataforma
\end{itemize}


\subsection{Ejemplos de Interfaces de Programaci\'on de Aplicaciones}

%!TEX root = ../../tesis.tex
\subsubsection{Web Speech \gls{api}}
\label{sec:webspeech}

La \foreign{Web Speech \gls{api}} define un est\'andar que especifica una interfaz de programaci\'on de aplicaciones para permitir
a los desarrolladores web incorporar s{\'\i}ntesis y reconocimiento del habla a sus sitios web.

\begin{itemize}
	\item \emph{Empresa o Instituci\'on Responsable:} La especificaci\'on de la interfaz fue
	publicada por el \foreign{Speech \gls{api} Community Group} del
	\mbox{\foreign{World Wide Web Consortium} \cite{GoogleWebSpeechAPI}.}
	La \'unica implementaci\'on disponible fue desarrollada por \foreign{Google} para su navegador \foreign{Chrome}.
	\item \emph{Precio:} la utilizaci\'on de esta herramienta a trav\'es de \foreign{Google Chrome} es gratuita e
	ilimitada en la actualidad. Aunque esto podr\'ia llegar a cambiar ya que se trata del producto de una empresa
    en particular.
	\item \emph{Soporte para m\'ultiples idiomas:} la interfaz ofrece soporte para m\'as de 30 idiomas.
	\item \emph{Soporte Offline:} la especificaci\'on publicada de la interfaz no limita ni reglamenta su
	implementaci\'on; sin embargo, la \'unica opci\'on funcional actualmente delega el procesamiento a los
	servidores de \foreign{Google}, por lo cual requiere de una conexi\'on a internet para su uso.
	\item \emph{Dependencia de Plataforma:} es necesario un navegador web para su utilización.
	Aunque actualmente solo puede utilizarse a trav\'es de \foreign{Chrome}, implementar la interfaz
	para otros navegadores es posible. No existe dependencia con el sistema operativo.
\end{itemize}


%!TEX root = ../../tesis.tex
\subsubsection{AT\&T Speech API}
\label{sec:att}

La \foreign{AT\&T Speech API} \cite{AttSpeech} define una interfaz de programaci\'on de aplicaciones para permitir
a los desarrolladores incorporar s{\'\i}ntesis y reconocimiento del habla en aplicaciones m\'oviles,
web o de escritorio. El procesamiento se lleva a cabo a trav\'es del motor \foreign{AT\&T Watson}.

\begin{itemize}
	\item \emph{Empresa o Instituci\'on Responsable:} \foreign{AT\&T Inc}.
	\item \emph{Precio:} un mill\'on de llamadas a la interfaz por 99\$ anuales, llamadas adicionales a un centavo
	cada una.
	\item \emph{Soporte para m\'ultiples idiomas:} el ingl\'es es el principal idioma soportado, aunque incluye
	soporte reducido para el espa\~nol.
	\item \emph{Soporte Offline:} al delegarse el procesamiento a los servidores de \foreign{AT\&T},
	una conexi\'on a internet es indispensable para su uso.
	\item \emph{Dependencia de Plataforma:} al ofrecer una interfaz REST, no existe dependencia con navegador
	ni con sistema operativo alguno. Adem\'as, se ofrecen \foreign{software development kits} o SDKs para facilitar
	el desarrollo en plataformas como \foreign{Windows}, \foreign{iOS} y \foreign{Android}.
\end{itemize}

\subsection{Microsoft Speech API}
\label{sec:microsoft}

%!TEX root = ../../tesis.tex
\subsubsection{WAMI}
\label{sec:wami}

WAMI es un \foreign{toolkit} desarrollado en el Instituto Tecnol\'ogico de
\mbox{Massachusetts \cite{WamiHome}}. T\'ecnicamente, es m\'as que una interfaz de programaci\'on
de aplicaciones; ofrece un conjunto de herramientas, para el lado servidor y el lado cliente,
que permiten al desarrollador ofrecer su propia implementaci\'on de la interfaz.

WAMI est\'a orientado a ofrecer una interfaz web para un determinado motor de reconocimiento del habla. 
A\'un as{\'\i}, el MIT ofrece un reconocedor del habla que implementa la interfaz para fines 
de prototipado como parte del proyecto.

\begin{itemize}
	\item \emph{Empresa o Instituci\'on Responsable:} Instituto Tecnol\'ogico de Massachusetts.
	\item \emph{Precio:} esta herramienta es gratuita y de c\'odigo abierto.
	\item \emph{Soporte para m\'ultiples idiomas:} la implementaci\'on para prototipado ofrecida
	por el MIT soporta los idiomas ingl\'es, chino y japon\'es. El soporte para idiomas de WAMI
	depende exclusivamente del reconocedor c on cual se integre.
	\item \emph{Soporte Offline:} WAMI se basa en la arquitectura cliente-servidor, es decir, el reconocimiento del habla 
    se realiza en el servidor y es enviado de vuelta al cliente. Por lo tanto, es necesaria una conexi\'on entre ambos.
	\item \emph{Dependencia de Plataforma:} al ofrecer una interfaz REST, no existe dependencia con
	navegador ni con sistema operativo alguno.
\end{itemize}

%!TEX root = ../../tesis.tex
\subsubsection{iSpeech \gls{api}}
\label{sec:ispeech}

La \foreign{iSpeech \gls{api}} \cite{iSpeech} define una interfaz de programaci\'on de aplicaciones para 
permitir a los desarrolladores incorporar s{\'\i}ntesis y reconocimiento del habla en aplicaciones 
m\'oviles, web o de escritorio.

\begin{itemize}
	\item \emph{Empresa o Instituci\'on Responsable:} \foreign{iSpeech}, una empresa estadounidense dedicada a
	proveer reconocimiento y s{\'\i}ntesis del habla como servicio. Su producto m\'as reconocido es la aplicaci\'on m\'ovil
	\foreign{Drivesafe.ly}, orientada a reducir las distracciones de un conductor.
	\item \emph{Precio:} Gratis para aplicaciones m\'oviles sin fines comerciales. De 0,02\$ a 0,005\$ por
        transacci\'on, dependiendo del tama\~no de la compra\footnote{\emph{iSpeech} var\'ia el precio por transacci\'on
    en base al tama\~no de la compra, es decir, la cantidad m\'axima de transacciones que ofrece el plan.}, para otros usos.
	\item \emph{Soporte para m\'ultiples idiomas:} soporta reconocimiento continuo (sin pausas) del habla para 6 idiomas,
	entre ellos el espa\~nol, y reconocimiento de vocabulario reducido para m\'as de 10 idiomas.
	\item \emph{Soporte Offline:} el procesamiento se realiza en los servidores de \foreign{iSpeech}, por lo cual
	una conexi\'on a internet es indispensable para su uso.
	\item \emph{Dependencia de Plataforma:} al ofrecer una interfaz \gls{rest}, no existe dependencia con navegador
	ni con sistema operativo alguno. Adem\'as, se ofrecen varios \gls{sdk} para facilitar
	el desarrollo en m\'ultiples plataformas y lenguajes de programaci\'on.
\end{itemize}
%!TEX root = ../../tesis.tex
\subsubsection{Dragon Mobile}
\label{sec:dragonmobile}

\foreign{Dragon Mobile} ofrece una interfaz de programaci\'on de aplicaciones para permitir
a los desarrolladores incorporar s{\'\i}ntesis y reconocimiento del habla a sus
\mbox{aplicaciones \cite{DragonMobile}}. Est\'a espec{\'\i}ficamente orientado al desarrollo de
aplicaciones para plataformas m\'oviles.

\begin{itemize}
	\item \emph{Empresa o Instituci\'on Responsable:} \foreign{Nuance Communications}.
	\item \emph{Precio:} el uso de esta herramienta es gratuito para aplicaciones sin fines comerciales.
	Desde 0.08\$ a 0.006\$ por transacci\'on, dependiendo del tama\~no de la compra, para otros usos.
	Para comercializar una aplicaci\'on que utiliza esta herramienta, es necesaria una inversi\'on inicial
	m{\'\i}nima de 3000\$.
	\item \emph{Soporte para m\'ultiples idiomas:} se ofrece soporte para m\'as de 20 idiomas, entre ellos
	el espa\~nol.
	\item \emph{Soporte Offline:} al delegarse el procesamiento a los servidores de \foreign{Nuance},
	una conexi\'on a internet es indispensable para su uso.
	\item \emph{Dependencia de Plataforma:} al ofrecer una interfaz \gls{rest}, no existe dependencia con navegador
	ni con sistema operativo alguno. Adem\'as, se ofrecen varios \gls{sdk} para facilitar
	el desarrollo en plataformas m\'oviles como \foreign{Windows Phone}, \foreign{iOS} y \foreign{Android}.
\end{itemize}

\section{Librer\'ias y Frameworks}
\label{sec:librerias}

Una librer{\'\i}a va m\'as all\'a de una especificaci\'on de interacci\'on y el c\'odigo fuente que permita cumplirla.
Una librer{\'\i}a incluye un conjunto de funcionalidades o m\'etodos cuyo modo de utilizaci\'on queda a criterio
del desarrollador. En el caso de un \foreign{framework}, se incluyen tambi\'en ciertos patrones de dise\~no para la aplicaci\'on.

Aunque la distinci\'on entre una librer{\'\i}a y una interfaz de programaci\'on de aplicaciones puede resultar
sutil, para este trabajo, se consideran el mayor grado de control y el menor nivel de abstracci\'on
como criterios para identificar una herramienta como una librer{\'\i}a.

As{\'\i}, una interfaz de programaci\'on de aplicaciones es una especificaci\'on de interacci\'on con un componente
reconocedor del habla completamente implementado, mientras que una librer{\'\i}a puede verse como un conjunto
de bloques que permiten al desarrollador construir su propio reconocedor del habla a medida.

De acuerdo a los criterios de evaluaci\'on generales, pueden mencionarse las 
siguientes caracter{\'\i}sticas :

\begin{itemize}
 	\item Conocimiento t\'ecnico necesario: la utilizaci\'on de librer{\'\i}as requiere conocimiento t\'ecnico
 	espec{\'\i}fico del \'area del reconocimiento del habla por parte del programador, debido a que el nivel de
 	abstracci\'on es menor al de las herramientas previamente descriptas.
 	\item Productividad: Debido al alto grado de control que ofrecen, el desarrollo de los distintos
 	elementos de un sistema de reconocimiento del habla requiere de m\'as trabajo de an\'alisis e implementaci\'on.
 	Esto resulta en una productividad baja en comparaci\'on con otras alternativas, que permiten al
 	programador abstraerse de los detalles del proceso de reconocimiento del habla.
 	A\'un as{\'\i} en ciertos \'ambitos, como el acad\'emico, la posibilidad de conocer y manipular los distintos
 	componentes del mencionado proceso representa una importante ventaja que ofrecen estas herramientas.
 	\item Flexibilidad: las librer{\'\i}as ofrecen un alto grado de flexibilidad, siendo la alternativa
 	m\'as adaptable de entre las categor{\'\i}as analizadas. Componentes importantes de un sistema
 	de reconocimiento del habla como el modelo de lenguaje, el modelo ac\'ustico y los algoritmos involucrados
 	pueden seleccionarse, modificarse o incluso reimplementarse de acuerdo al criterio del desarrollador.
 \end{itemize}

\subsection{Criterios Espec\'ificos de Evaluaci\'on}

Los criterios espec{\'\i}ficos seg\'un los cuales se evaluar\'an las opciones disponibles en esta
categor{\'\i}a son:

\begin{itemize}
	\item Empresa o Instituci\'on Responsable
	\item Precio
	\item Licencia
	\item Soporte para m\'ultiples idiomas
	\item Dependencia de Plataforma
	\item Modelos de Lenguaje Aceptados
	\item Modelos Ac\'usticos Aceptados
	\item Uso de Memoria
\end{itemize}


\subsection{Ejemplos de librer{\'\i}as o frameworks}

\subsection{Sphinx 4}
\label{sec:sphinx}

%!TEX root = ../../tesis.tex
\subsubsection{PocketSphinx}
\label{sec:pocketsphinx}

La librer{\'\i}a Pocketsphinx es desarrollada en paralelo a Sphinx 4 en la Universidad
\mbox{\foreign{Carnegie Mellon} \cite{PocketSphinxHomePage}}. Esta herramienta de c\'odigo abierto
est\'a implementada en el lenguaje C, logr\'andose con esto en un mayor rendimiento y facilidad
para desarrollar \foreign{bindings}\footnote{En programaci\'on, un \emph{Binding} es una adaptaci\'on de una librer\'ia para ser usada en un 
lenguaje de programaci\'on distinto de aquel en el que ha sido escrita.} 
para otros lenguajes de programaci\'on.

Pocketsphinx est\'a orientado a la optimizaci\'on del rendimiento y a la portabilidad a distintas
plataformas, resultando particularmente adecuada para la implementaci\'on de sistemas empotrados
y sistemas en tiempo \mbox{real \cite{HugginsDainesPocketSphinx2006}}.

\begin{itemize}
	\item \emph{Empresa o Instituci\'on Responsable:} Universidad \foreign{Carnegie Mellon}.
	\item \emph{Precio:} esta librer\'ia puede utilizarse de forma gratuita.
	\item \emph{Licencia:} Pocketsphinx  se distribuye bajo una versi\'on simplificada de la licencia
	\gls{bsd}, considerada de c\'odigo abierto.
	\item \emph{Soporte para m\'ultiples idiomas:} Pocketsphinx utiliza los mismos modelos
	que \mbox{Sphinx 4}, por lo que el soporte para idiomas es similar. Se ofrece soporte para 8
	idiomas, aunque puede extenderse a otros mediante las herramientas prove{\'\i}das.
	\item \emph{Dependencia de Plataforma :} esta librer\'ia puede utilizarse en sistemas operativos
	basados en Unix y en Windows. Adem\'as, cabe destacar la documentaci\'on existente para su instalaci\'on
	en plataformas m\'oviles como Android, iOS y el Kindle de Amazon.
	\item \emph{Modelos de Lenguaje Aceptados:} se utilizan los mismos modelos de lenguaje que \mbox{Sphinx 4},
	basados en gram\'aticas en formato JGSF y modelos estad{\'\i}sticos en formato \gls{arpa}.
	\item \emph{Modelos Ac\'usticos Aceptados:} tambi\'en los modelos ac\'usticos se comparten con \mbox{Sphinx 4}.
	Existen modelos para 8 idiomas ya construidos, y pueden construirse modelos adicionales.
	\item \emph{Uso de Memoria:} el consumo de memoria de Pocketsphinx es muy reducido en comparaci\'on con
        Sphinx 4\cite{SphinxVersions}.
\end{itemize}
%!TEX root = ../../tesis.tex
\subsubsection{HTK}
\label{sec:htk}

\gls{htk}, es un conjunto de herramientas
desarrolladas por el Departamento de Ingenier{\'\i}a de la Universidad de Cambridge. Est\'a constituido
por varias librer{\'\i}as y herramientas ejecutables implementadas en el \mbox{lenguaje C \cite{HTKHomePage}}.

Esta herramienta ofrece un alto grado de personalizaci\'on, por lo que se requiere un nivel considerable
de conocimiento t\'ecnico para implementar un sistema de reconocimiento del habla con la misma.
Cabe destacar que la \'ultima versi\'on estable de \gls{htk} fue lanzada en el a\~no 2009.

\begin{itemize}
	\item \emph{Empresa o Instituci\'on Responsable:} Universidad de Cambridge.
	\item \emph{Precio:} esta herramienta puede utilizarse de forma gratuita.
	\item \emph{Licencia:} aunque el c\'odigo fuente de \gls{htk} puede obtenerse y modificarse,
	la redistribuci\'on del mismo est\'a prohibida. Los modelos generados con \gls{htk} pueden
	redistribuirse libremente.
	\item \emph{Soporte para m\'ultiples idiomas:} \gls{htk} no provee soporte para ning\'un
	idioma directamente, pero proporciona herramientas para la definici\'on de los modelos necesarios.
	\item \emph{Dependencia de Plataforma:} esta herramienta puede utilizarse en sistemas operativos
	basados en Unix y en Windows.
	\item \emph{Modelos de Lenguaje Aceptados:} se utilizan los mismos modelos de lenguaje que \mbox{Sphinx 4},
	basados en gram\'aticas y modelos estad{\'\i}sticos en formato \gls{arpa}.
	\item \emph{Modelos Ac\'usticos Aceptados:} \gls{htk} define su propio formato de modelos ac\'usticos.
	\item \emph{Uso de Memoria:} la documentaci\'on de \gls{htk} hace \'enfasis en que el consumo de memoria depende en
	gran medida de la aplicaci\'on, mencionando 150 MB para la construcci\'on de modelos para un sistema de
	dictado como un ejemplo de niveles altos de utilizaci\'on de este recurso.
	Esta cantidad de memoria no resulta muy elevada para el \foreign{hardware} de una m\'aquina promedio actual.
\end{itemize}
%!TEX root = ../../tesis.tex
\subsubsection{Julius}
\label{sec:julius}

Julius es un motor de reconocimiento del habla altamente configurable desarrollado por un grupo
de universidades e institutos de investigaci\'on en Jap\'on. Est\'a orientando al reconocimiento
del habla continua para vocabularios grandes, como ocurre en el dictado.
Julius est\'a implementado en el \mbox{lenguaje C \cite{JuliusHomePage}}.

\begin{itemize}
	\item \emph{Empresa o Instituci\'on Responsable:} \foreign{Interactive Speech Technology Consortium},
	una dependencia del \foreign{Advanced Scientific Technology \& Management Research Institute of KYOTO}.
	\item \emph{Precio:} esta herramienta puede utilizarse de forma gratuita.
	\item \emph{Licencia:} se distribuye bajo una versi\'on revisada de la licencia
	\gls{bsd}, considerada de c\'odigo abierto.
	\item \emph{Soporte para m\'ultiples idiomas:} existen modelos disponibles para
	el japon\'es y el ingl\'es. Los formatos de los modelos siguen los est\'andares de otras
	herramientas, de modo a extender el soporte a otros idiomas. De esta manera,
	se han realizado investigaciones utilizando Julius para el esloveno, franc\'es, tailand\'es
	y otros idiomas.
	\item \emph{Dependencia de Plataforma:} esta herramienta puede utilizarse en sistemas operativos
	basados en \foreign{Unix} y en \foreign{Windows}.
	\item \emph{Modelos de Lenguaje Aceptados:} se aceptan modelos de lenguaje de palabras aisladas,
	basados en gram\'aticas y modelos estad{\'\i}sticos en formato \gls{arpa}.
	\item \emph{Modelos Ac\'usticos Aceptados:} utiliza el mismo formato de modelo ac\'ustico que \gls{htk}.
	\item \emph{Uso de Memoria:} el consumo de memoria de Julius es bajo, cit\'andose como ejemplo
	un consumo menor a 64 MB para reconocimiento sobre un vocabulario de 20 mil palabras utilizando
	un modelo de lenguaje basado en trigramas almacenado en memoria.
\end{itemize}
%!TEX root = ../../tesis.tex
\subsubsection{Kaldi}
\label{sec:kaldi}

Kaldi es un \foreign{toolkit} de reconocimiento del habla que comenz\'o a desarrollarse durante un taller
en la Universidad \foreign{John Hopkins} en el a\~no 2009. Su desarrollo continu\'o con el apoyo del gobierno
de Rep\'ublica Checa en la \mbox{Universidad de Brno \cite{Povey_ASRU2011}}.

Kaldi est\'a constituida por un conjunto de librer{\'\i}as C++ y herramientas ejecutables. Por su orientaci\'on y
alcance es comparable al proyecto \gls{htk}, resultando tambi\'en necesario un alto nivel de conocimiento t\'ecnico
para su utilizaci\'on.

Cabe destacar su integraci\'on con \gls{fst} para la
construcci\'on de modelos de lenguaje, la cual lo distingue de las dem\'as herramientas analizadas en esta secci\'on.

\begin{itemize}
	\item \emph{Empresa o Instituci\'on Responsable:} Agencia Tecnol\'ogica de Rep\'ublica Checa.
	\item \emph{Precio:} esta herramienta puede utilizarse de forma gratuita.
	\item \emph{Licencia:} Kaldi se distribuye bajo la licencia Apache 2.0, considerada de c\'odigo 
	abierto.
	\item \emph{Soporte para m\'ultiples idiomas:} Kaldi no provee soporte para ning\'un
	idioma directamente, pero proporciona herramientas para la definici\'on de los modelos 
	\mbox{necesarios.}
	\item \emph{Dependencia de Plataforma:} esta herramienta puede utilizarse en sistemas operativos
	basados en \foreign{Unix} y en \foreign{Windows}.
	\item \emph{Modelos de Lenguaje Aceptados:} se proveen herramientas para convertir modelos en formato
	\gls{arpa} a \gls{fst}s.
	\item \emph{Modelos Ac\'usticos Aceptados:} Kaldi define su propio formato de modelos ac\'usticos.
	\item \emph{Uso de Memoria:} no se encontraron datos espec{\'\i}ficos respecto al consumo de memoria
	de Kaldi.
\end{itemize}


\section{Resumen}

Para resumir lo expuesto en este cap\'itulo, \'esta secci\'on presentar\'a tablas que resuman los criterios generales y espec\'ificos
para cada categor\'ia. A continuaci\'on en la tabla~\ref{sec:resumen-herramientas} se pueden observar las distintas
herramientas presentadas.

\begin{table}[H]
\centering
\footnotesize
\begin{tabular}{|p{3.5cm}|>{\centering}p{3.5cm}|>{\centering}p{3.5cm}|>{\centering}p{3.5cm}|}
\hline
                               & Aplicaciones             &  \gls{api}s                            & Librer\'ias/Frameworks \tabularnewline
\hline
Conocimiento T\'ecnico         &     Bajo                    & Medio                            & Alto    \tabularnewline
Productividad                  &     Alto                    & Medio                            & Bajo    \tabularnewline
Flexibilidad                   &     Bajo                    & Medio                            & Alto    \tabularnewline
Alternativas Propietarias      & \begin{itemize} \item Dragon Natural Speaking \end{itemize}  & \begin{itemize} \item Web Speech \gls{api} \item \gls{att} Speech \gls{api} \item Microsoft Speech \gls{api} \item iSpeech \gls{api} \item Dragon Mobile \end{itemize}  &  \tabularnewline
Alternativas de Código abierto & \begin{itemize} \item Simon \item Palaver \end{itemize}          &                                  & \begin{itemize} \item Sphinx 4 \item PocketSphinx \item \gls{htk} \item Julius \item Kaldi \end{itemize} \tabularnewline
\hline
\end{tabular}
\caption{Resumen general de las herramientas}
\label{sec:resumen-herramientas}
\end{table}

\subsection{Aplicaciones}

A continuaci\'on se puede observar una comparaci\'on entre las aplicaciones presentadas respecto a los criterios espec\'ificos
para \'esta categor\'ia


\begin{table}[H]
\centering
\footnotesize
\begin{tabular}{|p{3.5cm}|p{3.5cm}|p{3.5cm}|p{3.5cm}|}
\hline
                                      &  Simon                                                       &  Palaver                                       & Dragon Natural Speaking \\
\hline
Precio                                & Gratuito, C\'odigo Abierto                                   & Gratuito, C\'odigo Abierto                     & Desde 99\$  \\
Soporte para m\'ultiples idiomas      & Si soporta                                                   & Si soporta                                     & Ingl\'es, Alem\'an, Franc\'es, Espa\~nol, Italiano y Holand\'es \\
Facilidad de configuraci\'on          & F\'acilmente configurable, posee interfaz de configuraci\'on & Reducida, no posee interfaz de configuraci\'on & F\'acilmente configurable, posee interfaz de configuraci\'on \\
Soporte para dispositivos m\'oviles   & Si soporta                                                   & No soporta                                     & Existen otros productos derivados para m\'oviles \\
\hline
\end{tabular}
\caption{Resumen de los criterios espec\'ificos de las aplicaciones}
\label{sec:resumen-aplicaciones}
\end{table}

\subsection{Interfaz de Programaci\'on de Aplicaciones}

En las tablas~\ref{sec:resumen-apis} y~\ref{sec:resumen-apis-2} se puede observar una comparaci\'on entre las \gls{api}s presentadas, respecto a los criterios espec\'ificos
para \'esta categor\'ia


\begin{table}[H]
\centering
\footnotesize
\begin{tabular}{|p{3.5cm}|p{3.5cm}|p{3.5cm}|p{3.5cm}|}
\hline
                                      &  Web Speech \gls{api} & \gls{att} Speech \gls{api} & Microsoft Speech \gls{api} \\
\hline
Empresa o Instituci\'on responsable & Especificaci\'on publicada por \foreign{Speech \gls{api} Community Group}. Implementado actualmente por Google  &  \gls{att} Inc.  & Microsoft\\
Precio                              & Gratuita, a trav\'es de Google Chrome  & Un mill\'on de llamadas a la \gls{api} por 99\$ anuales. Un centavo por llamada extra  & Gratis\\
Soporte para m\'ultiples idiomas    & Soporta m\'as  30 idiomas & Ingl'es (principal) y Espa\~nol (reducido) & Soporta 26 idiomas\\
Soporte Offline                     & No por el momento  & No  & Si \\
Dependencia de Plataforma           & No es dependiente  & No es dependiente & Si es dependiente, solo para \emph{Microsoft Windows} \\
\hline
\end{tabular}
\caption{Resumen de los criterios espec\'ificos de las APIs}
\label{sec:resumen-apis}
\end{table}


\begin{table}[H]
\centering
\footnotesize
\begin{tabular}{|p{3.5cm}|p{3.5cm}|p{3.5cm}|p{3.5cm}|}
\hline
                                      &  \gls{wami} & iSpeech \gls{api} & Dragon Mobile \\
\hline
Empresa o Instituci\'on responsable & Instituto Tecnológico de Massachusetts & \foreign{iSpeech}  & \foreign{Nuance Communications} \\
Precio &  Gratuita, C\'odigo Abierto  & Gratis para aplicaciones no comerciales. De 0.02\$ a 0.0001\$ para otros usos & Gratis para aplicaciones no comerciales. De 0.08\$ a 0.006\$ para otros usos. Inversi\'on m\'inima de 3000\$ para comercializar una aplicaci\'on \\
Soporte para m\'ultiples idiomas  & Ingl\'es, Chino y Japon\'es. Depende del reconocedor utilizado & Reconocimiento Cont\'inuo para 6 idiomas. Reconocmiento de vocabulario reducido para 10 idiomas & Soporta m\'as de 20 idiomas \\
Soporte Offline & No & No & No\\
Dependencia de Plataforma & No es dependiente & No es dependiente & No es dependiente\\
\hline
\end{tabular}
\caption{Resumen de los criterios espec\'ificos de las APIs}
\label{sec:resumen-apis-2}
\end{table}


\subsection{Librer\'ias/Frameworks}

A continuaci\'on, las tablas~\ref{sec:resumen-libs} y~\ref{sec:resumen-libs-2} muestran una comparaci\'on entre las librer\'ias presentadas, respecto a los criterios espec\'ificos
para \'esta categor\'ia


\begin{table}[H]
\centering
\footnotesize
\begin{tabular}{|p{3.5cm}|p{3.5cm}|p{3.5cm}|p{3.5cm}|}
\hline
                                  &  Sphinx 4 & PocketSphinx & \gls{htk} \\
\hline
Empresa o Instituci\'on Responsable & Universidad \foreign{Carnegie Mellon} & Universidad \foreign{Carnegie Mellon} & Universidad de Cambridge \\
Precio & Gratuito & Gratuito & Gratuito \\
Licencia & \gls{bsd}, aunque el componente \foreign{Java Speech \gls{api}} no es de C\'odigo Abierto & \gls{bsd} simplificada & C\'odigo fuente modificable, se prohibe su redistribuci\'on. Los generados pueden redistribuirse\\
Soporte para m\'ultiples idiomas & 8 idiomas. Se pueden incorporar m\'as modelos para otros idiomas & Mismos modelos que Sphinx 4 & No brinda soporte, pero provee las herramientas necesarias\\
Dependencia de Plataforma & No es dependiente, solo depende de la M\'aquina Virtual de Java & Sistemas basados en Unix, Android, Windows, iOS y Kindle &  Basados en Unix y Windows \\
Modelos de Lenguaje Aceptados & Basados en gram\'aticas JGSF, en bigramas y trigamas en formato \gls{arpa} &  Mismos modelos que Sphinx 4 &  Mismos modelos que Sphinx 4 \\
Modelos Ac\'usticos Aceptados & Modelo propio & Igual que Sphinx 4 &  Modelo propio \\
Uso de Memoria & No se recomienda para sistemas de poca memoria & Muy reducido en comparaci\'on a Sphinx 4 & Dependiente de la aplicaci\'on. Por ejemplo: 150 MB para generar modelos para un sistema de dictado \\
\hline
\end{tabular}
\caption{Resumen de los criterios espec\'ificos de las Librer\'ias/Frameworks}
\label{sec:resumen-libs}
\end{table}

\begin{table}[H]
\centering
\footnotesize
\begin{tabular}{|p{3.5cm}|p{3.5cm}|p{3.5cm}|}
\hline
                                  &  Julius & Kaldi \\
\hline
Empresa o Instituci\'on Responsable &  \foreign{Interactive Speech Technology Consortium} & Agencia Tecnológica de República Checa \\
Precio & Gratuito & Gratuito \\
Licencia & \gls{bsd} & Apache 2.0 \\
Soporte para m\'ultiples idiomas &  Japon\'es e Ingl\'es. Puede extenderse para otros idiomas &  No brinda soporte, pero provee las herramientas necesarias \\
Dependencia de Plataforma & Basados en Unix y Windows & Basados en Unix y Windows \\
Modelos de Lenguaje Aceptados & Basados en gram\'aticas y modelos en formato \gls{arpa} & Modelos en formato \gls{fst} \\
Modelos Acústicos Aceptados & Igual que \gls{htk} & Modelo propio \\
Uso de Memoria & Bajo, 64 MB para un vocabulario de 20 mil palabras & \\
\hline
\end{tabular}
\caption{Resumen de los criterios espec\'ificos de las Librer\'ias/Frameworks}
\label{sec:resumen-libs-2}
\end{table}

%!TEX root = ../tesis.tex
\chapter{Definici\'on del Problema}
\label{sec:problema}

% introduccion

%!TEX root = ../tesis.tex
\section{Descripci\'on General}
\label{sec:problema-general}

El reconocimiento del habla (tambi\'en conocido como reconocimiento autom\'atico del habla) es el proceso
de convertir una se\~nal de voz en una secuencia de palabras, mediante un algoritmo implementado
como programa \mbox{computacional \cite{JaisalAReview2012}}.

Las interfaces mediante voz del usuario son sistemas computaciones especializados que permiten la
interacci\'on entre seres humanos y computadoras (otros sistemas computacionales) a trav\'es de
s{\'\i}ntesis y reconocimiento del habla. Estas interfaces presentan ciertas caracter{\'\i}sticas que las
distinguen de las interfaces visuales \cite{GabrielVoice2007}:

\begin{itemize}
	\item Transitoriedad: la voz desaparece tan pronto como se termina de pronunciar una oraci\'on,
	lo cual obliga a recordar lo que se dijo. Las interfaces visuales, por otro lado, son persistentes.
	\item Invisibilidad: la voz no es visible, lo cual hace dif{\'\i}cil indicar al usuario las opciones
	disponibles y los comandos necesarios para ejecutarlas. En las interfaces visuales los men\'ues
	cumplen esta funci\'on.
	\item Asimetr{\'\i}a: la voz puede producirse r\'apidamente, pero comprender lo que se escucha requiere
	m\'as tiempo. As{\'\i}, un usuario puede hablar m\'as r\'apido de lo que escribe con un teclado; sin embargo,
	al usuario le toma m\'as tiempo comprender lo que escucha que lo que lee.
\end{itemize}

Estas caracter{\'\i}sticas suponen un desaf{\'\i}o adicional al momento de dise\~nar interfaces mediante voz del
usuario. A\'un as{\'\i}, estas interfaces poseen un gran potencial en situaciones en las cuales la
combinaci\'on tradicional de teclado, rat\'on y monitor resulta problem\'atica \cite{NielsenVoice2003}.
Pueden mencionarse como ejemplos los siguientes casos:

\begin{itemize}
	\item Usuarios con discapacidades, las cuales les impiden manejar apropiadamente el rat\'on y/o
	el teclado o visualizar la informaci\'on en el monitor.
	\item Usuarios en situaciones de manos y vista ocupadas: como la conducci\'on de un veh{\'\i}culo o
	la reparaci\'on de equipamiento complejo.
	\item Usuarios sin acceso a un teclado o monitor: en este caso los usuarios podr{\'\i}an acceder
	a un sistema a trav\'es de un tel\'efono convencional.
\end{itemize}

El dise\~no de una interfaz por voz del usuario plantea varias cuestiones interesantes para
su estudio, entre las cuales pueden mencionarse:

\begin{itemize}
	\item Dominio de la aplicaci\'on: cabe preguntarse si existen dominios m\'as o menos adecuados para su
	integraci\'on con el reconocimiento del habla. Y, de ser as{\'\i}, {?`}c\'omo pueden identificarse?
	\item Nivel de interactividad de la aplicaci\'on: el grado de interacci\'on entre el usuario y
	la aplicaci\'on podr{\'\i}a ser un elemento que merece consideraci\'on.
	
	Las aplicaciones que requieren que el usuario pronuncie uno o pocos comandos para cumplir con su
	prop\'osito podr\'ian parecer m\'as apropiadas para la implementaci\'on de una interfaz por voz del usuario.
	Sin embargo, {?`}es posible utilizar interfaces por voz de usuario para aplicaciones altamente
	interactivas sin que la productividad del usuario se vea perjudicada?

	\item Tama\~no del Lenguaje: relacionado a la usabilidad de la aplicaci\'on.

	Se puede argumentar a favor de un lenguaje extenso con gran cantidad de comandos reconocidos, 
	teniendo en cuenta la naturalidad de la interacci\'on con el usuario que esto permitir{\'\i}a.

	Sin embargo, un gran n\'umero de comandos reconocidos podr{\'\i}a resultar dif{\'\i}cil de recordar para
	el usuario. Por tanto, {?`}c\'omo puede medirse el efecto del tama\~no del lenguaje sobre la interfaz? 
	Adem\'as, {?`}es posible estimar un tama\~no ideal del lenguaje utilizado?

	\item Longitud de los comandos: con respecto a la usabilidad de la aplicaci\'on.

	El debate con respecto a la cantidad de palabras que forman un comando es bastante similar
	al punto anterior. De igual manera es v\'alido preguntarse {?`}c\'omo medir el efecto de la longitud
	de los comandos sobre la interfaz? y, de ser posible, {?`}c\'omo estimar una longitud recomendada?

	\item Duraci\'on de la interacci\'on: se refiere al tiempo que el usuario interact\'ua con la aplicaci\'on
	a trav\'es de la interfaz por voz del usuario.

	Las caracter{\'\i}sticas propias de este tipo de interfaces, anteriormente mencionadas, podr{\'\i}an sugerir
	que son m\'as adecuadas para interacciones breves. Sin embargo, {?`}es posible utilizarlas para 
	interacciones m\'as prolongadas? y adem\'as, {?`}puede medirse el efecto de la duraci\'on sobre la
	productividad del usuario? 

\end{itemize}



\section{Problema Espec\'ifico}
\label{sec:problema-especifico}

% intro

\subsection{Justificaci\'on}

\section{Motivaci\'on}
\label{sec:motivacion}


%!TEX root = ../tesis.tex
\chapter{Soluci\'on Propuesta}
\label{sec:solucion}

Numerosas aplicaciones en el \'area del reconocimiento del habla han sido desarrolladas: 
\emph{Siri} \cite{AppleSiri}, \emph{Google Now} \cite{GoogleNow}, 
\emph{Dragon Naturally Speaking} \cite{DragonNaturallySpeaking}, etc. Cada una de estas
aplicaciones satisface distintos tipos de requerimientos pensados o inspirados en las necesidades de los usuarios finales. 

En el cap\'itulo anterior se presentó el desafío que supone el diseño de interfaces basadas 
en reconocimiento del habla, así como los requerimientos básicos de la aplicación a ser implementada como 
parte de este trabajo.

Este cap\'itulo presenta, justifica y describe la soluci\'on espec\'ifica al problema de implementar 
una interfaz basada en reconocimiento del habla para la aplicaci\'on escogida.


%!TEX root = ../tesis.tex

\section{Aplicaci\'on Desarrollada}
\label{sec:aplicacion-desarrollada}

Como se expone en la secci\'on~\ref{sec:problema-especifico}, se propone el dise\~no
de una interfaz para componer m\'usica utilizando la voz. Adem\'as, este trabajo 
opta por la metodolog\'ia de trabajo de c\'odigo abierto, lo cual implica: 

\begin{itemize}
    \item Utilizar tecnolog\'ias de c\'odigo abierto para el desarrollo de la interfaz.
    \item Establecer un proceso de desarrollo transparente y abierto.
\end{itemize}

La metodolog\'ia adoptada para implementar la interfaz permite utilizar un proyecto
existente como punto de partida, y as\'i dise\~nar e implementar una interfaz alternativa que permite
controlar la aplicaci\'on por comandos de voz. Adem\'as,
el hecho de incorporar una nueva interfaz, permite analizar
una de las motivaciones de la secci\'on~\ref{sec:motivacion}: un programa de composici\'on
musical, que recibe comandos sonoros y emite tambi\'en un resultado sonoro, podr\'ia ser
m\'as natural para el usuario. 

La interfaz desarrollada se denomina \emph{TamTam Listens} y toma como punto de partida la 
aplicaci\'on \emph{TamTam Edit} de la plataforma educativa \foreign{Sugar}.

\subsection{TamTam Edit}
\label{sec:tamtam-edit}

La m\'usica es a menudo descrita como la forma m\'as pura de representaci\'on matem\'atica, es m\'as,
te\'oricos de la m\'usica han utilizado las matem\'aticas para resolver problemas musicales
\cite{TheSoundOfNumbers}. Esta fue la inspiraci\'on para la creaci\'on del compendio de 
actividades\footnote{Una Actividad, es una aplicaci\'on en el entorno de escritorio \emph{Sugar}.}
conocido como \emph{TamTam} desarrollado para la computadora XO\footnote{La XO, es una computadora 
port\'atil de bajo costo y consumo desarrollada por el proyecto \gls{olpc}.},
con los siguientes objetivos:

\begin{itemize}
    \item Proveer a los ni\~nos un ambiente de informaci\'on cultural construyendo m\'usica y sonidos.
    \item Brindar una experiencia sonora/musical divertida para usuarios sin conocimientos musicales.
    \item Promover un camino hacia experiencias musicales m\'as sofisticadas.
    \item Promover un instrumento musical con su propio ``sonido''.
    \item Desarrollar un ambiente din\'amico y mutable que propone la simpleza y permite la complejidad.
    \item Favorecer la creaci\'on de m\'usica grupalmente.
    \item Introducir los conceptos musicales y otros como: programaci\'on y audio.
\end{itemize}

\emph{TamTam Edit} es una aplicaci\'on, parte del conjunto de actividades musicales 
\emph{TamTam}, que proporciona una interfaz intuitiva para crear, modificar y organizar notas ubicadas 
en pistas virtuales.
Adem\'as incluye una paleta de casi cien tipos de sonidos y modelos de construcci\'on musical que permite 
crear distintos tipos de variaciones en estilos musicales \cite{TamTamWiki}.


Las secciones principales del programa se pueden observar en la figura~\ref{figure:ui-tamtam} 

\begin{figure}[H]
\centering
\includegraphics[width=0.8\textwidth]{./graphics/ui-tamtam-edit.png}
\caption{Interfaz de \emph{Tamtam Edit} y sus secciones principales.}
\label{figure:ui-tamtam}
\end{figure}

Como se puede apreciar en la figura~\ref{figure:ui-tamtam}, la interfaz se encuentra organizada en 
cinco pistas y cada pista (1) tiene asociada un instrumento (la quinta pista esta reservada para
instrumentos de tipo bater{\'\i}a). Cada pista se divide en 4 
compases (del comp\'as uno al cuatro) y cada comp\'as (2) se divide en 12 tiempos (del uno al doce). 
Las notas (3) se dibujan en los compases, como se puede ver, la longitud de la nota indica su 
duraci\'on y su altura el tipo de nota. Adem\'as la aplicaci\'on esta organizada en partituras (4), que 
son como hojas de cuaderno, y son \'utiles para componer m\'usicas largas.

En general, la interacci\'on entre el usuario y \foreign{TamTam Edit} que se presenta en
la figura~\ref{figure:ui-tamtam} se realiza de modo tradicional.
El usuario utiliza el teclado y el rat\'on para interactuar con la aplicaci\'on a trav\'es de elementos
gr\'aficos como botones y men\'us. 

Este trabajo de grado busca incorporar un medio de interacci\'on alternativo al propuesto 
por \emph{TamTam Edit}. A continuaci\'on se presentan los motivos que determinaron la elecci\'on 
de esta aplicaci\'on como punto de partida:

\begin{itemize}
    \item Impacto social: \emph{TamTam Edit} es una aplicaci\'on que forma parte de la plataforma establecida
    por el proyecto \gls{olpc}, el cual busca potenciar el proceso educativo \cite{OLPC}. 
    Los ni\~nos con alg\'un tipo de discapacidad tienen a menudo problemas para utilizar una computadora mediante
    dispositivos perif\'ericos tradicionales como el mouse o el teclado.

    En estos casos, la posibilidad de operar esta actividad mediante la voz ser{\'\i}a de gran beneficio 
    para los ni\~nos, mejorando la accesibilidad de la plataforma, y generando por tanto un impacto social
    positivo importante.
    \item Naturalidad de interacci\'on: utilizar la voz para interactuar con una aplicaci\'on podr{\'\i}a
    ofrecer una mayor naturalidad con respecto al enfoque tradicional de interacci\'on, teniendo en cuenta
    que es el medio de interacci\'on entre las personas.
    \item No reinventar la rueda: como el objetivo es incorporar una interfaz basada en reconocimiento del
    habla a una aplicaci\'on, se elige \emph{TamTam Edit} para no construir la aplicaci\'on desde cero,
    sino extender las capacidades de una ya existente.
    \item C\'odigo abierto: el motivo anterior es posible gracias a que \emph{TamTam Edit} es un 
    proyecto de c\'odigo abierto, lo cual brinda a los desarrolladores la libertad de extender las 
    funcionalidades de la aplicaci\'on.
    \item Lenguaje de programaci\'on: la actividad se encuentra implementada en el lenguaje de 
    programaci\'on Python. La librer{\'\i}a de reconocimiento del habla a ser utilizada proporciona 
    \emph{bindings} 
    \footnote{Un \emph{binding} es un componente \emph{software}
   que permite hacer uso de las funcionalidades prove{\'\i}das por una librer{\'\i}a, implementada
    en un determinado lenguaje de programaci\'on, utilizando un lenguaje de programaci\'on diferente. 
    En este caso particular, \emph{Pocketsphinx} est\'a implementado en C y C++.} para 
    este lenguaje, lo cual supone una ventaja importante para la implementaci\'on de la interfaz.
\end{itemize}

\subsection{Tamtam Listens}
\label{sec:tamtam-listens}

La interfaz de interacci\'on alternativa para la aplicaci\'on \emph{Tamtam Edit} se denomina  
\emph{Tamtam Listens}. \emph{Tamtam Listens} permite al usuario componer m\'usica utilizando comandos
de voz para acceder a las diferentes funcionalidades ofrecidas por \mbox{\emph{Tamtam Edit}.}

Al ofrecer al usuario final un medio de interacci\'on humano-computadora diferente, \emph{TamTam Listens}
debe ser intuitivo y f\'acil de usar. De modo a lograr esto, la soluci\'on a implementar no debe 
ofrecer al usuario la posibilidad de controlar el \foreign{mouse} o los componentes de la interfaz 
gr\'afica de \emph{TamTam Edit} utilizando la voz. 

Una correspondencia directa entre los comandos de voz y la interfaz gr\'afica puede resultar en 
un flujo de interacci\'on poco natural e inadecuado, motivado \'unicamente por la intenci\'on err\'onea 
de imitar el flujo de interacci\'on de las interfaces de escritorio tradicionales. 
La necesidad de ofrecer un flujo de interacci\'on diferente y apropiado para una interfaz mediante voz
como parte de \emph{TamTam Listens} se hizo clara durante la fase de dise\~no.

Para comprender la diferencia, puede considerarse el siguiente ejemplo. 
Crear una nota exige una secuencia de operaciones con el \foreign{mouse} al utilizarse la
interfaz tradicional: presionar el bot\'on de la herramienta correspondiente, seleccionar la pista
donde se desea crear la nota, utilizar el \foreign{mouse} para definir la 
duraci\'on de la nota, etc. Al utilizarse una interfaz mediante voz del usuario, la misma operaci\'on
puede realizarse pronunciando un comando como ``crear nota do''.

\subsection{Ejemplo: Componiendo una escala simple}
\label{sec:ejemplo-escala}

Como presentaci\'on del modelo de interacci\'on de la interfaz propuesta, se transcribe a 
continuaci\'on un tutorial de uso incluido en el manual de uso de \emph{Tamtam Listens}. 
Esto, de modo a ejemplificar una breve interacci\'on entre el usuario y la aplicaci\'on.

El tutorial sirve de gu{\'\i}a al usuario para realizar una sencilla composici\'on musical.
Para componer una escala simple con \foreign{TamTam Listens}, pueden seguirse los siguientes pasos:

\begin{enumerate}
  \item Para empezar, debemos obtener una partitura en blanco. Lo conseguimos pronunciando el comando: 
  ``Crear Nueva M\'usica''.

  \item Seleccionamos los instrumentos que queremos utilizar, diciendo:
    \begin{itemize}
      \item ``Piano en Pista Uno''
      \item ``Guitarra El\'ectrica en Pista Dos''
      \item ``Teclado en Pista Tres''
      \item ``Flauta en Pista Cuatro''
    \end{itemize}

  \item Antes de crear las notas, debemos ubicarnos en el punto donde queremos empezar.
  Para ubicarnos en el tiempo uno, del comp\'as uno de la pista uno:
  \begin{itemize}
    \item ``Pista Uno''
  \end{itemize}
        Si quisi\'esemos empezar en el tiempo uno del comp\'as dos, bastar{\'\i}a con decir:
  \begin{itemize}
    \item ``Comp\'as Dos''
  \end{itemize}
        En caso de querer empezar en el tiempo siete, decimos:
  \begin{itemize}
    \item ``Tiempo Siete''
  \end{itemize}

  \item Ya seleccionado el punto inicial, estamos listos para crear las notas:
  \begin{itemize}
    \item ``Crear Nota Do''
    \item ``Crear Nota Re''
    \item ``Crear Nota Mi''
    \item ``Crear Nota Fa''
    \item ``Crear Nota Sol''
    \item ``Crear Nota La''
    \item ``Crear Nota Si''
  \end{itemize}


  \item Como no queremos trabajar de m\'as, duplicamos las pistas para escuchar los dem\'as instrumentos:
  \begin{itemize}
    \item ``Duplicar Pista Uno en Pista Dos''
    \item ``Duplicar Pista Dos en Pista Tres''
    \item ``Duplicar Pista Tres en Pista Cuatro''
  \end{itemize}

  \item Para escuchar nuestra m\'usica: ``Reproducir M\'usica''.

\end{enumerate}


\subsection{Comandos V\'alidos de la Aplicaci\'on}
\label{sec:comandos-validos}

Como puede apreciarse en el ejemplo anterior, la interacci\'on entre el usuario y la aplicaci\'on
ocurre {\'\i}ntegramente a trav\'es de comandos de voz. Estos comandos son pronunciados por el usuario
e interpretados por \emph{TamTam Listens}, haciendo posible de este modo la composici\'on musical. 

Los comandos de voz que el usuario puede utilizar con \emph{TamTam Listens} se clasifican en:

\begin{itemize}
    \item Comandos Generales (G): independientes del contexto de la aplicaci\'on, es decir, no dependen
    de la pista o el comp\'as seleccionados. Por ejemplo: ``Crear Nueva M\'usica''.
    \item Comandos de Pista (P): dependientes de la pista seleccionada. Por ejemplo, al pronunciar 
    ``Comp\'as Dos'', el comp\'as espec{\'\i}fico seleccionado depende de la pista 
    previamente seleccionada.
    \item Comandos de Comp\'as (C): dependientes de la pista y el comp\'as seleccionados. Por ejemplo, 
    al pronunciar ``Crear Nota Do'', la ubicaci\'on espec{\'\i}fica de la nota creada depende de la
    pista y el comp\'as seleccionados.
\end{itemize}


A continuaci\'on se presentan los distintos comandos soportados, utilizando grafos para representarlos y 
as\'i facilitar su comprensi\'on. El nodo coloreado hace referencia a la \'ultima palabra de un comando 
v\'alido para la aplicaci\'on.

\subsubsection{Comandos Generales}

Los comandos generales son independientes con respecto a otros comandos de la aplicaci\'on, a continuaci\'on se presentan los distintos
comandos disponibles en esta categor\'ia. 
\begin{figure}[H] 
\centering
\includegraphics[width=0.5\textwidth]{./graphics/cmd-musica.png}
\caption{Comandos para reproducir, pausar, parar, exportar y crear una m\'usica}
\label{figure:cmd-crear-musica}
\end{figure}

En la figura~\ref{figure:cmd-crear-musica} se pueden observar los comandos m\'as b\'asicos
de la aplicaci\'on, explicados a continuaci\'on:

\begin{itemize}
\item \emph{reproducir m\'usica}: permite reproducir la m\'usica creada.
\item \emph{pausar m\'usica}: permite pausar la reproducci\'on actual dejando la l{\'\i}nea de reproducci\'on en el
punto de pausa.
\item \emph{parar m\'usica}: permite parar la reproducci\'on actual y ubica la l{\'\i}nea de reproducci\'on al inicio de
la m\'usica.
\item \emph{crear nueva m\'usica}:  permite crear una nueva composici\'on, dejando como resultado una
partitura en blanco.
\item \emph{exportar m\'usica}: permite guardar la m\'usica creada en un archivo para que reproducirse en un
reproductor multimedia.
\end{itemize}

\begin{figure}[H]
\begin{minipage}[b]{0.5\linewidth}
\centering
\includegraphics[width=0.6\linewidth]{./graphics/salir.png}
\caption{Comando para salir de la aplicaci\'on}
\label{figure:cmd-salir}
\end{minipage}
\quad
\begin{minipage}[b]{0.5\linewidth}
\centering
\includegraphics[width=0.6\linewidth]{./graphics/cmd-vol.png}
\caption{Comandos para aumentar/disminuir el volumen general}
\label{figure:cmd-vol}
\end{minipage}
\end{figure}

El comando de la figura~\ref{figure:cmd-salir} permite salir de \emph{TamTam Listens}, basta con decir
``salir de tamtam''. Por otro lado, los comandos de la figura~\ref{figure:cmd-vol}
y~\ref{figure:cmd-tempo} permiten controlar, respectivamente, el volumen y tempo general de la aplicaci\'on. Por ejemplo: ``aumentar volumen'', ``disminuir tempo''.

\begin{figure}[H]
\begin{minipage}[b]{0.5\linewidth}
\centering
\includegraphics[width=0.6\linewidth]{./graphics/cmd-tempo.png}
\caption{Comandos para aumentar/disminuir el tempo general de la aplicaci\'on}
\label{figure:cmd-tempo}
\end{minipage}
\quad
\begin{minipage}[b]{0.5\linewidth}
\centering
\includegraphics[width=0.6\linewidth]{./graphics/partitura-1.png}
\caption{Comandos para crear, limpiar y duplicar la partitura actual}
\label{figure:cmd-partitura-1}
\end{minipage}
\end{figure}

Los comandos de las figuras~\ref{figure:cmd-partitura-1} y~\ref{figure:cmd-partitura-2} afectan a
la partitura actual, como se explican a continuaci\'on:

\begin{itemize}
    \item \emph{crear nueva  partitura}:  permite crear una nueva partitura en blanco. Utilizaci\'on, 
    ``crear nueva partitura''.
    \item \emph{limpiar  partitura}: permite limpiar el contenido de la partitura actual, es decir, 
    borrar todas las notas. Utilizaci\'on, ``limpiar partitura''.
    \item \emph{duplicar partitura}: crea una nueva partitura con el mismo contenido que la partitura 
    actual. Utilizaci\'on, ``duplicar partitura''.
\end{itemize}

\begin{figure}[H]
\begin{minipage}[b]{0.5\linewidth}
\centering
\includegraphics[width=0.6\linewidth]{./graphics/partitura-2.png}
\caption{Comandos para navegar entre partituras}
\label{figure:cmd-partitura-2}
\end{minipage}
\quad
\begin{minipage}[b]{0.5\linewidth}
\centering
\includegraphics[width=0.6\linewidth]{./graphics/cmd-pista-1.png}
\caption{Comando para ubicarse en una pista}
\label{figure:cmd-pista-1}
\end{minipage}
\end{figure} 

En la figura~\ref{figure:cmd-pista-1} puede apreciarse el comando que permite al usuario ubicarse 
en una pista en particular. Por ejemplo, para ubicarse en la pista tres debe decir “pista tres”. 
Adem\'as de controlar el volumen general de la aplicaci\'on, en 
la figura~\ref{figure:cmd-vol-pista} se pueden ver los comandos para controlar el volumen de una pista en 
particular. Para aumentar el volumen de la pista tres, el usuario debe 
decir ``aumentar volumen de pista tres''.

\begin{figure}[H] 
\begin{minipage}[b]{0.5\linewidth}
\centering
\includegraphics[width=0.8\linewidth]{./graphics/vol-pista.png}
\caption{Comandos para aumentar/disminuir el volumen de una pista en particular}
\label{figure:cmd-vol-pista}
\end{minipage}
\quad
\begin{minipage}[b]{0.5\linewidth}
\centering
\includegraphics[width=0.9\linewidth]{./graphics/rep-pista.png}
\caption{Comandos para reproducir, silenciar, habilitar y limpiar una pista en particular}
\label{figure:cmd-rep-pista}
\end{minipage}
\end{figure}

Los comandos de la figura~\ref{figure:cmd-rep-pista} permiten: reproducir, silenciar, habilitar y limpiar el contenido de una
pista en particular. Por ejemplo, para reproducir las notas de la pista uno, el usuario debe decir ``reproducir pista uno''. 
Para poder generar distintos tipos de sonidos con \emph{TamTam Listens}, los usuarios de pueden asignar
instrumentos a cada una de las pistas de
la aplicaci\'on, esto se puede realizar con los comandos de la figura~\ref{figure:cmd-inst-p1-4} y~\ref{figure:cmd-inst-p5}. Para
asignar el piano a la pista dos, basta con decir ``piano en pista dos''.


\begin{figure}[H]
\centering
\includegraphics[width=0.8\textwidth]{./graphics/inst-p1-4.png}
\caption{Selecci\'on de instrumento para pistas del uno al cuatro}
\label{figure:cmd-inst-p1-4}
\end{figure}

\begin{figure}[H] 
\begin{minipage}[b]{0.5\linewidth}
\centering
\includegraphics[width=1\linewidth]{./graphics/inst-p5.png}
\caption{Selecci\'on de instrumento para la pista cinco}
\label{figure:cmd-inst-p5}
\end{minipage}
\quad
\begin{minipage}[b]{0.5\linewidth}
\centering
\includegraphics[width=1\linewidth]{./graphics/dup-pista.png}
\caption{Comando para duplicar las notas de una pista en otra}
\label{figure:cmd-dup-pista}
\end{minipage}
\end{figure}

Generalmente las composiciones presentan cierta secuencia de notas que se repiten para varios 
instrumentos. El comando de la figura \ref{figure:cmd-dup-pista} permite duplicar el contenido, 
es decir las notas, de una pista en otra. Por ejemplo, 
``duplicar pista uno en pista dos'' permite duplicar las notas de la pista uno en la pista dos.

\subsubsection{Comandos de Pista} 

Este comando ubica al usuario dentro de una pista en particular. Por lo tanto, se debe
seleccionar una pista para poder utilizar este comando.

\begin{figure}[H] 
\centering
\includegraphics[width=0.4\linewidth]{./graphics/cmd-compas.png}
\caption{Comando para ubicarse en un comp\'as}
\label{figure:cmd-compas}
\quad
\end{figure}

\subsubsection{Comandos de Comp\'as}

Estos comandos son muy importantes para la aplicaci\'on ya que permiten crear, modificar, 
eliminar las notas musicales. El comando de la figura~\ref{figure:cmd-crear-nota} permite crear notas en el  
comp\'as actual, por ejemplo ``crear nota do'' crea la nota do en el comp\'as previamente seleccionado. Por otro lado, en la figura~\ref{figure:cmd-tiempo-compas} 
se muestra el comando que permite al usuario ubicarse en un  
tiempo en espec\'ifico dentro del comp\'as actual. Esto es \'util para crear una nota a partir de ese punto o 
para seleccionar una nota que se encuentre en ese tiempo.

\begin{figure}[H]
\begin{minipage}[b]{0.5\linewidth}
\centering
\includegraphics[width=1\linewidth]{./graphics/cmd-crear-nota.png}
\caption{Comando para crear una nota}
\label{figure:cmd-crear-nota}
\end{minipage}
\quad
\begin{minipage}[b]{0.5\linewidth}
\centering
\includegraphics[width=1.1\linewidth]{./graphics/cmd-tiempo-compas.png}
\caption{Comando para ubicarse en un tiempo dado, dentro de un comp\'as}
\label{figure:cmd-tiempo-compas}
\end{minipage}
\end{figure}

As\'i como puede duplicarse notas de una pista a otra, tambi\'en puede duplicarse una nota de un comp\'as a otro utilizando el comando
de la figura~\ref{figure:cmd-dup-nota}, por ejemplo ``duplicar en pista uno compas dos'' permite duplicar una nota en el segundo comp\'as de la pista 
uno. Para poder eliminar una nota, previamente seleccionada, el usuario debe utilizar el comando
de la figura~\ref{figure:cmd-del-nota}.

\begin{figure}[H]
\begin{minipage}[b]{0.5\linewidth}
\centering
\includegraphics[width=1.2\linewidth]{./graphics/cmd-dup-nota.png}
\caption{Comando para duplicar una nota previamente seleccionada}
\label{figure:cmd-dup-nota}
\end{minipage}
\quad
\begin{minipage}[b]{0.5\linewidth}
\centering
\includegraphics[width=0.5\linewidth]{./graphics/del-note.png}
\caption{Comando para eliminar un nota previamente seleccionada}
\label{figure:cmd-del-nota}
\end{minipage}
\end{figure}

En la figura~\ref{figure:cmd-dur} se pude observar el comando que permite modificar la duraci\'on de 
una nota inmediatamente despu\'es de haberla creado o una nota previamente seleccionada. Finalmente, el comando
presentado en la figura~\ref{figure:cmd-note-tiempo} permite modificar el tiempo en el que inicia la 
nota inmediatamente despu\'es de haberla creado o una nota previamente seleccionada.

\begin{figure}[H]
\begin{minipage}[b]{0.5\linewidth}
\centering
\includegraphics[width=0.9\linewidth]{./graphics/cmd-dur.png}
\caption{Comando que permite configurar la duraci\'on de una nota}
\label{figure:cmd-dur}
\end{minipage}
\quad
\begin{minipage}[b]{0.5\linewidth}
\centering
\includegraphics[width=1.1\linewidth]{./graphics/cmd-note-tiempo.png}
\caption{Comando que permite configurar el inicio de una nota dentro del comp\'as}
\label{figure:cmd-note-tiempo}
\end{minipage}
\end{figure}
\section{Tecnolog\'ia a Utilizar}
\label{sec:tecnologia-utilizada}
Para implementar el reconocimiento  de comandos de voz necesario para \foreign{TamTam Listens}, 
se utilizan los proyectos de c\'odigo abierto \emph{Voxforge} \cite{Voxforge} y 
\emph{PocketSphinx} \cite{PocketSphinxHomePage}. 

\subsection{Voxforge}
\label{sec:voxforge-solucion}

\foreign{Voxforge} es un proyecto que busca recopilar grabaciones de voz de modo a crear 
y ofrecer varios corpus de habla bajo una licencia que permita su libre utilizaci\'on. 

A partir de estos corpus, es posible construir modelos ac\'usticos para su uso con motores de 
reconocimiento del habla de c\'odigo abierto, como \foreign{Pocketsphinx}.

\foreign{Voxforge} cuenta con grabaciones en diferentes idiomas, entre ellos ingĺ\'es, franc\'es, 
alem\'an y espa\~nol.


\subsection{PocketSphinx}
\label{sec:pocketsphinx-solucion}

El cap\'itulo~\ref{sec:tecnologias} clasifica a las herramientas que permiten implementar soluciones
basadas en reconocimiento del habla en tres categor\'ias: Aplicaciones, \gls{api}s y Librer\'ias. 
Dada la naturaleza de este trabajo, se opta por elegir una librer\'ia para implementar la interfaz
alternativa a la ofrecida por \emph{TamTam Edit}.

Para la implementaci\'on de la interfaz operable a trav\'es de la voz se elige la librer\'ia 
\emph{PocketSphinx} que, como se menciona en la secci\'on~\ref{sec:pocketsphinx}, es un motor de 
reconocimiento del habla orientado a la optimizaci\'on del rendimiento y la portabilidad.

Uno de los objetivos espec\'ificos de este trabajo es aplicar y contrastar en la pr\'actica
los conocimientos te\'oricos adquiridos. Como indica la secci\'on~\ref{sec:librerias}, una librer\'ia es
el tipo de herramienta que requiere un conocimiento t\'ecnico espec\'ifico del \'area, adem\'as de
brindar una alta flexibilidad permitiendo al programador manipular los distintos componentes del
proceso de reconocimiento del habla. Por este motivo, una librer\'ia se considera la herramienta 
m\'as adecuada para cumplir el objetivo mencionado anteriormente.

A continuaci\'on se presentan los motivos t\'ecnicos que determinaron la elecci\'on de esta librer\'ia:

\begin{itemize}
    \item \emph{PocketSphinx} est\'a orientada a la optimizaci\'on del rendimiento, resultando adecuada 
    para sistemas con recursos limitados, como la computadora \emph{XO}.
    \item La librer\'ia es un proyecto de c\'odigo abierto, por lo tanto se cumple la metodolog\'ia 
    de trabajo adoptada.
    \item \emph{PocketSphinx} ofrece soporte \emph{offline}, lo cual permite su utilizaci\'on en ambientes
    sin conexi\'on a internet, como ocurre frecuentemente con las \emph{XO}.
    \item Existen \foreign{bindings} para el lenguaje de programaci\'on Python, lo cual hace muy 
    sencilla la tarea de utilizar la librer\'ia desde el c\'odigo Python.
\end{itemize}

\section{Descripci\'on de la Implementaci\'on}
\label{sec:descripcion-implementacion}

En esta secci\'on se describen los detalles de la implementaci\'on de \foreign{TamTam Listens}.
Esto incluye la arquitectura y los componentes propios del sistema de reconocimiento del habla
los cuales, en conjunto con las herramientas mencionadas en las secciones anteriores, hacen posible
el funcionamiento de la interfaz mediante voz. 

\subsection{Arquitectura de TamTam Listens}
\label{sec:arquitectura-solucion}

La figura~\ref{figure:tamtam-listens-arq} muestra la arquitectura b\'asica de \emph{TamTam Listens}. Se 
pueden observar los componentes que forman parte de la soluci\'on desarrollada.

\begin{figure}[H] 
\centering
\includegraphics[width=0.7\textwidth]{./graphics/tamtam-listens-arq.png}
\caption{Arquitectura b\'asica de \emph{TamTam Listens}}
\label{figure:tamtam-listens-arq}
\end{figure}

Como sugiere la figura~\ref{figure:tamtam-listens-arq}, \emph{TamTam Listens} a\~nade soporte 
de reconocimiento del habla a \emph{TamTam Edit} utilizando la librer\'ia \emph{PocketSphinx}.
A favor de una soluci\'on modular, el reconocimiento de comandos de voz de \emph{PocketSphinx} 
es expuesto como como un  servicio del sistema, a trav\'es de llamadas a \emph{D-Bus}\cite{Dbus2013},
el cual es un mecanismo para comunicaci\'on entre procesos de Linux\footnote{Linux es un sistema operativo
cuyo \foreign{kernel} y paquetes de software son desarrollados bajo el modelo de c\'odigo abierto\cite{LinuxGuideCert}}.

\emph{TamTam Listens} consume como un servicio el reconocimiento del habla expuesto a trav\'es de 
\emph{D-Bus}, utilizando un protocolo de paso de mensajes.

La arquitectura propuesta, implementada como parte de \emph{TamTam Listens}, favorece la modularidad
y facilita la reutilizaci\'on de la soluci\'on para otros proyectos de caracter{\'\i}sticas similares.

\subsection{Proceso del Reconocimiento del Habla}
\label{sec:proceso-solucion}

\foreign{PocketSphinx} implementa el proceso del reconocimiento del habla basado en el enfoque 
estad{\'\i}stico, el cual se present\'o en el cap{\'\i}tulo~\ref{sec:proceso}. La librer{\'\i}a incluye
los diferentes algoritmos necesarios para las fases del proceso.

\begin{figure}[H] 
\centering
\includegraphics[width=0.8\textwidth]{./graphics/pocketsphinx.png}
\caption{Reconocimiento del habla mediante PocketSphinx. Gr\'afico basado en \cite{VerenichASR}.}
\label{figure:proceso-pocketsphinx}
\end{figure}

Como conjunto de caracter{\'\i}sticas espectrales, \foreign{Pocketsphinx} utiliza los coeficientes
espectrales en las frecuencias de mel (MFCC por sus siglas en ingl\'es).

En los \gls{hmm} del modelo ac\'ustico, utiliza mezclas Gaussianas para definir las probabilidades de 
transici\'on entre estados. El diccionario fon\'etico debe ser definido por el desarrollador.

Aunque \foreign{PocketSphinx} soporta modelos de lenguaje basados en n-gramas, para este trabajo 
se utiliza un modelo de lenguaje basado en gram\'atica. Esto se debe a la simpleza del lenguaje y a la 
falta de datos de entrenamiento para un modelo estad{\'\i}stico del mismo.

Como decodificador se utiliza el algoritmo de Viterbi, haciendo uso de la t\'ecnica de b\'usqueda en haz
para reducir el espacio de b\'usqueda.

Para hacer posible el reconocimiento del habla debe definirse el vocabulario que forma parte del 
dominio de la aplicaci\'on, su pronunciaci\'on y el conjunto de oraciones (en este caso, comandos 
v\'alidos) soportados por la aplicaci\'on. En otras palabras, \emph{Pocketsphinx} requiere un
modelo ac\'ustico y un modelo de lenguaje para funcionar.

\subsection{Modelo Ac\'ustico}
\label{sec:acustico-solucion}

El modelo ac\'ustico utilizado tiene como base las grabaciones de \foreign{Voxforge} en idioma espa\~nol,
y se encuentra disponible entre los recursos del proyecto \foreign{Pocketsphinx} para su utilizaci\'on
con el motor de reconocimiento del habla.

El diccionario fon\'etico define la secuencia de fonemas de cada palabra del lenguaje.
Este diccionario es utilizado por \emph{PocketSphinx} para asociar los fonemas reconocidos por
el modelo ac\'ustico a palabras v\'alidas del dominio de la aplicaci\'on.
En la figura~\ref{figure:fragmento-dic} se puede observar un fragmento del diccionario fon\'etico
utilizado por \emph{TamTam Listens}.

\lstset{
  basicstyle=\scriptsize,        % the size of the fonts that are used for the code
  breakatwhitespace=false,         % sets if automatic breaks should only happen at whitespace
  frame=single,                    % adds a frame around the code
  language=Octave,                 % the language of the code
  numbersep=5pt,                   % how far the line-numbers are from the code
  showstringspaces=false,          % underline spaces within strings only
  stepnumber=2,                    % the step between two line-numbers. If it's 1, each line will be numbered
  tabsize=2                       % sets default tabsize to 2 spaces
}

\begin{figure}[H]
\begin{lstlisting}
REPRODUCIR RR E P R O D U S I R
PAUSAR P A U S A R
PARAR  P A R A R
GENERAR  J E N E R A R
PARTITURA P A R T I T U R A
SIGUIENTE S I G I E N T E
ANTERIOR A N T E R I O R
\end{lstlisting}
\caption{Fragmento del diccionario fon\'etico utilizado en \emph{Tamtam Listens}.}
\label{figure:fragmento-dic}
\end{figure}

Como se puede observar, el formato del diccionario es bastante simple. Primero se define la palabra y 
a continuaci\'on los fonemas que la componen separados por espacios.
Una vez definidas las palabras de la aplicaci\'on, lo siguiente es determinar si una secuencia de
palabras es o no un comando v\'alido.

\subsection{Modelo de Lenguaje}
\label{sec:lenguaje-solucion}

Una gram\'atica en formato JSGF \cite{JSGF2000} fue utilizada como modelo de lenguaje, lo cual permiti\'o
definir los comandos soportados por la aplicaci\'on de una manera sencilla.
En la figura~\ref{figure:fragmento-gram} puede observarse un fragmento de la gram\'atica utilizada por 
la aplicaci\'on.

\begin{figure}[H]
\begin{lstlisting}
#JSGF V1.0;
grammar tamtam;

public <tamtam-listens> = <comando> | <pagina> | <pista-a>     | <pista-b>  | 
                          <seleccionar-compas> | <crear-nota>  | <seleccionar-nota> | 
                          <duplicar-nota>      | <borrar-nota> | <volumen> | <tempo> | 
                          <configurar-nota>    | <loop>;

<comando>  = REPRODUCIR MUSICA  | PAUSAR MUSICA   | PARAR MUSICA | GENERAR MUSICA | 
             CREAR NUEVA MUSICA | EXPORTAR MUSICA | SALIR DE TAMTAM;
<pagina>   = ( CREAR NUEVA | DUPLICAR | LIMPIAR ) PARTITURA | PARTITURA <orden>;
<orden>    = ( ANTERIOR | SIGUIENTE );
<loop>   (COMODIN)+;
\end{lstlisting}
\caption{Fragmento de la gram\'atica utilizada en \emph{Tamtam Listens}.}
\label{figure:fragmento-gram}
\end{figure} 


\subsection{Palabras fuera del Vocabulario}
\label{sec:oov}

La pronunciaci\'on de palabras fuera del vocabulario ocurre frecuentemente en numerosas aplicaciones
del reconocimiento del habla, y es una fuente conocida de errores en el reconocimiento \cite{Bazzi00Modeling}.

De modo a minimizar la cantidad de errores como consecuencia de las palabras fuera del vocabulario,
\foreign{TamTam Listens} integra a su gram\'atica un modelo de palabra gen\'erica. Este componente se
define mediante el no terminal $<loop>$ de la figura \ref{figure:fragmento-gram}.

Un modelo de palabra gen\'erica admite cualquier palabra que pueda pronunciarse, es decir, permite
cualquier secuencia arbitraria de fonemas \cite{Bazzi00Modeling}. 
En el caso de \foreign{TamTam Listens}, el no terminal $<loop>$ corresponde a una o m\'as repeticiones del
terminal $COMODIN$.

A su vez, el terminal $COMODIN$ est\'a asociado a cada uno de los fonemas en el diccionario fon\'etico de
\foreign{TamTam Listens}, como se observa en la figura \ref{figure:fragmento-comodin}.

\begin{figure}[H]
\begin{lstlisting}
COMODIN   A
COMODIN(2)  B
COMODIN(3)  C
COMODIN(4)  CH
COMODIN(5)  D
\end{lstlisting}
\caption{Definici\'on parcial de COMODIN en el diccionario de \emph{Tamtam Listens}.}
\label{figure:fragmento-comodin}
\end{figure}

La combinaci\'on de los componentes presentados en este cap\'itulo hizo posible implementar el motor 
de reconocimiento del habla como servicio de \emph{D-Bus} y as\'i ofrecer \emph{TamTam Listens} como una 
interfaz alternativa a la propuesta por \emph{TamTam Edit}.

En el siguiente cap\'itulo se presenta la evaluaci\'on de la interfaz implementada, teniendo en cuenta 
aspectos t\'ecnicos y funcionales.


%!TEX root = ../tesis.tex
\chapter{Evaluaci\'on}
\label{sec:evaluacion}


% introduccion
En el cap{\'\i}tulo anterior se describi\'o la soluci\'on propuesta para el estudio de interfaces basadas en
reconocimiento del habla, la cual fue implementada como parte de este trabajo de grado.

Con el fin de realizar una evaluaci\'on tanto de la aplicaci\'on misma como de ciertos conceptos 
e hip\'otesis planteados, se realiz\'o una prueba de usabilidad con 12 usuarios. 
Como parte de la misma, se midieron y analizaron factores relacionados a la aplicaci\'on,
al usuario y a la interacci\'on entre ambos.

Este cap{\'\i}tulo describe los principales aspectos del estudio realizado: objetivos, metodolog{\'\i}a
y m\'etricas utilizadas.

%!TEX root = ../tesis.tex
\section{Objetivos}
\label{sec:objetivos-estudio}

La prueba de usabilidad planteada apunta al aprendizaje de cuestiones relacionadas
a la aplicaci\'on implementada y a las interfaces mediante voz en general.

Una prueba con usuarios de la aplicaci\'on implementada puede colaborar al esclarecimiento de 
las interrogantes respecto a la influencia del tama\~no del lenguaje, la longitud de
los comandos, la duraci\'on de la interacci\'on y otros factores caracter{\'\i}sticos de una
interfaz mediante voz, que se presentaron en la secci\'on \ref{sec:problema-general}.

Los objetivos del estudio de usabilidad de \foreign{TamTam Listens} son los siguientes:

\begin{itemize}
	\item Evaluar experimentalmente la aplicaci\'on desarrollada, teniendo en cuenta 
	aspectos t\'ecnicos y funcionales.
	\item Identificar problemas que se presentan al utilizar una interfaz mediante voz y,
	 de ser posible, sugerir posibles soluciones.
	\item Proponer criterios que puedan utilizarse para la evaluaci\'on de interfaces 
	mediante voz.
	\item Verificar y validar la utilidad de los criterios propuestos, de acuerdo a 
	la informaci\'on y las conclusiones que pueden obtenerse mediante los mismos.
	\item Analizar la correlaci\'on entre los factores medidos, de modo a identificar
	posibles relaciones entre los mismos. 
	\item Obtener conclusiones relacionadas a factores externos a la aplicaci\'on, 
	como la memoria del usuario, el tiempo de entrenamiento previo y el tiempo
	total de uso.
\end{itemize}
%!TEX root = ../tesis.tex
\section{Metodolog{\'\i}a}
\label{sec:metodolog{\'\i}a}
A continuaci\'on se describen brevemente distintos aspectos relacionados al
estudio de usabilidad planteado.

\subsection{Muestra}
Las pruebas se realizan con 12 usuarios, estudiantes de Ingenier{\'\i}a Inform\'atica
de la Facultad Polit\'ecnica, Universidad Nacional de Asunci\'on. Ninguno de los usuarios 
ha utilizado anteriormente la aplicaci\'on evaluada, de manera a evitar los efectos
de la experiencia previa.

La cantidad de sujetos de prueba se elige de acuerdo a la recomendaci\'on realizada
en \cite{Hwang:2010} para este tipo de estudios, la cual es de 12. 
Adem\'as, de acuerdo a \cite[p.~267]{Rubin2008}, este n\'umero es el necesario de modo a obtener
resultados estad{\'\i}sticos significativos.

\subsection{Duraci\'on}
Cada sesi\'on de pruebas tiene una duraci\'on aproximada de una hora.
Las sesiones se realizan a lo largo de 2 semanas.

\subsection{Ambiente de Pruebas}
De modo a realizar el estudio en un ambiente controlado, todas las sesiones se
realizan en un Laboratorio de Inform\'atica de la Facultad Poĺit\'ecnica. Esto
permite minimizar los efectos de la interferencia sonora y humana durante el
desarrollo de las pruebas.

\subsection{Roles}
Cada sesi\'on cuenta con 3 participantes, cada uno desempe\~nando uno de los siguientes roles:
	\begin{itemize}
		\item Facilitador: responsable de administrar los distintos pasos del estudio, explicarlos al usuario 
		y aclarar las dudas que se presenten durante la fase inicial. 
		\item Observador: responsable de tomar nota de los eventos que sucedan durante las pruebas, principalmente
		tiempo de inicio y fin de cada paso y otros sucesos que aporten valor al estudio (errores, preguntas, etc.).
		\item Usuario: persona sin conocimientos del sistema. Los detalles del aprendizaje y primeras interacciones
		(tareas) con la aplicaci\'on se registran para su posterior an\'alisis.	
	\end{itemize}

\subsection{Fases}
Cada sesi\'on est\'a compuesta de 5 fases, las cuales se completan en orden secuencial.
A continuaci\'on, se describe brevemente cada una de las fases de la prueba de usabilidad.

\subsubsection{Fase 1: Test de Memoria}
Previamente a la realizaci\'on de las tareas, se lleva a cabo un test de memoria del usuario, basado en
el Test de Aprendizaje Auditivo Verbal de Rey \cite{Lopez1998}. 

Para la prueba, el facilitador lee una lista
de 15 palabras al usuario, luego de lo cual este \'ultimo debe repetir aquellas que recuerda.
El proceso se repite 5 veces en total, registr\'andose las palabras que el usuario recuerda y el orden 
en que lo hace.

\subsubsection{Fase 2: Entrenamiento}
La fase de entrenamiento tiene como objetivo la introducci\'on del usuario a los distintos comandos por voz disponibles
para interactuar con TamTam Listens. 
Para evitar la influencia de las variaciones que pueden presentarse en una
explicaci\'on directa del facilitador al usuario, fue grabada una serie de videos instructivos 
que se presentan al usuario en esta etapa.
Adem\'as, cada usuario recibe una copia del manual de la aplicaci\'on, el cual describe en mayor detalle las
funcionalidades de la aplicaci\'on.

\subsubsection{Fase 3: Tareas}
Cada usuario realiza un total de 4 tareas durante la prueba de usabilidad:
	\begin{itemize}			
		\item Tareas 1 y 2: actividades simples y breves que buscan llevar a la pr\'actica los conceptos 
		aprendidos en los videos instructivos. Se realizan con ayuda del facilitador y el manual de aplicaci\'on.
		\item Tarea 3: actividad de mayor complejidad, la cual se realiza ya sin asistencia del facilitador.
		El usuario puede, sin embargo, consultar el manual de aplicaci\'on.
		\item Tarea 4: actividad m\'as compleja de la sesi\'on, la cual se realiza sin asistencia del facilitador
		ni el manual de aplicaci\'on. El resultado de esta tarea, si se lleva a cabo correctamente, es una sencilla pieza musical.
	\end{itemize}

La dificultad de las tareas, la cual va en aumento durante el desarrollo de la prueba, 
est\'a definida en funci\'on de la cantidad y diversidad de los comandos necesarios para completarlas.

\subsubsection{Fase 4: Encuesta de Usabilidad}
Luego de finalizadas las tareas, los usuarios completan una breve encuesta relacionada a la usabilidad de
la aplicaci\'on y a las interfaces por voz del usuario en general.
Se recoge la opini\'on de los usuarios con respecto a las palabras y comandos utilizados, la duraci\'on del
entrenamiento y su predisposici\'on a utilizar interfaces de usuario basadas en el habla. Las respuestas se
registran utilizando una escala de Likert \cite{Allen:2007}.
Adem\'as, en cada punto, se solicitan sugerencias de parte del usuario. 	

\subsubsection{Fase 5: Recopilaci\'on y An\'alisis de Datos} 
Al finalizar cada sesi\'on, se recopilan los siguientes datos:
	\begin{itemize}			
		\item Registro de resultados del test de memoria.
		\item Registro de las notas del observador.
		\item Grabaci\'on de las tareas realizadas por el usuario.
		\item Encuesta completada por el usuario.
	\end{itemize}

Una vez finalizadas todas las pruebas, se procede a la sumarizaci\'on y an\'alisis de los datos disponibles
a partir de estos materiales. Los factores que se tienen en cuenta en esta etapa se describen en la
siguiente secci\'on.


%!TEX root = ../tesis.tex
\section{Factores Analizados}
\label{sec:factores}

Esta sección describe los factores que se tuvieron en cuenta para el ánalisis realizado
en base a los datos recopilados a partir de las pruebas de usabilidad que se realizaron.
Además, se definen las métricas utilizadas para la cuantificación de cada factor estudiado.

\subsection{Memoria del Usuario}
Factor relacionado a la capacidad de retención de información del sujeto del experimento.
Se consideró interesante su análisis debido a que:
\begin{itemize}
	\item TamTam Listens se trataba de una aplicación completamente nueva para el usuario.
	\item Los comandos de voz deben ser memorizados por el usuario para la interacción con
	la apliación.
\end{itemize}
\subsubsection{Métrica Utilizada}
La memoria del usuario se midió a través de los resultados del test de memoria que se llevó
a cabo con cada usuario como parte de la prueba de usabilidad.
El resultado del test de memoria administrado es la cantidad de palabras, de un total de 15,
que el usuario consiguió recordar al cabo de 5 repeticiones, siendo 12--13 el resultado promedio
esperado.

\subsection{Correctitud de la aplicación}
Factor relacionado al comportamiento esperado por parte de un sistema de reconocimiento del
habla. Se busca responder a la pregunta: ¿Qué tan correctamente se reconocen los comandos 
del usuario?
\subsubsection{Métricas Utilizadas}
\begin{itemize}
	\item Tasa de Aciertos de Comandos: se considera que el sistema reconoció correctamente
	un comando, si el usuario efectivamente pronunció el mismo.

	La tasa de aciertos es la razón entre la cantidad de comandos correctamente reconocidos 
	y la cantidad de comandos correctamente pronunciados por el usuario.
	
	Sean:

	\begin{itemize}
		\item $r_{ij}$ la cantidad de veces que la aplicación reconoció correctamente el \mbox{comando $i$}
		al usuario $j$.
		\item $c_{ij}$ la cantidad de veces que el usuario $j$ pronunció correctamente el \mbox{comando $i$.}
	\end{itemize}
	La tasa de aciertos del comando $i$ para el usuario $j$ se define como: 

	\begin{equation*}
		a_{{comando}_{ij}}=\frac{r_{ij}}{c_{ij}}
	\end{equation*}


	Sea $N_{j}$ la cantidad de comandos diferentes pronunciados por el usuario $j$,
	la tasa de aciertos para el usuario  se define como:
	
	\begin{equation*}
		a_{{usuario}_j}=\frac{\sum_k^{N_{j}}a_{{comando}_{kj}}}{N_{j}}
	\end{equation*}

	La tasa de aciertos del sistema se obtuvo hallando el promedio de los valores correspondientes
	a cada usuario.


	\item Tasa de Falsa Alarma: un falso positivo es un comando incorrecto (por causa del error humano)
	reconocido como válido por la aplicación.
	La tasa de falsa alarma es la razón entre la cantidad de falsos positivos y la cantidad
	de comandos incorrectos. 

	Sean:

	\begin{itemize}
		\item $m_{ij}$ la cantidad de veces que la aplicación reconoció incorrectamente el \mbox{comando $i$}
		al usuario $j$.
		\item $p_{ij}$ la cantidad de veces que el usuario $j$ pronunció incorrectamente el \mbox{comando $i$.}
	\end{itemize}
	La tasa de falsa alarma del comando $i$ para el usuario $j$ se define como: 

	\begin{equation*}
		f_{{comando}_{ij}}=\frac{m_{ij}}{p_{ij}}
	\end{equation*}

	El cálculo de la tasa de falsa alarma de cada usuario y del sistema se llevó a cabo de forma análoga 
	a la tasa de aciertos.
\end{itemize}

\subsection{Error Humano}
Factor relacionado a las equivocaciones por parte del usuario en su interacción con
TamTam Listens.
Un comando se consideró erróneo o incorrecto por parte del usuario si:
\begin{itemize}
	\item El usuario no respetó el orden de las palabras: esto es, pronunció las palabras de un comando en el orden incorrecto.
	\item El usuario se confundió entre dos comandos: esto es, utilizó palabras correspondientes a más de un comando.
	\item El usuario juntó dos o más comandos: esto es, pronunció de una sola vez dos o más comandos.
	\item El usuario dudó mientras pronunciaba el comando: esto es, titubeó o repitió una palabra del comando.
	\item El usuario utilizó comandos incompletos: esto es, pronunciando solo algunas palabras del comando.
	\item El usuario utilizó palabras que no forman parte del lenguaje.
\end{itemize}
\subsubsection{Métricas Utilizadas}
\begin{itemize}
	\item Cantidad de Errores: sumatoria de errores por comando y por usuario, clasificados en
	alguno de los tipos previamente mencionados.
	\item Tasa de Error Humano: se define como la razón entre la cantidad de comandos incorrectos
	y la cantidad de comandos pronunciados.
	Sean:

	\begin{itemize}
		\item $p_{ij}$ la cantidad de veces que el usuario $j$ pronunció incorrectamente el \mbox{comando $i$.}
		\item $t_{ij}$ la cantidad de veces que el usuario $j$ pronunció (correcta o incorrectamente) el 
		\mbox{comando $i$.}
	\end{itemize}
	La tasa de error humano del comando $i$ para el usuario $j$ se define como: 

	\begin{equation*}
		e_{{comando}_{ij}}=\frac{p_{ij}}{t_{ij}}
	\end{equation*}

	El cálculo de la tasa de error humano de cada usuario se llevó a cabo de forma análoga 
	a la tasa de aciertos.
\end{itemize}

\subsection{Eficiencia}
Factor relacionado al grado de productividad que logra el usuario de la aplicación.
\subsubsection{Métricas Utilizadas}
\begin{itemize}
	\item Duración de Tareas Uno y Dos: es la suma del tiempo que le tomó al usuario
	terminar las primeras dos tareas de la prueba, las cuales se realizaron con asistencia por
	parte del facilitador.
	\item Duración de Tareas Tres y Cuatro: es la suma del tiempo que le tomó al usuario
	terminar las últimas dos tareas de la prueba, las cuales se realizaron sin asistencia por
	parte del facilitador.
	\item Correctitud de la Tarea Cuatro: para él cálculo de esta medida, se descompuso la
	última tarea en lo distintos pasos necesarios para completarla correctamente.
	De esta manera, la tarea cuatro se consideró compuesta por 72 elementos:
	\begin{itemize}
		\item Creación de 54 notas.
		\item Selección de 4 instrumentos.
		\item Cambios en la duración de 12 notas.
		\item Reproducción de la música.
		\item Exportación de la música.
	\end{itemize}
	La correctitud de la tarea 4 para cada usuario se define como la razón entre la cantidad de operaciones
	correctamente realizadas y el total de operaciones (72).
\end{itemize}

\subsection{Satisfacción del Usuario}
Factor relacionado a la opinión del usuario sobre su interacción con la aplicación y las interfaces
basadas en reconocimiento del habla en general.
\subsubsection{Métricas Utilizadas}
La opinión del usuario se midió a través de los resultados de la encuesta realizada como parte de
la prueba de usabilidad, posterior a la finalización de las tareas. Las respuestas se
registraron utilizando una escala de Likert \cite{Allen:2007} con valores del 1 al 7.

Se solicitó la opinión del usuario con respecto a:
\begin{itemize}
	\item Las palabras utilizadas.
	\item Los comandos utilizados.
	\item Duración del entrenamiento.
	\item Interfaces por Voz del Usuario.
\end{itemize}






%!TEX root = ../tesis.tex
\chapter{Resultados Experimentales}
\label{sec:resultados}


% introduccion
Las pruebas experimentales, descritas en el cap\'itulo anterior, arrojaron una serie de
resultados interesantes que ser\'an presentados en este cap\'itulo para su posterior an\'alisis. Los
distintos valores obtenidos para las variables consideradas ser\'an presentados de manera tabular, dichos
valores ser\'an analizados para identificar correlaciones.

En la tabla~\ref{sec:tabla-encuesta}
se puede observar un resumen de la encuesta realizada a cada sujeto despu\'es de haber realizado el experimento. Por otro lado
la tabla~\ref{sec:tabla-resumen-experimento} muestra los n\'umeros que resumen las pruebas experimentales realizadas donde 
una fila de la tabla hace referencia a un sujeto del experimento y 
las columnas a las m\'etricas consideradas.

\begin{table}[H]
\centering
\footnotesize
\begin{tabular}{|r|r|r|r|r|}
\hline
    Sujeto & Palabras & Comandos & Entrenamiento & Interfaz por Voz \\
\hline
    1 & 6 & 7 & 6 & 7 \\
    2 & 6 & 7 & 7 & 7 \\
    3 & 7 & 7 & 7 & 7 \\
    4 & 6 & 6 & 4 & 5 \\
    5 & 7 & 7 & 7 & 4 \\
    6 & 6 & 6 & 5 & 5 \\
    7 & 7 & 7 & 7 & 5 \\
    8 & 6 & 7 & 6 & 7  \\
    9 & 6 & 7 & 7 & 5  \\
    10 & 5 & 6 & 6 & 6  \\
    11 & 6 & 6 & 6 & 7  \\
    12 & 6 & 6 & 7 & 5  \\
\hline
\end{tabular}
\caption{Resumen de la encuesta realizada despu\'es del experimento}
\label{sec:tabla-encuesta}
\end{table}

\begin{table}[H]
\centering
\footnotesize
\begin{tabular}{|p{1.6cm}|p{1.6cm}|p{1.6cm}|p{1.6cm}|p{1.6cm}|p{1.6cm}|p{1.6cm}|}
\hline
    Tasa de acierto & \% Error Humano & T1 + T2 & T3 + T4 & Cantidad de Errores & Correctitud T4 & Memoria \\
    \hline
    58,45 & 0 & 10,65 & 17,43 & 4 & 58,33 & 14 \\
    70,14 & 6,92 & 19,68 & 24,03 & 19 & 91,67 & 10 \\
    93,73 & 2,75 & 10,88 & 13,52 & 9 & 94,44 & 14 \\
    66,73 & 0 & 16,02 & 33,78 & 6 & 69,44 & 11 \\
    93,35 & 8,92 & 18,97 & 23,77 & 22 & 100 & 8 \\
    80,51 & 3,46 & 14,6 & 18,23 & 4 & 91,67 & 14 \\
    87,62 & 4,98  &  11,5 & 9,97 & 6 & 100 & 13 \\
    90,45 & 7,48 & 7,02 & 11,78 & 6 & 100 & 12 \\
    93,49 & 13,14 & 15,27 & 18,4 & 21 & 100 & 14 \\
    87,64 & 5,12 & 7,15 & 9,1 & 7 & 70,83 & 14 \\
    89,32 & 1,22 & 12,92 & 18 & 6 & 100 & 15 \\
    94,42 & 9,83 & 21,42 & 22,27 & 25 & 73,61 & 14 \\
\hline
\end{tabular}
\caption{Resumen de los valores de las variables consideradas en el experimento}
\label{sec:tabla-resumen-experimento}
\end{table}

A partir de los valores en la tabla~\ref{sec:tabla-resumen-experimento} y utilizando el \emph{Coeficiente de Correlaci\'on de Pearson}\cite{BoslaughStatistics2008} se llev\'o a cabo un an\'alisis  para
identificar correlaciones entre las variables consideradas en el experimento. El coeficiente de Pearson es una medida
del grado de correlaci\'on lineal o dependencia entre dos variables $X$ e $Y$. El valor del coeficiente se encuentra entre
-1 y 1 inclusive. El valor -1 indica que las variables est\'an correlacionadas negativamente (cuando $X$ crece, $Y$ decrece y viceversa),
0 indica que no existe correlaci\'on y 1 que existe una correlaci\'on positiva (cuando $X$ crece, $Y$ crece)

En la tabla que se muestran a continuaci\'on se pueden visualizar los coeficientes de correlaci\'on entre las variables estudiadas.

\begin{table}[H]
\centering
\footnotesize
\begin{tabular}{|p{1.8cm}|p{1.6cm}|p{1.6cm}|p{1.6cm}|p{1.6cm}|p{1.6cm}|p{1.6cm}|p{1.6cm}|}
\hline
                    &    Tasa de Acierto & \% Error Humano & T1 + T2 & T3 + T4 & Cantidad de Errores & Correctitud T4 & Memoria \\
\hline
\% Tasa de Acierto     & 1 & 0.65 & 0.14 & -0.09 & 0.71 & 0.50 & 0.22 \\
\% Error Humano        & 0.65 & 1 & 0.34 & 0.12 & 0.75 & 0.45 & -0.23 \\
T1 + T2                & 0.14 & 0.34 & 1 & 0.87 & 0.58 & -0.04 & -0.31 \\
T3 + T4                & -0.09 & 0.12 & 0.87 & 1 & 0.36 & -0.12 & -0.42 \\
Cantidad de Errores    & 0.71 & 0.75 & 0.58 & 0.36 & 1 & 0.23 & -0.23 \\
Correctitud T4         & 0.50 & 0.45 & -0.036 & -0.16 & 0.23 & 1 & -0.08 \\
Memoria                & 0.22 & -0.23 & -0.31 & -0.42 & -0.23 & -0.08 & 1 \\
\hline
\end{tabular}
\caption{Coeficientes de correlaci\'on para las variables del experimento}
\label{sec:tabla-correlacion}
\end{table}

Como se puede observar, existen correlaciones entre alguna de las variables presentadas, siendo algunas positivas y otras negativas. A 
continuaci\'on se listan las correlaciones identificadas:

\begin{itemize}
    \item Tasa de Acierto y \% Error Humano, 0.49 correlaci\'on positiva fuerte
    \item Tasa de Acierto y Cantidad de Errores, 0.71 correlaci\'on positiva muy fuerte
    \item Tasa de Acierto y Correctitud T4, 0.50 correlaci\'on positiva fuerte 
    \item \% de Error Humano y T1 + T2, 0.34 correlaci\'on positiva moderada
    \item \% de Error Humano y Correctitud T4, 0.45 correlaci\'on positiva fuerte
    \item T1 + T2 y T3 + T4, 0.87 correlaci\'on positiva muy fuerte
    \item T1 + T2 y Memoria, -0.31 correlaci\'on negativa moderada
    \item T1 + T2 y Cantidad de Errores, 0.58 correlaci\'on positiva fuerte
    \item T3 + T4 y Cantidad de Errores, 0.36 correlaci\'on positiva moderada
    \item T3 + T4 y Memoria, -0.42 correlaci\'on negativa fuerte
    \item Cantidad de Errores y Memoria, -0.23 correlaci\'on negativa d\'ebil
\end{itemize}

%!TEX root = ../tesis.tex
\chapter{Conclusiones}
\label{sec:conclusiones}

La realizaci\'on de este trabajo final de grado supuso primeramente la necesidad de estudiar
el \'area de reconocimiento del habla: su historia, el estado del arte y los fundamentos
te\'oricos involucrados en el proceso b\'asico de reconocimiento.

Se realiz\'o posteriormente una evaluaci\'on de varias herramientas disponibles para la
implementaci\'on de un sistema de reconocimiento del habla, con respecto a un conjunto
criterios propuestos, de modo a facilitar la selecci\'on de la herramienta adecuada de
acuerdo al caso.

Para llevar a la pr\'actica lo aprendido, se dise\~n\'o e implement\'o una interfaz por voz
del usuario para una aplicaci\'on simple de composici\'on musical y se llevaron a cabo
pruebas de la misma con usuarios.

Este cap{\'\i}tulo presenta las conclusiones que se obtuvieron durante el transcurso
del trabajo realizado, acerca del reconocimiento del habla y su aplicaci\'on a las
interfaces de usuarios.

%!TEX root = ../tesis.tex
\section{Dise\~no de la Interfaz}
\label{sec:disenho-interfaz}

El principal objetivo de la fase de dise\~no de la interfaz por voz del usuario consisti\'o
en definir el lenguaje que se utilizar{\'\i}a para la misma y el protocolo de interacci\'on entre
el usuario y la aplicaci\'on.

Es decir, se definieron los comandos que tendr{\'\i}an significado para el sistema y la manera
en que el usuario utilizar{\'\i}a los mismos para acceder a las funcionalidades implementadas.

Las conclusiones relacionadas a esta etapa se mencionan y describen a continuaci\'on.

\subsection{La naturalidad del lenguaje es de gran importancia para la interfaz}
Al utilizar una interfaz basada en reconocimiento del habla el usuario debe recordar los
comandos disponibles, de modo a utilizarlos para interactuar con el sistema.

La utilizaci\'on de comandos poco apropiados puede llevar a una interacci\'on innecesariamente
complicada entre la persona y la aplicaci\'on, dificultando el aprendizaje
de los comandos por parte del usuario.

Por el contrario, la utilizaci\'on de comandos naturales (palabras intuitivas) para el usuario puede colaborar
con la fluidez de la interacci\'on, facilitando el aprendizaje de los comandos por parte del usuario. 

Por este motivo, el primer punto que se consider\'o importante fue la naturalidad de los comandos
de acuerdo al dominio de la aplicaci\'on.

En el caso particular de \emph{TamTam Listens} esto signific\'o buscar t\'erminos musicales
que pudiesen asociarse con los elementos que formaban parte de la aplicaci\'on: partitura,
pista, comp\'as, nota, etc.

\subsection{Interactuar con la aplicaci\'on, no con la interfaz gr\'afica}
Un error que puede cometerse al intentar dise\~nar una interfaz por voz del usuario,
el cual estaba presente en el primer prototipo de \emph{TamTam Listens}, es el intento
de asociar comandos de voz a acciones que se realizar{\'\i}an normalmente con la interfaz
gr\'afica.

El reemplazar directamente clics y pulsaciones de teclas por comandos de voz da
como resultado una interfaz poco natural y dif{\'\i}cil de utilizar. Por tanto, fue
necesario replantear por completo el modo de acceso a las funcionalidades de manera
independiente a la interfaz gr\'afica.

En base a la experiencia, resulta recomendable asociar un comando a una funcionalidad
y no a una \'unica acci\'on en la interfaz tradicional. A modo de ejemplo, cambiar el
volumen en \emph{TamTam Edit} requer{\'\i}a presionar un bot\'on para desplegar un submen\'u 
y utilizar un {slider}, mientras que en \emph{TamTam Listens} era necesario un
\'unico comando ``Aumentar/Disminuir Volumen''.

\subsection{Utilizar el sonido como medio de retroalimentaci\'on}
El protocolo de interacci\'on entre el usuario y la aplicaci\'on puede pensarse de forma similar a una 
conversaci\'on entre personas. El usuario accede a las funcionalidades a trav\'es de comandos de voz.
A su vez, la aplicaci\'on puede utilizar la voz, o el sonido en general, para comunicar un mensaje al usuario.

Esta \'ultima posibilidad puede ser \'util como medio de retroalimentaci\'on entre la aplicaci\'on y el usuario.
Por ejemplo, se pueden utilizar notificaciones de audio para confirmar la realizaci\'on de una determinada
operaci\'on. Esto resulta especialmente importante teniendo en cuenta el error propio de todo sistema
de reconocimiento del habla.

Adem\'as, utilizando s{\'\i}ntesis del habla, se puede implementar un recordatorio de los comandos
disponibles. Esta es una herramienta v\'alida para facilitar el aprendizaje del lenguaje por parte del
usuario.

En el caso particular de \emph{TamTam Listens}, se utilizaron notificaciones de audio para confirmar
ciertas operaciones. Por ejemplo, al crear o editar una nota, se reproduc{\'\i}a la misma a modo de
confirmaci\'on.


%!TEX root = ../tesis.tex
\section{Implementaci\'on de la Interfaz}
\label{sec:implementacion-interfaz}

Una vez que se tuvo en claro lo que se buscaba implementar, el siguiente paso
fue la selecci\'on de las herramientas que se utilizar{\'\i}an para hacerlo. 

La herramienta elegida tiene gran influencia sobre el posterior proceso de
desarrollo, por lo cual esta decisi\'on debe tomarse analizando las caracter{\'\i}sticas
propias del proyecto en cuesti\'on.

A modo de ejemplo, algunas cuestiones que pueden tomarse en consideraci\'on son:

\begin{itemize}
	\item El tiempo del que se dispone.
	\item El dinero del que se dispone.
	\item El conocimiento t\'ecnico del equipo de desarrolladores.
	\item La plataforma sobre la cual debe ejecutarse el sistema.
	\item La necesidad de que el sistema funcione sin conexi\'on a internet.
	\item El soporte existente para el idioma que se busca reconocer.
\end{itemize}

La evaluaci\'on de varias opciones disponibles para la implementaci\'on de un sistema
basado en reconocimiento del habla, cuyos resultados se incluyen como parte de este
trabajo, permiti\'o realizar la selecci\'on de manera debidamente informada y justificada.

Habiendo seleccionado Pocketsphinx y el modelo ac\'ustico Voxforge para
la implementaci\'on, el mayor problema durante la fase de desarrollo estuvo relacionado
al plugin para \emph{Gstreamer} que se planeaba utilizar para integrar \emph{TamTam Edit}
con el motor de reconocimiento.

Luego de numerosos intentos fallidos, finalmente se opt\'o por desechar el mencionado plugin
e implementar una soluci\'on utilizando \emph{DBus} para la integraci\'on. Al exponerse los
resultados del reconocimiento del habla mediante un demonio, este enfoque permite integrar
Pocketsphinx con cualquier aplicaci\'on que utilice \emph{DBus} como m\'etodo de comunicaci\'on
entre procesos.

Otro inconveniente que puede mencionarse es la elevada tasa de error que se obtuvo en las
pruebas preliminares para ciertos comandos. Este problema se presentaba especialmente para
los comandos de una o dos palabras y oblig\'o a realizar modificaciones sobre el lenguaje
inicialmente planteado.
%!TEX root = ../tesis.tex
\section{Prueba con Usuarios}
\label{sec:prueba}

En esta sección se presentan las conclusiones que pudieron obtenerse a partir de los
resultados de las pruebas de \emph{TamTam Listens} realizadas con usuarios.

\subsection{Correlación}
\section{An\'alisis del Error}
\section{Encuesta}





%!TEX root = ../tesis.tex
\chapter{Trabajos Futuros}
\label{sec:trabajos-futuros}

%\section{Aplicaciones}
\label{sec:aplicaciones}

% introduccion

Las soluciones de reconocimiento del habla, catalogadas como aplicaciones, propocionan
a los usuarios finales los medios necesarios para realizar determinadas tareas mediante
la voz: dictado autom\'atico, navegaci\'on de interfaces, etc\'etera; sin
la necesidad de tener un conocimiento previo de los conceptos relacionados al
reconocimiento del habla. \'Estas aplicaciones presentan las siguientes 
caracter\'isticas en cuanto a los criterios generales de evaluaci\'on:

\begin{itemize}
    \item Conocimiento t\'ecnico: requieren de poco conocimiento t\'ecnico debido a que \'estas
        soluciones se ofrecen como un producto a los usuarios finales, por lo tanto, sus funcionalidades
        pueden utilizarse directamente sin necesidad de entender los componentes que intervienen
         en el sistema.
    \item Productividad: teniendo en cuenta el criterio anterior, cabe resaltar que con
        las aplicaciones se puede lograr un buen grado de productividad, porque los esfuerzos
        de los usuarios se enfocan en realizar determinadas tareas utilizando las funcionalidades
         disponibles.
    \item Flexibilidad: la flexibilidad para esta categor\'ia de soluciones es reducida, porque las
        funcionalidades que ofrece cada aplicaci\'on se encuentran orientadas a resolver tareas
        espec\'ificas: dictado autom\'atico, para la transcripci\'on de documentos; reconocimiento
        de comandos, para navegar interfaces; reconocimiento del habla, para 
        automatizaci\'on de servicios de operador. Cada aplicaci\'on tiene un \'ambito de aplicabilidad
        lo cual impone un l\'imite en su flexibilidad, en comparaci\'on a otras categor\'ias que
         se ver\'an m\'as adelante.
\end{itemize}

\subsection{Criterios Espec\'ificos de Evaluaci\'on}

Los criterios espec\'ificos nos permiten evaluar a las aplicaciones en cuanto a factores
particulares y relevantes para esta categor\'ia. A continuaci\'on se presentan los criterios 
espec\'ificos de evaluaci\'on para las aplicaciones:

\begin{itemize}
    \item Precio
    \item Soporte para m\'ultiples idiomas
\end{itemize}

\subsection{Ejemplos de Aplicaciones}

\subsection{Simon}
\label{sec:simon}

\subsubsection{Palaver}
\label{sec:palaver}

\subsection{Nuance}
\label{sec:nuance}



\appendix   % inician los apendices de tu tesis
	
% los cap'itulos que incluyas a partir de aqu'i aparecen 
% como ap'endices
%!TEX root = ../tesis.tex
\chapter{Documentos del Experimento}
%!TEX root = ../tesis.tex

\section{Descripci\'on del Estudio de Usabilidad}

\subsubsection{Introducci\'on}

La suite TamTam es un conjunto de actividades que permite componer m\'usicas sencillas en el entorno de 
escritorio Sugar. TamTam Edit es una actividad que forma parte de TamTam, la cual presenta una interfaz 
gr\'afica basada en una grilla de pistas y compases para la composici\'on musical. 
\foreign{TamTam Listens} es un proyecto basado en TamTam Edit, que a\~nade una interfaz mediante de 
voz del usuario a la aplicaci\'on.

Como parte del trabajo de grado de los autores, se realizar\'a una prueba con usuarios a fin de evaluar los conceptos e hip\'otesis planteados. Las variables a ser medidas para su posterior an\'alisis.

\subsubsection{Objetivos}

\begin{itemize}
\item Evaluar la aplicaci\'on desarrollada, teniendo en cuenta aspectos t\'ecnicos y funcionales.
\item Identificar problemas que se presentan al utilizar una interfaz mediante voz y, de ser posible, sugerir posibles soluciones.
\item Proponer criterios que puedan utilizarse para la evaluaci\'on de interfaces mediante voz.
\item Verificar y validar la utilidad de los criterios propuestos, de acuerdo a la informaci\'on y las conclusiones que pueden obtenerse mediante los mismos.
\item Obtener conclusiones relacionadas a factores externos a la aplicaci\'on, como la memoria del usuario y el tiempo de entrenamiento previo.
\end{itemize}

\subsubsection{Metodolog\'ia}
Al inicio de la sesión se tomará un test de memoria del usuario
Luego de una serie de videos instructivos introductorios, el usuario llevar\'a a cabo:

\begin{itemize}
    \item 2 tareas con asistencia del manual de la aplicaci\'on y el responsable de la prueba.
    \item 1 tarea con asistencia del manual de aplicaci\'on \'unicamente.
    \item 1 tarea sin asistencia.
\end{itemize}

Las tareas ir\'an en orden incremental de dificultad y su desarrollo se grabar\'a para su posterior 
an\'alisis.

Finalmente, el usuario completará una encuesta de modo a compartir su apreciación subjetiva de la sesión.

\subsubsection{Ambiente de Pruebas}
Las sesiones se realizarán en un Laboratorio de Inform\'atica de la Facultad Poĺit\'ecnica. Esto
permitirá minimizar los efectos de la interferencia sonora y humana durante el
desarrollo de las pruebas.

\subsubsection{Usuarios}
Las pruebas se realizarán con 12 usuarios, estudiantes de Ingenier{\'\i}a Inform\'atica
de la Facultad Polit\'ecnica, Universidad Nacional de Asunci\'on. Ninguno de los usuarios 
deberá contar con experiencia previa con la aplicación evaluada.

\subsubsection{Variables a considerar: {?`}Qu\'e se quiere medir?}

\begin{itemize}
    \item Memoria del usuario: capacidad de retenci\'on de informaci\'on del sujeto de la prueba. 
    Podr\'ia ser importante al tratarse de una interfaz nueva para el mismo. 
    \item Precisi\'on: relacionada al sistema de reconocimiento autom\'atico del habla. 
    {?`}Qu\'e tan correctamente se reconocen los comandos del usuario?
    \item Eficiencia: relacionada al grado de productividad que logra el usuario al realizar tareas con
    la aplicaci\'on.
    \item Error del usuario: {?`}Qu\'e tan seguido comete errores el usuario? {?`}Qu\'e tan f\'acil le es recuperarse? {?`}Qu\'e factores podrían estar relacionados al error humano?
    \begin{itemize}
        \item Complejidad de la gram\'atica.
        \item Duración de la interacción.
    \end{itemize}
    \item Satisfacci\'on del usuario: relacionada a la opini\'on del usuario sobre su interacci\'on con la aplicaci\'on.
    \begin{itemize}
        \item Naturalidad de las palabras.
        \item Naturalidad de los comandos (secuencias de palabras).
        \item Duración del entrenamiento.
        \item Predisposición a utilizar interfaces por voz del usuario.
    \end{itemize}
\end{itemize}



\subsubsection{M\'etricas a utilizar: {?`}C\'omo se va a medir?}

\begin{itemize}
    \item Test de Memoria de Rey.
    \item Cuestionario del Responsable de la Prueba (se completar\'a luego de observar la grabaci\'on)
    Se clasifica cada comando pronunciado, incluyendo una marca de tiempo:
    \begin{itemize}
        \item Comandos correctamente pronunciados y reconocidos.
        \item Comandos correctamente pronunciados y no reconocidos.
        \item Comandos incorrectamente pronunciados y reconocidos.
        \item Comandos incorrectamente pronunciados y no reconocidos.
    \end{itemize}
    En base a estos valores puede calcularse la tasa de error del sistema y del usuario.
    \item Grado de correctitud de la tarea 4.
    \begin{itemize}
    \item $ \frac{Operaciones \:correctamente \:realizadas}{Total \:de \:Operaciones}$
    \end{itemize}
    \item Cuestionario al Sujeto:
    \begin{itemize}
        \item Las palabras son adecuadas/intuitivas 1-7.
        \item Los comandos son adecuados/intuitivos 1-7.
        \item El tiempo de entrenamiento result\'o 1-7.
        \item Utilizaría una interfaz por voz del usuario cotidianamente 1-7.
    \end{itemize}
\end{itemize}

\subsubsection{An\'alisis: {?`}C\'omo se obtienen las conclusiones?}
\begin{itemize}
    \item Memoria del Usuario:
    \begin{itemize}
        \item Test de memoria.
    \end{itemize}
    \item Precisi\'on:
    \begin{itemize}
        \item \% de error de comandos.
    \end{itemize}
    \item Eficiencia:
    \begin{itemize}
        \item Grado de correctitud de la tarea 4.
    \end{itemize}
    \item Error del Usuario:
    \begin{itemize}
        \item Conteo y Clasificaci\'on de Errores por Tarea.
    \end{itemize}
    \item Satisfacci\'on del Usuario:
    \begin{itemize}
        \item Cuestionario Sujeto: naturalidad palabras / comandos - tiempo de entrenamiento.   
    \end{itemize}
    \item Correlaci\'on entre variables estudiadas.
\end{itemize}

Para información más detallada acerca de la prueba de usabilidad, consultar el capítulo 
\ref{sec:evaluacion}.
\section{Protocolo del Experimentador}

Este documento representa la gu\'ia a ser utilizada por el experimentador para llevar a cabo el 
experimento. 
Describe los pasos a seguir para la preparaci\'on, realizaci\'on y registro de resultados del experimento en cuesti\'on.

\subsection{Antes del Experimento}

\begin{enumerate}
    \item Instale el equipamiento del experimento (computadora port\'atil, \foreign{headset}, materiales, etc.) en 
    el lugar f\'isico escogido para su realizaci\'on. Tenga en cuenta que el nivel de interferencia 
    sonora debe ser lo m\'as bajo posible.
    \item Encienda la computadora port\'atil y configurar el equipamiento. Seleccione el \foreign{headset Microsoft LifeChat LX-3000} como dispositivo de sonido, con el volumen de salida al 100% y el volumen de entrada al 90%.
    \item Verifique la disponibilidad de los materiales del experimento:
    \begin{itemize}
        \item Screencasts de la aplicaci\'on.
        \item Protocolo del Sujeto del Experimento.
        \item Gu\'ias para las tareas asistidas y no asistidas.
        \item Manual de la aplicaci\'on.
        \item Formulario para el Sujeto del Experimento.
    \end{itemize}
    \item Realice una prueba b\'asica de \foreign{TamTam Listens} de modo a verificar su 
    correcto funcionamiento. 
\end{enumerate}

\subsection{Durante el Experimento}

\begin{enumerate}
    \item De la bienvenida al sujeto del experimento, present\'andose, para proceder a ubicarlo en el lugar correspondiente (sentado frente a la computadora port\'atil).
    \item De una breve introducci\'on de \foreign{TamTam Listens} y el objetivo del experimento. 
    De ser necesario, puede utilizarse el manual de la aplicaci\'on y el documento del experimento como 
    material de apoyo.
    \item Explique y realice el test de memoria. Utilice como gu\'ia para este paso el documento titulado 
    ‘Test de Memoria - Protocolo’.
    \item Reproduzca el primer screencast introductorio de TamTam Listens en la computadora port\'atil, 
    estando el sujeto observando el monitor y escuchando a trav\'es del \foreign{headset}.
    \item Una vez terminada la reproducci\'on del screencast, consulte al sujeto si tiene dudas. De 
    existir alguna, acl\'arela.
    \item Entregue al sujeto el manual de aplicaciòn y perm\'itale leerlo. De existir alguna, acl\'arela.
    \item De inicio a la grabaci\'on, utilizando el script record-screen.sh, de las tareas
    correspondientes al experimento. Recuerde que deber\'a abrir y cerrar la aplicaci\'on entre 
    cada tarea.
    \item Entregue al sujeto de experimento la gu\'ia correspondiente a las tareas asistidas. De una 
    breve explicaci\'on de ser necesaria.
    \item Aguarde en silencio y ligeramente alejado mientras el sujeto del experimento realiza la tarea 
    asistida n\'umero 1. Durante estas tareas se admiten preguntas del sujeto, las cuales debe responder 
    con la mayor claridad posible. El tiempo l\'imite para la realizaci\'on de esta tarea es de 
    10 minutos.
    \item Reproduzca el segundo screencast de \foreign{TamTam Listens} en la computadora port\'atil,
    estando el sujeto observando el monitor y escuchando a trav\'es del \foreign{headset}. De nuevo, aclare 
    cualquier duda que se presente.
    \item Aguarde en silencio y ligeramente alejado mientras el sujeto del experimento realiza la tarea asistida n\'umero 2. Durante esta tarea se admiten preguntas del sujeto, las cuales debe responder con la mayor claridad posible. El tiempo l\'imite para la realizaci\'on de esta tarea es de 10 minutos.
    \item El sujeto debe realizar la tarea n\'umero 3 ayudado solamente por el manual de la aplicaci\'on y le corresponde anunciar cuando haya finalizado. El tiempo l\'imite para la realizaci\'on de esta tarea es de 10 minutos.
    \item Retire la gu\'ia correspondiente a las tareas asistidas y el manual de la aplicaci\'on. Entregue la gu\'ia correspondiente a la \'ultima tarea. No existe tiempo l\'imite para esta tarea, aunque el sujeto puede decidir no continuar. Haga \'enfasis en este punto.
    \item El sujeto deber\'a realizar la \'ultima tarea  sin asistencia alguna, y le corresponde avisar cuando haya finalizado o se haya dado por vencido.
    \item Entregue el formulario para el sujeto del experimento, donde se esperan capturar las impresiones de \'este. Aguarde sin interferir mientras se completa el formulario.
    \item Agradezca la colaboraci\'on del sujeto del experimento.
\end{enumerate}


\subsection{Despu\'es del Experimento}

 Por cada tarea asistida realizada por el sujeto del experimento,deber\'a observar la grabaci\'on correspondiente y completar los siguientes apartados en el formulario del experimentador:

\subsubsection{Conteo de Resultados}

 Los comandos reconocidos por TamTam se clasifican en cuatro grupos:

\begin{itemize}
    \item Comandos Generales
    \begin{itemize}
        \item  Reproducir
        \item Pausar
        \item Parar
        \item Exportar
        \item Salir
        \item Reproducir Pista
        \item Habilitar/Silenciar Pista
    \end{itemize}
    \item Comandos de Configuraci\'on
    \begin{itemize}
        \item Control de Volumen
        \item Control de Tempo
        \item Selecci\'on de Instrumento
    \end{itemize}
    \item Comandos de Navegaci\'on
    \begin{itemize}
        \item Selecci\'on de Pista
        \item Selecci\'on de Comp\'as
        \item Selecci\'on de Tiempo
    \end{itemize}
    \item Comandos de Composici\'on
    \begin{itemize}
        \item Crear M\'usica
        \item Crear Nota
        \item Crear P\'agina
        \item Duplicar P\'agina
        \item Duplicar Nota
        \item Borrar Nota
        \item Editar Nota
    \end{itemize}
\end{itemize}

Por cada comando que haya pronunciado el sujeto, determine el grupo al que pertenece. Luego, en la tabla dispuesta para tal efecto, registre el resultado del comando, el cual puede ser uno de los siguientes:
\begin{itemize}
    \item Comando bien pronunciado y reconocido: comportamiento correcto del sistema.
    \item Comando bien pronunciado y no reconocido: se consideran como parte de este grupo los comandos v\'alidos que no hayan sido reconocidos o hayan sido reconocidos incorrectamente (como otro comando).
    \item Comando mal pronunciado y reconocido: palabras o frases fuera del lenguaje, comandos mal pronunciados, u otros errores del usuario que hayan sido reconocidos como un comando v\'alido (falso positivo).
    \item Comando mal pronunciado y no reconocido: comportamiento correcto del sistema. Error del usuario que se reconoce como tal.
\end{itemize}

Como resultado de este apartado se obtendr\'a la cantidad de ocurrencias para cada par <grupo, resultado>.
La precisi\'on debe calcularse sumando la cantidad de ocurrencias correspondientes a un comportamiento incorrecto del sistema y dividiendo sobre la cantidad total de ocurrencias.

Deben considerarse solamente los casos en que se pronunciaron correctamente los comandos, de modo a excluir el error humano del c\'alculo.


\subsubsection{An\'alisis del Error Humano}
En esta secci\'on, se clasifican los comandos mal pronunciados registrados en el apartado anterior en una de las siguientes categor\'ias:
\begin{itemize}
    \item El usuario no respet\'o el orden de las palabras: esto es, pronunci\'o las palabras de un comando en el orden incorrecto.
    \item El usuario se confundi\'o entre dos comandos: esto es, utiliz\'o palabras correspondientes a m\'as de un comando.
    \item El usuario dud\'o mientras pronunciaba el comando: esto es, titube\'o o repiti\'o una palabra del comando.
    \item El usuario utiliz\'o palabras que no forman parte del lenguaje.
\end{itemize}

Este conteo podr\'ia ayudar a identificar las principales dificultades del usuario con la interfaz.

\subsubsection{Observaciones}

En este apartado debe registrarse cualquier detalle adicional que pudiese, a criterio del experimentador, llevar a la obtenci\'on de conclusiones \'utiles.

Para la \'ultima tarea, adem\'as de los puntos ya mencionados, registre:
\begin{enumerate}
    \item Si el usuario termin\'o la tarea \\
    La tarea se considera terminada si el usuario sigui\'o una secuencia l\'ogica de comandos que llevar\'ian a su culminaci\'on (m\'as all\'a de los errores del reconocimiento).
    \item Grado de correctitud del resultado \\
    Para este punto son de utilidad las siguientes definiciones:
    \begin{itemize}
        \item Inserci\'on: operaci\'on de a\\\'unnadir una nota.
        \item Sustituci\'on: operaci\'on de reemplazar una nota por otra de distinto tono o duraci\'on.
        \item Eliminaci\'on: operaci\'on de eliminar una nota.
    \end{itemize}
\end{enumerate}

El grado se correctitud del resultado se estimar\'a mediante la Distancia de Levenshtein, es decir, la cantidad de inserciones + sustituciones + eliminaciones necesarias para llegar al resultado correcto (menos es mejor).

\section{Protocolo del Sujeto de la Prueba}

Este documento representa una gu\'ia a ser utilizada para llevar a cabo la prueba de usabilidad. 
Describe los pasos a seguir para cada etapa de la misma.

\subsection{Presentaci\'on y Test de Memoria}

\begin{enumerate}
    \item En primer lugar, el responsable de la prueba le comentar\'a brevemente acerca de 
    \foreign{TamTam Listens}, la prueba de la cual formar\'a parte y sus objetivos.
    \item Luego de la introducci\'on, realizar\'a un breve test de memoria guiado por el responsable de la prueba. Tiempo Aproximado: 10 minutos
    \newcounter{enumTemp}
    \setcounter{enumTemp}{\theenumi}
\end{enumerate}

\subsection{Videotutorial y Primera Tarea Asistida}
\begin{enumerate}
    \setcounter{enumi}{\theenumTemp}
    \item A continuaci\'on, observar\'a un videotutorial relacionado a la aplicaci\'on. En el mismo, se
     explicar\'an los conceptos b\'asicos y algunos comandos disponibles para componer m\'usica utilizando 
    \foreign{TamTam Listens}. 
    Tiempo Aproximado: 6 minutos
    \item Una vez terminado el video, es momento de preguntas. Consulte cualquier punto que no le haya quedado claro al responsable de la prueba, quien le ayudar\'a a aclarar sus dudas.
    \item El facilitador le entregar\'a el manual de la aplicaci\'on. Este documento servir\'a de ayuda memoria para dar sus primeros con \foreign{TamTam Listens}. Lea el manual y consulte cualquier duda con el responsable de la prueba.
        Tiempo aproximado: 10 minutos
    \item Hora de poner a prueba lo aprendido. Se le entregar\'an las gu\'ias correspondientes a las tareas asistidas. Con la ayuda del manual, intente realizar la primera tarea. En caso de duda, consulte al facilitador. Como se trata de su primera vez usando TamTam, la tarea ser\'a muy simple.
    Tiempo Aproximado: 10 minutos
    \setcounter{enumTemp}{\theenumi}
\end{enumerate}

\subsection{Videotutorial y otras Tareas Asistidas}

\begin{enumerate}
    \setcounter{enumi}{\theenumTemp}
    \item Seguido, observar\'a otro videotutorial de \foreign{TamTam Listens}. Se introducir\'an algunos comandos adicionales para controlar la aplicaci\'on. Tiempo Aproximado: 5 minutos
    \item Momento de realizar la tarea n\'umero 2. Esta tarea le permitir\'a practicar los comandos adicionales. Utilice el manual para realizar la tarea. En caso de duda, consulte al facilitador.
    Tiempo Aproximado: 10 minutos
    \item A continuaci\'on deber\'a realizar la tarea n\'umero 3. Esta vez, deber\'a realizarla sin hacer preguntas al responsable de la prueba.
    Tiempo Aproximado: 10 minutos
    \item Al terminar la tarea 3, avise al responsable de la prueba. \'el retirar\'a la gu\'ia correspondiente. Tambi\'en deber\'a entregar el manual de la aplicaci\'on. La siguiente ser\'a la \'ultima tarea, por lo que deber\'a realizarla sin ayuda.
    \setcounter{enumTemp}{\theenumi}
\end{enumerate}

\subsection{Tarea Final sin Asistencia}

\begin{enumerate}
    \setcounter{enumi}{\theenumTemp}
    \item Ya casi termina. Para esta tarea, la n\'umero 4, no se preocupe del tiempo. Lo importante es terminar el trabajo de la mejor manera. 
    \item El resultado de la tarea 4, si lo hizo bien, ser\'a una sencilla m\'usica infantil bastante conocida. Reproduzca su obra. ¿Reconoce la melod\'ia?
    \item Al terminar, o si decide rendirse (no lo haga sin haberlo intentado mucho, por favor), avise al responsable de la prueba.
    \setcounter{enumTemp}{\theenumi}
\end{enumerate}

\subsection{Formulario y Fin de la Prueba}

\begin{enumerate}
    \setcounter{enumi}{\theenumTemp}
    \item Por \'ultimo, el facilitador le entregar\'a un formulario con unas cuantas preguntas acerca de su experiencia con \foreign{TamTam Listens}. T\'omese su tiempo, consulte de ser necesario, y por favor responda con sinceridad: su opini\'on es muy importante para la prueba.
    \item Lleg\'o al final. Muchas gracias por haber colaborado con el estudio de usabilidad! 
\end{enumerate}

\section{Tareas Asistidas}

Estas tareas tienen como objetivo ayudarle a aprender como utilizar TamTam Listens.
Intente seguir la secuencia de pasos. Si tiene dudas, consulte el manual de la aplicaci\'on.
Si contin\'ua con dificultades, puede hacer preguntas al responsable del experimento.

\subsection{Tarea 1}
\begin{enumerate}
    \item Crear una nueva m\'usica (debe obtenerse una sola partitura completamente vac\'ia)
    \item En la pista uno, seleccionar como instrumento la Guitarra El\'ectrica
    \item En la pista dos, seleccionar como instrumento el Clarinete
    \item En la pista cuatro, seleccionar como instrumento el Tri\'angulo
    \item En el comp\'as uno de la pista uno, dibujar tres notas de duraci\'on 4
    \item En el comp\'as cuatro de la pista uno, dibujar dos notas de duraci\'on 4
    \item Reproducir
    \item Salir
\end{enumerate}

\subsection{Tarea 2}
\begin{enumerate}
    \item Crear una nueva m\'usica
    \item En el comp\'as uno de la pista uno, dibujar dos notas de duraci\'on 6
    \item Cambiar el inicio de la \'ultima nota creada al tiempo 9.
    \item En el comp\'as tres de la pista cuatro, dibujar una nota de duraci\'on 4
    \item Seleccionar la primera nota del comp\'as uno
    \item Reemplazar la nota seleccionada por otra
    \item Duplicar la pista uno en la tres
    \item Duplicar la partitura
    \item Eliminar las notas de la pista tres
    \item Reproducir la pista uno
    \item Reproducir todo
    \item Exportar el trabajo realizado
    \item Salir
\end{enumerate}

A partir de esta tarea no se admiten preguntas sobre comandos al responsable del experimento.
Intente seguir la gu\'ia, usando el manual de la aplicaci\'on en caso de dudas.

\subsection{Tarea 3}

\begin{enumerate}
    \item Crear una nueva m\'usica
    \item En la pista uno, seleccionar como instrumento la Flauta
    \item En la pista dos, seleccionar como instrumento el Piano
    \item En la pista cuatro, seleccionar como instrumento la Guitarra El\'ectrica
    \item En la pista cinco, seleccionar como instrumento la Bater\'ia
    \item En el comp\'as uno de la pista uno, crear un nota de duraci\'on 4
    \item En el comp\'as cuatro de la pista uno, crear dos notas de duraci\'on 6
    \item Copiar el contenido de la pista uno en la pista dos
    \item En el comp\'as cuatro de la pista dos, seleccionar la segunda nota y disminuir su duraci\'on.
    \item Borrar la \'unica nota del comp\'as uno de la pista dos
    \item Crear una nueva partitura en blanco
    \item En el comp\'as uno de la pista cuatro, crear una nota de duraci\'on 10
    \item En el comp\'as uno de la pista cinco, crear dos notas
    \item En el comp\'as cuatro de la pista cinco, crear tres notas
    \item Reproducir todo
    \item Exportar el trabajo realizado
    \item Salir
\end{enumerate}

\section{Tarea sin Asistencia}

\subsection{Tarea 4: Haciendo M\'usica con TamTam Listens}

El siguiente ejercicio busca evaluar lo aprendido durante el entrenamiento:

Componga la siguiente m\'usica utilizando TamTam Listens.

Las notas se presentan especificando su nombre y la duraci\'on entre par\'entesis.
Cree las notas en el orden en que se presentan a continuaci\'on.
Las notas deben crearse a partir del inicio de la pista y de seguido.
Atenci\'on: Cada parte de la m\'usica corresponde a una partitura y deben reproducirse una despu\'es de la otra.

\subsubsection{Primera Parte}
Instrumentos: Guitarra El\'ectrica (Pista Uno) y Flauta (Pista Dos)
Crear las siguientes notas en ambas pistas:

$sol (4) - sol (4) - do\:agudo (4) - sol (4) - sol (4) - mi (4)$

$sol (4)  - sol (4) - la (2) - sol (2) - la (4) - si (4) - do\:agudo (4)$

\subsubsection{Segunda Parte}
En otra partitura.

Instrumentos: Piano (Pista Tres) y Clarinete (Pista 4)

Crear las siguientes notas en ambas pistas:

$sol (4) - sol (4) - do\:agudo (4) - sol (4) - sol (4) - mi (4)$

$sol (4)  - sol (4) - la (2) - sol (2) - la (4) - si (4) - do\:agudo (2) - do\:agudo (2)$

Reproduzca la obra que acaba de componer. ¿Reconoce la melod\'ia?

Una vez finalizado el ejercicio, guarde la m\'usica. 

\section{Manual del Usuario. Tamtam Listens}

\subsection{¿Qu\'e es TamTam?}

TamTam es un compendio de cuatro actividades relacionadas con la m\'usica y sonido para la XO.
TamTam Edit es una de \'estas actividades y proporciona una interfaz para crear, modificar y organizar
notas ubicadas en cinco ``pistas'' virtuales. Adem\'as incluye una paleta de casi cien tipos de sonidos y
modelos de construcci\'on musical que permite crear distintos tipos de variaciones en estilos musicales.
Las secciones principales del programa se pueden observar en la siguiente figura:


\begin{figure}[H] 
\centering
\includegraphics[width=1\textwidth]{./graphics/ui-tamtam.png}
\caption{Interfaz de TamTam Edit y sus secciones principales}
\label{figure:ui-tamtam}
\end{figure}

Como se puede apreciar la interfaz se encuentra organizada en cinco pistas y cada pista tiene asociada
un instrumento (la quinta pista esta reservada para instrumento de tipo bater\'ia). Cada pista se divide en
4 compases (del comp\'as uno al cuatro) y cada comp\'as se divide en 12 tiempos (del uno al doce). Las
notas se dibujan en los compases, como se puede ver, la longitud de la nota indica su duraci\'on y su
altura el tipo de nota.  Adem\'as la aplicaci\'on esta organizada en partituras, que son como hojas  de
cuaderno, y son \'utiles para componer m\'usicas largas.

\subsection{TamTam Listens}
Tamtam Listens es una interfaz alternativa para la aplicaci\'on TamTam Edit, en la cual se puede
componer m\'usica utilizando comandos de voz. 

\subsubsection{¿C\'omo interactuar con la interfaz?}
Para que la aplicaci\'on pueda reconocer los comandos de voz sin mayores inconvenientes, los mismos
deben ser dichos de manera clara y sin pausas largas entre palabras. Adem\'as en la Figura 1 se puede
observar una caja de texto que sirve para notificar al usuario el estado del reconocedor. El color verde
indica que se acaba de procesar el comando  de voz   y,  en caso de  \'exito, se muestra el comando
reconocido, en caso contrario, se muestra un mensaje pidiendo repetir el comando.  Adem\'as, el color
verde indica que la aplicaci\'on esta lista para reconocer otro comando. Por otro lado, el color rojo indica
que se esta procesando el comando actual y la aplicaci\'on no esta disponible para nuevos comandos.
A continuaci\'on se muestran los comandos disponibles en TamTa TamTam Listens

\subsection{Comandos Generales}

\begin{figure}[H] 
\centering
\includegraphics[width=0.5\textwidth]{./graphics/cmd-musica.png}
\caption{Comandos para reproducir, pausar, parar, exportar y crear una m\'usica}
\label{figure:cmd-crear-musica}
\end{figure}

\begin{itemize}
\item \emph{reproducir m\'usica}: permite reproducir la m\'usica creada.
\item \emph{pausar m\'usica}: permite pausar la reproducci\'on actual dejando la l{\'\i}nea de reproducci\'on en el
punto de pausa.
\item \emph{parar m\'usica}: permite parar la reproducci\'on actual y ubica la l{\'\i}nea de reproducci\'on al inicio de
la m\'usica.
\item \emph{crear nueva m\'usica}:  permite crear una nueva composici\'on, dejando como resultado una
partitura en blanco.
\item \emph{exportar m\'usica}: permite guardar la m\'usica creada en un archivo para que reproducirse en un
reproductor multimedia.
\end{itemize}

Cabe destacar que cuando se esta reproduciendo una m\'usica, la aplicaci\'on solo acepta los comandos de
voz pausar m\'usica o parar m\'usica.


\begin{figure}[H]
\begin{minipage}[b]{0.5\linewidth}
\centering
\includegraphics[width=0.6\linewidth]{./graphics/salir.png}
\caption{Comando para salir de la aplicaci\'on}
\label{figure:cmd-salir}
\end{minipage}
\quad
\begin{minipage}[b]{0.5\linewidth}
\centering
\includegraphics[width=0.6\linewidth]{./graphics/cmd-vol.png}
\caption{Comandos para aumentar/disminuir el volumen general}
\label{figure:cmd-vol}
\end{minipage}
\end{figure}

El comando de la figura~\ref{figure:cmd-salir} permite salir de \emph{TamTam Listens}, simplemente hay que decir ``salir de tamtam''. Por otro lado, los comandos de la figura~\ref{figure:cmd-vol}
y~\ref{figure:cmd-tempo} permiten controlar, respectivamente, el volumen y tempo general de la aplicaci\'on. Por ejemplo: ``aumentar volumen'', ``disminuir tempo''.

\begin{figure}[H]
\begin{minipage}[b]{0.5\linewidth}
\centering
\includegraphics[width=0.6\linewidth]{./graphics/cmd-tempo.png}
\caption{Comandos para aumentar/disminuir el tempo general de al aplicaci\'on}
\label{figure:cmd-tempo}
\end{minipage}
\end{figure}

\subsection{Comandos de Partitura}

Los comandos de las figuras~\ref{figure:cmd-partitura-1} y~\ref{figure:cmd-partitura-2} afectan a
la partitura actual, como se explican a continuaci\'on:

\begin{itemize}
    \item \emph{crear nueva  partitura}:  permite crear una nueva partitura en blanco. Utilizaci\'on, ``crear nueva partitura''
    \item \emph{limpiar  partitura}: permite limpiar el contenido de la partitura actual, es decir, borrar todas las notas. Utilizaci\'on, ``limpiar partitura''
    \item \emph{duplicar partitura}: crea una nueva partitura con el mismo contenido que la partitura actual. Utilizaci\'on, ``duplicar partitura''.
\end{itemize}

\begin{figure}[H]
\begin{minipage}[b]{0.5\linewidth}
\centering
\includegraphics[width=0.6\linewidth]{./graphics/partitura-1.png}
\caption{Comandos para crear, limpiar y duplicar la partitura actual}
\label{figure:cmd-partitura-1}
\end{minipage}
\quad
\begin{minipage}[b]{0.5\linewidth}
\centering
\includegraphics[width=0.6\linewidth]{./graphics/partitura-2.png}
\caption{Comandos para navegar entre partituras}
\label{figure:cmd-partitura-2}
\end{minipage}
\end{figure} 

Los comandos de la figura~\ref{figure:cmd-partitura-2} permiten navegar entre partituras.

\subsection{Comandos de Pista}
En la figura~\ref{figure:cmd-pista-1} puede apreciarse el comando que permite al usuario ubicarse en una pista en particular. Por  
ejemplo, para ubicarse en la pista tres debe decir “pista tres”. Adem\'as de controlar el volumen general de la aplicaci\'on, en 
la figura~\ref{figure:cmd-vol-pista} se pueden ver los comandos para controlar el volumen de una pista en particular. Para aumentar
el volumen de la pista tres, el usuario debe decir ``aumentar volumen de pista tres''.

\begin{figure}[H] 
\centering
\includegraphics[width=0.6\linewidth]{./graphics/cmd-pista-1.png}
\caption{Comando para ubicarse en una pista}
\label{figure:cmd-pista-1}
\end{figure}

\begin{figure}[H] 
\begin{minipage}[b]{0.5\linewidth}
\centering
\includegraphics[width=0.8\linewidth]{./graphics/vol-pista.png}
\caption{Comandos para aumentar/disminuir el volumen de una pista en particular}
\label{figure:cmd-vol-pista}
\end{minipage}
\quad
\begin{minipage}[b]{0.5\linewidth}
\centering
\includegraphics[width=0.9\linewidth]{./graphics/rep-pista.png}
\caption{Comandos para reproducir, silenciar, habilitar y limpiar una pista en particular}
\label{figure:cmd-rep-pista}
\end{minipage}
\end{figure}

Los comandos de la figura~\ref{figure:cmd-rep-pista} permiten: reproducir, silenciar, habilitar y limpiar el contenido de una
pista en particular. Por ejemplo, para reproducir las notas de la pista uno, el usuario debe decir ``reproducir pista uno''. 
Para poder generar distintos tipos de sonidos con \emph{TamTam Listens}, los usuarios de pueden asignar
instrumentos a cada una de las pistas de
la aplicaci\'on, \'esto se puede realizar con los comandos de la figura~\ref{figure:cmd-inst-p1-4} y~\ref{figure:cmd-inst-p5}. Para
asignar el piano a la pista dos, basta con decir ``piano en pista dos''.


\begin{figure}[H]
\centering
\includegraphics[width=0.8\textwidth]{./graphics/inst-p1-4.png}
\caption{Selecci\'on de instrumento para pistas del uno al cuatro}
\label{figure:cmd-inst-p1-4}
\end{figure}

\begin{figure}[H] 
\begin{minipage}[b]{0.5\linewidth}
\centering
\includegraphics[width=1\linewidth]{./graphics/inst-p5.png}
\caption{Selecci\'on de instrumento para la pista cinco}
\label{figure:cmd-inst-p5}
\end{minipage}
\quad
\begin{minipage}[b]{0.5\linewidth}
\centering
\includegraphics[width=1\linewidth]{./graphics/dup-pista.png}
\caption{Comando para duplicar las notas de una pista en otra}
\label{figure:cmd-dup-pista}
\end{minipage}
\end{figure}

Generalmente las composiciones tienen cierta secuencia de notas que se repiten para varios instrumentos. El comando de la 
figure~\ref{figure:cmd-dup-pista} permite duplicar el contenido, es decir las notas, de una pista en otra. Por ejemplo, 
``duplicar pista uno en pista dos'' permite duplicar las notas de la pista uno en la pista dos.

\subsection{Comandos de Comp\'as}

Estos comandos son muy importantes para la aplicaci\'on ya que permiten crear, modificar, 
eliminar las notas musicales. El comando de la figura~\ref{figure:cmd-compas} ubica al usuario dentro de una pista en particular. 
Por lo tanto, se debe seleccionar una pista para poder utilizar este comando.

\begin{figure}[H] 
\centering
\includegraphics[width=0.4\linewidth]{./graphics/cmd-compas.png}
\caption{Comando para ubicarse en un comp\'as}
\label{figure:cmd-compas}
\quad
\end{figure}

El comando de la figura~\ref{figure:cmd-crear-nota} permite crear notas en el  
comp\'as actual, por ejemplo ``crear nota do'' crea la nota do en el comp\'as previamente seleccionado. Por otro lado, en la figura~\ref{figure:cmd-tiempo-compas} se presenta el comando que permite al usuario ubicarse en un  
tiempo en particular dentro del comp\'as actual. Esto es \'util para crear una nota a partir de ese punto o 
para seleccionar una nota que se encuentre en ese tiempo.

\begin{figure}[H]
\begin{minipage}[b]{0.5\linewidth}
\centering
\includegraphics[width=1\linewidth]{./graphics/cmd-crear-nota.png}
\caption{Comando para crear una nota}
\label{figure:cmd-crear-nota}
\end{minipage}
\quad
\begin{minipage}[b]{0.5\linewidth}
\centering
\includegraphics[width=1.1\linewidth]{./graphics/cmd-tiempo-compas.png}
\caption{Comando para ubicarse en un tiempo dado, dentro de un comp\'as}
\label{figure:cmd-tiempo-compas}
\end{minipage}
\end{figure}

As\'i como puede duplicarse notas de una pista a otra, tambi\'en puede duplicarse una nota de un comp\'as a otro utilizando el comando
de la figura~\ref{figure:cmd-dup-nota}, por ejemplo ``duplicar en pista uno compas dos'' permite duplicar una nota en el segundo comp\'as de la pista 
uno. Para poder eliminar una nota, previamente seleccionada, el usuario debe utilizar el comando
de la figura~\ref{figure:cmd-del-nota}.

\begin{figure}[H]
\begin{minipage}[b]{0.5\linewidth}
\centering
\includegraphics[width=1.2\linewidth]{./graphics/cmd-dup-nota.png}
\caption{Comando para duplicar una nota previamente seleccionada}
\label{figure:cmd-dup-nota}
\end{minipage}
\quad
\begin{minipage}[b]{0.5\linewidth}
\centering
\includegraphics[width=0.5\linewidth]{./graphics/del-note.png}
\caption{Comando para eliminar un nota previamente seleccionada}
\label{figure:cmd-del-nota}
\end{minipage}
\end{figure}

En la figura~\ref{figure:cmd-dur} se pude observar el comando que permite modificar la duraci\'on de 
una nota inmediatamente despu\'es de haberla creado o una nota previamente seleccionada. Finalmente, el comando
presentado en la figura~\ref{figure:cmd-note-tiempo} permite modificar el tiempo en el que inicia la 
nota inmediatamente despu\'es de haberla creado o una nota previamente seleccionada.

\begin{figure}[H]
\begin{minipage}[b]{0.5\linewidth}
\centering
\includegraphics[width=0.9\linewidth]{./graphics/cmd-dur.png}
\caption{Comando que permite configurar la duraci\'on de una nota}
\label{figure:cmd-dur}
\end{minipage}
\quad
\begin{minipage}[b]{0.5\linewidth}
\centering
\includegraphics[width=1.1\linewidth]{./graphics/cmd-note-tiempo.png}
\caption{Comando que permite configurar el inicio de una nota dentro del comp\'as}
\label{figure:cmd-note-tiempo}
\end{minipage}
\end{figure}

\subsection{Diagramas de Interacci\'on}

En esta secci\'on se incluyen algunos diagramas que muestran los pasos necesarios para realizar cierto 
tipo de acciones dentro de la aplicaci\'on.

\subsubsection{Pasos para crear una nota}

En el siguiente diagrama se puede observar los pasos necesarios para crear una nota

\begin{figure}[H]
\includegraphics[width=0.9\linewidth]{./graphics/pasos-crear-nota2.png}
\end{figure}

\subsubsection{Pasos para modificar una nota reci\'en creada}
Una nota reci\'en creada puede autom\'aticamente ser modificada alterando su duraci\'on y/o 
tiempo de inicio en el comp\'as, como se puede ver en el siguiente diagrama.

\begin{figure}[H]
\includegraphics[width=0.9\linewidth]{./graphics/pasos-modificar-nota-recien-creada.png}
\end{figure}

\subsubsection{Pasos para modificar una nota}
Para poder modificar una nota en particular (no necesariamente una que acabamos de crear) debemos 
primero seleccionar la nota, para luego seguir de manera similar al diagrama anterior.

\begin{figure}[H]
\includegraphics[width=0.9\linewidth]{./graphics/pasos-modificar-nota-seleccionar.png}
\end{figure}

\subsection{Tutoriales de Uso}
En esta secci\'on se incluyen explicaciones breves de algunos ejemplos de uso de la aplicaci\'on, de modo a facilitar el proceso de aprendizaje.

\subsubsection{Componiendo una escala simple}

\begin{enumerate}
\item Para empezar, debemos obtener una partitura en blanco. Lo conseguimos pronunciando el comando: ``Crear Nueva M\'usica''.
\item Seleccionamos los instrumentos que queremos utilizar, diciendo:
\begin{itemize}
    \item ``Piano en Pista Uno''
    \item ``Guitarra El\'ectrica en Pista Dos''
    \item ``Teclado en Pista Tres''
    \item ``Flauta en Pista Cuatro''
\end{itemize}
\item Antes de crear las notas, debemos ubicarnos en el punto donde queremos empezar:
\begin{itemize}
\item ``Pista Uno'' : el comando nos ubica en el tiempo uno, del comp\'as uno de la pista uno.
\end{itemize}
    Si quisi\'esemos empezar en el tiempo uno del comp\'as dos, bastar\'ia con decir:
\begin{itemize}
\item ``Comp\'as Dos''
\end{itemize}
    En caso de querer empezar en el tiempo siete, decimos:
\begin{itemize}
\item ``Tiempo Siete''
\end{itemize}
\item Ya seleccionado el punto inicial, estamos listos para crear las notas :
\begin{itemize}
    \item ``Crear Nota Do''
    \item ``Crear Nota Re''
    \item ``Crear Nota Mi''
    \item ``Crear Nota Fa''
    \item ``Crear Nota Sol''
    \item ``Crear Nota La''
    \item ``Crear Nota Si''
\end{itemize}
\item Como no queremos trabajar de m\'as, duplicamos las pistas para escuchar los dem\'as instrumentos.
\begin{itemize}
    \item ``Duplicar Pista Uno en Pista Dos''
    \item  ``Duplicar Pista Dos en Pista Tres''
    \item ``Duplicar Pista Tres en Pista Cuatro''
\end{itemize}
\item As\'i de f\'acil. Para escuchar nuestra m\'usica: ``Reproducir M\'usica''.
\end{enumerate}

\subsubsection{Modificando una Nota}

Si queremos cambiar una nota luego de su creaci\'on, estos son los pasos a seguir:

\begin{enumerate}
\item Primero, debemos seleccionar la nota. Si la acabamos de crear, la nota se 
    encuentra seleccionada, por lo que este paso puede omitirse.
    En caso contrario, basta con seleccionar alg\'un punto ocupado por la nota en la partitura.
    Por ejemplo, para seleccionar una nota en el comienzo del segundo comp\'as de la pista tres,  dir\'iamos:
    \begin{itemize}
        \item ``Pista Tres''
        \item ``Comp\'as Dos''
        \item ``Tiempo Uno''
    \end{itemize}
\item Una vez seleccionada la nota, podemos cambiar su duraci\'on (por defecto 4 tiempos) a 9 tiempos, podemos decir: 
    ``Duraci\'on Nueve''.
\item Si queremos desplazar el inicio de la nota al quinto tiempo, decimos ``Inicio en Tiempo Cinco''.
\item Supongamos que la nota existente es un Fa y queremos cambiarla por un Sol. Para reemplazar la 
    nota por otra, basta con crear la nueva nota en su lugar diciendo ``Crear Nota Sol''.
\item Si nada nos convence, y preferimos eliminar la nota seleccionada, decimos ``Eliminar Nota''.
\end{enumerate}

% estos comandos generan la bilbiograf'ia
\printbibliography

\end{document}
