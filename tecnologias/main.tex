%!TEX root = ../tesis.tex
 \chapter{Tecnolog\'ias y Herramientas}
\label{sec:tecnologias}

En la actualidad existe un gran n\'umero de herramientas de software que pueden utilizarse para la implementaci\'on de un
sistema de reconocimiento del habla. Si bien esta diversidad representa una ventaja, esto requiere
la selecci\'on de la tecnolog{\'\i}a adecuada. Es selecci\'on conforma la primera tarea a llevar a
cabo por las personas que deseen trabajar en esta \'area.

Este cap{\'\i}tulo presenta, categoriza y eval\'ua varias alternativas tecnolog{\'\i}cas disponibles, relacionadas
al reconocimiento del habla, con el fin de facilitar la elecci\'on de la herramienta adecuada
de acuerdo a las necesidades de cada caso. La siguiente clasificaci\'on y criterios que ser\'an presentados son
una propuesta que forma parte de este trabajo de grado.

Inicialmente, se clasifican las herramientas en las siguientes categor{\'\i}as:
\begin{itemize}
	\item Aplicaciones: herramientas software que permiten al usuario realizar una o
	varias tareas \mbox{espec{\'\i}ficas \cite{GoodwillComputer}}.
    \item \gls{api}: interfaces entre una base de c\'odigo y
	el programador \cite{DoucetteOnApi}. Esta abstracci\'on puede verse como una caja negra
	(los detalles de la implementaci\'on se mantienen ocultos) que provee una determinada funcionalidad.
	\item Librer{\'\i}as/Frameworks: con junto de m\'etodos o funciones a los que se puede invocar.
	En el caso de un framework, incluye tambi\'en patrones de dise\~no
	\mbox{predefinidos \cite{FowlerInversion}}.
\end{itemize}

Las categor{\'\i}as citadas anteriormente representan, diferentes enfoques para realizar el trabajo
de reconocimiento del habla. A modo de discernir entre los mismos, se establecen
los siguientes criterios generales de evaluaci\'on:
\begin{itemize}
	\item Conocimiento t\'ecnico necesario: nivel de conocimiento relativo al \'area necesario para la
	utilizaci\'on de la herramienta. 
	\item Productividad: raz\'on entre las funcionalidades que pueden implementarse,o tareas que pueden
	realizarse, utilizando la herramienta y recursos que toma la implementaci\'on.
	\item Flexibilidad: facilidad de adaptaci\'on de la herramienta para la soluci\'on de diferentes
	tipos de problemas.
\end{itemize}

La siguiente figura presenta de forma esquematizada el contenido del cap{\'\i}tulo, y puede tomarse como
gu\'ia para la lectura de las secciones del mismo.

\begin{figure}[H]
\centering
\includegraphics[width=1\linewidth]{./graphics/esquema-herramientas.png}
\caption{Esquema que presenta el contenido del cap\'itulo}
\label{figure:esquema-herramientas}
\end{figure}

\section{Aplicaciones}
\label{sec:aplicaciones}

% introduccion

Las soluciones de reconocimiento del habla, catalogadas como aplicaciones, propocionan
a los usuarios finales los medios necesarios para realizar determinadas tareas mediante
la voz: dictado autom\'atico, navegaci\'on de interfaces, etc\'etera; sin
la necesidad de tener un conocimiento previo de los conceptos relacionados al
reconocimiento del habla. \'Estas aplicaciones presentan las siguientes 
caracter\'isticas en cuanto a los criterios generales de evaluaci\'on:

\begin{itemize}
    \item Conocimiento t\'ecnico: requieren de poco conocimiento t\'ecnico debido a que \'estas
        soluciones se ofrecen como un producto a los usuarios finales, por lo tanto, sus funcionalidades
        pueden utilizarse directamente sin necesidad de entender los componentes que intervienen
         en el sistema.
    \item Productividad: teniendo en cuenta el criterio anterior, cabe resaltar que con
        las aplicaciones se puede lograr un buen grado de productividad, porque los esfuerzos
        de los usuarios se enfocan en realizar determinadas tareas utilizando las funcionalidades
         disponibles.
    \item Flexibilidad: la flexibilidad para esta categor\'ia de soluciones es reducida, porque las
        funcionalidades que ofrece cada aplicaci\'on se encuentran orientadas a resolver tareas
        espec\'ificas: dictado autom\'atico, para la transcripci\'on de documentos; reconocimiento
        de comandos, para navegar interfaces; reconocimiento del habla, para 
        automatizaci\'on de servicios de operador. Cada aplicaci\'on tiene un \'ambito de aplicabilidad
        lo cual impone un l\'imite en su flexibilidad, en comparaci\'on a otras categor\'ias que
         se ver\'an m\'as adelante.
\end{itemize}

\subsection{Criterios Espec\'ificos de Evaluaci\'on}

Los criterios espec\'ificos nos permiten evaluar a las aplicaciones en cuanto a factores
particulares y relevantes para esta categor\'ia. A continuaci\'on se presentan los criterios 
espec\'ificos de evaluaci\'on para las aplicaciones:

\begin{itemize}
    \item Precio
    \item Soporte para m\'ultiples idiomas
\end{itemize}

\subsection{Ejemplos de Aplicaciones}

\subsection{Simon}
\label{sec:simon}

\subsubsection{Palaver}
\label{sec:palaver}

\subsection{Nuance}
\label{sec:nuance}


%!TEX root = ../../tesis.tex
\section{Interfaces de Programaci\'on de Aplicaciones}
\label{apis}

% introduccion

Una interfaz de programaci\'on de aplicaciones (tambi\'en conocida como \gls{api} por sus siglas 
en ingl\'es) proporciona a los desarrolladores los medios necesarios para integrar funcionalidades de
reconocimiento del habla al \foreign{software} que implementan, sin necesidad de poseer
conocimiento en detalle del \'area. De acuerdo a los criterios de evaluaci\'on generales, 
pueden mencionarse las siguientes caracter{\'\i}sticas:

\begin{itemize}
 	\item Conocimiento t\'ecnico necesario: aunque no requieren conocimiento sobre detalles de implementaci\'on
 	del \'area de reconocimiento del habla, si resultan necesarios conocimientos b\'asicos de
 	programaci\'on, por lo cual resulta inadecuada su utilizaci\'on por parte de usuarios finales.
 	\item Productividad: la utilizaci\'on de una interfaz de programaci\'on de aplicaciones permite
 	al desarrollador abstraerse de la complejidad subyacente del problema, lo cual resulta en
 	un alto grado de productividad. Sin embargo, el ocultamiento de los detalles del proceso
 	puede ser un factor negativo en algunos \'ambitos, como el \mbox{acad\'emico}.
 	\item Flexibilidad: estas herramientas presentan un buen grado de flexibilidad, debido a
 	que no est\'an orientadas a una tarea en particular. La selecci\'on del problema espec{\'\i}fico
 	que se soluciona utilizando reconocimiento del habla es responsabilidad del desarrollador
 	que utiliza la interfaz. Cabe destacar, sin embargo, que los componentes propios de la
 	implementaci\'on de la interfaz de programaci\'on, al estar ocultos, no
 	pueden modificarse.
 \end{itemize}


\subsection{Criterios Espec\'ificos de Evaluaci\'on}

Los criterios espec{\'\i}ficos seg\'un los cuales se evaluar\'an las opciones disponibles en esta
categor{\'\i}a son:

\begin{itemize}
	\item Empresa o Instituci\'on Responsable
	\item Precio o Inversi\'on econ\'omica
	\item Soporte para m\'ultiples idiomas
	\item Soporte Offline
	\item Dependencia de Plataforma
\end{itemize}


\subsection{Ejemplos de Interfaces de Programaci\'on de Aplicaciones}

%!TEX root = ../../tesis.tex
\subsubsection{Web Speech \gls{api}}
\label{sec:webspeech}

La \foreign{Web Speech \gls{api}} define un est\'andar que especifica una interfaz de programaci\'on de aplicaciones para permitir
a los desarrolladores web incorporar s{\'\i}ntesis y reconocimiento del habla a sus sitios web.

\begin{itemize}
	\item \emph{Empresa o Instituci\'on Responsable:} La especificaci\'on de la interfaz fue
	publicada por el \foreign{Speech \gls{api} Community Group} del
	\mbox{\foreign{World Wide Web Consortium} \cite{GoogleWebSpeechAPI}.}
	La \'unica implementaci\'on disponible fue desarrollada por \foreign{Google} para su navegador \foreign{Chrome}.
	\item \emph{Precio:} la utilizaci\'on de esta herramienta a trav\'es de \foreign{Google Chrome} es gratuita e
	ilimitada en la actualidad. Aunque esto podr\'ia llegar a cambiar ya que se trata del producto de una empresa
    en particular.
	\item \emph{Soporte para m\'ultiples idiomas:} la interfaz ofrece soporte para m\'as de 30 idiomas.
	\item \emph{Soporte Offline:} la especificaci\'on publicada de la interfaz no limita ni reglamenta su
	implementaci\'on; sin embargo, la \'unica opci\'on funcional actualmente delega el procesamiento a los
	servidores de \foreign{Google}, por lo cual requiere de una conexi\'on a internet para su uso.
	\item \emph{Dependencia de Plataforma:} es necesario un navegador web para su utilización.
	Aunque actualmente solo puede utilizarse a trav\'es de \foreign{Chrome}, implementar la interfaz
	para otros navegadores es posible. No existe dependencia con el sistema operativo.
\end{itemize}


%!TEX root = ../../tesis.tex
\subsubsection{AT\&T Speech API}
\label{sec:att}

La \foreign{AT\&T Speech API} \cite{AttSpeech} define una interfaz de programaci\'on de aplicaciones para permitir
a los desarrolladores incorporar s{\'\i}ntesis y reconocimiento del habla en aplicaciones m\'oviles,
web o de escritorio. El procesamiento se lleva a cabo a trav\'es del motor \foreign{AT\&T Watson}.

\begin{itemize}
	\item \emph{Empresa o Instituci\'on Responsable:} \foreign{AT\&T Inc}.
	\item \emph{Precio:} un mill\'on de llamadas a la interfaz por 99\$ anuales, llamadas adicionales a un centavo
	cada una.
	\item \emph{Soporte para m\'ultiples idiomas:} el ingl\'es es el principal idioma soportado, aunque incluye
	soporte reducido para el espa\~nol.
	\item \emph{Soporte Offline:} al delegarse el procesamiento a los servidores de \foreign{AT\&T},
	una conexi\'on a internet es indispensable para su uso.
	\item \emph{Dependencia de Plataforma:} al ofrecer una interfaz REST, no existe dependencia con navegador
	ni con sistema operativo alguno. Adem\'as, se ofrecen \foreign{software development kits} o SDKs para facilitar
	el desarrollo en plataformas como \foreign{Windows}, \foreign{iOS} y \foreign{Android}.
\end{itemize}

\subsection{Microsoft Speech API}
\label{sec:microsoft}

%!TEX root = ../../tesis.tex
\subsubsection{WAMI}
\label{sec:wami}

WAMI es un \foreign{toolkit} desarrollado en el Instituto Tecnol\'ogico de
\mbox{Massachusetts \cite{WamiHome}}. T\'ecnicamente, es m\'as que una interfaz de programaci\'on
de aplicaciones; ofrece un conjunto de herramientas, para el lado servidor y el lado cliente,
que permiten al desarrollador ofrecer su propia implementaci\'on de la interfaz.

WAMI est\'a orientado a ofrecer una interfaz web para un determinado motor de reconocimiento del habla. 
A\'un as{\'\i}, el MIT ofrece un reconocedor del habla que implementa la interfaz para fines 
de prototipado como parte del proyecto.

\begin{itemize}
	\item \emph{Empresa o Instituci\'on Responsable:} Instituto Tecnol\'ogico de Massachusetts.
	\item \emph{Precio:} esta herramienta es gratuita y de c\'odigo abierto.
	\item \emph{Soporte para m\'ultiples idiomas:} la implementaci\'on para prototipado ofrecida
	por el MIT soporta los idiomas ingl\'es, chino y japon\'es. El soporte para idiomas de WAMI
	depende exclusivamente del reconocedor c on cual se integre.
	\item \emph{Soporte Offline:} WAMI se basa en la arquitectura cliente-servidor, es decir, el reconocimiento del habla 
    se realiza en el servidor y es enviado de vuelta al cliente. Por lo tanto, es necesaria una conexi\'on entre ambos.
	\item \emph{Dependencia de Plataforma:} al ofrecer una interfaz REST, no existe dependencia con
	navegador ni con sistema operativo alguno.
\end{itemize}

%!TEX root = ../../tesis.tex
\subsubsection{iSpeech \gls{api}}
\label{sec:ispeech}

La \foreign{iSpeech \gls{api}} \cite{iSpeech} define una interfaz de programaci\'on de aplicaciones para 
permitir a los desarrolladores incorporar s{\'\i}ntesis y reconocimiento del habla en aplicaciones 
m\'oviles, web o de escritorio.

\begin{itemize}
	\item \emph{Empresa o Instituci\'on Responsable:} \foreign{iSpeech}, una empresa estadounidense dedicada a
	proveer reconocimiento y s{\'\i}ntesis del habla como servicio. Su producto m\'as reconocido es la aplicaci\'on m\'ovil
	\foreign{Drivesafe.ly}, orientada a reducir las distracciones de un conductor.
	\item \emph{Precio:} Gratis para aplicaciones m\'oviles sin fines comerciales. De 0,02\$ a 0,005\$ por
        transacci\'on, dependiendo del tama\~no de la compra\footnote{\emph{iSpeech} var\'ia el precio por transacci\'on
    en base al tama\~no de la compra, es decir, la cantidad m\'axima de transacciones que ofrece el plan.}, para otros usos.
	\item \emph{Soporte para m\'ultiples idiomas:} soporta reconocimiento continuo (sin pausas) del habla para 6 idiomas,
	entre ellos el espa\~nol, y reconocimiento de vocabulario reducido para m\'as de 10 idiomas.
	\item \emph{Soporte Offline:} el procesamiento se realiza en los servidores de \foreign{iSpeech}, por lo cual
	una conexi\'on a internet es indispensable para su uso.
	\item \emph{Dependencia de Plataforma:} al ofrecer una interfaz \gls{rest}, no existe dependencia con navegador
	ni con sistema operativo alguno. Adem\'as, se ofrecen varios \gls{sdk} para facilitar
	el desarrollo en m\'ultiples plataformas y lenguajes de programaci\'on.
\end{itemize}
%!TEX root = ../../tesis.tex
\subsubsection{Dragon Mobile}
\label{sec:dragonmobile}

\foreign{Dragon Mobile} ofrece una interfaz de programaci\'on de aplicaciones para permitir
a los desarrolladores incorporar s{\'\i}ntesis y reconocimiento del habla a sus
\mbox{aplicaciones \cite{DragonMobile}}. Est\'a espec{\'\i}ficamente orientado al desarrollo de
aplicaciones para plataformas m\'oviles.

\begin{itemize}
	\item \emph{Empresa o Instituci\'on Responsable:} \foreign{Nuance Communications}.
	\item \emph{Precio:} el uso de esta herramienta es gratuito para aplicaciones sin fines comerciales.
	Desde 0.08\$ a 0.006\$ por transacci\'on, dependiendo del tama\~no de la compra, para otros usos.
	Para comercializar una aplicaci\'on que utiliza esta herramienta, es necesaria una inversi\'on inicial
	m{\'\i}nima de 3000\$.
	\item \emph{Soporte para m\'ultiples idiomas:} se ofrece soporte para m\'as de 20 idiomas, entre ellos
	el espa\~nol.
	\item \emph{Soporte Offline:} al delegarse el procesamiento a los servidores de \foreign{Nuance},
	una conexi\'on a internet es indispensable para su uso.
	\item \emph{Dependencia de Plataforma:} al ofrecer una interfaz \gls{rest}, no existe dependencia con navegador
	ni con sistema operativo alguno. Adem\'as, se ofrecen varios \gls{sdk} para facilitar
	el desarrollo en plataformas m\'oviles como \foreign{Windows Phone}, \foreign{iOS} y \foreign{Android}.
\end{itemize}

\section{Librer\'ias y Frameworks}
\label{sec:librerias}

Una librer{\'\i}a va m\'as all\'a de una especificaci\'on de interacci\'on y el c\'odigo fuente que permita cumplirla.
Una librer{\'\i}a incluye un conjunto de funcionalidades o m\'etodos cuyo modo de utilizaci\'on queda a criterio
del desarrollador. En el caso de un \foreign{framework}, se incluyen tambi\'en ciertos patrones de dise\~no para la aplicaci\'on.

Aunque la distinci\'on entre una librer{\'\i}a y una interfaz de programaci\'on de aplicaciones puede resultar
sutil, para este trabajo, se consideran el mayor grado de control y el menor nivel de abstracci\'on
como criterios para identificar una herramienta como una librer{\'\i}a.

As{\'\i}, una interfaz de programaci\'on de aplicaciones es una especificaci\'on de interacci\'on con un componente
reconocedor del habla completamente implementado, mientras que una librer{\'\i}a puede verse como un conjunto
de bloques que permiten al desarrollador construir su propio reconocedor del habla a medida.

De acuerdo a los criterios de evaluaci\'on generales, pueden mencionarse las 
siguientes caracter{\'\i}sticas :

\begin{itemize}
 	\item Conocimiento t\'ecnico necesario: la utilizaci\'on de librer{\'\i}as requiere conocimiento t\'ecnico
 	espec{\'\i}fico del \'area del reconocimiento del habla por parte del programador, debido a que el nivel de
 	abstracci\'on es menor al de las herramientas previamente descriptas.
 	\item Productividad: Debido al alto grado de control que ofrecen, el desarrollo de los distintos
 	elementos de un sistema de reconocimiento del habla requiere de m\'as trabajo de an\'alisis e implementaci\'on.
 	Esto resulta en una productividad baja en comparaci\'on con otras alternativas, que permiten al
 	programador abstraerse de los detalles del proceso de reconocimiento del habla.
 	A\'un as{\'\i} en ciertos \'ambitos, como el acad\'emico, la posibilidad de conocer y manipular los distintos
 	componentes del mencionado proceso representa una importante ventaja que ofrecen estas herramientas.
 	\item Flexibilidad: las librer{\'\i}as ofrecen un alto grado de flexibilidad, siendo la alternativa
 	m\'as adaptable de entre las categor{\'\i}as analizadas. Componentes importantes de un sistema
 	de reconocimiento del habla como el modelo de lenguaje, el modelo ac\'ustico y los algoritmos involucrados
 	pueden seleccionarse, modificarse o incluso reimplementarse de acuerdo al criterio del desarrollador.
 \end{itemize}

\subsection{Criterios Espec\'ificos de Evaluaci\'on}

Los criterios espec{\'\i}ficos seg\'un los cuales se evaluar\'an las opciones disponibles en esta
categor{\'\i}a son:

\begin{itemize}
	\item Empresa o Instituci\'on Responsable
	\item Precio
	\item Licencia
	\item Soporte para m\'ultiples idiomas
	\item Dependencia de Plataforma
	\item Modelos de Lenguaje Aceptados
	\item Modelos Ac\'usticos Aceptados
	\item Uso de Memoria
\end{itemize}


\subsection{Ejemplos de librer{\'\i}as o frameworks}

\subsection{Sphinx 4}
\label{sec:sphinx}

%!TEX root = ../../tesis.tex
\subsubsection{PocketSphinx}
\label{sec:pocketsphinx}

La librer{\'\i}a Pocketsphinx es desarrollada en paralelo a Sphinx 4 en la Universidad
\mbox{\foreign{Carnegie Mellon} \cite{PocketSphinxHomePage}}. Esta herramienta de c\'odigo abierto
est\'a implementada en el lenguaje C, logr\'andose con esto en un mayor rendimiento y facilidad
para desarrollar \foreign{bindings}\footnote{En programaci\'on, un \emph{Binding} es una adaptaci\'on de una librer\'ia para ser usada en un 
lenguaje de programaci\'on distinto de aquel en el que ha sido escrita.} 
para otros lenguajes de programaci\'on.

Pocketsphinx est\'a orientado a la optimizaci\'on del rendimiento y a la portabilidad a distintas
plataformas, resultando particularmente adecuada para la implementaci\'on de sistemas empotrados
y sistemas en tiempo \mbox{real \cite{HugginsDainesPocketSphinx2006}}.

\begin{itemize}
	\item \emph{Empresa o Instituci\'on Responsable:} Universidad \foreign{Carnegie Mellon}.
	\item \emph{Precio:} esta librer\'ia puede utilizarse de forma gratuita.
	\item \emph{Licencia:} Pocketsphinx  se distribuye bajo una versi\'on simplificada de la licencia
	\gls{bsd}, considerada de c\'odigo abierto.
	\item \emph{Soporte para m\'ultiples idiomas:} Pocketsphinx utiliza los mismos modelos
	que \mbox{Sphinx 4}, por lo que el soporte para idiomas es similar. Se ofrece soporte para 8
	idiomas, aunque puede extenderse a otros mediante las herramientas prove{\'\i}das.
	\item \emph{Dependencia de Plataforma :} esta librer\'ia puede utilizarse en sistemas operativos
	basados en Unix y en Windows. Adem\'as, cabe destacar la documentaci\'on existente para su instalaci\'on
	en plataformas m\'oviles como Android, iOS y el Kindle de Amazon.
	\item \emph{Modelos de Lenguaje Aceptados:} se utilizan los mismos modelos de lenguaje que \mbox{Sphinx 4},
	basados en gram\'aticas en formato JGSF y modelos estad{\'\i}sticos en formato \gls{arpa}.
	\item \emph{Modelos Ac\'usticos Aceptados:} tambi\'en los modelos ac\'usticos se comparten con \mbox{Sphinx 4}.
	Existen modelos para 8 idiomas ya construidos, y pueden construirse modelos adicionales.
	\item \emph{Uso de Memoria:} el consumo de memoria de Pocketsphinx es muy reducido en comparaci\'on con
        Sphinx 4\cite{SphinxVersions}.
\end{itemize}
%!TEX root = ../../tesis.tex
\subsubsection{HTK}
\label{sec:htk}

\gls{htk}, es un conjunto de herramientas
desarrolladas por el Departamento de Ingenier{\'\i}a de la Universidad de Cambridge. Est\'a constituido
por varias librer{\'\i}as y herramientas ejecutables implementadas en el \mbox{lenguaje C \cite{HTKHomePage}}.

Esta herramienta ofrece un alto grado de personalizaci\'on, por lo que se requiere un nivel considerable
de conocimiento t\'ecnico para implementar un sistema de reconocimiento del habla con la misma.
Cabe destacar que la \'ultima versi\'on estable de \gls{htk} fue lanzada en el a\~no 2009.

\begin{itemize}
	\item \emph{Empresa o Instituci\'on Responsable:} Universidad de Cambridge.
	\item \emph{Precio:} esta herramienta puede utilizarse de forma gratuita.
	\item \emph{Licencia:} aunque el c\'odigo fuente de \gls{htk} puede obtenerse y modificarse,
	la redistribuci\'on del mismo est\'a prohibida. Los modelos generados con \gls{htk} pueden
	redistribuirse libremente.
	\item \emph{Soporte para m\'ultiples idiomas:} \gls{htk} no provee soporte para ning\'un
	idioma directamente, pero proporciona herramientas para la definici\'on de los modelos necesarios.
	\item \emph{Dependencia de Plataforma:} esta herramienta puede utilizarse en sistemas operativos
	basados en Unix y en Windows.
	\item \emph{Modelos de Lenguaje Aceptados:} se utilizan los mismos modelos de lenguaje que \mbox{Sphinx 4},
	basados en gram\'aticas y modelos estad{\'\i}sticos en formato \gls{arpa}.
	\item \emph{Modelos Ac\'usticos Aceptados:} \gls{htk} define su propio formato de modelos ac\'usticos.
	\item \emph{Uso de Memoria:} la documentaci\'on de \gls{htk} hace \'enfasis en que el consumo de memoria depende en
	gran medida de la aplicaci\'on, mencionando 150 MB para la construcci\'on de modelos para un sistema de
	dictado como un ejemplo de niveles altos de utilizaci\'on de este recurso.
	Esta cantidad de memoria no resulta muy elevada para el \foreign{hardware} de una m\'aquina promedio actual.
\end{itemize}
%!TEX root = ../../tesis.tex
\subsubsection{Julius}
\label{sec:julius}

Julius es un motor de reconocimiento del habla altamente configurable desarrollado por un grupo
de universidades e institutos de investigaci\'on en Jap\'on. Est\'a orientando al reconocimiento
del habla continua para vocabularios grandes, como ocurre en el dictado.
Julius est\'a implementado en el \mbox{lenguaje C \cite{JuliusHomePage}}.

\begin{itemize}
	\item \emph{Empresa o Instituci\'on Responsable:} \foreign{Interactive Speech Technology Consortium},
	una dependencia del \foreign{Advanced Scientific Technology \& Management Research Institute of KYOTO}.
	\item \emph{Precio:} esta herramienta puede utilizarse de forma gratuita.
	\item \emph{Licencia:} se distribuye bajo una versi\'on revisada de la licencia
	\gls{bsd}, considerada de c\'odigo abierto.
	\item \emph{Soporte para m\'ultiples idiomas:} existen modelos disponibles para
	el japon\'es y el ingl\'es. Los formatos de los modelos siguen los est\'andares de otras
	herramientas, de modo a extender el soporte a otros idiomas. De esta manera,
	se han realizado investigaciones utilizando Julius para el esloveno, franc\'es, tailand\'es
	y otros idiomas.
	\item \emph{Dependencia de Plataforma:} esta herramienta puede utilizarse en sistemas operativos
	basados en \foreign{Unix} y en \foreign{Windows}.
	\item \emph{Modelos de Lenguaje Aceptados:} se aceptan modelos de lenguaje de palabras aisladas,
	basados en gram\'aticas y modelos estad{\'\i}sticos en formato \gls{arpa}.
	\item \emph{Modelos Ac\'usticos Aceptados:} utiliza el mismo formato de modelo ac\'ustico que \gls{htk}.
	\item \emph{Uso de Memoria:} el consumo de memoria de Julius es bajo, cit\'andose como ejemplo
	un consumo menor a 64 MB para reconocimiento sobre un vocabulario de 20 mil palabras utilizando
	un modelo de lenguaje basado en trigramas almacenado en memoria.
\end{itemize}
%!TEX root = ../../tesis.tex
\subsubsection{Kaldi}
\label{sec:kaldi}

Kaldi es un \foreign{toolkit} de reconocimiento del habla que comenz\'o a desarrollarse durante un taller
en la Universidad \foreign{John Hopkins} en el a\~no 2009. Su desarrollo continu\'o con el apoyo del gobierno
de Rep\'ublica Checa en la \mbox{Universidad de Brno \cite{Povey_ASRU2011}}.

Kaldi est\'a constituida por un conjunto de librer{\'\i}as C++ y herramientas ejecutables. Por su orientaci\'on y
alcance es comparable al proyecto \gls{htk}, resultando tambi\'en necesario un alto nivel de conocimiento t\'ecnico
para su utilizaci\'on.

Cabe destacar su integraci\'on con \gls{fst} para la
construcci\'on de modelos de lenguaje, la cual lo distingue de las dem\'as herramientas analizadas en esta secci\'on.

\begin{itemize}
	\item \emph{Empresa o Instituci\'on Responsable:} Agencia Tecnol\'ogica de Rep\'ublica Checa.
	\item \emph{Precio:} esta herramienta puede utilizarse de forma gratuita.
	\item \emph{Licencia:} Kaldi se distribuye bajo la licencia Apache 2.0, considerada de c\'odigo 
	abierto.
	\item \emph{Soporte para m\'ultiples idiomas:} Kaldi no provee soporte para ning\'un
	idioma directamente, pero proporciona herramientas para la definici\'on de los modelos 
	\mbox{necesarios.}
	\item \emph{Dependencia de Plataforma:} esta herramienta puede utilizarse en sistemas operativos
	basados en \foreign{Unix} y en \foreign{Windows}.
	\item \emph{Modelos de Lenguaje Aceptados:} se proveen herramientas para convertir modelos en formato
	\gls{arpa} a \gls{fst}s.
	\item \emph{Modelos Ac\'usticos Aceptados:} Kaldi define su propio formato de modelos ac\'usticos.
	\item \emph{Uso de Memoria:} no se encontraron datos espec{\'\i}ficos respecto al consumo de memoria
	de Kaldi.
\end{itemize}


\section{Resumen}

Para resumir lo expuesto en este cap\'itulo, \'esta secci\'on presentar\'a tablas que resuman los criterios generales y espec\'ificos
para cada categor\'ia. A continuaci\'on en la tabla~\ref{sec:resumen-herramientas} se pueden observar las distintas
herramientas presentadas.

\begin{table}[H]
\centering
\footnotesize
\begin{tabular}{|p{3.5cm}|>{\centering}p{3.5cm}|>{\centering}p{3.5cm}|>{\centering}p{3.5cm}|}
\hline
                               & Aplicaciones             &  \gls{api}s                            & Librer\'ias/Frameworks \tabularnewline
\hline
Conocimiento T\'ecnico         &     Bajo                    & Medio                            & Alto    \tabularnewline
Productividad                  &     Alto                    & Medio                            & Bajo    \tabularnewline
Flexibilidad                   &     Bajo                    & Medio                            & Alto    \tabularnewline
Alternativas Propietarias      & \begin{itemize} \item Dragon Natural Speaking \end{itemize}  & \begin{itemize} \item Web Speech \gls{api} \item \gls{att} Speech \gls{api} \item Microsoft Speech \gls{api} \item iSpeech \gls{api} \item Dragon Mobile \end{itemize}  &  \tabularnewline
Alternativas de Código abierto & \begin{itemize} \item Simon \item Palaver \end{itemize}          &                                  & \begin{itemize} \item Sphinx 4 \item PocketSphinx \item \gls{htk} \item Julius \item Kaldi \end{itemize} \tabularnewline
\hline
\end{tabular}
\caption{Resumen general de las herramientas}
\label{sec:resumen-herramientas}
\end{table}

\subsection{Aplicaciones}

A continuaci\'on se puede observar una comparaci\'on entre las aplicaciones presentadas respecto a los criterios espec\'ificos
para \'esta categor\'ia


\begin{table}[H]
\centering
\footnotesize
\begin{tabular}{|p{3.5cm}|p{3.5cm}|p{3.5cm}|p{3.5cm}|}
\hline
                                      &  Simon                                                       &  Palaver                                       & Dragon Natural Speaking \\
\hline
Precio                                & Gratuito, C\'odigo Abierto                                   & Gratuito, C\'odigo Abierto                     & Desde 99\$  \\
Soporte para m\'ultiples idiomas      & Si soporta                                                   & Si soporta                                     & Ingl\'es, Alem\'an, Franc\'es, Espa\~nol, Italiano y Holand\'es \\
Facilidad de configuraci\'on          & F\'acilmente configurable, posee interfaz de configuraci\'on & Reducida, no posee interfaz de configuraci\'on & F\'acilmente configurable, posee interfaz de configuraci\'on \\
Soporte para dispositivos m\'oviles   & Si soporta                                                   & No soporta                                     & Existen otros productos derivados para m\'oviles \\
\hline
\end{tabular}
\caption{Resumen de los criterios espec\'ificos de las aplicaciones}
\label{sec:resumen-aplicaciones}
\end{table}

\subsection{Interfaz de Programaci\'on de Aplicaciones}

En las tablas~\ref{sec:resumen-apis} y~\ref{sec:resumen-apis-2} se puede observar una comparaci\'on entre las \gls{api}s presentadas, respecto a los criterios espec\'ificos
para \'esta categor\'ia


\begin{table}[H]
\centering
\footnotesize
\begin{tabular}{|p{3.5cm}|p{3.5cm}|p{3.5cm}|p{3.5cm}|}
\hline
                                      &  Web Speech \gls{api} & \gls{att} Speech \gls{api} & Microsoft Speech \gls{api} \\
\hline
Empresa o Instituci\'on responsable & Especificaci\'on publicada por \foreign{Speech \gls{api} Community Group}. Implementado actualmente por Google  &  \gls{att} Inc.  & Microsoft\\
Precio                              & Gratuita, a trav\'es de Google Chrome  & Un mill\'on de llamadas a la \gls{api} por 99\$ anuales. Un centavo por llamada extra  & Gratis\\
Soporte para m\'ultiples idiomas    & Soporta m\'as  30 idiomas & Ingl'es (principal) y Espa\~nol (reducido) & Soporta 26 idiomas\\
Soporte Offline                     & No por el momento  & No  & Si \\
Dependencia de Plataforma           & No es dependiente  & No es dependiente & Si es dependiente, solo para \emph{Microsoft Windows} \\
\hline
\end{tabular}
\caption{Resumen de los criterios espec\'ificos de las APIs}
\label{sec:resumen-apis}
\end{table}


\begin{table}[H]
\centering
\footnotesize
\begin{tabular}{|p{3.5cm}|p{3.5cm}|p{3.5cm}|p{3.5cm}|}
\hline
                                      &  \gls{wami} & iSpeech \gls{api} & Dragon Mobile \\
\hline
Empresa o Instituci\'on responsable & Instituto Tecnológico de Massachusetts & \foreign{iSpeech}  & \foreign{Nuance Communications} \\
Precio &  Gratuita, C\'odigo Abierto  & Gratis para aplicaciones no comerciales. De 0.02\$ a 0.0001\$ para otros usos & Gratis para aplicaciones no comerciales. De 0.08\$ a 0.006\$ para otros usos. Inversi\'on m\'inima de 3000\$ para comercializar una aplicaci\'on \\
Soporte para m\'ultiples idiomas  & Ingl\'es, Chino y Japon\'es. Depende del reconocedor utilizado & Reconocimiento Cont\'inuo para 6 idiomas. Reconocmiento de vocabulario reducido para 10 idiomas & Soporta m\'as de 20 idiomas \\
Soporte Offline & No & No & No\\
Dependencia de Plataforma & No es dependiente & No es dependiente & No es dependiente\\
\hline
\end{tabular}
\caption{Resumen de los criterios espec\'ificos de las APIs}
\label{sec:resumen-apis-2}
\end{table}


\subsection{Librer\'ias/Frameworks}

A continuaci\'on, las tablas~\ref{sec:resumen-libs} y~\ref{sec:resumen-libs-2} muestran una comparaci\'on entre las librer\'ias presentadas, respecto a los criterios espec\'ificos
para \'esta categor\'ia


\begin{table}[H]
\centering
\footnotesize
\begin{tabular}{|p{3.5cm}|p{3.5cm}|p{3.5cm}|p{3.5cm}|}
\hline
                                  &  Sphinx 4 & PocketSphinx & \gls{htk} \\
\hline
Empresa o Instituci\'on Responsable & Universidad \foreign{Carnegie Mellon} & Universidad \foreign{Carnegie Mellon} & Universidad de Cambridge \\
Precio & Gratuito & Gratuito & Gratuito \\
Licencia & \gls{bsd}, aunque el componente \foreign{Java Speech \gls{api}} no es de C\'odigo Abierto & \gls{bsd} simplificada & C\'odigo fuente modificable, se prohibe su redistribuci\'on. Los generados pueden redistribuirse\\
Soporte para m\'ultiples idiomas & 8 idiomas. Se pueden incorporar m\'as modelos para otros idiomas & Mismos modelos que Sphinx 4 & No brinda soporte, pero provee las herramientas necesarias\\
Dependencia de Plataforma & No es dependiente, solo depende de la M\'aquina Virtual de Java & Sistemas basados en Unix, Android, Windows, iOS y Kindle &  Basados en Unix y Windows \\
Modelos de Lenguaje Aceptados & Basados en gram\'aticas JGSF, en bigramas y trigamas en formato \gls{arpa} &  Mismos modelos que Sphinx 4 &  Mismos modelos que Sphinx 4 \\
Modelos Ac\'usticos Aceptados & Modelo propio & Igual que Sphinx 4 &  Modelo propio \\
Uso de Memoria & No se recomienda para sistemas de poca memoria & Muy reducido en comparaci\'on a Sphinx 4 & Dependiente de la aplicaci\'on. Por ejemplo: 150 MB para generar modelos para un sistema de dictado \\
\hline
\end{tabular}
\caption{Resumen de los criterios espec\'ificos de las Librer\'ias/Frameworks}
\label{sec:resumen-libs}
\end{table}

\begin{table}[H]
\centering
\footnotesize
\begin{tabular}{|p{3.5cm}|p{3.5cm}|p{3.5cm}|}
\hline
                                  &  Julius & Kaldi \\
\hline
Empresa o Instituci\'on Responsable &  \foreign{Interactive Speech Technology Consortium} & Agencia Tecnológica de República Checa \\
Precio & Gratuito & Gratuito \\
Licencia & \gls{bsd} & Apache 2.0 \\
Soporte para m\'ultiples idiomas &  Japon\'es e Ingl\'es. Puede extenderse para otros idiomas &  No brinda soporte, pero provee las herramientas necesarias \\
Dependencia de Plataforma & Basados en Unix y Windows & Basados en Unix y Windows \\
Modelos de Lenguaje Aceptados & Basados en gram\'aticas y modelos en formato \gls{arpa} & Modelos en formato \gls{fst} \\
Modelos Acústicos Aceptados & Igual que \gls{htk} & Modelo propio \\
Uso de Memoria & Bajo, 64 MB para un vocabulario de 20 mil palabras & \\
\hline
\end{tabular}
\caption{Resumen de los criterios espec\'ificos de las Librer\'ias/Frameworks}
\label{sec:resumen-libs-2}
\end{table}
