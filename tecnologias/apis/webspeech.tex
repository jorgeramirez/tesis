%!TEX root = ../../tesis.tex
\subsubsection{Web Speech \gls{api}}
\label{sec:webspeech}

La \foreign{Web Speech \gls{api}} define un est\'andar que especifica una interfaz de programaci\'on de aplicaciones para permitir
a los desarrolladores web incorporar s{\'\i}ntesis y reconocimiento del habla a sus sitios web.

\begin{itemize}
	\item \emph{Empresa o Instituci\'on Responsable:} La especificaci\'on de la interfaz fue
	publicada por el \foreign{Speech \gls{api} Community Group} del
	\mbox{\foreign{World Wide Web Consortium} \cite{GoogleWebSpeechAPI}.}
	La \'unica implementaci\'on disponible fue desarrollada por \foreign{Google} para su navegador \foreign{Chrome}.
	\item \emph{Precio:} la utilizaci\'on de esta herramienta a trav\'es de \foreign{Google Chrome} es gratuita e
	ilimitada en la actualidad. Aunque esto podr\'ia llegar a cambiar ya que se trata del producto de una empresa
    en particular.
	\item \emph{Soporte para m\'ultiples idiomas:} la interfaz ofrece soporte para m\'as de 30 idiomas.
	\item \emph{Soporte Offline:} la especificaci\'on publicada de la interfaz no limita ni reglamenta su
	implementaci\'on; sin embargo, la \'unica opci\'on funcional actualmente delega el procesamiento a los
	servidores de \foreign{Google}, por lo cual requiere de una conexi\'on a internet para su uso.
	\item \emph{Dependencia de Plataforma:} es necesario un navegador web para su utilización.
	Aunque actualmente solo puede utilizarse a trav\'es de \foreign{Chrome}, implementar la interfaz
	para otros navegadores es posible. No existe dependencia con el sistema operativo.
\end{itemize}

