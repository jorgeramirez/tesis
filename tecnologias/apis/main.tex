%!TEX root = ../../tesis.tex
\section{Interfaces de Programaci\'on de Aplicaciones}
\label{apis}

% introduccion

Una interfaz de programaci\'on de aplicaciones (tambi\'en conocida como \gls{api} por sus siglas 
en ingl\'es) proporciona a los desarrolladores los medios necesarios para integrar funcionalidades de
reconocimiento del habla al \foreign{software} que implementan, sin necesidad de poseer
conocimiento en detalle del \'area. De acuerdo a los criterios de evaluaci\'on generales, 
pueden mencionarse las siguientes caracter{\'\i}sticas:

\begin{itemize}
 	\item Conocimiento t\'ecnico necesario: aunque no requieren conocimiento sobre detalles de implementaci\'on
 	del \'area de reconocimiento del habla, si resultan necesarios conocimientos b\'asicos de
 	programaci\'on, por lo cual resulta inadecuada su utilizaci\'on por parte de usuarios finales.
 	\item Productividad: la utilizaci\'on de una interfaz de programaci\'on de aplicaciones permite
 	al desarrollador abstraerse de la complejidad subyacente del problema, lo cual resulta en
 	un alto grado de productividad. Sin embargo, el ocultamiento de los detalles del proceso
 	puede ser un factor negativo en algunos \'ambitos, como el \mbox{acad\'emico}.
 	\item Flexibilidad: estas herramientas presentan un buen grado de flexibilidad, debido a
 	que no est\'an orientadas a una tarea en particular. La selecci\'on del problema espec{\'\i}fico
 	que se soluciona utilizando reconocimiento del habla es responsabilidad del desarrollador
 	que utiliza la interfaz. Cabe destacar, sin embargo, que los componentes propios de la
 	implementaci\'on de la interfaz de programaci\'on, al estar ocultos, no
 	pueden modificarse.
 \end{itemize}


\subsection{Criterios Espec\'ificos de Evaluaci\'on}

Los criterios espec{\'\i}ficos seg\'un los cuales se evaluar\'an las opciones disponibles en esta
categor{\'\i}a son:

\begin{itemize}
	\item Empresa o Instituci\'on Responsable
	\item Precio o Inversi\'on econ\'omica
	\item Soporte para m\'ultiples idiomas
	\item Soporte Offline
	\item Dependencia de Plataforma
\end{itemize}


\subsection{Ejemplos de Interfaces de Programaci\'on de Aplicaciones}

%!TEX root = ../../tesis.tex
\subsubsection{Web Speech API}
\label{sec:webspeech}

La \foreign{Web Speech API} define un est\'andar que especifica una interfaz de programaci\'on de aplicaciones para permitir
a los desarrolladores web incorporar s{\'\i}ntesis y reconocimiento del habla a sus sitios web.

\begin{itemize}
	\item \emph{Empresa o Instituci\'on Responsable:} La especificaci\'on de la interfaz fue
	publicada por el \foreign{Speech API Community Group} del
	\mbox{\foreign{World Wide Web Consortium} \cite{GoogleWebSpeechAPI}.}
	La \'unica implementaci\'on disponible fue desarrollada por \foreign{Google} para su navegador \foreign{Chrome}.
	\item \emph{Precio:} la utilizaci\'on de esta herramienta a trav\'es de \foreign{Google Chrome} es gratuita e
	ilimitada en la actualidad. Aunque esto podr\'ia llegar a cambiar ya que se trata del producto de una empresa
    en particular.
	\item \emph{Soporte para m\'ultiples idiomas:} la interfaz ofrece soporte para m\'as de 30 idiomas.
	\item \emph{Soporte Offline:} la especificaci\'on publicada de la interfaz no limita ni reglamenta su
	implementaci\'on; sin embargo, la \'unica opci\'on funcional actualmente delega el procesamiento a los
	servidores de \foreign{Google}, por lo cual requiere de una conexi\'on a internet para su uso.
	\item \emph{Dependencia de Plataforma:} es necesario un navegador web para su utilización.
	Aunque actualmente solo puede utilizarse a trav\'es de \foreign{Chrome}, implementar la interfaz
	para otros navegadores es posible. No existe dependencia con el sistema operativo.
\end{itemize}


%!TEX root = ../../tesis.tex
\subsubsection{\gls{att} Speech \gls{api}}
\label{sec:att}

La \foreign{\gls{att} Speech \gls{api}} \cite{AttSpeech} define una interfaz de programaci\'on de aplicaciones para permitir
a los desarrolladores incorporar s{\'\i}ntesis y reconocimiento del habla en aplicaciones m\'oviles,
web o de escritorio. El procesamiento se lleva a cabo a trav\'es del motor \foreign{\gls{att} Watson}.

\begin{itemize}
	\item \emph{Empresa o Instituci\'on Responsable:} \foreign{\gls{att} Inc}.
	\item \emph{Precio:} un mill\'on de llamadas a la interfaz por 99\$ anuales, llamadas adicionales a un centavo
	cada una.
	\item \emph{Soporte para m\'ultiples idiomas:} el ingl\'es es el principal idioma soportado, aunque incluye
	soporte reducido para el espa\~nol.
	\item \emph{Soporte Offline:} al delegarse el procesamiento a los servidores de \foreign{\gls{att}},
	una conexi\'on a internet es indispensable para su uso.
	\item \emph{Dependencia de Plataforma:} al ofrecer una interfaz \gls{rest}, no existe dependencia con navegador
	ni con sistema operativo alguno. Adem\'as, se ofrecen varios \gls{sdk}s para facilitar
	el desarrollo en plataformas como \foreign{Windows}, \foreign{iOS} y \foreign{Android}.
\end{itemize}

%!TEX root = ../../tesis.tex
\subsubsection{Microsoft Speech \gls{api}}
\label{sec:microsoft}

La \foreign{Microsoft Speech \gls{api}} es una interfaz de programaci\'on de aplicaciones orientada a
facilitar la integraci\'on de s{\'\i}ntesis y reconocimiento del habla con el sistema operativo \foreign{Windows}
y las aplicaciones que se ejecutan sobre el mismo \cite{MicrosoftSpeech}. Aplicaciones como \foreign{Microsoft Office}
utilizan esta interfaz para ofrecer modelos de interacci\'on multimodal\footnote{Las interfaces multimodales
ofrecen al usuario final distintas formas de interactuar con el sistema.}con el usuario.

\begin{itemize}
	\item \emph{Empresa o Instituci\'on Responsable:} \foreign{Microsoft}.
	\item \emph{Precio:} esta herramienta est\'a incluida en el sistema operativo \foreign{Windows}, aunque
	tambi\'en puede descargarse por separado de manera gratuita.
	\item \emph{Soporte para m\'ultiples idiomas:} esta herramienta ofrece soporte para 26 idiomas,
	entre los cuales se encuentra el espa\~nol.
	\item \emph{Soporte Offline:} esta herramienta se encuentra integrada con el sistema operativo
	y realiza el procesamiento de manera local, por lo cual puede utilizarse sin conexi\'on a internet.
	\item \emph{Dependencia de Plataforma:} esta interfaz es estrictamente dependiende del sistema operativo,
	siendo utilizable solo con \foreign{Microsoft Windows}.
\end{itemize}

%!TEX root = ../../tesis.tex
\subsubsection{WAMI}
\label{sec:wami}

WAMI es un \foreign{toolkit} desarrollado en el Instituto Tecnol\'ogico de
\mbox{Massachusetts \cite{WamiHome}}. T\'ecnicamente, es m\'as que una interfaz de programaci\'on
de aplicaciones; ofrece un conjunto de herramientas, para el lado servidor y el lado cliente,
que permiten al desarrollador ofrecer su propia implementaci\'on de la interfaz.

WAMI est\'a orientado a ofrecer una interfaz web para un determinado reconocedor del habla,
como Sphinx, Julius, u otro. A\'un as{\'\i}, el MIT ofrece un reconocedor del habla que implementa
la interfaz para fines de prototipado como parte del proyecto.

\begin{itemize}
	\item \emph{Empresa o Instituci\'on Responsable:} Instituto Tecnol\'ogico de Massachusetts.
	\item \emph{Precio:} esta herramienta es gratuita y de c\'odigo abierto.
	\item \emph{Soporte para m\'ultiples idiomas:} la implementaci\'on para prototipado ofrecida
	por el MIT soporta los idiomas ingl\'es, chino y japon\'es. El soporte para idiomas de WAMI
	depende exclusivamente del reconocedor con cual se integre.
	\item \emph{Soporte Offline:} WAMI se basa en la arquitectura cliente--servidor, por lo cual
	es necesaria una conexi\'on entre ambos.
	\item \emph{Dependencia de Plataforma:} al ofrecer una interfaz REST, no existe dependencia con
	navegador ni con sistema operativo alguno.
\end{itemize}

%!TEX root = ../../tesis.tex
\subsubsection{iSpeech \gls{api}}
\label{sec:ispeech}

La \foreign{iSpeech \gls{api}} \cite{iSpeech} define una interfaz de programaci\'on de aplicaciones para permitir
a los desarrolladores incorporar s{\'\i}ntesis y reconocimiento del habla en aplicaciones m\'oviles,
web o de escritorio.

\begin{itemize}
	\item \emph{Empresa o Instituci\'on Responsable:} \foreign{iSpeech}, una empresa estadounidense dedicada a
	proveer reconocimiento y s{\'\i}ntesis del habla como servicio. Su producto m\'as reconocido es la aplicaci\'on m\'ovil
	\foreign{Drivesafe.ly}, orientada a reducir las distracciones de un conductor.
	\item \emph{Precio:} Gratis para aplicaciones m\'oviles sin fines comerciales. De 0,02\$ a 0,0001\$ por
	transacci\'on, dependiendo del tama\~no de la compra, para otros usos.
	\item \emph{Soporte para m\'ultiples idiomas:} soporta reconocimiento cont{\'\i}nuo (sin pausas) del habla para 6 idiomas,
	entre ellos el espa\~nol, y reconocimiento de vocabulario reducido para m\'as de 10 idiomas.
	\item \emph{Soporte Offline:} el procesamiento se realiza en los servidores de \foreign{iSpeech}, por lo cual
	una conexi\'on a internet es indispensable para su uso.
	\item \emph{Dependencia de Plataforma:} al ofrecer una interfaz \gls{rest}, no existe dependencia con navegador
	ni con sistema operativo alguno. Adem\'as, se ofrecen varios \gls{sdk} para facilitar
	el desarrollo en m\'ultiples plataformas y lenguajes de programaci\'on.
\end{itemize}
%!TEX root = ../../tesis.tex
\subsubsection{Dragon Mobile}
\label{sec:dragonmobile}

\foreign{Dragon Mobile} ofrece una interfaz de programaci\'on de aplicaciones para permitir
a los desarrolladores incorporar s{\'\i}ntesis y reconocimiento del habla a sus
\mbox{aplicaciones \cite{DragonMobile}}. Est\'a espec{\'\i}ficamente orientado al desarrollo de
aplicaciones para plataformas m\'oviles.

\begin{itemize}
	\item \emph{Empresa o Instituci\'on Responsable:} \foreign{Nuance Communications}.
	\item \emph{Precio:} el uso de esta herramienta es gratuito para aplicaciones sin fines comerciales.
	Desde 0.08\$ a 0.006\$ por transacci\'on, dependiendo del tama\~no de la compra, para otros usos.
	Para comercializar una aplicaci\'on que utiliza esta herramienta, es necesaria una inversi\'on inicial
	m{\'\i}nima de 3000\$.
	\item \emph{Soporte para m\'ultiples idiomas:} se ofrece soporte para m\'as de 20 idiomas, entre ellos
	el espa\~nol.
	\item \emph{Soporte Offline:} al delegarse el procesamiento a los servidores de \foreign{Nuance},
	una conexi\'on a internet es indispensable para su uso.
	\item \emph{Dependencia de Plataforma:} al ofrecer una interfaz \gls{rest}, no existe dependencia con navegador
	ni con sistema operativo alguno. Adem\'as, se ofrecen varios \gls{sdk} para facilitar
	el desarrollo en plataformas m\'oviles como \foreign{Windows Phone}, \foreign{iOS} y \foreign{Android}.
\end{itemize}
