%!TEX root = ../../tesis.tex
\subsubsection{\gls{att} Speech \gls{api}}
\label{sec:att}

La \foreign{\gls{att} Speech \gls{api}} \cite{AttSpeech} define una interfaz de programaci\'on de aplicaciones para permitir
a los desarrolladores incorporar s{\'\i}ntesis y reconocimiento del habla en aplicaciones m\'oviles,
web o de escritorio. El procesamiento se lleva a cabo a trav\'es del motor \foreign{\gls{att} Watson}.

\begin{itemize}
	\item \emph{Empresa o Instituci\'on Responsable:} \foreign{\gls{att} Inc}.
	\item \emph{Precio:} un mill\'on de llamadas a la interfaz por 99\$ anuales, llamadas adicionales a un centavo
	cada una.
	\item \emph{Soporte para m\'ultiples idiomas:} el ingl\'es es el principal idioma soportado, aunque incluye
	soporte reducido para el espa\~nol.
	\item \emph{Soporte Offline:} al delegarse el procesamiento a los servidores de \foreign{\gls{att}},
	una conexi\'on a internet es indispensable para su uso.
    \item \emph{Dependencia de Plataforma:} al ofrecer una interfaz \gls{rest}\footnote{REST es una t\'ecnica de
        arquitectura de software para sistemas hipermedia distribuidos. Provee un conjunto de restricciones arquitecturales
        que, cuando se aplican como un todo, enfatizan la escalabilidad de interacci\'on entre componentes, generalidad
        de interfaces, despliegue independiente de componentes, reforzar la seguridad y encapsular sistemas heredados\cite{Fielding2000}.}, 
    no existe dependencia con navegador
	ni con sistema operativo alguno. Adem\'as, se ofrecen varios \gls{sdk}s para facilitar
	el desarrollo en plataformas como \foreign{Windows}, \foreign{iOS} y \foreign{Android}.
\end{itemize}
