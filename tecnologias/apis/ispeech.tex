%!TEX root = ../../tesis.tex
\subsubsection{iSpeech API}
\label{sec:ispeech}

La \foreign{iSpeech API} \cite{iSpeech} define una interfaz de programaci\'on de aplicaciones para permitir
a los desarrolladores incorporar s{\'\i}ntesis y reconocimiento del habla en aplicaciones m\'oviles,
web o de escritorio.

\begin{itemize}
	\item \emph{Empresa o Instituci\'on Responsable:} \foreign{iSpeech}, una empresa estadounidense dedicada a
	proveer reconocimiento y s{\'\i}ntesis del habla como servicio. Su producto m\'as reconocido es la aplicaci\'on m\'ovil
	\foreign{Drivesafe.ly}, orientada a reducir las distracciones de un conductor.
	\item \emph{Precio:} Gratis para aplicaciones m\'oviles sin fines comerciales. De 0,02\$ a 0,0001\$ por
	transacci\'on, dependiendo del tama\~no de la compra, para otros usos.
	\item \emph{Soporte para m\'ultiples idiomas:} soporta reconocimiento cont{\'\i}nuo (sin pausas) del habla para 6 idiomas,
	entre ellos el espa\~nol, y reconocimiento de vocabulario reducido para m\'as de 10 idiomas.
	\item \emph{Soporte Offline:} el procesamiento se realiza en los servidores de \foreign{iSpeech}, por lo cual
	una conexi\'on a internet es indispensable para su uso.
	\item \emph{Dependencia de Plataforma:} al ofrecer una interfaz REST, no existe dependencia con navegador
	ni con sistema operativo alguno. Adem\'as, se ofrecen \foreign{software development kits} o SDKs para facilitar
	el desarrollo en m\'ultiples plataformas y lenguajes de programaci\'on.
\end{itemize}