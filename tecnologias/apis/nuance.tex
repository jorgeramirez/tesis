%!TEX root = ../../tesis.tex
\subsubsection{Dragon Mobile}
\label{sec:dragonmobile}

\foreign{Dragon Mobile} ofrece una interfaz de programaci\'on de aplicaciones para permitir
a los desarrolladores incorporar s{\'\i}ntesis y reconocimiento del habla a sus
\mbox{aplicaciones \cite{DragonMobile}}. Est\'a espec{\'\i}ficamente orientado al desarrollo de
aplicaciones para plataformas m\'oviles.

\begin{itemize}
	\item \emph{Empresa o Instituci\'on Responsable:} \foreign{Nuance Communications}.
	\item \emph{Precio:} el uso de esta herramienta es gratuito para aplicaciones sin fines comerciales.
	Desde 0.08\$ a 0.006\$ por transacci\'on, dependiendo del tama\~no de la compra, para otros usos.
	Para comercializar una aplicaci\'on que utiliza esta herramienta, es necesaria una inversi\'on inicial
	m{\'\i}nima de 3000\$.
	\item \emph{Soporte para m\'ultiples idiomas:} se ofrece soporte para m\'as de 20 idiomas, entre ellos
	el espa\~nol.
	\item \emph{Soporte Offline:} al delegarse el procesamiento a los servidores de \foreign{Nuance},
	una conexi\'on a internet es indispensable para su uso.
	\item \emph{Dependencia de Plataforma:} al ofrecer una interfaz REST, no existe dependencia con navegador
	ni con sistema operativo alguno. Adem\'as, se ofrecen \foreign{software development kits} o SDKs para facilitar
	el desarrollo en plataformas m\'oviles como \foreign{Windows Phone}, \foreign{iOS} y \foreign{Android}.
\end{itemize}