%!TEX root = ../../tesis.tex
\subsubsection{WAMI}
\label{sec:wami}

WAMI es un \foreign{toolkit} desarrollado en el Instituto Tecnol\'ogico de
\mbox{Massachusetts \cite{WamiHome}}. T\'ecnicamente, es m\'as que una interfaz de programaci\'on
de aplicaciones; ofrece un conjunto de herramientas, para el lado servidor y el lado cliente,
que permiten al desarrollador ofrecer su propia implementaci\'on de la interfaz.

WAMI est\'a orientado a ofrecer una interfaz web para un determinado motor de reconocimiento del habla. 
A\'un as{\'\i}, el MIT ofrece un reconocedor del habla que implementa la interfaz para fines 
de prototipado como parte del proyecto.

\begin{itemize}
	\item \emph{Empresa o Instituci\'on Responsable:} Instituto Tecnol\'ogico de Massachusetts.
	\item \emph{Precio:} esta herramienta es gratuita y de c\'odigo abierto.
	\item \emph{Soporte para m\'ultiples idiomas:} la implementaci\'on para prototipado ofrecida
	por el MIT soporta los idiomas ingl\'es, chino y japon\'es. El soporte para idiomas de WAMI
	depende exclusivamente del reconocedor c on cual se integre.
	\item \emph{Soporte Offline:} WAMI se basa en la arquitectura cliente-servidor, es decir, el reconocimiento del habla 
    se realiza en el servidor y es enviado de vuelta al cliente. Por lo tanto, es necesaria una conexi\'on entre ambos.
	\item \emph{Dependencia de Plataforma:} al ofrecer una interfaz REST, no existe dependencia con
	navegador ni con sistema operativo alguno.
\end{itemize}
