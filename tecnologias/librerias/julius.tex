%!TEX root = ../../tesis.tex
\subsubsection{Julius}
\label{sec:julius}

Julius es un motor de reconocimiento del habla altamente configurable desarrollado por un grupo
de universidades e institutos de investigaci\'on en Jap\'on. Est\'a orientando al reconocimiento
del habla continua para vocabularios grandes, como ocurre en el dictado.
Julius est\'a implementado en el \mbox{lenguaje C \cite{JuliusHomePage}}.

\begin{itemize}
	\item \emph{Empresa o Instituci\'on Responsable:} \foreign{Interactive Speech Technology Consortium},
	una dependencia del \foreign{Advanced Scientific Technology \& Management Research Institute of KYOTO}.
	\item \emph{Precio:} esta herramienta puede utilizarse de forma gratuita.
	\item \emph{Licencia:} se distribuye bajo una versi\'on revisada de la licencia
	\gls{bsd}, considerada de c\'odigo abierto.
	\item \emph{Soporte para m\'ultiples idiomas:} existen modelos disponibles para
	el japon\'es y el ingl\'es. Los formatos de los modelos siguen los est\'andares de otras
	herramientas, de modo a extender el soporte a otros idiomas. De esta manera,
	se han realizado investigaciones utilizando Julius para el esloveno, franc\'es, tailand\'es
	y otros idiomas.
	\item \emph{Dependencia de Plataforma:} esta herramienta puede utilizarse en sistemas operativos
	basados en \foreign{Unix} y en \foreign{Windows}.
	\item \emph{Modelos de Lenguaje Aceptados:} se aceptan modelos de lenguaje de palabras aisladas,
	basados en gram\'aticas y modelos estad{\'\i}sticos en formato \gls{arpa}.
	\item \emph{Modelos Ac\'usticos Aceptados:} utiliza el mismo formato de modelo ac\'ustico que \gls{htk}.
	\item \emph{Uso de Memoria:} el consumo de memoria de Julius es bajo, cit\'andose como ejemplo
	un consumo menor a 64 MB para reconocimiento sobre un vocabulario de 20 mil palabras utilizando
	un modelo de lenguaje basado en trigramas almacenado en memoria.
\end{itemize}