\section{Librer\'ias y \emph{Frameworks}}
\label{sec:librerias}

Una librer{\'\i}a va m\'as all\'a de una especificaci\'on de interacci\'on y el c\'odigo fuente que permita cumplirla.
Una librer{\'\i}a incluye un conjunto de funcionalidades o m\'etodos cuyo modo de utilizaci\'on queda a criterio
del desarrollador. En el caso de un \foreign{framework}, se incluyen tambi\'en ciertos patrones de dise\~no para la aplicaci\'on.

Aunque la distinci\'on entre una librer{\'\i}a y una interfaz de programaci\'on de aplicaciones puede resultar
sutil, para este trabajo, se consideran el mayor grado de control y el menor nivel de abstracci\'on
como criterios para identificar una herramienta como una librer{\'\i}a.

As{\'\i}, una interfaz de programaci\'on de aplicaciones es una especificaci\'on de interacci\'on con un componente
reconocedor del habla completamente implementado, mientras que una librer{\'\i}a puede verse como un conjunto
de bloques que permiten al desarrollador construir su propio reconocedor del habla a medida.

De acuerdo a los criterios de evaluaci\'on generales, pueden mencionarse las 
siguientes caracter{\'\i}sticas :

\begin{itemize}
 	\item Conocimiento t\'ecnico necesario: la utilizaci\'on de librer{\'\i}as requiere conocimiento t\'ecnico
 	espec{\'\i}fico del \'area del reconocimiento del habla por parte del programador, debido a que el nivel de
 	abstracci\'on es menor al de las herramientas previamente descriptas.
 	\item Productividad: Debido al alto grado de control que ofrecen, el desarrollo de los distintos
 	elementos de un sistema de reconocimiento del habla requiere de m\'as trabajo de an\'alisis e implementaci\'on.
 	Esto resulta en una productividad baja en comparaci\'on con otras alternativas, que permiten al
 	programador abstraerse de los detalles del proceso de reconocimiento del habla.
 	A\'un as{\'\i} en ciertos \'ambitos, como el acad\'emico, la posibilidad de conocer y manipular los distintos
 	componentes del mencionado proceso representa una importante ventaja que ofrecen estas herramientas.
 	\item Flexibilidad: las librer{\'\i}as ofrecen un alto grado de flexibilidad, siendo la alternativa
 	m\'as adaptable de entre las categor{\'\i}as analizadas. Componentes importantes de un sistema
 	de reconocimiento del habla como el modelo de lenguaje, el modelo ac\'ustico y los algoritmos involucrados
 	pueden seleccionarse, modificarse o incluso reimplementarse de acuerdo al criterio del desarrollador.
 \end{itemize}

\subsection{Criterios Espec\'ificos de Evaluaci\'on}

Los criterios espec{\'\i}ficos seg\'un los cuales se evaluar\'an las opciones disponibles en esta
categor{\'\i}a son:

\begin{itemize}
	\item Empresa o Instituci\'on Responsable
	\item Precio
	\item Licencia
	\item Soporte para m\'ultiples idiomas
	\item Dependencia de Plataforma
	\item Modelos de Lenguaje Aceptados
	\item Modelos Ac\'usticos Aceptados
	\item Uso de Memoria
\end{itemize}


\subsection{Ejemplos de librer{\'\i}as o \emph{frameworks}}

%!TEX root = ../../tesis.tex
\subsubsection{Sphinx 4}
\label{sec:sphinx}

La librer{\'\i}a Sphinx 4 es un sistema de reconocimiento del habla de c\'odigo abierto desarrollado en
la Universidad \mbox{\foreign{Carnegie Mellon} \cite{Sphinx4}}. La misma est\'a implementada
completamente en el lenguaje de programaci\'on Java, y permite la utilizaci\'on de modelos de lenguaje,
modelos ac\'usticos y diccionarios fon\'eticos construidos por el propio desarrollador utilizando las
herramientas proveídas por la librer\'ia.

Esta herramienta est\'a orientada principalmente a la escalabilidad y a la integraci\'on con
componente externos, facilitando la implementaci\'on, por ejemplo, de sistema de reconocimiento del
habla en la nube.

\begin{itemize}
	\item \emph{Empresa o Instituci\'on Responsable:} Universidad \foreign{Carnegie Mellon}.
	\item \emph{Precio:} esta herramienta puede utilizarse de forma gratuita.
	\item \emph{Licencia:} Sphinx 4 se distribuye bajo una licencia basada en la \gls{bsd}, considerada
	de c\'odigo abierto. Un \'unico componente, la \foreign{Java Speech \gls{api}} debe descargarse por
	separado por no ser de c\'odigo abierto, aunque puede obtenerse de manera gratuita.
	\item \emph{Soporte para m\'ultiples idiomas:} existen modelos ac\'usticos construidos para
	8 idiomas, entre ellos el espa\~nol, aunque la calidad de los mismos var{\'\i}a.
	De todas formas, el desarrollador puede construir sus propios modelos con las herramientas
	que provee la librer{\'\i}a, con lo cual puede brindarse soporte para otros idiomas.
	\item \emph{Dependencia de Plataforma:} puede utilizarse cualquier plataforma para la cual
	exista una versi\'on de la m\'aquina virtual de Java disponible.
	\item \emph{Modelos de Lenguaje Aceptados:} se aceptan modelos de lenguaje basados en una
        gram\'atica, en formato JGSF (\foreign{Java Speech Grammar Format}), y modelos de 
        lenguaje estad{\'\i}sticos, basados en bigramas y trigramas en formato \gls{arpa} (\foreign{Advanced Research Projects Agency}).
	\item \emph{Modelos Ac\'usticos Aceptados:} Sphinx define su propio formato de modelos ac\'usticos,
	en el cual existen modelos ya construidos para 8 idiomas. Adem\'as, se ofrecen las herramientas
	necesarias para construir un modelo ac\'ustico propio.
	\item \emph{Uso de Memoria:} esta herramienta no se recomienda para sistemas con poca memoria,
	debido a que depende de la m\'aquina virtual de Java para ejecutarse.
\end{itemize}

%!TEX root = ../../tesis.tex
\subsubsection{PocketSphinx}
\label{sec:pocketsphinx}

La librer{\'\i}a Pocketsphinx es desarrollada en paralelo a Sphinx 4 en la Universidad
\mbox{\foreign{Carnegie Mellon} \cite{PocketSphinxHomePage}}. Esta herramienta de c\'odigo abierto
est\'a implementada en el lenguaje C, logr\'andose con esto en un mayor rendimiento y facilidad
para desarrollar \foreign{bindings} para otros lenguajes de programaci\'on.

Pocketsphinx est\'a orientado a la optimizaci\'on del rendimiento y a la portabilidad a distintas
plataformas, resultando particularmente adecuada para la implementaci\'on de sistemas empotrados
y sistemas en tiempo \mbox{real \cite{HugginsDainesPocketSphinx2006}}.

\begin{itemize}
	\item \emph{Empresa o Instituci\'on Responsable:} Universidad \foreign{Carnegie Mellon}.
	\item \emph{Precio:} esta librer\'ia puede utilizarse de forma gratuita.
	\item \emph{Licencia:} Pocketsphinx  se distribuye bajo una versi\'on simplificada de la licencia
	\gls{bsd}, considerada de c\'odigo abierto.
	\item \emph{Soporte para m\'ultiples idiomas:} Pocketsphinx utiliza los mismos modelos
	que \mbox{Sphinx 4}, por lo que el soporte para idiomas es similar. Se ofrece soporte para 8
	idiomas, aunque puede extenderse a otros mediante las herramientas prove{\'\i}das.
	\item \emph{Dependencia de Plataforma :} esta librer\'ia puede utilizarse en sistemas operativos
	basados en Unix y en Windows. Adem\'as, cabe destacar la documentaci\'on existente para su instalaci\'on
	en plataformas m\'oviles como Android, iOS y el Kindle de Amazon.
	\item \emph{Modelos de Lenguaje Aceptados:} se utilizan los mismos modelos de lenguaje que \mbox{Sphinx 4},
	basados en gram\'aticas en formato JGSF y modelos estad{\'\i}sticos en formato \gls{arpa}.
	\item \emph{Modelos Ac\'usticos Aceptados:} tambi\'en los modelos ac\'usticos se comparten con \mbox{Sphinx 4}.
	Existen modelos para 8 idiomas ya construidos, y pueden construirse modelos adicionales.
	\item \emph{Uso de Memoria:} el consumo de memoria de Pocketsphinx es muy reducido en comparaci\'on con
        Sphinx 4\cite{SphinxVersions}.
\end{itemize}
%!TEX root = ../../tesis.tex
\subsubsection{HTK}
\label{sec:htk}

\gls{htk}, es un conjunto de herramientas
desarrolladas por el Departamento de Ingenier{\'\i}a de la Universidad de Cambridge. Est\'a constituido
por varias librer{\'\i}as y herramientas ejecutables implementadas en el \mbox{lenguaje C \cite{HTKHomePage}}.

Esta herramienta ofrece un alto grado de personalizaci\'on, por lo que se requiere un nivel considerable
de conocimiento t\'ecnico para implementar un sistema de reconocimiento del habla con la misma.
Cabe destacar que la \'ultima versi\'on estable de \gls{htk} fue lanzada en el a\~no 2009.

\begin{itemize}
	\item \emph{Empresa o Instituci\'on Responsable:} Universidad de Cambridge.
	\item \emph{Precio:} esta herramienta puede utilizarse de forma gratuita.
	\item \emph{Licencia:} aunque el c\'odigo fuente de \gls{htk} puede obtenerse y modificarse,
	la redistribuci\'on del mismo est\'a prohibida. Los modelos generados con \gls{htk} pueden
	redistribuirse libremente.
	\item \emph{Soporte para m\'ultiples idiomas:} \gls{htk} no provee soporte para ning\'un
	idioma directamente, pero proporciona herramientas para la definici\'on de los modelos necesarios.
	\item \emph{Dependencia de Plataforma:} esta herramienta puede utilizarse en sistemas operativos
	basados en Unix y en Windows.
	\item \emph{Modelos de Lenguaje Aceptados:} se utilizan los mismos modelos de lenguaje que \mbox{Sphinx 4},
	basados en gram\'aticas y modelos estad{\'\i}sticos en formato \gls{arpa}.
	\item \emph{Modelos Ac\'usticos Aceptados:} \gls{htk} define su propio formato de modelos ac\'usticos.
	\item \emph{Uso de Memoria:} la documentaci\'on de \gls{htk} hace \'enfasis en que el consumo de memoria depende en
	gran medida de la aplicaci\'on, mencionando 150 MB para la construcci\'on de modelos para un sistema de
	dictado como un ejemplo de niveles altos de utilizaci\'on de este recurso.
	Esta cantidad de memoria no resulta muy elevada para el \foreign{hardware} de una m\'aquina promedio actual.
\end{itemize}
\subsection{Julius}
\label{sec:julius}

%!TEX root = ../../tesis.tex
\subsubsection{Kaldi}
\label{sec:kaldi}

Kaldi es un \foreign{toolkit} de reconocimiento del habla que comenz\'o a desarrollarse durante un taller
en la Universidad \foreign{John Hopkins} en el a\~no 2009. Su desarrollo continu\'o con el apoyo del gobierno
de Rep\'ublica Checa en la \mbox{Universidad de Brno \cite{Povey_ASRU2011}}.

Kaldi est\'a constituida por un conjunto de librer{\'\i}as C++ y herramientas ejecutables. Por su orientaci\'on y
alcance es comparable al proyecto \gls{htk}, resultando tambi\'en necesario un alto nivel de conocimiento t\'ecnico
para su utilizaci\'on.

Cabe destacar su integraci\'on con \gls{fst} para la
construcci\'on de modelos de lenguaje, la cual lo distingue de las dem\'as herramientas analizadas en esta secci\'on.

\begin{itemize}
	\item \emph{Empresa o Instituci\'on Responsable:} Agencia Tecnol\'ogica de Rep\'ublica Checa.
	\item \emph{Precio:} esta herramienta puede utilizarse de forma gratuita.
	\item \emph{Licencia:} Kaldi se distribuye bajo la licencia Apache 2.0, considerada de c\'odigo 
	abierto.
	\item \emph{Soporte para m\'ultiples idiomas:} Kaldi no provee soporte para ning\'un
	idioma directamente, pero proporciona herramientas para la definici\'on de los modelos 
	\mbox{necesarios.}
	\item \emph{Dependencia de Plataforma:} esta herramienta puede utilizarse en sistemas operativos
	basados en \foreign{Unix} y en \foreign{Windows}.
	\item \emph{Modelos de Lenguaje Aceptados:} se proveen herramientas para convertir modelos en formato
	\gls{arpa} a \gls{fst}s.
	\item \emph{Modelos Ac\'usticos Aceptados:} Kaldi define su propio formato de modelos ac\'usticos.
	\item \emph{Uso de Memoria:} no se encontraron datos espec{\'\i}ficos respecto al consumo de memoria
	de Kaldi.
\end{itemize}
