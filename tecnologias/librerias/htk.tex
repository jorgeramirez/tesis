%!TEX root = ../../tesis.tex
\subsubsection{HTK}
\label{sec:htk}

\gls{htk}, es un conjunto de herramientas
desarrolladas por el Departamento de Ingenier{\'\i}a de la Universidad de Cambridge. Est\'a constituido
por varias librer{\'\i}as y herramientas ejecutables implementadas en el \mbox{lenguaje C \cite{HTKHomePage}}.

Esta herramienta ofrece un alto grado de personalizaci\'on, por lo que se requiere un nivel considerable
de conocimiento t\'ecnico para implementar un sistema de reconocimiento del habla con la misma.
Cabe destacar que la \'ultima versi\'on estable de \gls{htk} fue lanzada en el a\~no 2009.

\begin{itemize}
	\item \emph{Empresa o Instituci\'on Responsable:} Universidad de Cambridge.
	\item \emph{Precio:} esta herramienta puede utilizarse de forma gratuita.
	\item \emph{Licencia:} aunque el c\'odigo fuente de \gls{htk} puede obtenerse y modificarse,
	la redistribuci\'on del mismo est\'a prohibida. Los modelos generados con \gls{htk} pueden
	redistribuirse libremente.
	\item \emph{Soporte para m\'ultiples idiomas:} \gls{htk} no provee soporte para ning\'un
	idioma directamente, pero proporciona herramientas para la definici\'on de los modelos necesarios.
	\item \emph{Dependencia de Plataforma:} esta herramienta puede utilizarse en sistemas operativos
	basados en Unix y en Windows.
	\item \emph{Modelos de Lenguaje Aceptados:} se utilizan los mismos modelos de lenguaje que \mbox{Sphinx 4},
	basados en gram\'aticas y modelos estad{\'\i}sticos en formato \gls{arpa}.
	\item \emph{Modelos Ac\'usticos Aceptados:} \gls{htk} define su propio formato de modelos ac\'usticos.
	\item \emph{Uso de Memoria:} la documentaci\'on de \gls{htk} hace \'enfasis en que el consumo de memoria depende en
	gran medida de la aplicaci\'on, mencionando 150 MB para la construcci\'on de modelos para un sistema de
	dictado como un ejemplo de niveles altos de utilizaci\'on de este recurso.
	Esta cantidad de memoria no resulta muy elevada para el \foreign{hardware} de una m\'aquina promedio actual.
\end{itemize}