%!TEX root = ../../tesis.tex
\subsubsection{PocketSphinx}
\label{sec:pocketsphinx}

La librer{\'\i}a Pocketsphinx es desarrollada en paralelo a Sphinx 4 en la Universidad
\mbox{\foreign{Carnegie Mellon} \cite{PocketSphinxHomePage}}. Esta herramienta de c\'odigo abierto
est\'a implementada en el lenguaje C, logr\'andose con esto en un mayor rendimiento y facilidad
para desarrollar \foreign{bindings} para otros lenguajes de programaci\'on.

Pocketsphinx est\'a orientado a la optimizaci\'on del rendimiento y a la portabilidad a distintas
plataformas, resultando particularmente adecuada para la implementaci\'on de sistemas empotrados
y sistemas en tiempo \mbox{real \cite{HugginsDainesPocketSphinx2006}}.

\begin{itemize}
	\item \emph{Empresa o Instituci\'on Responsable:} Universidad \foreign{Carnegie Mellon}.
	\item \emph{Precio:} esta herramienta puede utilizarse de forma gratuita.
	\item \emph{Licencia:} Pocketsphinx se distribuye bajo una versi\'on simplificada de la licencia
	BSD, considerada de c\'odigo abierto.
	\item \emph{Soporte para m\'ultiples idiomas:} Pocketsphinx utiliza los mismos modelos
	que \mbox{Sphinx 4}, por lo que el soporte para idiomas es similar. Se ofrece soporte para 8
	idiomas, aunque puede extenderse a otros mediante las herramientas prove{\'\i}das.
	\item \emph{Dependencia de Plataforma:} esta herramienta puede utilizarse en sistemas operativos
	basados en Unix y en Windows. Adem\'as, cabe destacar la documentaci\'on existente para su instalaci\'on
	en plataformas m\'oviles como Android, iOS y el Kindle de Amazon.
	\item \emph{Modelos de Lenguaje Aceptados:} se utilizan los mismos modelos de lenguaje que \mbox{Sphinx 4},
	basados en gram\'aticas en formato JGSF y modelos estad{\'\i}sticos en formato ARPA.
	\item \emph{Modelos Ac\'usticos Aceptados:} tambi\'en los modelos ac\'usticos se comparten con \mbox{Sphinx 4}.
	Existen modelos para 8 idiomas ya construidos, y pueden construirse modelos adicionales.
	\item \emph{Uso de Memoria:} el consumo de memoria de Pocketsphinx es muy reducido en comparaci\'on con
	Sphinx 4.
\end{itemize}