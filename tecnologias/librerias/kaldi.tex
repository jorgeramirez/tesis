%!TEX root = ../../tesis.tex
\subsubsection{Kaldi}
\label{sec:kaldi}

Kaldi es un \foreign{toolkit} de reconocimiento del habla que comenz\'o a desarrollarse durante un taller
en la Universidad \foreign{John Hopkins} en el a\~no 2009. Su desarrollo continu\'o con el apoyo del gobierno
de Rep\'ublica Checa en la \mbox{Universidad de Brno \cite{Povey_ASRU2011}}.

Kaldi est\'a constituida por un conjunto de librer{\'\i}as C++ y herramientas ejecutables. Por su orientaci\'on y
alcance es comparable al proyecto \gls{htk}, resultando tambi\'en necesario un alto nivel de conocimiento t\'ecnico
para su utilizaci\'on.

Cabe destacar su integraci\'on con \gls{fst} para la
construcci\'on de modelos de lenguaje, la cual lo distingue de las dem\'as herramientas analizadas en esta secci\'on.

\begin{itemize}
	\item \emph{Empresa o Instituci\'on Responsable:} Agencia Tecnol\'ogica de Rep\'ublica Checa.
	\item \emph{Precio:} esta herramienta puede utilizarse de forma gratuita.
	\item \emph{Licencia:} Kaldi se distribuye bajo la licencia Apache 2.0, considerada de c\'odigo 
	abierto.
	\item \emph{Soporte para m\'ultiples idiomas:} Kaldi no provee soporte para ning\'un
	idioma directamente, pero proporciona herramientas para la definici\'on de los modelos 
	\mbox{necesarios.}
	\item \emph{Dependencia de Plataforma:} esta herramienta puede utilizarse en sistemas operativos
	basados en \foreign{Unix} y en \foreign{Windows}.
	\item \emph{Modelos de Lenguaje Aceptados:} se proveen herramientas para convertir modelos en formato
	\gls{arpa} a \gls{fst}s.
	\item \emph{Modelos Ac\'usticos Aceptados:} Kaldi define su propio formato de modelos ac\'usticos.
	\item \emph{Uso de Memoria:} no se encontraron datos espec{\'\i}ficos respecto al consumo de memoria
	de Kaldi.
\end{itemize}