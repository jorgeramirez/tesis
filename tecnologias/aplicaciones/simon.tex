\subsubsection{Simon}
\label{sec:simon}

\emph{Simon} es un programa de reconocimiento del habla de c\'odigo abierto que tiene como objetivo
reemplazar el teclado y mouse \cite{SimonListen}. Esta dise\~nado para ser muy flexible y permite la
personalizaci\'on para cualquier aplicaci\'on que necesite incorporar reconocimiento
del habla. Adem\'as, \'este programa pretende ayudar a personas con alguna discapacidad, debido
a que busca brindarles la posibilidad de escribir correos electr\'onicos, navegar internet, etc.

\begin{itemize}
    \item \emph{Precio:} esta herramienta es gratuita y de c\'odigo abierto.
    \item \emph{Soporte para m\'ultiples idiomas:} \emph{Simon} es una herramienta independiente del idioma,
    debido a que permite incorporar nuevos modelos estad\'isticos. Si se desea
	utilizar el espa\~nol, por ejemplo, simplemente hay que incorporar el modelo estad\'istico que representa
    a los sonidos que forman las palabras de dicho idioma.
    \item \emph{Facilidad de configuraci\'on:} esta herramienta est\'a dise\~nada para que sea
	f\'acilmente configurable desde su interfaz gr\'afica. Utiliza el concepto de Escenarios para identificar a un contexto (aplicaci\'on)
	que se desea operar a trav\'es de la voz. Brinda, por defecto, un modelo base para el idioma ingl\'es (el
    modelo prove\'ido por Voxforge\cite{Voxforge}), pero permite incorporar otros modelos e incluso generar nuevos.
    \item \emph{Soporte para dispositivos m\'oviles:} \emph{Simone} es un cliente de Simon para dispositivos m\'oviles. Aunque las
	capacidades de este cliente son limitadas, actualmente permite: controlar una computadora a trav\'es de una red
	local inal\'ambrica (utilizar el tel\'efono como micr\'ofono), realizar llamadas a contactos y navegar el software. Actualmente
	solo soporta Meego\footnote{Meego es un sistema operativo de c\'odigo abierto que resulta de la uni\'on de
	los sistemas operativos \emph{Maemo} de Nokia y \emph{Moblin} de Intel, con la intenci\'on de competir
	con el sistema operativo \emph{Android} de Google}, aunque el soporte para Android se encuentra en desarrollo.

\end{itemize}
