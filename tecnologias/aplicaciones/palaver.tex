\subsubsection{Palaver}
\label{sec:palaver}

\emph{Palaver} es una aplicaci\'on, actualmente disponible solo para Linux, orientada principalmente
en el control de funcionalidades de las computadoras a trav\'es de la voz, pero tambi\'en
puede utilizarse para transcripci\'on \cite{Palaver}. \emph{Palaver} utiliza un diccionario para
definir los comandos que soporta y , adem\'as, permite definir nuevos comandos a trav\'es
de un sistema de plugins.

\begin{itemize}
    \item \emph{Precio:} esta herramienta es gratuita y de c\'odigo abierto.
    \item \emph{Soporte para m\'ultiples idiomas:} esta herramienta delega la interpretaci\'on de
	la voz a los servidores de Google, por lo tanto, es posible utilizar \emph{Palaver} en otros idiomas. Pero
	esto implica la modificaci\'on del diccionario de comandos, ya que por defecto \emph{Palaver} provee
	un diccionario para el idioma ingl\'es. Adem\'as, al delegar el reconocimiento a servidores externos, es
    necesario tener conexi\'on a internet para utilizar esta herramienta.
    \item \emph{Facilidad de configuraci\'on:} esta herramienta permite extender los comandos soportados a trav\'es de
	un sistema de plugins, adem\'as permite configurar otros aspectos de la aplicaci\'on (como el idioma). Sin embargo,
	no existe una interfaz gr\'afica para realizar las configuraciones, lo que reduce en cierta medida la facilidad
	de configuraci\'on.
    \item \emph{Soporte para dispositivos m\'oviles:} actualmente esta herramienta no soporta dispositivos m\'oviles.
\end{itemize}
