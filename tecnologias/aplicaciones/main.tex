\section{Aplicaciones}
\label{sec:aplicaciones}

% introduccion

Las soluciones de reconocimiento del habla, catalogadas como aplicaciones, propocionan
a los usuarios finales los medios necesarios para realizar determinadas tareas mediante
la voz: dictado autom\'atico, navegaci\'on de interfaces, etc\'etera; sin
la necesidad de tener un conocimiento previo de los conceptos relacionados al
reconocimiento del habla. \'Estas aplicaciones presentan las siguientes 
caracter\'isticas en cuanto a los criterios generales de evaluaci\'on:

\begin{itemize}
    \item Conocimiento t\'ecnico: requieren de poco conocimiento t\'ecnico debido a que \'estas
        soluciones se ofrecen como un producto a los usuarios finales, por lo tanto, sus funcionalidades
        pueden utilizarse directamente sin necesidad de entender los componentes que intervienen
         en el sistema.
    \item Productividad: teniendo en cuenta el criterio anterior, cabe resaltar que con
        las aplicaciones se puede lograr un buen grado de productividad, porque los esfuerzos
        de los usuarios se enfocan en realizar determinadas tareas utilizando las funcionalidades
         disponibles.
    \item Flexibilidad: la flexibilidad para esta categor\'ia de soluciones es reducida, porque las
        funcionalidades que ofrece cada aplicaci\'on se encuentran orientadas a resolver tareas
        espec\'ificas: dictado autom\'atico, para la transcripci\'on de documentos; reconocimiento
        de comandos, para navegar interfaces; reconocimiento del habla, para 
        automatizaci\'on de servicios de operador. Cada aplicaci\'on tiene un \'ambito de aplicabilidad
        lo cual impone un l\'imite en su flexibilidad, en comparaci\'on a otras categor\'ias que
         se ver\'an m\'as adelante.
\end{itemize}

\subsection{Criterios Espec\'ificos de Evaluaci\'on}

Los criterios espec\'ificos nos permiten evaluar a las aplicaciones en cuanto a factores
particulares y relevantes para esta categor\'ia. A continuaci\'on se presentan los criterios 
espec\'ificos de evaluaci\'on para las aplicaciones:

\begin{itemize}
    \item Precio
    \item Soporte para m\'ultiples idiomas
\end{itemize}

\subsection{Ejemplos de Aplicaciones}

\subsection{Simon}
\label{sec:simon}

\subsubsection{Palaver}
\label{sec:palaver}

\subsection{Nuance}
\label{sec:nuance}

