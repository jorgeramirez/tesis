\section{Aplicaciones}
\label{sec:aplicaciones}

% introduccion

Las soluciones de reconocimiento del habla, catalogadas como aplicaciones, propocionan
a los usuarios finales los medios necesarios para realizar determinadas tareas mediante
la voz: dictado autom\'atico, navegaci\'on de interfaces, etc\'etera; sin
la necesidad de tener un conocimiento previo de los conceptos relacionados al
reconocimiento del habla. \'Estas aplicaciones presentan las siguientes 
caracter\'isticas en cuanto a los criterios generales de evaluaci\'on:

\begin{itemize}
    \item Conocimiento t\'ecnico: requieren de poco conocimiento t\'ecnico debido a que \'estas
        soluciones se ofrecen como un producto a los usuarios finales, por lo tanto, sus funcionalidades
        pueden utilizarse directamente sin necesidad de entender los componentes que intervienen
         en el sistema.
    \item Productividad: teniendo en cuenta el criterio anterior, cabe resaltar que con
        las aplicaciones se puede lograr un buen grado de productividad, porque los esfuerzos
        de los usuarios se enfocan en realizar determinadas tareas utilizando las funcionalidades
         disponibles.
    \item Flexibilidad: la flexibilidad para esta categor\'ia de soluciones es reducida, porque las
        funcionalidades que ofrece cada aplicaci\'on se encuentran orientadas a resolver tareas
        espec\'ificas: dictado autom\'atico, para la transcripci\'on de documentos; reconocimiento
        de comandos, para navegar interfaces; reconocimiento del habla, para 
        automatizaci\'on de servicios de operador. Cada aplicaci\'on tiene un \'ambito de aplicabilidad
        lo cual impone un l\'imite en su flexibilidad, en comparaci\'on a otras categor\'ias que
         se ver\'an m\'as adelante.
\end{itemize}

\subsection{Criterios Espec\'ificos de Evaluaci\'on}

Los criterios espec\'ificos nos permiten evaluar a las aplicaciones en cuanto a factores
particulares y relevantes para esta categor\'ia. A continuaci\'on se presentan los criterios 
espec\'ificos de evaluaci\'on para las aplicaciones:

\begin{itemize}
    \item Precio
    \item Soporte para m\'ultiples idiomas
    \item Facilidad de configuraci\'on
    \item Soporte para dispositivos m\'oviles
\end{itemize}

\subsection{Ejemplos de Aplicaciones}

\subsubsection{Simon}
\label{sec:simon}

\emph{Simon} es un programa de reconocimiento del habla de c\'odigo abierto que tiene como objetivo
reemplazar el teclado y mouse \cite{SimonListen}. Esta dise\~nado para ser muy flexible y permite la
personalizaci\'on para cualquier aplicaci\'on que necesite incorporar reconocimiento
del habla. Adem\'as, \'este programa pretende ayudar a personas con alguna discapacidad, debido
a que busca brindarles la posibilidad de escribir correos electr\'onicos, navegar internet, etc.

\begin{itemize}
    \item \emph{Precio:} esta herramienta es gratuita y de c\'odigo abierto.
    \item \emph{Soporte para m\'ultiples idiomas:} \emph{Simon} es una herramienta independiente del idioma,
    debido a que permite incorporar nuevos modelos estad\'isticos. Si se desea
	utilizar el espa\~nol, por ejemplo, simplemente hay que incorporar el modelo estad\'istico que representa
    a los sonidos que forman las palabras de dicho idioma.
    \item \emph{Facilidad de configuraci\'on:} esta herramienta est\'a dise\~nada para que sea
	f\'acilmente configurable desde su interfaz gr\'afica. Utiliza el concepto de Escenarios para identificar a un contexto (aplicaci\'on)
	que se desea operar a trav\'es de la voz. Brinda, por defecto, un modelo base para el idioma ingl\'es (el
    modelo prove\'ido por Voxforge\cite{Voxforge}), pero permite incorporar otros modelos e incluso generar nuevos.
    \item \emph{Soporte para dispositivos m\'oviles:} \emph{Simone} es un cliente de Simon para dispositivos m\'oviles. Aunque las
	capacidades de este cliente son limitadas, actualmente permite: controlar una computadora a trav\'es de una red
	local inal\'ambrica (utilizar el tel\'efono como micr\'ofono), realizar llamadas a contactos y navegar el software. 
    La versi\'on inicial de \emph{Simone} se desarroll\'o para 
	Meego\footnote{Meego fue un sistema operativo de c\'odigo abierto que result\'o de la uni\'on de
	los sistemas operativos \emph{Maemo} de Nokia y \emph{Moblin} de Intel, con la intenci\'on de competir
	con el sistema operativo \emph{Android} de Google.}, aunque el soporte para Android se encuentra en desarrollo.

\end{itemize}

\subsubsection{Palaver}
\label{sec:palaver}

\emph{Palaver} es una aplicaci\'on, actualmente disponible solo para Linux orientada principalmente
al control de funcionalidades de las computadoras a trav\'es de la voz, aunque tambi\'en
puede utilizarse para transcripci\'on \cite{Palaver}. \emph{Palaver} utiliza un diccionario para
definir los comandos que soporta y, adem\'as, permite definir nuevos comandos a trav\'es
de un sistema de \foreign{plugins}.

\begin{itemize}
    \item \emph{Precio:} esta herramienta es gratuita y de c\'odigo abierto.
    \item \emph{Soporte para m\'ultiples idiomas:} esta herramienta delega la interpretaci\'on de
	la voz a los servidores de Google, por lo tanto, es posible utilizar \emph{Palaver} en otros idiomas.
    Sin embargo, esto implica la modificaci\'on del diccionario de comandos, ya que por 
    defecto \emph{Palaver} provee un diccionario para el idioma ingl\'es. Adem\'as, al delegar el
    reconocimiento a servidores externos, es necesario tener conexi\'on a internet para utilizar
    esta herramienta.
    \item \emph{Facilidad de configuraci\'on:} esta herramienta permite extender los comandos soportados a 
    trav\'es de un sistema de plugins, adem\'as es posible configurar otros aspectos de la aplicaci\'on
    (como el idioma). 
    Sin embargo, no existe una interfaz gr\'afica para realizar las configuraciones, lo que reduce en
    cierta medida la facilidad de configuraci\'on.
    \item \emph{Soporte para dispositivos m\'oviles:} actualmente esta herramienta no ofrece soporte
    para dispositivos m\'oviles.
\end{itemize}

\subsubsection{Dragon Naturally Speaking}
\label{sec:nuance}

\foreign{Dragon Naturally Speaking} es una aplicaci\'on de la empresa \emph{Nuance Communications} 
que permite interactuar con las computadoras a trav\'es de la voz \cite{DragonNaturallySpeaking}. Permite
crear documentos, escribir correos electr\'onicos, lanzar aplicaciones, abrir archivos, controlar
el mouse, etc. Esta herramienta presenta una interfaz de usuario minimalista y posee tres
\'areas principales de aplicaci\'on: dictado, s\'intesis de habla y reconocimiento de comandos. 
El usuario puede dictar y obtener la transcripci\'on del habla en forma de texto escrito, obtener
una versi\'on sintetizada de un documento como un flujo de audio, y emitir comandos de voz que
son reconocidos por el sistema.

\begin{itemize}
    \item \emph{Precio:} esta herramienta viene en varias ediciones, la \foreign{Home Edition} se puede
        obtener desde \$99.99 y la versi\'on \foreign{Premium} desde \$199.99
    \item \emph{Soporte para m\'ultiples idiomas:} este programa soporta los siguientes idiomas: Ingl\'es, Franc\'es,
	Alem\'an, Espa\~nol, Italiano, Holand\'es. Adem\'as cabe destacar que esta herramienta soporta 
    m\'ultiples acentos.
    \item \emph{Facilidad de configuraci\'on:} esta herramienta presenta una interfaz minimalista a 
    trav\'es de la cual se puede modificar la configuraci\'on del sistema.
    \item \emph{Soporte para dispositivos m\'oviles:} \foreign{Dragon Remote Mic} es una aplicaci\'on para 
    \emph{iOS} y \emph{Android} que permite convertir un tel\'efono celular en un microfono inal\'ambrico 
    para el sistema. Adem\'as, la empresa \foreign{Nuance Communications} proporciona otros productos, 
    derivados de \foreign{Dragon}, orientados a dispositivos m\'oviles.
\end{itemize}

