\section{Aplicaciones}
\label{sec:aplicaciones}

Las aplicaciones son soluciones de reconocimiento del habla que propocionan
a los usuarios finales los medios necesarios para realizar determinadas tareas mediante
la voz: dictado autom\'atico, navegaci\'on de interfaces, etc\'etera; sin
la necesidad de tener un conocimiento previo de los conceptos relacionados al
reconocimiento del habla. Las aplicaciones presentan las siguientes 
caracter\'isticas en cuanto a los criterios generales de evaluaci\'on:

\begin{itemize}
    \item Conocimiento t\'ecnico: requieren de poco conocimiento  t\'ecnico debido a que estas
        soluciones se ofrecen como un producto a los usuarios finales, por lo tanto, sus funcionalidades
        pueden utilizarse directamente sin necesidad de conocer los detalles t\'ecnicos de
        los componentes que intervienen en el sistema.
    \item Productividad: desde la perspectiva del usuario final y teniendo en cuenta el criterio anterior, 
        cabe resaltar que con
        las aplicaciones se puede lograr un alto grado de productividad, porque los esfuerzos
        de los usuarios se enfocan en realizar determinadas tareas utilizando las funcionalidades
        disponibles.
    \item Flexibilidad : la flexibilidad para esta categor\'ia de soluciones es reducida, porque las
        funcionalidades que ofrece cada aplicaci\'on se encuentran orientadas a resolver tareas
        espec\'ificas: dictado autom\'atico, para la transcripci\'on de documentos; reconocimiento
        de comandos, etc. Cada aplicaci\'on tiene un \'ambito de aplicabilidad
        lo cual impone un l\'imite en cuanto a su flexibilidad, en comparaci\'on a otras categor\'ias que
        se analizarán m\'as adelante.
\end{itemize}

\subsection{Criterios Espec\'ificos de Evaluaci\'on}

Los criterios espec\'ificos nos permiten evaluar a las aplicaciones con respecto a factores
particulares y relevantes para esta categor\'ia. Los criterios 
espec\'ificos de evaluaci\'on para las aplicaciones son:

\begin{itemize}
    \item Precio o Inversi\'on econ\'omica
    \item Soporte para m\'ultiples idiomas
    \item Facilidad de configuraci\'on
    \item Soporte para dispositivos m\'oviles
\end{itemize}

\subsection{Ejemplos de Aplicaciones}

\subsection{Simon}
\label{sec:simon}

\subsubsection{Palaver}
\label{sec:palaver}

\subsection{Nuance}
\label{sec:nuance}

