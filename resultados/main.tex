%!TEX root = ../tesis.tex
\chapter{Resultados Obtenidos}
\label{sec:resultados}


% introduccion
Las pruebas con usuarios, descritas en el cap\'itulo anterior, arrojaron una serie de
resultados interesantes que ser\'an presentados en este cap\'itulo para su posterior an\'alisis. Los
distintos valores obtenidos para las variables consideradas ser\'an presentados de manera tabular, dichos
valores ser\'an analizados para identificar correlaciones.

Una de las primeras actividades de la prueba de usabilidad fue el Test de Memoria, en la tabla~\ref{sec:tabla-memoria}
se puede ver el desempeño de cada sujeto en esta actividad.

\begin{table}[H]
\centering
\footnotesize
\begin{tabular}{|p{1.6cm}|p{1.6cm}|}
\hline
    Sujeto & Memoria \\
    \hline 
    1 & 14 \\
    2 & 10 \\
    3 & 14 \\
    4 & 11 \\
    5 & 8 \\
    6 & 14 \\
    7 & 13 \\
    8 & 12 \\
    9 & 14 \\
    10 & 14 \\
    11 & 15 \\
    12 & 14 \\
\hline
\end{tabular}
\caption{Resumen del Test de Memoria}
\label{sec:tabla-memoria}
\end{table}

Como se mencion\'o anteriormente, 12 a 13 palabras recordadas es el promedio esperado. Luego del Test de Memoria
cada sujeto realiz\'o dos tareas simple de manera tal a poder llevar a la pr\'actica lo aprendido en el entrenamiento,
como se explic\'o en el cap\'itulo anterior. A continuaci\'on la tabla~\ref{sec:tabla-T1} presenta el tiempo $T1$ de
cada sujeto, siendo $T1$ la suma del tiempo de la tarea uno y dos (en minutos).

\begin{table}[H]
\centering
\footnotesize
\begin{tabular}{|p{1.6cm}|p{1.6cm}|}
\hline
    Sujeto & $T1$ \\
    \hline 
    1 & 10,65 \\
    2 & 19,68 \\
    3 & 10,88 \\
    4 & 16,02 \\
    5 & 18,97 \\
    6 & 14,6 \\
    7 & 11,5 \\
    8 & 7,02 \\
    9 & 15,27 \\
    10 & 7,15 \\
    11 & 12,92 \\
    12 & 21,42 \\
\hline
\end{tabular}
\caption{Resumen del tiempo $T1$ de cada sujeto}
\label{sec:tabla-T1}
\end{table}

La tabla~\ref{sec:tabla-t1-memoria} muestra a las tablas~\ref{sec:tabla-memoria} y~\ref{sec:tabla-T1} en una
sola, y as\'i ver si podr\'ia llegar a existir correlaci\'on entre estas variables. La existencia 
de correlaci\'on permitir\'a corroborar si una buena memoria ayuda a un buen tiempo $T1$.

\begin{table}[H]
\centering
\footnotesize
\begin{tabular}{|p{1.6cm}|p{1.6cm}|p{1.6cm}|}
\hline
    Sujeto & Memoria & $T1$ \\
    \hline 
    1 & 14 & 10,65 \\
    2 & 10 & 19,68 \\
    3 & 14 & 10,88 \\
    4 & 11 & 16,02 \\
    5 & 8 & 18,97 \\
    6 & 14 & 14,6 \\
    7 & 13 & 11,5 \\
    8 & 12 & 7,02 \\
    9 & 14 & 15,27 \\
    10 & 14 & 7,15 \\
    11 & 15 & 12,92 \\
    12 & 14 & 21,42 \\
\hline
\end{tabular}
\caption{Resumen del Test de memoria y tiempo $T1$ de cada sujeto}
\label{sec:tabla-t1-memoria}
\end{table}

Una vez terminadas las dos tareas iniciales cada sujeto realiz\'o dos tareas m\'as sin ayuda del facilitador. A
continuaci\'on la tabla~\ref{sec:tabla-t2-memoria} presenta el tiempo $T2$ de cada sujeto, siendo $T2$ 
la suma del tiempo de la tarea tres y cuatro (en minutos). Adem\'as se vuelve a mostrar el desempeño del sujeto
en el Test de Memoria por el mismo motivo expuesto anteriormente.

\begin{table}[H]
\centering
\footnotesize
\begin{tabular}{|p{1.6cm}|p{1.6cm}|p{1.6cm}|}
\hline
    Sujeto & Memoria & T2 \\
    \hline 
    1 & 14 &  17,43 \\
    2 & 10 &  24,03  \\
    3 & 14 &  13,52 \\
    4 & 11 &  33,78 \\
    5 & 8 &   23,77 \\
    6 & 14 &  18,23 \\
    7 & 13 &  9,97 \\
    8 & 12 &  11,78 \\
    9 & 14 &  18,4    \\
    10 & 14 & 9,1 \\
    11 & 15 & 18  \\
    12 & 14 & 22,27 \\
\hline
\end{tabular}
\caption{Resumen del Tiempo $T2$ y Test de memoria de cada sujeto}
\label{sec:tabla-t2-memoria}
\end{table}

En base a la interacci\'on del sujeto con el sistema de reconocimiento en las tareas tres y cuatro, se pudieron
extraer las siguientes m\'etricas: Tasa de Aciertos (TA), Cantidad de Errores (CE), Tasa de Error Humano (EH) y
correctitud de la Tarea Cuatro (C). Los valores para estas m\'etricas pueden observarse en la tabla~\ref{sec:tabla-metricas-t4}

\begin{table}[H]
\centering
\footnotesize
\begin{tabular}{|p{1.6cm}|p{1.6cm}|p{1.6cm}|p{1.6cm}|p{1.6cm}|p{1.6cm}|}
\hline
    Sujeto & T2 & TA   &  EH  & CE &  C\\
    \hline 
    1  & 17,43 & 58,25 & 2,20  & 4  & 58,33 \\
    2  & 24,03 & 70,14 & 6,92  & 19 & 91,67 \\
    3  & 13,52 & 93,73 & 2,75  & 9  & 94,44 \\
    4  & 33,78 & 66,73 & 3,35  & 6  & 69,44 \\
    5  & 23,77 & 93,35 & 8,92  & 22 & 100   \\
    6  & 18,23 & 80,51 & 3,46  & 4  & 91,67 \\
    7  & 9,97  & 87,62 & 4,98  & 6  & 100   \\
    8  & 11,78 & 90,45 & 7,48  & 6  & 100   \\
    9  & 18,4  & 93,49 & 13,14 & 21 & 100   \\
    10 & 9,1   & 87,64 & 5,12  & 7  & 70,83 \\
    11 & 18    & 89,32 & 1,22  & 6  & 100   \\
    12 & 22,27 & 94,42 & 9,83  & 25 & 73,61 \\
\hline
\end{tabular}
\caption{Resumen de las m\'etricas relacionadas a las tareas 3 y 4}
\label{sec:tabla-metricas-t4}
\end{table}


Luego de haber finalizado la tarea cuatro, cada sujeto complet\'o una encuesta de manera tal a
opinar sobre su experiencia interactuando con el sistema de reconocimiento de voz. En la tabla~\ref{sec:tabla-encuesta}
se puede observar un resumen de la encuesta realizada.


\begin{table}[H] 
\centering
\footnotesize
\begin{tabular}{|r|r|r|r|r|}
\hline
    Sujeto & Palabras & Comandos & Entrenamiento & Interfaz por Voz \\
    \hline
    1 & 6 & 7 & 6 & 7 \\
    2 & 6 & 7 & 7 & 7 \\
    3 & 7 & 7 & 7 & 7 \\
    4 & 6 & 6 & 4 & 5 \\
    5 & 7 & 7 & 7 & 4 \\
    6 & 6 & 6 & 5 & 5 \\
    7 & 7 & 7 & 7 & 5 \\
    8 & 6 & 7 & 6 & 7  \\
    9 & 6 & 7 & 7 & 5  \\
    10 & 5 & 6 & 6 & 6  \\
    11 & 6 & 6 & 6 & 7  \\
    12 & 6 & 6 & 7 & 5  \\
\hline
\end{tabular}
\caption{Resumen de la encuesta realizada.}
\label{sec:tabla-encuesta}
\end{table}

Finalmente, se presenta la tabla que resume la prueba de usabilidad realizada. En la tabla~\ref{sec:tabla-resumen-prueba}
se pueden observar todas las m\'etricas juntas, la cual sirve como entrada para el analisis de correlaci\'on a realizar.

\begin{table}[H]
\centering
\footnotesize
\begin{tabular}{|p{1.6cm}|p{1.6cm}|p{1.6cm}|p{1.6cm}|p{1.6cm}|p{1.6cm}|p{1.6cm}|p{1.6cm}|}
\hline
    Sujeto & TA    &    EH & T1    & T2  & CE &  C    & Memoria \\
    \hline 
    1      & 58,25 & 2,20  & 10,65 & 17,43 & 4  & 58,33 & 14 \\
    2      & 70,14 & 6,92  & 19,68 & 24,03 & 19 & 91,67 & 10 \\
    3      & 93,73 & 2,75  & 10,88 & 13,52 & 9  & 94,44 & 14 \\
    4      & 66,73 & 3,35  & 16,02 & 33,78 & 6  & 69,44 & 11 \\
    5      & 93,35 & 8,92  & 18,97 & 23,77 & 22 & 100   & 8  \\
    6      & 80,51 & 3,46  & 14,6  & 18,23 & 4  & 91,67 & 14 \\
    7      & 87,62 & 4,98  &  11,5 & 9,97  & 6  & 100   & 13 \\
    8      & 90,45 & 7,48  & 7,02  & 11,78 & 6  & 100   & 12 \\
    9      & 93,49 & 13,14 & 15,27 & 18,4  & 21 & 100   & 14 \\
    10     & 87,64 & 5,12  & 7,15  & 9,1   & 7  & 70,83 & 14 \\
    11     & 89,32 & 1,22  & 12,92 & 18    & 6  & 100   & 15 \\
    12     & 94,42 & 9,83  & 21,42 & 22,27 & 25 & 73,61 & 14 \\
\hline
\end{tabular}
\caption{Resumen de los valores de las métricas tenidas en cuenta.}
\label{sec:tabla-resumen-prueba}
\end{table}

A partir de los valores en la tabla~\ref{sec:tabla-resumen-prueba} y utilizando el \emph{Coeficiente de 
Correlaci\'on de Pearson}\cite{BoslaughStatistics2008} se llev\'o a cabo un an\'alisis  para identificar posibles
correlaciones entre las variables consideradas. El coeficiente de Pearson es una medida del grado de 
correlaci\'on lineal o dependencia entre dos variables $X$ e $Y$. El valor del coeficiente se encuentra entre
-1 y 1 inclusive. El valor -1 indica que las variables est\'an correlacionadas negativamente 
(cuando $X$ crece, $Y$ decrece y viceversa), 0 indica que no existe correlaci\'on y 1 que existe una correlaci\'on
positiva (cuando $X$ crece, $Y$ crece).

En la tabla que se muestra a continuaci\'on se pueden visualizar los coeficientes de correlaci\'on entre las m\'etricas estudiadas.

\begin{table}[H]
\centering
\footnotesize
\begin{tabular}{|p{1.6cm}|p{1.6cm}|p{1.6cm}|p{1.6cm}|p{1.6cm}|p{1.6cm}|p{1.6cm}|p{1.6cm}|}
\hline
          &    TA & EH    & T1     & T2    & CE    & C      & Memoria \\
\hline
TA        & 1     & 0.51  & 0.14   & -0.09 & 0.71  & 0.50   & 0.22  \\
EH        & 0.51  & 1     & 0.40   & 0.23  & 0.72  & 0.29   & -0.38 \\
T1        & 0.14  & 0.40  & 1      & 0.87  & 0.58  & -0.04  & -0.31 \\
T2        & -0.09 & 0.23  & 0.87   & 1     & 0.36  & -0.16  & -0.42 \\
CE        & 0.71  & 0.72  & 0.58   & 0.36  & 1     & 0.23   & -0.23 \\
C         & 0.50  & 0.29  & -0.036 & -0.16 & 0.23  & 1      & -0.08 \\
Memoria   & 0.22  & -0.38 & -0.31  & -0.42 & -0.23 & -0.08  & 1     \\
\hline
\end{tabular}
\caption{Coeficientes de correlaci\'on para las m\'etricas consideradas.}
\label{sec:tabla-correlacion}
\end{table}

Como se puede observar, existen correlaciones entre alguna de las m\'etricas presentadas, 
siendo algunas positivas y otras negativas. A continuaci\'on se listan las correlaciones identificadas:

\begin{itemize}
    \item TA y EH, 0.51 correlaci\'on positiva fuerte
    \item TA y CE, 0.71 correlaci\'on positiva muy fuerte
    \item TA y C, 0.50 correlaci\'on positiva fuerte 
    \item EH y T1, 0.40 correlaci\'on positiva fuerte
    \item EH y C, 0.29 correlaci\'on positiva d\'ebil
    \item T1 y T2, 0.87 correlaci\'on positiva muy fuerte
    \item T1 y Memoria, -0.31 correlaci\'on negativa moderada
    \item T1 y CE, 0.58 correlaci\'on positiva fuerte
    \item T2 y CE, 0.36 correlaci\'on positiva moderada
    \item T2 y Memoria, -0.42 correlaci\'on negativa fuerte
    \item CE y Memoria, -0.23 correlaci\'on negativa d\'ebil
\end{itemize}
