%!TEX root = ../tesis.tex
\chapter{Resultados Experimentales}
\label{sec:resultados}


% introduccion
Las pruebas experimentales, descritas en el cap\'itulo anterior, arrojaron una serie de
resultados interesantes que ser\'an presentados en este cap\'itulo para su posterior an\'alisis. Los
distintos valores obtenidos para las variables consideradas ser\'an presentados de manera tabular, dichos
valores ser\'an analizados para identificar correlaciones.

En la tabla~\ref{sec:tabla-encuesta}
se puede observar un resumen de la encuesta realizada a cada sujeto despu\'es de haber realizado el experimento. Por otro lado
la tabla~\ref{sec:tabla-resumen-experimento} muestra los n\'umeros que resumen las pruebas experimentales realizadas donde 
una fila de la tabla hace referencia a un sujeto del experimento y 
las columnas a las m\'etricas consideradas.

\begin{table}[H]
\centering
\footnotesize
\begin{tabular}{|r|r|r|r|r|}
\hline
    Sujeto & Palabras & Comandos & Entrenamiento & Interfaz por Voz \\
\hline
    1 & 6 & 7 & 6 & 7 \\
    2 & 6 & 7 & 7 & 7 \\
    3 & 7 & 7 & 7 & 7 \\
    4 & 6 & 6 & 4 & 5 \\
    5 & 7 & 7 & 7 & 4 \\
    6 & 6 & 6 & 5 & 5 \\
    7 & 7 & 7 & 7 & 5 \\
    8 & 6 & 7 & 6 & 7  \\
    9 & 6 & 7 & 7 & 5  \\
    10 & 5 & 6 & 6 & 6  \\
    11 & 6 & 6 & 6 & 7  \\
    12 & 6 & 6 & 7 & 5  \\
\hline
\end{tabular}
\caption{Resumen de la encuesta realizada despu\'es del experimento}
\label{sec:tabla-encuesta}
\end{table}

\begin{table}[H]
\centering
\footnotesize
\begin{tabular}{|p{1.6cm}|p{1.6cm}|p{1.6cm}|p{1.6cm}|p{1.6cm}|p{1.6cm}|p{1.6cm}|}
\hline
    Tasa de acierto & \% Error Humano & T1 + T2 & T3 + T4 & Cantidad de Errores & Correctitud T4 & Memoria \\
    \hline
    58,45 & 0 & 10,65 & 17,43 & 4 & 58,33 & 14 \\
    70,14 & 6,92 & 19,68 & 24,03 & 19 & 91,67 & 10 \\
    93,73 & 2,75 & 10,88 & 13,52 & 9 & 94,44 & 14 \\
    66,73 & 0 & 16,02 & 33,78 & 6 & 69,44 & 11 \\
    93,35 & 8,92 & 18,97 & 23,77 & 22 & 100 & 8 \\
    80,51 & 3,46 & 14,6 & 18,23 & 4 & 91,67 & 14 \\
    87,62 & 4,98  &  11,5 & 9,97 & 6 & 100 & 13 \\
    90,45 & 7,48 & 7,02 & 11,78 & 6 & 100 & 12 \\
    93,49 & 13,14 & 15,27 & 18,4 & 21 & 100 & 14 \\
    87,64 & 5,12 & 7,15 & 9,1 & 7 & 70,83 & 14 \\
    89,32 & 1,22 & 12,92 & 18 & 6 & 100 & 15 \\
    94,42 & 9,83 & 21,42 & 22,27 & 25 & 73,61 & 14 \\
\hline
\end{tabular}
\caption{Resumen de los valores de las variables consideradas en el experimento}
\label{sec:tabla-resumen-experimento}
\end{table}

A partir de los valores en la tabla~\ref{sec:tabla-resumen-experimento} y utilizando el \emph{Coeficiente de Correlaci\'on de Pearson}\cite{BoslaughStatistics2008} se llev\'o a cabo un an\'alisis  para
identificar correlaciones entre las variables consideradas en el experimento. El coeficiente de Pearson es una medida
del grado de correlaci\'on lineal o dependencia entre dos variables $X$ e $Y$. El valor del coeficiente se encuentra entre
-1 y 1 inclusive. El valor -1 indica que las variables est\'an correlacionadas negativamente (cuando $X$ crece, $Y$ decrece y viceversa),
0 indica que no existe correlaci\'on y 1 que existe una correlaci\'on positiva (cuando $X$ crece, $Y$ crece)

En la tabla que se muestran a continuaci\'on se pueden visualizar los coeficientes de correlaci\'on entre las variables estudiadas.

\begin{table}[H]
\centering
\footnotesize
\begin{tabular}{|p{1.8cm}|p{1.6cm}|p{1.6cm}|p{1.6cm}|p{1.6cm}|p{1.6cm}|p{1.6cm}|p{1.6cm}|}
\hline
                    &    Tasa de Acierto & \% Error Humano & T1 + T2 & T3 + T4 & Cantidad de Errores & Correctitud T4 & Memoria \\
\hline
\% Tasa de Acierto     & 1 & 0.65 & 0.14 & -0.09 & 0.71 & 0.50 & 0.22 \\
\% Error Humano        & 0.65 & 1 & 0.34 & 0.12 & 0.75 & 0.45 & -0.23 \\
T1 + T2                & 0.14 & 0.34 & 1 & 0.87 & 0.58 & -0.04 & -0.31 \\
T3 + T4                & -0.09 & 0.12 & 0.87 & 1 & 0.36 & -0.12 & -0.42 \\
Cantidad de Errores    & 0.71 & 0.75 & 0.58 & 0.36 & 1 & 0.23 & -0.23 \\
Correctitud T4         & 0.50 & 0.45 & -0.036 & -0.16 & 0.23 & 1 & -0.08 \\
Memoria                & 0.22 & -0.23 & -0.31 & -0.42 & -0.23 & -0.08 & 1 \\
\hline
\end{tabular}
\caption{Coeficientes de correlaci\'on para las variables del experimento}
\label{sec:tabla-correlacion}
\end{table}

Como se puede observar, existen correlaciones entre alguna de las variables presentadas, siendo algunas positivas y otras negativas. A 
continuaci\'on se listan las correlaciones identificadas:

\begin{itemize}
    \item Tasa de Acierto y \% Error Humano, 0.49 correlaci\'on positiva fuerte
    \item Tasa de Acierto y Cantidad de Errores, 0.71 correlaci\'on positiva muy fuerte
    \item Tasa de Acierto y Correctitud T4, 0.50 correlaci\'on positiva fuerte 
    \item \% de Error Humano y T1 + T2, 0.34 correlaci\'on positiva moderada
    \item \% de Error Humano y Correctitud T4, 0.45 correlaci\'on positiva fuerte
    \item T1 + T2 y T3 + T4, 0.87 correlaci\'on positiva muy fuerte
    \item T1 + T2 y Memoria, -0.31 correlaci\'on negativa moderada
    \item T1 + T2 y Cantidad de Errores, 0.58 correlaci\'on positiva fuerte
    \item T3 + T4 y Cantidad de Errores, 0.36 correlaci\'on positiva moderada
    \item T3 + T4 y Memoria, -0.42 correlaci\'on negativa fuerte
    \item Cantidad de Errores y Memoria, -0.23 correlaci\'on negativa d\'ebil
\end{itemize}
