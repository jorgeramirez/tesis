%!TEX root = ../tesis.tex
\chapter{Resultados Obtenidos}
\label{sec:resultados}


% introduccion
Las pruebas con usuarios, descritas en el cap\'itulo anterior, arrojaron una serie de
resultados interesantes que ser\'an presentados en este cap\'itulo para su posterior an\'alisis. Los
distintos valores obtenidos para las variables consideradas ser\'an presentados de manera tabular, dichos
valores ser\'an analizados para identificar correlaciones.


\subsubsection{Test de Memoria}

Una de las primeras actividades de la prueba de usabilidad fue el Test de Memoria ($M$), en la tabla~\ref{sec:tabla-memoria}
se puede ver el desempeño de cada sujeto en esta actividad.

\begin{table}[H]
\centering
\footnotesize
\begin{tabular}{|p{1.6cm}|p{1.6cm}|}
\hline
    Sujeto & $M$ \\
    \hline 
    1 & 14 \\
    2 & 10 \\
    3 & 14 \\
    4 & 11 \\
    5 & 8 \\
    6 & 14 \\
    7 & 13 \\
    8 & 12 \\
    9 & 14 \\
    10 & 14 \\
    11 & 15 \\
    12 & 14 \\
\hline
\end{tabular}
\caption{Resumen del Test de Memoria}
\label{sec:tabla-memoria}
\end{table}

Como se mencion\'o anteriormente, 12 a 13 palabras recordadas es el promedio esperado. 

\subsubsection{Entrenamiento}

Luego del Test de Memoria cada sujeto realiz\'o dos tareas simples de manera tal a poder 
llevar a la pr\'actica los conocimientos te\'oricos aprendidos.
A continuaci\'on la tabla~\ref{sec:tabla-t1-memoria} presenta el tiempo $T_1$ de
cada sujeto, siendo $T_1$ la suma del tiempo de la tarea uno y dos (en minutos), adem\'as se muestra
Memoria del sujeto y as\'i ver si podr\'ia llegar a existir correlaci\'on entre estas variables. La existencia 
de correlaci\'on permitir\'a corroborar si una buena memoria ayuda a un buen tiempo $T_1$.

\begin{table}[H]
\centering
\footnotesize
\begin{tabular}{|p{1.6cm}|p{1.6cm}|p{1.6cm}|}
\hline
    Sujeto & $M$ & $T_1$ \\
    \hline 
    1 & 14 & 10,65 \\
    2 & 10 & 19,68 \\
    3 & 14 & 10,88 \\
    4 & 11 & 16,02 \\
    5 & 8 & 18,97 \\
    6 & 14 & 14,6 \\
    7 & 13 & 11,5 \\
    8 & 12 & 7,02 \\
    9 & 14 & 15,27 \\
    10 & 14 & 7,15 \\
    11 & 15 & 12,92 \\
    12 & 14 & 21,42 \\
\hline
\end{tabular}
\caption{Resumen del Test de memoria y tiempo $T1$ de cada sujeto}
\label{sec:tabla-t1-memoria}
\end{table}

Una vez terminadas las dos tareas iniciales cada sujeto realiz\'o dos tareas m\'as sin ayuda del facilitador.

\subsubsection{Tarea Tres}

La tarea tres es la primera de las dos tareas que el sujeto realiz\'o sin asistencia del facilitador, como
se explic\'o anteriormente, el sujeto tuvo la posibilidad de consultar el manual durante esta actividad.
A continuaci\'on la tabla~\ref{sec:tabla-tarea3} presenta las distintas variables consideradas con relaci\'on
a la tarea tres: Tasa de Aciertos ($A$), Error de la M\'aquina ($E_1$),  
Tasa de Error Humano ($E_2$), Duraci\'on ($T_3$), Cantidad de Errores ($E_3$) y Cantidad de Comandos Utilizados ($U$).

\begin{table}[H]
\centering
\footnotesize
\begin{tabular}{|p{1.6cm}|p{1.6cm}|p{1.6cm}|p{1.6cm}|p{1.6cm}|p{1.6cm}|p{1.6cm}|p{1.6cm}|}
\hline
    Sujeto &  $A$     & $E_1$    & $E_2$   & $T_3$      & $E_3$ & $M$ & $U$ \\
    \hline 
    1      & 53,42  & 46,58  & 2,83 & 0:10:39 & 4  & 14      &  28 \\
    2      & 68,46  & 31,54  & 5,96 & 0:10:28 & 11 & 10      &  26 \\
    3      & 90,99  & 9,01   & 4,11 & 0:07:14 & 9  & 14      &  28 \\
    4      & 66,84  & 33,16  & 0,81 & 0:16:52 & 1  & 11      &  31 \\
    5      & 93,47  & 6,53   & 3,08 & 0:12:53 & 3  & 8       &  30 \\
    6      & 81,17  & 18,83  & 3,06 & 0:09:04 & 3  & 14      &  30 \\
    7      & 86,86  & 13,14  & 6,48 & 0:05:04 & 4  & 13      &  27 \\
    8      & 95,15  & 4,85   & 5,17 & 0:05:00 & 2  & 12      &  29 \\
    9      & 95,61  & 4,39   & 9,23 & 0:12:31 & 9  & 14      &  28 \\
    10     & 87,32  & 12,68  & 6,51 & 0:05:26 & 8  & 14      &  32 \\
    11     & 84,70  & 15,30  & 1,79 & 0:10:58 & 5  & 15      &  28 \\
    12     & 95,53  & 4,47   & 9,55 & 0:12:05 & 18 & 14      &  37 \\
\hline
\end{tabular}
\caption{Resumen de las variables relacionadas con la tarea tres}
\label{sec:tabla-tarea3}
\end{table}

\subsubsection{Tarea Cuatro}

La tarea cuatro es la \'ultima tarea realizada por el sujeto. Como se mencion\'o anteriormente, es la tarea
de mayor dificultad ya que el sujeto la realiza sin ning\'un tipo de asistencia. A continuaci\'on se presenta
la tabla~\ref{sec:tabla-tarea4} que resume la tarea cuatro, incluyendo las variables presentadas en la tarea tres y, adem\'as, se
agrega la variable que mide la correctitud de la tarea cuatro ($C$)

\begin{table}[H]
\centering
\footnotesize
\begin{tabular}{|p{1.4cm}|p{1.4cm}|p{1.4cm}|p{1.4cm}|p{1.4cm}|p{1.4cm}|p{1.4cm}|p{1.4cm}|p{1.4cm}|}
\hline
Sujeto& $A$ & $E_1$ & $E_2$  & $T_4$      & $E_3$ & $C$ &  $M$ & $U$ \\
 \hline 
1  &  71,06 & 28,94 &  0     &  0:06:47   &  0  &  58,33  & 14 &  20  \\ 
2  &  73,23 & 26,77 &  3,80  &  0:13:34   &  8  &  91,67  & 10 &  22 \\
3  &  93,25 & 6,75  &  2,38  &  0:06:17   &  0  &  94,44  & 14 &  21 \\
4  &  68,07 & 31,93 &  6,05  &  0:16:55   &  5  &  69,44  & 11 &  22 \\
5  &  90,84 & 9,16  &  18,05 &  0:10:53   &  26 &  100      & 8  &  23 \\
6  &  74,38 & 25,62 &  3,51  &  0:09:10   &  1  &  91,67  & 14 &  19  \\
7  &  87,79 & 12,21 &  3,21  &  0:04:54   &  2  &  100      & 13 &  20  \\
8  &  83,87 & 16,13 &  7,47  &  0:06:47   &  4  &  100      & 12 &  22  \\
9  &  91,67 & 8,33  &  15    &  0:05:53   &  12 &  100      & 14 &  20  \\
10 &  88,57 & 11,43 &  2,78  &  0:03:40   &  2  &  70,83  & 14 &  18  \\
11 &  94,17 & 5,83  &  0     &  0:07:02   &  1  &  100      & 15 &  22  \\
12 &  93,43 & 6,57  &  9,92  &  0:10:11   &  7  &  73,61  & 14 &  29  \\
    \hline
\end{tabular}
\caption{Resumen de las variables relacionadas con la tarea cuatro}
\label{sec:tabla-tarea4}
\end{table}


\subsubsection{Resumen}

Finalmente, se presenta la tabla que resume la prueba de usabilidad realizada. En la tabla~\ref{sec:tabla-resumen-prueba}
se pueden observar todas las m\'etricas juntas, la cual sirve como entrada 
para el analisis de correlaci\'on a realizar.

\begin{table}[H]
\centering
\footnotesize
\begin{tabular}{|p{1.2cm}|p{1.2cm}|p{1.2cm}|p{1.2cm}|p{1.2cm}|p{1.2cm}|p{1.2cm}|p{1.2cm}|p{1.2cm}|p{1.2cm}|}
\hline
Sujeto  &   $A$  &   $E_1$ &  $E_2$  &  $T_{1+2}$  & $T_{3+4}$     & $E_3$ & $C$ &  $M$ & $U$ \\
\hline 
1 &  58,25 & 41,75 & 2,20  & 10,65 & 17,43 & 4  & 58,33 & 14 & 36 \\
2 &  70,01 & 29,99 & 6,12  & 19,68 & 24,03 & 19 & 91,67 & 10 & 39 \\
3 &  93,73 & 6,27  & 2,75  & 10,88 & 13,52 & 9  & 94,44 & 14 & 40 \\
4 &  66,35 & 33,65 & 3,27  & 16,02 & 33,78 & 6  & 69,44 & 11 & 41 \\
5 &  93,32 & 6,68  & 9,42  & 18,97 & 23,77 & 29 & 100     & 8  & 43 \\
6 &  79,97 & 20,03 & 3,46  & 14,6  & 18,23 & 4  & 91,67 & 14 & 38 \\
7 &  87,62 & 12,38 & 4,98  & 11,5  & 9,97  & 6  & 100     & 13 & 38 \\
8 &  90,45 & 9,55  & 7,48  & 7,02  & 11,78 & 6  & 100     & 12 & 42 \\
9 &  93,49 & 6,51  & 13,14 & 15,27 & 18,4  & 21 & 100     & 14 & 38 \\
10 & 87,64 & 12,36 & 5,12  & 7,15  & 9,1   & 7  & 70,83 & 14 & 41 \\
11 & 89,32 & 10,68 & 1,22  & 12,92 & 18      & 6  & 100     & 15 & 41 \\
12 & 94,30 & 5,70  & 11,79 & 21,42 & 22,27 & 25 & 73,61 & 14 & 51 \\
\hline
\end{tabular}
\caption{Resumen de las variables de la prueba de usabilidad}
\label{sec:tabla-resumen-prueba}
\end{table}


\subsubsection{Correlaci\'on}

A partir de los valores en la tabla~\ref{sec:tabla-resumen-prueba} y utilizando el \emph{Coeficiente de 
Correlaci\'on de Pearson}\cite{BoslaughStatistics2008} se llev\'o a cabo un an\'alisis  para identificar posibles
correlaciones entre las variables consideradas. El coeficiente de Pearson es una medida del grado de 
correlaci\'on lineal o dependencia entre dos variables $X$ e $Y$. El valor del coeficiente se encuentra entre
-1 y 1 inclusive. El valor -1 indica que las variables est\'an correlacionadas negativamente 
(cuando $X$ crece, $Y$ decrece y viceversa), 0 indica que no existe correlaci\'on y 1 que existe una correlaci\'on
positiva (cuando $X$ crece, $Y$ crece).

En la tabla que se muestra a continuaci\'on se pueden visualizar los coeficientes de correlaci\'on entre las m\'etricas estudiadas.

\begin{table}[H] 
\centering
\footnotesize
\begin{tabular}{|p{1.2cm}|p{1.2cm}|p{1.2cm}|p{1.2cm}|p{1.2cm}|p{1.2cm}|p{1.2cm}|p{1.2cm}|p{1.2cm}|p{1.2cm}|}
\hline
&  $A$  &   $E_1$ &  $E_2$  &  $T_{1+2}$  & $T_{3+4}$     & $E_3$ & $C$ &  $M$ & $U$ \\
\hline
$A$  &  1  &  -1  &  0,51  &  0,14  &  -0,09  &  0,69  &  0,5  &  0,22  &  0,53 \\
$E_1$  &  -1  &  1  &  -0,51  &  -0,14  &  0,09  &  -0,69  &  -0,5  &  -0,22  &  -0,53 \\
$E_2$  &  0,51  &  -0,51  &  1  &  0,4  &  0,23  &  0,71  &  0,29  &  -0,38  &  0,35  \\
$T_{1+2}$  &  0,14  &  -0,14  &  0,4  &  1  &  0,87  &  0,57  &  -0,04  &  -0,31  &  0,25 \\
$T_{3+4}$  &  -0,09  &  0,09  &  0,23  &  0,87  &  1  &  0,37  &  -0,16  &  -0,42  &  0,2 \\
$E_3$  &  0,69  &  -0,69  &  0,71  &  0,57  &  0,37  &  1  &  0,28  &  -0,29  &  0,52 \\
$C$  &  0,5  &  -0,5  &  0,29  &  -0,04  &  -0,16  &  0,28  &  1  &  -0,08  &  0,13 \\
$M$  &  0,22  &  -0,22  &  -0,38  &  -0,31  &  -0,42  &  -0,29  &  -0,08  &  1  &  -0,23 \\
$U$  &  0,53  &  -0,53  &  0,35  &  0,25  &  0,2  &  0,52  &  0,13  &  -0,23  &  1 \\
\hline
\end{tabular}
\caption{Coeficientes de correlaci\'on para las m\'etricas consideradas.}
\label{sec:tabla-correlacion}
\end{table}

Como se puede observar, existen correlaciones entre alguna de las m\'etricas presentadas, 
siendo algunas positivas y otras negativas. A continuaci\'on se listan las correlaciones identificadas:

\begin{itemize}
    \item $A$ y $E_2$, 0,51 correlaci\'on positiva fuerte
    \item $A$ y $E_3$, 0,69 correlaci\'on positiva fuerte
    \item $A$ y $C$, 0,50 correlaci\'on positiva fuerte 
    \item $E_2$ y $T_{1+2}$, 0,40 correlaci\'on positiva fuerte
    \item $E_2$ y $C$, 0,29 correlaci\'on positiva d\'ebil
    \item $T_{1+2}$ y $T_{3+4}$, 0,87 correlaci\'on positiva muy fuerte
    \item $T_{1+2}$ y $M$, -0,31 correlaci\'on negativa moderada
    \item $T_{1+2}$ y $E_3$, 0,57 correlaci\'on positiva fuerte
    \item $T_{3+4}$ y $E_3$, 0,37 correlaci\'on positiva moderada
    \item $T_{3+4}$ y $M$, -0,42 correlaci\'on negativa fuerte
    \item $E_3$ y $M$, -0,29 correlaci\'on negativa d\'ebil
\end{itemize}


Luego de haber finalizado la tarea cuatro, cada sujeto complet\'o una encuesta de manera tal a
opinar sobre su experiencia interactuando con el sistema de reconocimiento de voz. En la tabla~\ref{sec:tabla-encuesta}
se puede observar un resumen de la encuesta realizada.


\begin{table}[H] 
\centering
\footnotesize
\begin{tabular}{|r|r|r|r|r|}
\hline
    Sujeto & Palabras & Comandos & Entrenamiento & Interfaz por Voz \\
    \hline
    1 & 6 & 7 & 6 & 7 \\
    2 & 6 & 7 & 7 & 7 \\
    3 & 7 & 7 & 7 & 7 \\
    4 & 6 & 6 & 4 & 5 \\
    5 & 7 & 7 & 7 & 4 \\
    6 & 6 & 6 & 5 & 5 \\
    7 & 7 & 7 & 7 & 5 \\
    8 & 6 & 7 & 6 & 7  \\
    9 & 6 & 7 & 7 & 5  \\
    10 & 5 & 6 & 6 & 6  \\
    11 & 6 & 6 & 6 & 7  \\
    12 & 6 & 6 & 7 & 5  \\
\hline
\end{tabular}
\caption{Resumen de la encuesta realizada.}
\label{sec:tabla-encuesta}
\end{table}



