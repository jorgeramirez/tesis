%!TEX root = ../tesis.tex
\chapter{Introducci\'on}
\label{sec:intro}

El habla, y m\'as concretamente el lenguaje como medio de comunicaci\'on, es una
de las caracter{\'\i}sticas fundamentales que diferencian al ser humano de los dem\'as
animales y representa un factor clave de su evoluci\'on \cite{SchepartzLanguage1993}. La misma es considerada
el principal modo de comunicaci\'on y la forma m\'as eficiente y natural de intercambio 
de informaci\'on entre seres humanos \cite{GaikwadAReview2010}.

Atendiendo la importancia de la voz y el habla en la comunicaci\'on entre personas,
resulta l\'ogico el inter\'es por desarrollar tecnolog{\'\i}as que permiten una interacci\'on
similar entre una persona y una computadora. Es decir, resulta interesante la idea 
de ``hablar'' con una computadora.

El reconocimiento del habla, tambi\'en conocido como reconocimiento autom\'atico del habla,
es el proceso de convertir una se\~nal de voz en una secuencia de
palabras, mediante un algoritmo implementado program\'aticamente \cite{JaisalAReview2012}. 
Su integraci\'on con interfaces de usuario busca una interacci\'on humano-computadora m\'as
natural, de manera a superar las limitaciones existentes en el modelo convencional
de interacci\'on.

Los asistentes virtuales de Apple \cite{AppleSiri} y Google \cite{GoogleNow}, 
los centros de mando en los autom\'oviles de Ford \cite{FordSync} y Toyota \cite{ToyotaEntune},
y los SmartTV de Samsung \cite{SamsungVoiceControl}; todos ellos capaces de reaccionar a 
comandos por voz del usuario, son ejemplos de la incorporaci\'on de interfaces basadas 
en reconocimiento del habla en actividades de la vida diaria.

Los resultados de los estudios realizados por la respetada consultora tecnol\'ogica
Gartner confirman esta tendencia. Estos ubican al reconocimiento del habla entre
las tecnolog{\'\i}as que se estabilizar\'an, y cuyos beneficios estar\'an ampliamente
demostrados, en los pr\'oximos 2 a 5 a\~nos \cite{Gartner2013}. 

El escenario descrito resulta propicio para la investigaci\'on en el \'area.
Teniendo esto en cuenta, resulta importante contar con un estudio te\'orico
y pr\'actico del reconocimiento del habla, desde sus fundamentos hasta su
aplicaci\'on. El estudio ser{\'\i}a de gran utilidad como material introductorio
para quien quisiese estudiar y aplicar el reconocimiento del habla, pudiendo
servir de base para numerosos trabajos futuros.

Este es el eje central de este trabajo final de grado, cuyos objetivos y organizaci\'on se
exponen a continuaci\'on.

\section{Objetivo General}
\label{sec:objgral}

Realizar un estudio del trasfondo hist\'{o}rico, los fundamentos te\'{o}ricos 
y el estado del arte del reconocimiento del habla de modo a comprender, describir 
e introducir esta \'{a}rea de investigaci\'{o}n.  


\section{Objetivos Espec\'{i}ficos}
\label{sec:objspec}

\begin{itemize}
	\item Presentar y describir los antecedentes hist\'oricos y el estado del arte del reconocimiento
	del habla, teniendo en cuenta las diversas \'areas de aplicaci\'on del mismo.

    \item Analizar, ordenar y caracterizar el proceso t\'{i}pico de un sistema de reconocimiento del habla, 
        incluyendo los aspectos te\'{o}ricos involucrados en cada paso del mismo.

    \item Evaluar y seleccionar las herramientas disponibles que permiten la implementaci\'{o}n de soluciones 
        relacionadas al reconocimiento del habla.
    
    \item Dise\~{n}ar e implementar una interfaz mediante voz del usuario de manera a aplicar y 
    contrastar en la pr\'{a}ctica los conocimientos te\'{o}ricos adquiridos.
    
    \item Evaluar la soluci\'{o}n implementada de modo a obtener datos cuantitativos y cualitativos que 
        permitan extraer conclusiones sobre la aplicabilidad del reconocimiento del habla a las interfaces 
        de usuario.
\end{itemize}

\section{Organizaci\'on del Trabajo}
\label{sec:organizacion}


El presente trabajo est\'a organizado como se describe a continuaci\'on:

\begin{itemize}
	\item En el cap{\'\i}tulo 2 se presenta un breve resumen de los antecedentes hist\'oricos del reconocimiento
	del habla, desde sus inicios hasta la actualidad.
	\item En el cap{\'\i}tulo 3 se presentan varias \'areas de aplicaci\'on del reconocimiento del habla, 
	mencionando avances y algunos trabajos de referencia en cada una.
	\item En el cap{\'\i}tulo 4 se presenta el proceso t\'{i}pico de un sistema de reconocimiento del habla,
	describiendo algunos fundamentos matem\'aticos y algoritmos relacionados a cada paso.
	\item En el cap{\'\i}tulo 5 se presentan algunas medidas de desempe\~no de sistemas de reconocimiento del
	habla, incluy\'endose resultados obtenidos en trabajos de investigaci\'on con respecto a cada una.
	\item En el cap{\'\i}tulo 6 se presentan los resultados de un estudio comparativo de las tecnolog{\'\i}as
	y herramientas disponibles para la implementaci\'on de sistemas de reconocimiento del habla.
	\item En el cap{\'\i}tulo 7 se presenta el problema planteado a modo de llevar a la pr\'actica los
	conocimiento te\'oricos aprendidos.
	\item En el cap{\'\i}tulo 8 se presenta la soluci\'on propuesta para el problema planteado, incluyendo
	las herramientas a utilizar y otros detalles de implementaci\'on.
	\item En el cap{\'\i}tulo 9 se presentan los aspectos relacionados a la evaluaci\'on realizada una vez
	implementada la soluci\'on. Se mencionan los objectivos y se describen la metodolog{\'\i}a utilizada
	y las variables que se tuvieron en cuenta.
	\item En el cap{\'\i}tulo 10 se presentan los resultados obtenidos una vez realizada la evaluaci\'on
	del trabajo.
	\item En el cap{\'\i}tulo 11 se presentan las conclusiones obtenidas a trav\'es de las distintas
	etapas de realizaci\'on del trabajo, desde las investigaciones iniciales hasta la evaluaci\'on.
	\item Finalmente, en el cap{\'\i}tulo 12 se presentan algunas oportunidades posibles para
	la realizaci\'on de trabajos futuros en el \'area.
\end{itemize}