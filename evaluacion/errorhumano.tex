%!TEX root = ../tesis.tex
\section{An\'alisis del Error Humano}
\label{sec:evaluacionError}
Todas las m\'etricas presentadas en la secci\'on anterior se calcularon para cada usuario,
pudiendo promediarse para obtener un valor representativo de la aplicaci\'on.

En el caso particular del error humano, se realiz\'o un estudio m\'as detallado
por considerarse que podr{\'\i}a llevar a obtener conclusiones interesantes. De esta manera,
se calcularon tambi\'en:

\begin{itemize}
	\item Tasa de Error Humano por Longitud del Comando: la longitud de un comando se define
	como la cantidad de palabras que lo componen. La longitud de los comandos disponibles en el 
	lenguaje de TamTam Listens va de 2 a 6 palabras.

	El c\'alculo de la tasa de error se llev\'o a cabo para cada longitud posible.  


	\item Tasa de Error Humano por Nivel Contextual del Comando: el nivel contextual de un comando
	se define de acuerdo al contexto en que se aplica el mismo.
	Para TamTam Listens, se definieron 3 niveles posibles:
		\begin{itemize}
			\item General: comandos que afectan la m\'usica en general. 

			Ejemplo: Crear Nueva M\'usica.
			\item Pista: comandos que dependen de una pista seleccionada para su aplicaci\'on. 

			Ejemplo: Piano en Pista 2.
			\item Comp\'as: comandos que que dependen de una pista y un comp\'as seleccionados 
			para su aplicaci\'on. 

			Ejemplo: Crear Nota Do.
		\end{itemize}
	
		
	El c\'alculo de la tasa de error se llev\'o a cabo para cada nivel contextual posible.
	\item Distribuci\'on del Error Humano con respecto al Tiempo transcurrido: para cada usuario, 
	se dividi\'o $T_{3+4}$ en 10 etapas (de 10\%¸de la duraci\'on cada una).

	Para cada etapa se calcul\'o el porcentaje de errores cometidos, con respecto al total de
	errores cometidos por el usuario ($E_2$).

	La distribuci\'on del error humano con respecto al tiempo transcurrido se obtuvo calculando 
	el promedio de los valores de los usuarios para cada etapa.  
\end{itemize}