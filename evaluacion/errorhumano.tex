%!TEX root = ../tesis.tex
\section{An\'alisis del Error Humano}
\label{sec:evaluacionError}
Todas las m\'etricas presentadas en la secci\'on anterior se calculan para cada usuario,
pudiendo promediarse para obtener un valor representativo de la aplicaci\'on.

En el caso particular del error humano, se realiza un estudio m\'as detallado
por considerarse que podr{\'\i}a llevar a obtener conclusiones importantes. 
De esta manera, se calculan tambi\'en:

\begin{itemize}
	\item Tasa de Error Humano por Longitud del Comando: la longitud de un comando se define
	como la cantidad de palabras que lo componen. La longitud de los comandos disponibles en el 
	lenguaje de TamTam Listens va de 2 a 6 palabras.

	El c\'alculo de la tasa de error se lleva a cabo para cada longitud posible.  


	\item Tasa de Error Humano por Nivel Contextual del Comando: el nivel contextual de un comando
	se define de acuerdo al contexto en que se aplica el mismo.
	Para TamTam Listens, se definieron 3 niveles posibles:
		\begin{itemize}
			\item General: comandos que afectan la m\'usica en general. 

			Ejemplo: Crear Nueva M\'usica.
			\item Pista: comandos que dependen de una pista seleccionada para su aplicaci\'on. 

			Ejemplo: Piano en Pista 2.
			\item Comp\'as: comandos que dependen de una pista y un comp\'as seleccionados 
			para su aplicaci\'on. 

			Ejemplo: Crear Nota Do.
		\end{itemize}
	
		
	El c\'alculo de la tasa de error se lleva a cabo para cada nivel contextual posible.

	\item Tasa de Error Humano por Comando: se calcula la tasa de error para cada uno
	de los comandos pronunciados por los usuarios.

	Esta medida en particular puede ser de gran utilidad para identificar los comandos de la
	aplicaci\'on que resultan problem\'aticos para los usuarios.
	
	\item Distribuci\'on del Error Humano por Etapas de la Sesi\'on: para cada usuario, 
	se divide $T_{3+4}$ en 10 etapas (de 10\% de la duraci\'on cada una).

	Para cada etapa se calcula el porcentaje de errores cometidos, con respecto al total de
	errores cometidos por el usuario ($E_3$).

	La distribuci\'on del error humano por etapas se obtiene calculando 
	el promedio de los valores de los usuarios para cada etapa.
   
\end{itemize}