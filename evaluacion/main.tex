%!TEX root = ../tesis.tex
\chapter{Evaluaci\'on}
\label{sec:evaluacion}


% introduccion
En el cap{\'\i}tulo anterior se describi\'o la soluci\'on propuesta para el estudio de interfaces basadas en
reconocimiento del habla, la cual fue implementada como parte de este trabajo de grado.

Con el fin de realizar una evaluaci\'on tanto de la aplicaci\'on misma como de ciertos conceptos 
e hip\'otesis planteados, se realiza una prueba de usabilidad. 
Como parte de la misma, se miden y analizan factores relacionados a la aplicaci\'on,
al usuario y a la interacci\'on entre ambos.

Este cap{\'\i}tulo presenta los principales aspectos de dise\~no del estudio propuesto: objetivos, metodolog{\'\i}a
y m\'etricas utilizadas. La \'ultima secci\'on describe con mayor detalle el an\'alisis con respecto
a uno de los factores estudiados, el error humano.


%!TEX root = ../tesis.tex
\section{Objetivos}
\label{sec:objetivos-estudio}

La prueba de usabilidad planteada apunta al aprendizaje de cuestiones relacionadas
a la aplicaci\'on implementada y a las interfaces mediante voz en general.

Una prueba con usuarios de la aplicaci\'on implementada puede colaborar al esclarecimiento de 
las interrogantes respecto a la influencia del tama\~no del lenguaje, la longitud de
los comandos, la duraci\'on de la interacci\'on y otros factores caracter{\'\i}sticos de una
interfaz mediante voz, que se presentaron en la secci\'on \ref{sec:problema-general}.

Los objetivos del estudio de usabilidad de \foreign{TamTam Listens} son los siguientes:

\begin{itemize}
	\item Evaluar experimentalmente la aplicaci\'on desarrollada, teniendo en cuenta 
	aspectos t\'ecnicos y funcionales.
	\item Identificar problemas que se presentan al utilizar una interfaz mediante voz y,
	 de ser posible, sugerir posibles soluciones.
	\item Proponer criterios que puedan utilizarse para la evaluaci\'on de interfaces 
	mediante voz.
	\item Verificar y validar la utilidad de los criterios propuestos, de acuerdo a 
	la informaci\'on y las conclusiones que pueden obtenerse mediante los mismos.
	\item Analizar la correlaci\'on entre los factores medidos, de modo a identificar
	posibles relaciones entre los mismos. 
	\item Obtener conclusiones relacionadas a factores externos a la aplicaci\'on, 
	como la memoria del usuario, el tiempo de entrenamiento previo y el tiempo
	total de uso.
\end{itemize}
%!TEX root = ../tesis.tex
\section{Metodolog{\'\i}a}
\label{sec:metodolog{\'\i}a}
A continuaci\'on se describen brevemente distintos aspectos relacionados al
estudio de usabilidad planteado.

\subsection{Muestra}
Las pruebas se realizan con 12 usuarios, estudiantes de Ingenier{\'\i}a Inform\'atica
de la Facultad Polit\'ecnica, Universidad Nacional de Asunci\'on. Ninguno de los usuarios 
ha utilizado anteriormente la aplicaci\'on evaluada, de manera a evitar los efectos
de la experiencia previa.

La cantidad de sujetos de prueba se elige de acuerdo a la recomendaci\'on realizada
en \cite{Hwang:2010} para este tipo de estudios, la cual es de 12. 
Adem\'as, de acuerdo a \cite[p.~267]{Rubin2008}, este n\'umero es el necesario de modo a obtener
resultados estad{\'\i}sticos significativos.

\subsection{Duraci\'on}
Cada sesi\'on de pruebas tiene una duraci\'on aproximada de una hora.
Las sesiones se realizan a lo largo de 2 semanas.

\subsection{Ambiente de Pruebas}
De modo a realizar el estudio en un ambiente controlado, todas las sesiones se
realizan en un Laboratorio de Inform\'atica de la Facultad Poĺit\'ecnica. Esto
permite minimizar los efectos de la interferencia sonora y humana durante el
desarrollo de las pruebas.

\subsection{Roles}
Cada sesi\'on cuenta con 3 participantes, cada uno desempe\~nando uno de los siguientes roles:
	\begin{itemize}
		\item Facilitador: responsable de administrar los distintos pasos del estudio, explicarlos al usuario 
		y aclarar las dudas que se presenten durante la fase inicial. 
		\item Observador: responsable de tomar nota de los eventos que sucedan durante las pruebas, principalmente
		tiempo de inicio y fin de cada paso y otros sucesos que aporten valor al estudio (errores, preguntas, etc.).
		\item Usuario: persona sin conocimientos del sistema. Los detalles del aprendizaje y primeras interacciones
		(tareas) con la aplicaci\'on se registran para su posterior an\'alisis.	
	\end{itemize}

\subsection{Fases}
Cada sesi\'on est\'a compuesta de 5 fases, las cuales se completan en orden secuencial.
A continuaci\'on, se describe brevemente cada una de las fases de la prueba de usabilidad.

\subsubsection{Fase 1: Test de Memoria}
Previamente a la realizaci\'on de las tareas, se lleva a cabo un test de memoria del usuario, basado en
el Test de Aprendizaje Auditivo Verbal de Rey \cite{Lopez1998}. 

Para la prueba, el facilitador lee una lista
de 15 palabras al usuario, luego de lo cual este \'ultimo debe repetir aquellas que recuerda.
El proceso se repite 5 veces en total, registr\'andose las palabras que el usuario recuerda y el orden 
en que lo hace.

\subsubsection{Fase 2: Entrenamiento}
La fase de entrenamiento tiene como objetivo la introducci\'on del usuario a los distintos comandos por voz disponibles
para interactuar con TamTam Listens. 
Para evitar la influencia de las variaciones que pueden presentarse en una
explicaci\'on directa del facilitador al usuario, fue grabada una serie de videos instructivos 
que se presentan al usuario en esta etapa.
Adem\'as, cada usuario recibe una copia del manual de la aplicaci\'on, el cual describe en mayor detalle las
funcionalidades de la aplicaci\'on.

\subsubsection{Fase 3: Tareas}
Cada usuario realiza un total de 4 tareas durante la prueba de usabilidad:
	\begin{itemize}			
		\item Tareas 1 y 2: actividades simples y breves que buscan llevar a la pr\'actica los conceptos 
		aprendidos en los videos instructivos. Se realizan con ayuda del facilitador y el manual de aplicaci\'on.
		\item Tarea 3: actividad de mayor complejidad, la cual se realiza ya sin asistencia del facilitador.
		El usuario puede, sin embargo, consultar el manual de aplicaci\'on.
		\item Tarea 4: actividad m\'as compleja de la sesi\'on, la cual se realiza sin asistencia del facilitador
		ni el manual de aplicaci\'on. El resultado de esta tarea, si se lleva a cabo correctamente, es una sencilla pieza musical.
	\end{itemize}

La dificultad de las tareas, la cual va en aumento durante el desarrollo de la prueba, 
est\'a definida en funci\'on de la cantidad y diversidad de los comandos necesarios para completarlas.

\subsubsection{Fase 4: Encuesta de Usabilidad}
Luego de finalizadas las tareas, los usuarios completan una breve encuesta relacionada a la usabilidad de
la aplicaci\'on y a las interfaces por voz del usuario en general.
Se recoge la opini\'on de los usuarios con respecto a las palabras y comandos utilizados, la duraci\'on del
entrenamiento y su predisposici\'on a utilizar interfaces de usuario basadas en el habla. Las respuestas se
registran utilizando una escala de Likert \cite{Allen:2007}.
Adem\'as, en cada punto, se solicitan sugerencias de parte del usuario. 	

\subsubsection{Fase 5: Recopilaci\'on y An\'alisis de Datos} 
Al finalizar cada sesi\'on, se recopilan los siguientes datos:
	\begin{itemize}			
		\item Registro de resultados del test de memoria.
		\item Registro de las notas del observador.
		\item Grabaci\'on de las tareas realizadas por el usuario.
		\item Encuesta completada por el usuario.
	\end{itemize}

Una vez finalizadas todas las pruebas, se procede a la sumarizaci\'on y an\'alisis de los datos disponibles
a partir de estos materiales. Los factores que se tienen en cuenta en esta etapa se describen en la
siguiente secci\'on.


%!TEX root = ../tesis.tex
\section{Factores Analizados}
\label{sec:factores}

Esta sección describe los factores que se tuvieron en cuenta para el ánalisis realizado
en base a los datos recopilados a partir de las pruebas de usabilidad que se realizaron.
Además, se definen las métricas utilizadas para la cuantificación de cada factor estudiado.

\subsection{Memoria del Usuario}
Factor relacionado a la capacidad de retención de información del sujeto del experimento.
Se consideró interesante su análisis debido a que:
\begin{itemize}
	\item TamTam Listens se trataba de una aplicación completamente nueva para el usuario.
	\item Los comandos de voz deben ser memorizados por el usuario para la interacción con
	la apliación.
\end{itemize}
\subsubsection{Métrica Utilizada}
La memoria del usuario se midió a través de los resultados del test de memoria que se llevó
a cabo con cada usuario como parte de la prueba de usabilidad.
El resultado del test de memoria administrado es la cantidad de palabras, de un total de 15,
que el usuario consiguió recordar al cabo de 5 repeticiones, siendo 12--13 el resultado promedio
esperado.

\subsection{Correctitud de la aplicación}
Factor relacionado al comportamiento esperado por parte de un sistema de reconocimiento del
habla. Se busca responder a la pregunta: ¿Qué tan correctamente se reconocen los comandos 
del usuario?
\subsubsection{Métricas Utilizadas}
\begin{itemize}
	\item Tasa de Aciertos de Comandos: se considera que el sistema reconoció correctamente
	un comando, si el usuario efectivamente pronunció el mismo.

	La tasa de aciertos es la razón entre la cantidad de comandos correctamente reconocidos 
	y la cantidad de comandos correctamente pronunciados por el usuario.
	
	Sean:

	\begin{itemize}
		\item $r_{ij}$ la cantidad de veces que la aplicación reconoció correctamente el \mbox{comando $i$}
		al usuario $j$.
		\item $c_{ij}$ la cantidad de veces que el usuario $j$ pronunció correctamente el \mbox{comando $i$.}
	\end{itemize}
	La tasa de aciertos del comando $i$ para el usuario $j$ se define como: 

	\begin{equation*}
		a_{{comando}_{ij}}=\frac{r_{ij}}{c_{ij}}
	\end{equation*}


	Sea $N_{j}$ la cantidad de comandos diferentes pronunciados por el usuario $j$,
	la tasa de aciertos para el usuario  se define como:
	
	\begin{equation*}
		a_{{usuario}_j}=\frac{\sum_k^{N_{j}}a_{{comando}_{kj}}}{N_{j}}
	\end{equation*}

	La tasa de aciertos del sistema se obtuvo hallando el promedio de los valores correspondientes
	a cada usuario.


	\item Tasa de Falsa Alarma: un falso positivo es un comando incorrecto (por causa del error humano)
	reconocido como válido por la aplicación.
	La tasa de falsa alarma es la razón entre la cantidad de falsos positivos y la cantidad
	de comandos incorrectos. 

	Sean:

	\begin{itemize}
		\item $m_{ij}$ la cantidad de veces que la aplicación reconoció incorrectamente el \mbox{comando $i$}
		al usuario $j$.
		\item $p_{ij}$ la cantidad de veces que el usuario $j$ pronunció incorrectamente el \mbox{comando $i$.}
	\end{itemize}
	La tasa de falsa alarma del comando $i$ para el usuario $j$ se define como: 

	\begin{equation*}
		f_{{comando}_{ij}}=\frac{m_{ij}}{p_{ij}}
	\end{equation*}

	El cálculo de la tasa de falsa alarma de cada usuario y del sistema se llevó a cabo de forma análoga 
	a la tasa de aciertos.
\end{itemize}

\subsection{Error Humano}
Factor relacionado a las equivocaciones por parte del usuario en su interacción con
TamTam Listens.
Un comando se consideró erróneo o incorrecto por parte del usuario si:
\begin{itemize}
	\item El usuario no respetó el orden de las palabras: esto es, pronunció las palabras de un comando en el orden incorrecto.
	\item El usuario se confundió entre dos comandos: esto es, utilizó palabras correspondientes a más de un comando.
	\item El usuario juntó dos o más comandos: esto es, pronunció de una sola vez dos o más comandos.
	\item El usuario dudó mientras pronunciaba el comando: esto es, titubeó o repitió una palabra del comando.
	\item El usuario utilizó comandos incompletos: esto es, pronunciando solo algunas palabras del comando.
	\item El usuario utilizó palabras que no forman parte del lenguaje.
\end{itemize}
\subsubsection{Métricas Utilizadas}
\begin{itemize}
	\item Cantidad de Errores: sumatoria de errores por comando y por usuario, clasificados en
	alguno de los tipos previamente mencionados.
	\item Tasa de Error Humano: se define como la razón entre la cantidad de comandos incorrectos
	y la cantidad de comandos pronunciados.
	Sean:

	\begin{itemize}
		\item $p_{ij}$ la cantidad de veces que el usuario $j$ pronunció incorrectamente el \mbox{comando $i$.}
		\item $t_{ij}$ la cantidad de veces que el usuario $j$ pronunció (correcta o incorrectamente) el 
		\mbox{comando $i$.}
	\end{itemize}
	La tasa de error humano del comando $i$ para el usuario $j$ se define como: 

	\begin{equation*}
		e_{{comando}_{ij}}=\frac{p_{ij}}{t_{ij}}
	\end{equation*}

	El cálculo de la tasa de error humano de cada usuario se llevó a cabo de forma análoga 
	a la tasa de aciertos.
\end{itemize}

\subsection{Eficiencia}
Factor relacionado al grado de productividad que logra el usuario de la aplicación.
\subsubsection{Métricas Utilizadas}
\begin{itemize}
	\item Duración de Tareas Uno y Dos: es la suma del tiempo que le tomó al usuario
	terminar las primeras dos tareas de la prueba, las cuales se realizaron con asistencia por
	parte del facilitador.
	\item Duración de Tareas Tres y Cuatro: es la suma del tiempo que le tomó al usuario
	terminar las últimas dos tareas de la prueba, las cuales se realizaron sin asistencia por
	parte del facilitador.
	\item Correctitud de la Tarea Cuatro: para él cálculo de esta medida, se descompuso la
	última tarea en lo distintos pasos necesarios para completarla correctamente.
	De esta manera, la tarea cuatro se consideró compuesta por 72 elementos:
	\begin{itemize}
		\item Creación de 54 notas.
		\item Selección de 4 instrumentos.
		\item Cambios en la duración de 12 notas.
		\item Reproducción de la música.
		\item Exportación de la música.
	\end{itemize}
	La correctitud de la tarea 4 para cada usuario se define como la razón entre la cantidad de operaciones
	correctamente realizadas y el total de operaciones (72).
\end{itemize}

\subsection{Satisfacción del Usuario}
Factor relacionado a la opinión del usuario sobre su interacción con la aplicación y las interfaces
basadas en reconocimiento del habla en general.
\subsubsection{Métricas Utilizadas}
La opinión del usuario se midió a través de los resultados de la encuesta realizada como parte de
la prueba de usabilidad, posterior a la finalización de las tareas. Las respuestas se
registraron utilizando una escala de Likert \cite{Allen:2007} con valores del 1 al 7.

Se solicitó la opinión del usuario con respecto a:
\begin{itemize}
	\item Las palabras utilizadas.
	\item Los comandos utilizados.
	\item Duración del entrenamiento.
	\item Interfaces por Voz del Usuario.
\end{itemize}





%!TEX root = ../tesis.tex
\section{An\'alisis del Error Humano}
\label{sec:evaluacionError}
Todas las m\'etricas presentadas en la secci\'on anterior se calcularon para cada usuario,
pudiendo promediarse para obtener un valor representativo de la aplicaci\'on.

En el caso particular del error humano, se realiz\'o un estudio m\'as detallado
por considerarse que podr{\'\i}a llevar a obtener conclusiones interesantes. De esta manera,
se calcularon tambi\'en:

\begin{itemize}
	\item Distribuci\'on del error humano con respecto al tiempo transcurrido: para cada usuario, 
	se dividi\'o $T_{3+4}$ en 10 etapas (de 10\%¸de la duraci\'on cada una).

	Para cada etapa se calcul\'o el porcentaje de errores cometidos, con respecto al total de
	errores cometidos por el usuario ($E_2$).

	La distribuci\'on del error humano con respecto al tiempo transcurrido se obtuvo calculando 
	el promedio de los valores de los usuarios para cada etapa.

	\item Tasa de error por longitud del comando pronunciado: la longitud de un comando se define
	como la cantidad de palabras que lo componen. La longitud de los comandos disponibles en el 
	lenguaje de TamTam Listens va de 2 a 6 palabras.

	El c\'alculo de la tasa de error se llev\'o a cabo para cada longitud posible.  


	\item Tasa de error por nivel contextual del comando pronunciado: el nivel contextual de un comando
	se define de acuerdo al contexto en que se aplica el mismo.
	Para TamTam Listens, se definieron 3 niveles posibles:
		\begin{itemize}
			\item General: comandos que afectan la m\'usica en general. 

			Ejemplo: Crear Nueva M\'usica.
			\item Pista: comandos que dependen de una pista seleccionada para su aplicaci\'on. 

			Ejemplo: Piano en Pista 2.
			\item Comp\'as: comandos que que dependen de una pista y un comp\'as seleccionados 
			para su aplicaci\'on. 

			Ejemplo: Crear Nota Do.
		\end{itemize}
		
	El c\'alculo de la tasa de error se llev\'o a cabo para cada nivel contextual posible.  

\end{itemize}