%!TEX root = ../tesis.tex
\chapter{Evaluaci\'on}
\label{sec:evaluacion}


% introduccion
En el cap{\'\i}tulo anterior se describi\'o la soluci\'on propuesta para el estudio de interfaces basadas en
reconocimiento del habla, la cual fue implementada como parte de este trabajo de grado.

Con el fin de realizar una evaluaci\'on tanto de la aplicaci\'on misma como de ciertos conceptos 
e hip\'otesis planteados, se realiza una prueba de usabilidad. 
Como parte de la misma, se miden y analizan factores relacionados a la aplicaci\'on,
al usuario y a la interacci\'on entre ambos.

Este cap{\'\i}tulo presenta los principales aspectos de dise\~no del estudio propuesto: objetivos, metodolog{\'\i}a
y m\'etricas utilizadas. La \'ultima secci\'on describe con mayor detalle el an\'alisis con respecto
a uno de los factores estudiados, el error humano.


%!TEX root = ../tesis.tex
\section{Objetivos}
\label{sec:objetivos-estudio}

Fueron definidos los siguientes objetivos para el estudio realizado:

\begin{itemize}
	\item Evaluar experimentalmente la aplicaci\'on desarrollada, teniendo en cuenta 
	aspectos t\'ecnicos y funcionales.
	\item Identificar problemas que se presentan al utilizar una interfaz mediante voz y,
	 de ser posible, sugerir posibles soluciones.
	\item Proponer criterios que puedan utilizarse para la evaluaci\'on de interfaces 
	mediante voz.
	\item Verificar y validar la utilidad de los criterios propuestos, de acuerdo a 
	la informaci\'on y las conclusiones que pueden obtenerse mediante los mismos.
	\item Analizar la correlación entre los factores medidos, de modo a identificar
	posibles relaciones entre los mismos. 
	\item Obtener conclusiones relacionadas a factores externos a la aplicaci\'on, 
	como la memoria del usuario y el tiempo de entrenamiento previo.
\end{itemize}
%!TEX root = ../tesis.tex
\section{Metodolog{\'\i}a}
\label{sec:metodolog{\'\i}a}
A continuaci\'on se describen brevemente distintos aspectos relacionados al
estudio realizado.

\subsection{Muestra}
Las pruebas fueron realizadas con 12 usuarios, estudiantes de Ingenier{\'\i}a Inform\'atica
de la Facultad Polit\'ecnica, Universidad Nacional de Asunci\'on. Ninguno de los usuarios 
hab{\'\i}a utilizado anteriormente la aplicaci\'on evaluada, de manera a evitar los efectos
de la experiencia previa.

La cantidad de sujetos de prueba fue elegida de acuerdo a la recomendaci\'on realizada
en \cite{Hwang:2010} para este tipo de estudios. Adem\'as, de acuerdo a \cite[p.~267]{Rubin2008}, 
este n\'umero es el necesario de modo a obtener resultados estad{\'\i}sticos significativos.

\subsection{Duraci\'on}
Cada sesi\'on de pruebas tuvo una duraci\'on aproximada de una hora.
Las sesiones fueron realizadas a lo largo de 2 semanas.

\subsection{Ambiente de Pruebas}
De modo a realizar el estudio en un ambiente controlado, todas las sesiones fueron
realizadas en un Laboratorio de Inform\'atica de la Facultad Poĺit\'ecnica. Esto
permiti\'o minimizar los efectos de la interferencia sonora y humana durante el
desarrollo de las pruebas.

\subsection{Roles}
Cada sesi\'on cont\'o con 3 participantes:
	\begin{itemize}
		\item Facilitador: responsable de administrar los distintos pasos del estudio, explicarlos al usuario 
		y aclarar las dudas que se presenten durante la fase inicial. 
		\item Observador: responsable de tomar nota de los eventos que sucedan durante las pruebas, principalmente
		tiempo de inicio y fin de cada paso y otros sucesos que aporten valor al estudio (errores, preguntas, etc.).
		\item Usuario: persona sin conocimientos del sistema. Los detalles del aprendizaje y primeras interacciones
		(tareas) con la aplicaci\'on se registran para su posterior an\'alisis.	
	\end{itemize}

\subsection{Test de Memoria}
Previamente a la realizaci\'on de las tareas, se llev\'o a cabo un test de memoria del usuario, basado en
el Test de Aprendizaje Auditivo Verbal de Rey \cite{Lopez1998}. 

Para la prueba, el facilitador le{\'\i}a una lista
de 15 palabras al usuario, luego de lo cual este \'ultimo deb{\'\i}a repetir aquellas que recordaba.
El proceso se repet{\'\i}a 5 veces en total, registr\'andose las palabras que el usuario recordaba y el orden en que
lo hac{\'\i}a.

\subsection{Entrenamiento}
La fase de entrenamiento tuvo como objetivo la introducci\'on del usuario a los distintos comandos por voz disponibles
para interactuar con TamTam Listens. 
Para evitar la influencia de las variaciones que pod{\'\i}an haberse presentado en una
explicaci\'on directa del facilitador al usuario, fue grabada una serie de videotutoriales que se presentaban al usuario
en esta etapa.
Adem\'as, cada usuario recibi\'o una copia del manual de la aplicaci\'on, el cual describ{\'\i}a en mayor detalles las
funcionalidades de la aplicaci\'on.

\subsection{Tareas}
Cada usuario realiz\'o un total de 4 tareas durante la prueba de usabilidad:
	\begin{itemize}			
		\item Tareas 1 y 2: tareas simples y breves que buscan llevar a la pr\'actica los conceptos aprendidos en los
		videotutoriales. Se realizan con ayuda del facilitador y el manual de aplicaci\'on.
		\item Tarea 3: tarea un poco m\'as compleja, la cual se realiza ya sin asistencia del facilitador. El usuario puede, sin embargo, consultar el manual de aplicaci\'on.
		\item Tarea 4: tarea de mayor dificultad, la cual se realiza sin asistencia del facilitador ni el manual de
		aplicaci\'on. El resultado de esta tarea, si se lleva a cabo correctamente, es una sencilla pieza musical.
	\end{itemize}

\subsection{Encuesta de Usabilidad}
Luego de finalizadas las tareas, los usuarios completaron una breve encuesta relacionada a la usabilidad de
la aplicaci\'on y a las interfaces por voz de usuario en general.
Se recogi\'o la opini\'on de los usuarios con respecto a las palabras y comandos utilizados, la duraci\'on del
entrenamiento y su predisposici\'on a utilizar interfaces de usuario basadas en el habla. Las respuestas se
registraron utilizando una escala de Likert \cite{Allen:2007}.
Adem\'as, en cada punto, se solicitaron sugerencias de parte del usuario. 	

\subsection{Recopilaci\'on y An\'alisis de Datos} 
Al finalizar cada sesi\'on, se recopilaron los siguientes datos:
	\begin{itemize}			
		\item Registro de resultados del test de memoria.
		\item Registro de las notas del observador.
		\item Grabaci\'on de las tareas realizadas por el usuario.
		\item Encuesta completada por el usuario.
	\end{itemize}

Una vez finalizadas todas las pruebas, se procedi\'o a la sumarizaci\'on y an\'alisis de los datos disponibles
a partir de estos materiales. Los factores que se tuvieron en cuenta en esta etapa se describen en la
siguiente secci\'on.


%!TEX root = ../tesis.tex
\section{Factores Analizados}
\label{sec:factores}

Esta secci\'on describe los factores que se tienen en cuenta para el \'analisis a realizarse
en base a los datos recopilados a partir de las pruebas de usabilidad.
Se definen tambi\'en las m\'etricas utilizadas para la cuantificaci\'on de cada factor estudiado.

\subsection{Memoria del Usuario}
\label{sec:memoria-del-usuario}
Factor relacionado a la capacidad de retenci\'on de informaci\'on del usuario.
Se considera interesante su an\'alisis debido a que:
\begin{itemize}
	\item TamTam Listens se trata de una aplicaci\'on completamente nueva para el usuario.
	\item Los comandos de voz deben ser memorizados por el usuario para la interacci\'on con
	la apliaci\'on.
\end{itemize}
\subsubsection{M\'etrica Utilizada}
La memoria del usuario se midie a trav\'es de los resultados del test de memoria que se lleva
a cabo con cada usuario como parte de la prueba de usabilidad.
El resultado del test de memoria administrado es la cantidad de palabras, de un total de 15,
que el usuario consigue recordar al cabo de 5 repeticiones, siendo 12--13 el resultado promedio
esperado. El resultado del test de memoria se representa como $M$.

\subsection{Correctitud de la aplicaci\'on}
Factor relacionado al comportamiento esperado por parte de un sistema de reconocimiento del
habla. Se busca responder a la pregunta: {?`}Qu\'e tan correctamente se reconocen los comandos 
del usuario?
Los valores de las m\'etricas relacionadas a la correctitud de la aplicaci\'on se calculan
en base a las tareas 3 y 4, por realizarse estas sin asistencia del facilitador.  
\subsubsection{M\'etricas Utilizadas}
\begin{itemize}
	\item Tasa de Aciertos de Comandos: se considera que el sistema reconoce correctamente
	un comando, si el usuario efectivamente ha pronunciado el mismo.

	La tasa de aciertos es la raz\'on entre la cantidad de comandos correctamente reconocidos 
	y la cantidad de comandos correctamente pronunciados por el usuario. 
	La misma se representa como $A$.
	
	Sean:

	\begin{itemize}
		\item $r_{ij}$ n\'umero de reconocimientos correctos del \mbox{comando $i$} pronunciado por el usuario $j$,
		por parte de la aplicaci\'on.
		\item $c_{ij}$ n\'umero de pronunciaciones correctas del \mbox{comando $i$} por parte del usuario $j$.
	\end{itemize}
	La tasa de aciertos del comando $i$ para el usuario $j$ se define como: 

	\begin{equation*}
		a_{{comando}_{ij}}=\frac{r_{ij}}{c_{ij}}
	\end{equation*}


	Sea $N_{j}$ la cantidad total de comandos diferentes pronunciados (correcta o incorrectamente) por el usuario $j$,
	la tasa de aciertos para el usuario  se define como:
	
	\begin{equation*}
		A=a_{{usuario}_j}=\frac{\sum_k^{N_{j}}a_{{comando}_{kj}}}{N_{j}}
	\end{equation*}

	La tasa de aciertos del sistema se obtiene hallando el promedio de los valores correspondientes
	a cada usuario.

	\item Tasa de Error de Comandos: es la raz\'on entre la cantidad de comandos correctos no reconocidos 
	y la cantidad de comandos correctamente pronunciados por el usuario. 
	La misma se representa como $E_1$ y es igual a $1-A$.

\end{itemize}

\subsection{Error Humano}
Factor relacionado a las equivocaciones por parte del usuario en su interacci\'on con
TamTam Listens.
Un comando se considera err\'oneo o incorrecto por parte del usuario si:
\begin{itemize}
	\item El usuario no respeta el orden de las palabras: esto es, pronuncia las palabras de un comando en el orden incorrecto.
	\item El usuario se confundie entre dos comandos: esto es, utiliza palabras correspondientes a m\'as
	de un comando.
	\item El usuario junta dos o m\'as comandos: esto es, pronuncia de una sola vez dos o m\'as comandos.
	\item El usuario duda mientras pronuncia el comando: esto es, titubea o repite una palabra del comando.
	\item El usuario utiliza comandos incompletos: esto es, pronunciando solo algunas palabras del comando.
	\item El usuario utiliza palabras que no forman parte del lenguaje.
	\item El usuario viola alguna otra regla impuesta por el lenguaje de la aplicaci\'on.
\end{itemize}
Los valores de las m\'etricas relacionadas al error humano se calculan
en base a las tareas 3 y 4, por realizarse estas sin asistencia del facilitador. 

\subsubsection{M\'etricas Utilizadas}
\begin{itemize}
	\item Cantidad de Errores: sumatoria de errores por usuario, clasificados en
	alguno de los tipos previamente mencionados. Se representa como $E_2$.
	\item Tasa de Error Humano: se define como la raz\'on entre la cantidad de comandos incorrectos
	y la cantidad de comandos pronunciados. Se representa como $E_3$.
	
	Sean:

	\begin{itemize}
		\item $p_{ij}$ n\'umero de pronunciaciones incorrectas del \mbox{comando $i$} por parte del usuario $j$.
		\item $t_{ij}$ n\'umero de pronunciaciones (correctas e incorrectas) del \mbox{comando $i$} por parte del usuario $j$.
	\end{itemize}
	La tasa de error humano del comando $i$ para el usuario $j$ se define como: 

	\begin{equation*}
		e_{{comando}_{ij}}=\frac{p_{ij}}{t_{ij}}
	\end{equation*}

	El c\'alculo de la tasa de error humano de cada usuario se lleva a cabo de forma an\'aloga 
	a la tasa de aciertos.
\end{itemize}

\subsection{Eficiencia}
Factor relacionado al grado de productividad que logra el usuario de la aplicaci\'on.
\subsubsection{M\'etricas Utilizadas}
\begin{itemize}
	\item Duraci\'on de Tareas Uno y Dos: es la suma del tiempo que le toma al usuario
	terminar la tarea 1 ($T_1$) y el que le toma terminar la tarea 2 ($T_2$). 
	Estas tareas se realizan con asistencia por parte del facilitador. 
	La duraci\'on se midi\'o en minutos. Se representa como $T_{1+2}$.
	\item Duraci\'on de Tareas Tres y Cuatro: es la suma del tiempo que le toma al usuario
	terminar la tarea 3 ($T_3$) y el que le toma terminar la tarea 4 ($T_4$). 
	Estas tareas se realizan sin asistencia por parte del facilitador. 
	La duraci\'on se mide en minutos. Se representa como $T_{3+4}$.
	\item Cantidad de Comandos Utilizados: es la cantidad total de comandos diferentes utilizados
	al menos una vez por el usuario durante las tareas 3 y 4. Se representa como $U$.
	\item Correctitud de la Tarea Cuatro: para \'el c\'alculo de esta medida, se descompone la
	\'ultima tarea en lo distintos pasos necesarios para completarla correctamente.
	De esta manera, la tarea cuatro se considera compuesta por 72 elementos:
	\begin{itemize}
		\item Creaci\'on de 54 notas.
		\item Selecci\'on de 4 instrumentos.
		\item Cambios en la duraci\'on de 12 notas.
		\item Reproducci\'on de la m\'usica.
		\item Exportaci\'on de la m\'usica.
	\end{itemize}
	La correctitud de la tarea 4 para cada usuario se define como la raz\'on entre la cantidad de operaciones
	correctamente realizadas y el total de operaciones (72). Se representa como $C$.
\end{itemize}

\subsection{Satisfacci\'on del Usuario}
Factor relacionado a la apreciaci\'on subjetiva del usuario sobre su interacci\'on con la aplicaci\'on y
las interfaces basadas en reconocimiento del habla en general.
\subsubsection{M\'etricas Utilizadas}
La opini\'on del usuario se mide a trav\'es de los resultados de la encuesta que se realiza como parte de
la prueba de usabilidad, posterior a la finalizaci\'on de las tareas. Las respuestas se
registran utilizando una escala de Likert \cite{Allen:2007} con valores del 1 al 7.

Se solicita la opini\'on del usuario con respecto a:
\begin{itemize}
	\item Que tan adecuadas son palabras utilizadas.
	\item Que tan adecuados son los comandos utilizados.
	\item Que tan adecuada resulta la duraci\'on del entrenamiento.
	\item Que tan frecuentemente utilizar{\'\i}a las interfaces por voz del usuario.
\end{itemize}

Posteriormente, de modo a eliminar el potencial efecto de los estilos de respuesta
de los usuarios \cite{Fischer2010}, se utiliza el m\'etodo de estandarizaci\'on 
por rango \cite{Pagolu2011} para reescalar los resultados de la encuesta.

Siendo:
\begin{itemize}
	\item $min_i$ la respuesta de menor valor del usuario $i$.
	\item $max_i$ larespuesta de mayor valor del usuario $j$.
\end{itemize}

Para cada respuesta $s$ del usuario $i$, el valor ajustado $s_a$ se define como:

\begin{equation*}
s_a=\frac{s-min_i}{max_i-min_i}
\end{equation*}


De esta manera, todas las respuestas pasan a estar en el rango entre 0 y 1.  





%!TEX root = ../tesis.tex
\section{An\'alisis del Error Humano}
\label{sec:evaluacionError}
Todas las m\'etricas presentadas en la secci\'on anterior se calcularon para cada usuario,
pudiendo promediarse para obtener un valor representativo de la aplicaci\'on.

En el caso particular del error humano, se realiz\'o un estudio m\'as detallado
por considerarse que podr{\'\i}a llevar a obtener conclusiones interesantes. De esta manera,
se calcularon tambi\'en:

\begin{itemize}
	\item Tasa de Error Humano por Longitud del Comando: la longitud de un comando se define
	como la cantidad de palabras que lo componen. La longitud de los comandos disponibles en el 
	lenguaje de TamTam Listens va de 2 a 6 palabras.

	El c\'alculo de la tasa de error se llev\'o a cabo para cada longitud posible.  


	\item Tasa de Error Humano por Nivel Contextual del Comando: el nivel contextual de un comando
	se define de acuerdo al contexto en que se aplica el mismo.
	Para TamTam Listens, se definieron 3 niveles posibles:
		\begin{itemize}
			\item General: comandos que afectan la m\'usica en general. 

			Ejemplo: Crear Nueva M\'usica.
			\item Pista: comandos que dependen de una pista seleccionada para su aplicaci\'on. 

			Ejemplo: Piano en Pista 2.
			\item Comp\'as: comandos que que dependen de una pista y un comp\'as seleccionados 
			para su aplicaci\'on. 

			Ejemplo: Crear Nota Do.
		\end{itemize}
	
		
	El c\'alculo de la tasa de error se llev\'o a cabo para cada nivel contextual posible.
	\item Distribuci\'on del Error Humano con respecto al Tiempo transcurrido: para cada usuario, 
	se dividi\'o $T_{3+4}$ en 10 etapas (de 10\%¸de la duraci\'on cada una).

	Para cada etapa se calcul\'o el porcentaje de errores cometidos, con respecto al total de
	errores cometidos por el usuario ($E_2$).

	La distribuci\'on del error humano con respecto al tiempo transcurrido se obtuvo calculando 
	el promedio de los valores de los usuarios para cada etapa.  
\end{itemize}