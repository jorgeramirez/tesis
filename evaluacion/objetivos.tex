%!TEX root = ../tesis.tex
\section{Objetivos}
\label{sec:objetivos-estudio}

La prueba de usabilidad planteada apunta al aprendizaje de cuestiones relacionadas
a la aplicaci\'on implementada y a las interfaces mediante voz en general.

Una prueba con usuarios de la aplicaci\'on implementada puede colaborar al esclarecimiento de 
las interrogantes respecto a la influencia del tama\~no del lenguaje, la longitud de
los comandos, la duraci\'on de la interacci\'on y otros factores caracter{\'\i}sticos de una
interfaz mediante voz, que se presentaron en la secci\'on \ref{sec:problema-general}.

Los objetivos del estudio de usabilidad de \foreign{TamTam Listens} son los siguientes:

\begin{itemize}
	\item Evaluar experimentalmente la aplicaci\'on desarrollada, teniendo en cuenta 
	aspectos t\'ecnicos y funcionales.
	\item Identificar problemas que se presentan al utilizar una interfaz mediante voz y,
	 de ser posible, sugerir posibles soluciones.
	\item Proponer criterios que puedan utilizarse para la evaluaci\'on de interfaces 
	mediante voz.
	\item Verificar y validar la utilidad de los criterios propuestos, de acuerdo a 
	la informaci\'on y las conclusiones que pueden obtenerse mediante los mismos.
	\item Analizar la correlaci\'on entre los factores medidos, de modo a identificar
	posibles relaciones entre los mismos. 
	\item Obtener conclusiones relacionadas a factores externos a la aplicaci\'on, 
	como la memoria del usuario, el tiempo de entrenamiento previo y el tiempo
	total de uso.
\end{itemize}