%!TEX root = ../tesis.tex
\section{Metodolog{\'\i}a}
\label{sec:metodolog{\'\i}a}
A continuaci\'on se describen brevemente distintos aspectos relacionados al
estudio de usabilidad planteado.

\subsection{Muestra}
Las pruebas se realizan con 12 usuarios, estudiantes de Ingenier{\'\i}a Inform\'atica
de la Facultad Polit\'ecnica, Universidad Nacional de Asunci\'on. Ninguno de los usuarios 
ha utilizado anteriormente la aplicaci\'on evaluada, de manera a evitar los efectos
de la experiencia previa.

La cantidad de sujetos de prueba se elige de acuerdo a la recomendaci\'on realizada
en \cite{Hwang:2010} para este tipo de estudios, la cual es de 12. 
Adem\'as, de acuerdo a \cite[p.~267]{Rubin2008}, este n\'umero es el necesario de modo a obtener
resultados estad{\'\i}sticos significativos.

\subsection{Duraci\'on}
Cada sesi\'on de pruebas tiene una duraci\'on aproximada de una hora.
Las sesiones se realizan a lo largo de 2 semanas.

\subsection{Ambiente de Pruebas}
De modo a realizar el estudio en un ambiente controlado, todas las sesiones se
realizan en un Laboratorio de Inform\'atica de la Facultad Poĺit\'ecnica. Esto
permite minimizar los efectos de la interferencia sonora y humana durante el
desarrollo de las pruebas.

\subsection{Roles}
Cada sesi\'on cuenta con 3 participantes, cada uno desempeñando uno de los siguientes roles:
	\begin{itemize}
		\item Facilitador: responsable de administrar los distintos pasos del estudio, explicarlos al usuario 
		y aclarar las dudas que se presenten durante la fase inicial. 
		\item Observador: responsable de tomar nota de los eventos que sucedan durante las pruebas, principalmente
		tiempo de inicio y fin de cada paso y otros sucesos que aporten valor al estudio (errores, preguntas, etc.).
		\item Usuario: persona sin conocimientos del sistema. Los detalles del aprendizaje y primeras interacciones
		(tareas) con la aplicaci\'on se registran para su posterior an\'alisis.	
	\end{itemize}

\subsection{Fases}
Cada sesión está compuesta de 5 fases, las cuales se completan en orden secuencial.
A continuación, se describe brevemente cada una de las fases de la prueba de usabilidad.

\subsubsection{Fase 1: Test de Memoria}
Previamente a la realizaci\'on de las tareas, se lleva a cabo un test de memoria del usuario, basado en
el Test de Aprendizaje Auditivo Verbal de Rey \cite{Lopez1998}. 

Para la prueba, el facilitador lee una lista
de 15 palabras al usuario, luego de lo cual este \'ultimo debe repetir aquellas que recuerda.
El proceso se repite 5 veces en total, registr\'andose las palabras que el usuario recuerda y el orden en que
lo hace.

\subsubsection{Fase 2: Entrenamiento}
La fase de entrenamiento tiene como objetivo la introducci\'on del usuario a los distintos comandos por voz disponibles
para interactuar con TamTam Listens. 
Para evitar la influencia de las variaciones que pueden presentarse en una
explicaci\'on directa del facilitador al usuario, fue grabada una serie de videotutoriales 
que se presentan al usuario en esta etapa.
Adem\'as, cada usuario recibe una copia del manual de la aplicaci\'on, el cual describe en mayor detalle las
funcionalidades de la aplicaci\'on.

\subsubsection{Fase 3: Tareas}
Cada usuario realiza un total de 4 tareas durante la prueba de usabilidad:
	\begin{itemize}			
		\item Tareas 1 y 2: actividades simples y breves que buscan llevar a la pr\'actica los conceptos aprendidos
		en los videotutoriales. Se realizan con ayuda del facilitador y el manual de aplicaci\'on.
		\item Tarea 3: actividad un poco m\'as compleja, la cual se realiza ya sin asistencia del facilitador. El usuario puede, sin embargo, consultar el manual de aplicaci\'on.
		\item Tarea 4: actividad de mayor dificultad, la cual se realiza sin asistencia del facilitador ni el manual de
		aplicaci\'on. El resultado de esta tarea, si se lleva a cabo correctamente, es una sencilla pieza musical.
	\end{itemize}

\subsubsection{Fase 4: Encuesta de Usabilidad}
Luego de finalizadas las tareas, los usuarios completan una breve encuesta relacionada a la usabilidad de
la aplicaci\'on y a las interfaces por voz de usuario en general.
Se recoge la opini\'on de los usuarios con respecto a las palabras y comandos utilizados, la duraci\'on del
entrenamiento y su predisposici\'on a utilizar interfaces de usuario basadas en el habla. Las respuestas se
registran utilizando una escala de Likert \cite{Allen:2007}.
Adem\'as, en cada punto, se solicitan sugerencias de parte del usuario. 	

\subsubsection{Fase 5: Recopilaci\'on y An\'alisis de Datos} 
Al finalizar cada sesi\'on, se recopilan los siguientes datos:
	\begin{itemize}			
		\item Registro de resultados del test de memoria.
		\item Registro de las notas del observador.
		\item Grabaci\'on de las tareas realizadas por el usuario.
		\item Encuesta completada por el usuario.
	\end{itemize}

Una vez finalizadas todas las pruebas, se procede a la sumarizaci\'on y an\'alisis de los datos disponibles
a partir de estos materiales. Los factores que se tienen en cuenta en esta etapa se describen en la
siguiente secci\'on.

