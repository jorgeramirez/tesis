%!TEX root = ../tesis.tex
\section{Metodolog{\'\i}a}
\label{sec:metodolog{\'\i}a}
A continuaci\'on se describen brevemente distintos aspectos relacionados al
estudio realizado.

\subsection{Muestra}
Las pruebas fueron realizadas con 12 usuarios, estudiantes de Ingenier{\'\i}a Inform\'atica
de la Facultad Polit\'ecnica, Universidad Nacional de Asunci\'on. Ninguno de los usuarios 
hab{\'\i}a utilizado anteriormente la aplicaci\'on evaluada, de manera a evitar los efectos
de la experiencia previa.

La cantidad de sujetos de prueba fue elegida de acuerdo a la recomendaci\'on realizada
en \cite{Hwang:2010} para este tipo de estudios. Adem\'as, de acuerdo a \cite[p.~267]{Rubin2008}, 
este n\'umero es el necesario de modo a obtener resultados estad{\'\i}sticos significativos.

\subsection{Duraci\'on}
Cada sesi\'on de pruebas tuvo una duraci\'on aproximada de una hora.
Las sesiones fueron realizadas a lo largo de 2 semanas.

\subsection{Ambiente de Pruebas}
De modo a realizar el estudio en un ambiente controlado, todas las sesiones fueron
realizadas en un Laboratorio de Inform\'atica de la Facultad Poĺit\'ecnica. Esto
permiti\'o minimizar los efectos de la interferencia sonora y humana durante el
desarrollo de las pruebas.

\subsection{Roles}
Cada sesi\'on cont\'o con 3 participantes:
	\begin{itemize}
		\item Facilitador: responsable de administrar los distintos pasos del estudio, explicarlos al usuario 
		y aclarar las dudas que se presenten durante la fase inicial. 
		\item Observador: responsable de tomar nota de los eventos que sucedan durante las pruebas, principalmente
		tiempo de inicio y fin de cada paso y otros sucesos que aporten valor al estudio (errores, preguntas, etc.).
		\item Usuario: persona sin conocimientos del sistema. Los detalles del aprendizaje y primeras interacciones
		(tareas) con la aplicaci\'on se registran para su posterior an\'alisis.	
	\end{itemize}

\subsection{Test de Memoria}
Previamente a la realizaci\'on de las tareas, se llev\'o a cabo un test de memoria del usuario, basado en
el Test de Aprendizaje Auditivo Verbal de Rey \cite{Lopez1998}. 

Para la prueba, el facilitador le{\'\i}a una lista
de 15 palabras al usuario, luego de lo cual este \'ultimo deb{\'\i}a repetir aquellas que recordaba.
El proceso se repet{\'\i}a 5 veces en total, registr\'andose las palabras que el usuario recordaba y el orden en que
lo hac{\'\i}a.

\subsection{Entrenamiento}
La fase de entrenamiento tuvo como objetivo la introducci\'on del usuario a los distintos comandos por voz disponibles
para interactuar con TamTam Listens. 
Para evitar la influencia de las variaciones que pod{\'\i}an haberse presentado en una
explicaci\'on directa del facilitador al usuario, fue grabada una serie de videotutoriales que se presentaban al usuario
en esta etapa.
Adem\'as, cada usuario recibi\'o una copia del manual de la aplicaci\'on, el cual describ{\'\i}a en mayor detalles las
funcionalidades de la aplicaci\'on.

\subsection{Tareas}
Cada usuario realiz\'o un total de 4 tareas durante la prueba de usabilidad:
	\begin{itemize}			
		\item Tareas 1 y 2: tareas simples y breves que buscan llevar a la pr\'actica los conceptos aprendidos en los
		videotutoriales. Se realizan con ayuda del facilitador y el manual de aplicaci\'on.
		\item Tarea 3: tarea un poco m\'as compleja, la cual se realiza ya sin asistencia del facilitador. El usuario puede, sin embargo, consultar el manual de aplicaci\'on.
		\item Tarea 4: tarea de mayor dificultad, la cual se realiza sin asistencia del facilitador ni el manual de
		aplicaci\'on. El resultado de esta tarea, si se lleva a cabo correctamente, es una sencilla pieza musical.
	\end{itemize}

\subsection{Encuesta de Usabilidad}
Luego de finalizadas las tareas, los usuarios completaron una breve encuesta relacionada a la usabilidad de
la aplicaci\'on y a las interfaces por voz de usuario en general.
Se recogi\'o la opini\'on de los usuarios con respecto a las palabras y comandos utilizados, la duraci\'on del
entrenamiento y su predisposici\'on a utilizar interfaces de usuario basadas en el habla. Las respuestas se
registraron utilizando una escala de Likert \cite{Allen:2007}.
Adem\'as, en cada punto, se solicitaron sugerencias de parte del usuario. 	

\subsection{Recopilaci\'on y An\'alisis de Datos} 
Al finalizar cada sesi\'on, se recopilaron los siguientes datos:
	\begin{itemize}			
		\item Registro de resultados del test de memoria.
		\item Registro de las notas del observador.
		\item Grabaci\'on de las tareas realizadas por el usuario.
		\item Encuesta completada por el usuario.
	\end{itemize}

Una vez finalizadas todas las pruebas, se procedi\'o a la sumarizaci\'on y an\'alisis de los datos disponibles
a partir de estos materiales. Los factores que se tuvieron en cuenta en esta etapa se describen en la
siguiente secci\'on.

