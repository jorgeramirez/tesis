%!TEX root = ../tesis.tex
\section{Factores Analizados}
\label{sec:factores}

Esta sección describe los factores que se tuvieron en cuenta para el ánalisis realizado
en base a los datos recopilados a partir de las pruebas de usabilidad que se realizaron.
Además, se definen las métricas utilizadas para la cuantificación de cada factor estudiado.

\subsection{Memoria del Usuario}
Factor relacionado a la capacidad de retención de información del sujeto del experimento.
Se consideró interesante su análisis debido a que:
\begin{itemize}
	\item TamTam Listens se trataba de una aplicación completamente nueva para el usuario.
	\item Los comandos de voz deben ser memorizados por el usuario para la interacción con
	la apliación.
\end{itemize}
\subsubsection{Métrica Utilizada}
La memoria del usuario se midió a través de los resultados del test de memoria que se llevó
a cabo con cada usuario como parte de la prueba de usabilidad.
El resultado del test de memoria administrado es la cantidad de palabras, de un total de 15,
que el usuario consiguió recordar al cabo de 5 repeticiones, siendo 12--13 el resultado promedio
esperado.

\subsection{Correctitud de la aplicación}
Factor relacionado al comportamiento esperado por parte de un sistema de reconocimiento del
habla. Se busca responder a la pregunta: ¿Qué tan correctamente se reconocen los comandos 
del usuario?
\subsubsection{Métricas Utilizadas}
\begin{itemize}
	\item Tasa de Aciertos de Comandos: se considera que el sistema reconoció correctamente
	un comando, si el usuario efectivamente pronunció el mismo.

	La tasa de aciertos es la razón entre la cantidad de comandos correctamente reconocidos 
	y la cantidad de comandos correctamente pronunciados por el usuario.
	
	Sean:

	\begin{itemize}
		\item $r_{ij}$ la cantidad de veces que la aplicación reconoció correctamente el \mbox{comando $i$}
		al usuario $j$.
		\item $c_{ij}$ la cantidad de veces que el usuario $j$ pronunció correctamente el \mbox{comando $i$.}
	\end{itemize}
	La tasa de aciertos del comando $i$ para el usuario $j$ se define como: 

	\begin{equation*}
		a_{{comando}_{ij}}=\frac{r_{ij}}{c_{ij}}
	\end{equation*}


	Sea $N_{j}$ la cantidad de comandos diferentes pronunciados por el usuario $j$,
	la tasa de aciertos para el usuario  se define como:
	
	\begin{equation*}
		a_{{usuario}_j}=\frac{\sum_k^{N_{j}}a_{{comando}_{kj}}}{N_{j}}
	\end{equation*}

	La tasa de aciertos del sistema se obtuvo hallando el promedio de los valores correspondientes
	a cada usuario.


	\item Tasa de Falsa Alarma: un falso positivo es un comando incorrecto (por causa del error humano)
	reconocido como válido por la aplicación.
	La tasa de falsa alarma es la razón entre la cantidad de falsos positivos y la cantidad
	de comandos incorrectos. 

	Sean:

	\begin{itemize}
		\item $m_{ij}$ la cantidad de veces que la aplicación reconoció incorrectamente el \mbox{comando $i$}
		al usuario $j$.
		\item $p_{ij}$ la cantidad de veces que el usuario $j$ pronunció incorrectamente el \mbox{comando $i$.}
	\end{itemize}
	La tasa de falsa alarma del comando $i$ para el usuario $j$ se define como: 

	\begin{equation*}
		f_{{comando}_{ij}}=\frac{m_{ij}}{p_{ij}}
	\end{equation*}

	El cálculo de la tasa de falsa alarma de cada usuario y del sistema se llevó a cabo de forma análoga 
	a la tasa de aciertos.
\end{itemize}

\subsection{Error Humano}
Factor relacionado a las equivocaciones por parte del usuario en su interacción con
TamTam Listens.
Un comando se consideró erróneo o incorrecto por parte del usuario si:
\begin{itemize}
	\item El usuario no respetó el orden de las palabras: esto es, pronunció las palabras de un comando en el orden incorrecto.
	\item El usuario se confundió entre dos comandos: esto es, utilizó palabras correspondientes a más de un comando.
	\item El usuario juntó dos o más comandos: esto es, pronunció de una sola vez dos o más comandos.
	\item El usuario dudó mientras pronunciaba el comando: esto es, titubeó o repitió una palabra del comando.
	\item El usuario utilizó comandos incompletos: esto es, pronunciando solo algunas palabras del comando.
	\item El usuario utilizó palabras que no forman parte del lenguaje.
\end{itemize}
\subsubsection{Métricas Utilizadas}
\begin{itemize}
	\item Cantidad de Errores: sumatoria de errores por comando y por usuario, clasificados en
	alguno de los tipos previamente mencionados.
	\item Tasa de Error Humano: se define como la razón entre la cantidad de comandos incorrectos
	y la cantidad de comandos pronunciados.
	Sean:

	\begin{itemize}
		\item $p_{ij}$ la cantidad de veces que el usuario $j$ pronunció incorrectamente el \mbox{comando $i$.}
		\item $t_{ij}$ la cantidad de veces que el usuario $j$ pronunció (correcta o incorrectamente) el 
		\mbox{comando $i$.}
	\end{itemize}
	La tasa de error humano del comando $i$ para el usuario $j$ se define como: 

	\begin{equation*}
		e_{{comando}_{ij}}=\frac{p_{ij}}{t_{ij}}
	\end{equation*}

	El cálculo de la tasa de error humano de cada usuario se llevó a cabo de forma análoga 
	a la tasa de aciertos.
\end{itemize}

\subsection{Eficiencia}
Factor relacionado al grado de productividad que logra el usuario de la aplicación.
\subsubsection{Métricas Utilizadas}
\begin{itemize}
	\item Duración de Tareas Uno y Dos: es la suma del tiempo que le tomó al usuario
	terminar las primeras dos tareas de la prueba, las cuales se realizaron con asistencia por
	parte del facilitador.
	\item Duración de Tareas Tres y Cuatro: es la suma del tiempo que le tomó al usuario
	terminar las últimas dos tareas de la prueba, las cuales se realizaron sin asistencia por
	parte del facilitador.
	\item Correctitud de la Tarea Cuatro: para él cálculo de esta medida, se descompuso la
	última tarea en lo distintos pasos necesarios para completarla correctamente.
	De esta manera, la tarea cuatro se consideró compuesta por 72 elementos:
	\begin{itemize}
		\item Creación de 54 notas.
		\item Selección de 4 instrumentos.
		\item Cambios en la duración de 12 notas.
		\item Reproducción de la música.
		\item Exportación de la música.
	\end{itemize}
	La correctitud de la tarea 4 para cada usuario se define como la razón entre la cantidad de operaciones
	correctamente realizadas y el total de operaciones (72).
\end{itemize}

\subsection{Satisfacción del Usuario}
Factor relacionado a la opinión del usuario sobre su interacción con la aplicación y las interfaces
basadas en reconocimiento del habla en general.
\subsubsection{Métricas Utilizadas}
La opinión del usuario se midió a través de los resultados de la encuesta realizada como parte de
la prueba de usabilidad, posterior a la finalización de las tareas. Las respuestas se
registraron utilizando una escala de Likert \cite{Allen:2007} con valores del 1 al 7.

Se solicitó la opinión del usuario con respecto a:
\begin{itemize}
	\item Las palabras utilizadas.
	\item Los comandos utilizados.
	\item Duración del entrenamiento.
	\item Interfaces por Voz del Usuario.
\end{itemize}




