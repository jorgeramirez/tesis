%!TEX root = ../tesis.tex
\section{Factores Analizados}
\label{sec:factores}

En la secci\'on \ref{sec:medidas-desempenho} se mencionaron algunas medidas de desempe\~no 
utilizadas normalmente para evaluar un sistema de reconocimiento del habla. Estas m\'etricas
resultan adecuadas para medir factores como la precisi\'on o el tiempo de respuesta de un
sistema, pero ignoran al actor principal de una prueba de usabilidad: el usuario.

Teniendo en cuenta que se eval\'ua una interfaz mediante voz del usuario, y de acuerdo a los
objetivos planteados en la secci\'on \ref{sec:objetivos-estudio}, se propone un an\'alisis m\'as
centrado en la interacci\'on entre el usuario y la aplicaci\'on.

Esta secci\'on describe los factores que se tienen en cuenta para el \'analisis a realizarse
en base a los datos recopilados a partir de las pruebas de usabilidad.
Se definen tambi\'en las m\'etricas utilizadas para la cuantificaci\'on de cada factor estudiado.

\subsection{Memoria del Usuario}
\label{sec:memoria-del-usuario}
Factor relacionado a la capacidad de retenci\'on de informaci\'on del usuario.
Se considera interesante su an\'alisis debido a que:
\begin{itemize}
	\item TamTam Listens se trata de una aplicaci\'on completamente nueva para el usuario.
	\item Los comandos de voz deben ser memorizados por el usuario para la interacci\'on con
	la aplicaci\'on.
\end{itemize}
\subsubsection{M\'etrica Utilizada}
La memoria del usuario se mide a trav\'es de los resultados del test de memoria que se lleva
a cabo con cada usuario como parte de la prueba de usabilidad.
El resultado del test de memoria administrado es la cantidad de palabras, de un total de 15,
que el usuario consigue recordar al cabo de 5 repeticiones, siendo 12--13 el resultado promedio
esperado. El resultado del test de memoria se representa como $M$.

\subsection{Correctitud de la aplicaci\'on}
Factor relacionado al comportamiento esperado por parte de un sistema de reconocimiento del
habla. Se busca responder a la pregunta: {?`}Qu\'e tan correctamente se reconocen los comandos 
del usuario?
Los valores de las m\'etricas relacionadas a la correctitud de la aplicaci\'on se calculan
en base a las tareas 3 y 4, por realizarse estas sin asistencia del facilitador.  
\subsubsection{M\'etricas Utilizadas}
\begin{itemize}
	\item Tasa de Aciertos de Comandos: se considera que el sistema reconoce correctamente
	un comando, si el usuario efectivamente ha pronunciado el mismo.

	La tasa de aciertos es la raz\'on entre la cantidad de comandos correctamente reconocidos 
	y la cantidad de comandos correctamente pronunciados por el usuario. 
	La misma se representa como $A$.
	
	Sean:

	\begin{itemize}
		\item $r_{ij}$ n\'umero de reconocimientos correctos del \mbox{comando $i$} pronunciado por el usuario $j$,
		por parte de la aplicaci\'on.
		\item $c_{ij}$ n\'umero de pronunciaciones correctas del \mbox{comando $i$} por parte del usuario $j$.
	\end{itemize}
	La tasa de aciertos del comando $i$ para el usuario $j$ se define como: 

	\begin{equation*}
		a_{{comando}_{ij}}=\frac{r_{ij}}{c_{ij}}
	\end{equation*}


	Sea $N_{j}$ la cantidad total de comandos diferentes pronunciados (correcta o incorrectamente) por el usuario $j$,
	la tasa de aciertos para el usuario  se define como:
	
	\begin{equation*}
		A=a_{{usuario}_j}=\frac{\sum_k^{N_{j}}a_{{comando}_{kj}}}{N_{j}}
	\end{equation*}

	La tasa de aciertos del sistema se obtiene hallando el promedio de los valores correspondientes
	a cada usuario.

	\item Tasa de Error de Comandos: es la raz\'on entre la cantidad de comandos correctos no reconocidos 
	y la cantidad de comandos correctamente pronunciados por el usuario. 
	La misma se representa como $E_1$ y es igual a $1-A$.

\end{itemize}

\subsection{Error Humano}
Factor relacionado a las equivocaciones por parte del usuario en su interacci\'on con
TamTam Listens.
Un comando se considera err\'oneo por parte del usuario si:
\begin{itemize}
	\item El usuario no respeta el orden de las palabras: esto es, pronuncia las palabras de un comando 
	en el orden incorrecto.
	\item El usuario se confunde entre dos comandos: esto es, utiliza palabras correspondientes a m\'as
	de un comando.
	\item El usuario junta dos o m\'as comandos: esto es, pronuncia de una sola vez dos o m\'as comandos.
	\item El usuario duda mientras pronuncia el comando: esto es, titubea o repite una palabra del 
	comando.
	\item El usuario utiliza comandos incompletos: esto es, pronunciando solo algunas palabras del 
	comando.
	\item El usuario utiliza palabras que no forman parte del lenguaje.
	\item El usuario viola alguna otra regla impuesta por el lenguaje de la aplicaci\'on.
\end{itemize}
Los valores de las m\'etricas relacionadas al error humano se calculan
en base a las tareas 3 y 4, por realizarse estas sin asistencia del facilitador. 

\subsubsection{M\'etricas Utilizadas}
\begin{itemize}
	\item Tasa de Error Humano: se define como la raz\'on entre la cantidad de comandos incorrectos
	y la cantidad de comandos pronunciados. Se representa como $E_2$.
	
	Sean:

	\begin{itemize}
		\item $p_{ij}$ n\'umero de pronunciaciones incorrectas del \mbox{comando $i$} por parte del 
		\mbox{usuario $j$.}
		\item $t_{ij}$ n\'umero de pronunciaciones (correctas e incorrectas) del \mbox{comando $i$} por parte del usuario $j$.
	\end{itemize}
	La tasa de error humano del comando $i$ para el usuario $j$ se define como: 

	\begin{equation*}
		e_{{comando}_{ij}}=\frac{p_{ij}}{t_{ij}}
	\end{equation*}

	El c\'alculo de la tasa de error humano de cada usuario se lleva a cabo de forma an\'aloga 
	a la tasa de aciertos.

	\item Cantidad de Errores: sumatoria de errores por usuario, clasificados en
	alguno de los tipos previamente mencionados. Se representa como $E_3$.

\end{itemize}

\subsection{Eficiencia}
Factor relacionado al grado de productividad que logra el usuario de la aplicaci\'on.
\subsubsection{M\'etricas Utilizadas}
\begin{itemize}
	\item Duraci\'on de Tareas Uno y Dos: es la suma del tiempo que le toma al usuario
	terminar la tarea 1 ($T_1$) y el que le toma terminar la tarea 2 ($T_2$). 
	Estas tareas se realizan con asistencia por parte del facilitador. 
	La duraci\'on se mide en minutos. Se representa como $T_{1+2}$.
	\item Duraci\'on de Tareas Tres y Cuatro: es la suma del tiempo que le toma al usuario
	terminar la tarea 3 ($T_3$) y el que le toma terminar la tarea 4 ($T_4$). 
	Estas tareas se realizan sin asistencia por parte del facilitador. 
	La duraci\'on se mide en minutos. Se representa como $T_{3+4}$.
	\item Cantidad de Comandos Utilizados: es la cantidad total de comandos diferentes utilizados
	al menos una vez por el usuario durante las tareas 3 y 4. Se representa como $U$.
	\item Correctitud de la Tarea Cuatro: para el c\'alculo de esta medida, se descompone la
	\'ultima tarea en lo distintos pasos necesarios para completarla correctamente.
	De esta manera, la tarea cuatro se considera compuesta por 72 elementos:
	\begin{itemize}
		\item Creaci\'on de 54 notas.
		\item Selecci\'on de 4 instrumentos.
		\item Cambios en la duraci\'on de 12 notas.
		\item Reproducci\'on de la m\'usica.
		\item Exportaci\'on de la m\'usica.
	\end{itemize}
	La correctitud de la tarea 4 para cada usuario se define como la raz\'on entre la cantidad de operaciones
	correctamente realizadas y el total de operaciones (72). Se representa como $C$.

	Cabe resaltar que cada usuario puede utilizar distintos comandos para llevar a cabo estas operaciones.
	Por ejemplo, un usuario podr{\'\i}a crear las 54 notas una por una, mientras otro usuario podr{\'\i}a
	duplicar las pistas con notas similares para completar la tarea con menos esfuerzo.

\end{itemize}

\subsection{Satisfacci\'on del Usuario}
Factor relacionado a la apreciaci\'on subjetiva del usuario sobre su interacci\'on con la aplicaci\'on y
las interfaces basadas en reconocimiento del habla en general.
\subsubsection{M\'etricas Utilizadas}
La opini\'on del usuario se mide a trav\'es de los resultados de la encuesta que se realiza como parte de
la prueba de usabilidad, posterior a la finalizaci\'on de las tareas. Las respuestas se
registran utilizando una escala de Likert \cite{Allen:2007} con valores del 1 al 7.

Se solicita la opini\'on del usuario con respecto a:
\begin{itemize}
	\item Que tan adecuadas son las palabras utilizadas.
	\item Que tan adecuados son los comandos utilizados.
	\item Que tan adecuada resulta la duraci\'on del entrenamiento.
	\item Que tan frecuentemente utilizar{\'\i}a las interfaces por voz del usuario.
\end{itemize}

Posteriormente, de modo a eliminar el potencial efecto de los estilos de respuesta
de los usuarios \cite{Fischer2010}, se utiliza el m\'etodo de estandarizaci\'on 
por rango \cite{Pagolu2011} para reescalar los resultados de la encuesta.

Siendo:
\begin{itemize}
	\item $min_i$ la respuesta de menor valor del usuario $i$.
	\item $max_i$ la respuesta de mayor valor del usuario $j$.
\end{itemize}

Para cada respuesta $s$ del usuario $i$, el valor ajustado $s_a$ se define como:

\begin{equation*}
s_a=\frac{s-min_i}{max_i-min_i}
\end{equation*}


De esta manera, todas las respuestas pasan a estar en el rango entre 0 y 1.  




