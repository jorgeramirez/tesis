\section{Clasificaci\'on General de los Sistemas de Reconocimiento del Habla}

Diferentes aplicaciones de las tecnolog\'ias del reconocimiento del habla generan, necesariamente, 
restricciones diferentes en el problema y esto lleva a diferentes algoritmos\cite{Jurafsky}. A continuaci\'on se presenta
la clasificaci\'on general de los sistemas de reconocimiento teniendo en cuenta lo mencionado anteriormente:

\begin{itemize}
    \item Reconocimiento de Palabras Aisladas vs Reconocimiento Cont\'inuo: el t\'ermino cont\'inuo significa
        que las palabras se dicen una despu\'es de la otra de manera natural. Sin embargo, en los sistemas
        que reconocen palabras aisladas, las palabras deben estar separadas por pausas.
    \item Dependiente del hablante vs Independiente del hablante: la dependencia del hablante hace referencia
        a que si el sistema debe entrenarse para cada usuario o puede llevar a cabo el reconocimiento del habla
        de usuarios ``extra\~nos'' al sistema.
    \item Tama\~no del vocabulario: dependiendo del tama\~no del vocabulario los reconocedores pueden dividirse
        en sistemas de vocabulario peque\~no (t\'ipicamente menos de 100 palabras) y sistemas de vocabulario grande (generalmente
        entre 5000 y 60000 palabras).
    \item Robustez: hace referencia a la capacidad de los sistemas de llevar a cabo el 
        reconocimiento del habla en condiciones ruidosas o no.
\end{itemize}