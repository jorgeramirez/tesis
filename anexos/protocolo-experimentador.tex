\section{Protocolo del Facilitador}

Este documento representa la gu\'ia a ser utilizada por el facilitador para llevar a cabo la
prueba. Describe los pasos a seguir para la preparaci\'on, realizaci\'on y registro de resultados de una sesi\'on de la prueba en cuesti\'on.

\subsection{Antes de la Prueba}

\begin{enumerate}
    \item Instale el equipamiento de la prueba (computadora port\'atil, \foreign{headset}, materiales, 
    etc.) en el lugar f\'isico escogido para su realizaci\'on. Tenga en cuenta que el nivel de
    interferencia sonora debe ser lo m\'as bajo posible.
    \item Encienda la computadora port\'atil y configurar el equipamiento. Seleccione el \foreign{headset 
    Microsoft LifeChat LX-3000} como dispositivo de sonido, con el volumen de salida al 100\% y el 
    volumen de entrada al 90\%.
    \item Verifique la disponibilidad de los materiales de la prueba:
    \begin{itemize}
        \item Videos instructivos de la aplicaci\'on.
        \item Protocolo del Sujeto de la Prueba.
        \item Gu\'ias para las tareas asistidas y no asistidas.
        \item Manual de la aplicaci\'on.
        \item Formulario para el Sujeto de la Prueba.
    \end{itemize}
    \item Realice una prueba b\'asica de \foreign{TamTam Listens} de modo a verificar su 
    correcto funcionamiento. 
\end{enumerate}

\subsection{Durante la Prueba}

\begin{enumerate}
    \item De la bienvenida al usuario, present\'andose, para proceder a ubicarlo en el lugar
    correspondiente (sentado frente a la computadora port\'atil).
    \item De una breve introducci\'on de \foreign{TamTam Listens} y el objetivo de la prueba. 
    De ser necesario, puede utilizarse el manual de la aplicaci\'on y el documento del estudio de 
    usabilidad como materiales de apoyo.
    \item Explique y realice el test de memoria. Utilice como gu\'ia para este paso el documento titulado 
    ‘Test de Memoria - Protocolo’.
    \item Reproduzca el primer video instructivo introductorio de TamTam Listens en la computadora 
    port\'atil, estando el sujeto observando el monitor y escuchando a trav\'es del \foreign{headset}.
    \item Una vez terminada la reproducci\'on del video instructivo, consulte al sujeto si tiene dudas.
    De existir alguna, acl\'arela.
    \item Entregue al sujeto el manual de aplicaci\'on y perm\'itale leerlo. De existir alguna, acl\'arela.
    \item De inicio a la grabaci\'on, utilizando el script record-screen.sh, de las tareas
    correspondientes a la sesi\'on. Recuerde que deber\'a abrir y cerrar la aplicaci\'on entre 
    cada tarea.
    \item Entregue al sujeto la gu\'ia correspondiente a las tareas asistidas. De una 
    breve explicaci\'on de ser necesaria.
    \item Aguarde en silencio y ligeramente alejado mientras el usuario realiza la tarea 
    asistida n\'umero 1. Durante estas tareas se admiten preguntas del sujeto, las cuales debe responder 
    con la mayor claridad posible. El tiempo l\'imite para la realizaci\'on de esta tarea es de 
    10 minutos.
    \item Reproduzca el segundo video instructivo de \foreign{TamTam Listens} en la computadora 
    port\'atil, estando el sujeto observando el monitor y escuchando a trav\'es del \foreign{headset}. 
    De nuevo, aclare cualquier duda que se presente.
    \item Aguarde en silencio y ligeramente alejado mientras el usuario realiza la tarea asistida 
    n\'umero 2. Durante esta tarea se admiten preguntas del sujeto, las cuales debe responder con la
    mayor claridad posible. El tiempo l\'imite para la realizaci\'on de esta tarea es de 10 minutos.
    \item El sujeto debe realizar la tarea n\'umero 3 ayudado solamente por el manual de la aplicaci\'on 
    y le corresponde anunciar cuando haya finalizado. El tiempo l\'imite para la realizaci\'on de esta 
    tarea es de 10 minutos.
    \item Retire la gu\'ia correspondiente a las tareas asistidas y el manual de la aplicaci\'on. 
    Entregue la gu\'ia correspondiente a la \'ultima tarea. No existe tiempo l\'imite para esta tarea, 
    aunque el sujeto puede decidir no continuar. Haga \'enfasis en este punto.
    \item El sujeto deber\'a realizar la \'ultima tarea  sin asistencia alguna, y le corresponde avisar 
    cuando haya finalizado o se haya dado por vencido.
    \item Entregue el formulario para el sujeto, donde se esperan capturar las 
    impresiones de \'este. Aguarde sin interferir mientras se completa el formulario.
    \item Agradezca la colaboraci\'on del usuario.
\end{enumerate}


\subsection{Despu\'es de la Prueba}

 Por cada tarea asistida realizada por el usuario, deber\'a observar la grabaci\'on correspondiente y registrar los siguientes datos:

\subsubsection{Conteo de Resultados}

Por cada comando que haya pronunciado el sujeto, registre en una tabla dispuesta para tal efecto el resultado obtenido, el cual puede ser uno de los siguientes:
\begin{itemize}
    \item Comando bien pronunciado y reconocido: comportamiento correcto del sistema.
    \item Comando bien pronunciado y no reconocido: se consideran como parte de este grupo los comandos 
    v\'alidos que no hayan sido reconocidos o hayan sido reconocidos incorrectamente (como otro comando).
    \item Comando mal pronunciado y reconocido: palabras o frases fuera del lenguaje, comandos mal 
    pronunciados, u otros errores del usuario que hayan sido reconocidos como un comando v\'alido (falso 
    positivo).
    \item Comando mal pronunciado y no reconocido: comportamiento correcto del sistema. Error del usuario 
    que se reconoce como tal.
\end{itemize}

La precisi\'on del sistema debe calcularse sumando la cantidad de ocurrencias correspondientes a un 
comportamiento incorrecto del sistema y dividiendo sobre la cantidad total de ocurrencias.
Deben considerarse solamente los casos en que se pronunciaron correctamente los comandos, de modo a 
excluir el error humano del c\'alculo.


\subsubsection{An\'alisis del Error Humano}
Para cada comando mal pronunciado registrado en el apartado anterior anotar:
\begin{itemize}
    \item El tiempo de la ocurrencia relativo al inicio de la sesi\'on.
    \item La longitud del comando.
    \item El tipo de comando, de acuerdo a la clasificaci\'on que se define en el manual de aplicaci\'on.
\end{itemize}

Este conteo podr\'ia ayudar a identificar las principales dificultades del usuario con la interfaz.

\subsubsection{Observaciones}

Debe registrarse cualquier detalle adicional que pudiese, a criterio del facilitador, llevar a la 
obtenci\'on de conclusiones \'utiles.

Para la \'ultima tarea, adem\'as de los puntos ya mencionados, registre:
\begin{enumerate}
    \item Si el usuario termin\'o la tarea \\
    La tarea se considera terminada si el usuario sigui\'o una secuencia l\'ogica de comandos que 
    llevar\'ian a su culminaci\'on (m\'as all\'a de los errores del reconocimiento).
    \item Grado de correctitud del resultado \\
    La tarea cuatro se considera compuesta por 72 elementos:
    \begin{itemize}
        \item Creaci\'on de 54 notas.
        \item Selecci\'on de 4 instrumentos.
        \item Cambios en la duraci\'on de 12 notas.
        \item Reproducci\'on de la m\'usica.
        \item Exportaci\'on de la m\'usica.
    \end{itemize}

\end{enumerate}

La correctitud de esta tarea para cada usuario se define como la raz\'on entre la cantidad de 
operaciones correctamente realizadas y el total de operaciones (72).
