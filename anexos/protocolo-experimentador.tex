\section{Protocolo del Experimentador}

Este documento representa la gu\'ia a ser utilizada por el experimentador para llevar a cabo el experimento. 
Describe los pasos a seguir para la preparaci\'on, realizaci\'on y registro de resultados del experimento en cuesti\'on.

\subsection{Antes del Experimento}

\begin{enumerate}
    \item Instale el equipamiento del experimento (computadora port\'atil, headset, materiales, etc.) en el lugar f\'isico escogido para su realizaci\'on. Tenga en cuenta que el nivel de interferencia sonora debe ser lo m\'as bajo posible.
    \item Encienda la computadora port\'atil y configurar el equipamiento. Seleccione el headset Microsoft LifeChat LX-3000 como dispositivo de sonido, con el volumen de salida al 100% y el volumen de entrada al 90%.
    \item Verifique la disponibilidad de los materiales del experimento:
    \begin{itemize}
        \item Screencasts de la aplicaci\'on.
        \item Protocolo del Sujeto del Experimento.
        \item Gu\'ias para las tareas asistidas y no asistidas.
        \item Manual de la aplicaci\'on.
        \item Formulario para el Sujeto del Experimento.
    \end{itemize}
    \item Realice una prueba b\'asica de TamTam Listens de modo a verificar su correcto funcionamiento. 
\end{enumerate}

\subsection{Durante el Experimento}

\begin{enumerate}
    \item De la bienvenida al sujeto del experimento, present\'andose, para proceder a ubicarlo en el lugar correspondiente (sentado frente a la computadora port\'atil).
    \item De una breve introducci\'on de TamTam Listens y el objetivo del experimento. De ser necesario, puede utilizarse el manual de la aplicaci\'on y el documento del experimento como material de apoyo.
    \item Explique y realice el test de memoria. Utilice como gu\'ia para este paso el documento titulado ‘Test de Memoria - Protocolo’.
    \item Reproduzca el primer screencast introductorio de TamTam Listens en la computadora port\'atil, estando el sujeto observando el monitor y escuchando a trav\'es del headset.
    \item Una vez terminada la reproducci\'on del screencast, consulte al sujeto si tiene dudas. De existir alguna, acl\'arela.
    \item Entregue al sujeto el manual de aplicaciòn y perm\'itale leerlo. De existir alguna, acl\'arela.
    \item De inicio a la grabaci\'on, utilizando el script record-screen.sh, de las tareas correspondientes al experimento. Recuerde que deber\'a abrir y cerrar la aplicaci\'on entre cada tarea.
    \item Entregue al sujeto de experimento la gu\'ia correspondiente a las tareas asistidas. De una breve explicaci\'on de ser necesaria.
    \item Aguarde en silencio y ligeramente alejado mientras el sujeto del experimento realiza la tarea asistida n\'umero 1. Durante estas tareas se admiten preguntas del sujeto, las cuales debe responder con la mayor claridad posible. El tiempo l\'imite para la realizaci\'on de esta tarea es de 10 minutos.
    \item Reproduzca el segundo screencast de TamTam Listens en la computadora port\'atil, estando el sujeto observando el monitor y escuchando a trav\'es del headset. De nuevo, aclare cualquier duda que se presente.
    \item Aguarde en silencio y ligeramente alejado mientras el sujeto del experimento realiza la tarea asistida n\'umero 2. Durante esta tarea se admiten preguntas del sujeto, las cuales debe responder con la mayor claridad posible. El tiempo l\'imite para la realizaci\'on de esta tarea es de 10 minutos.
    \item El sujeto debe realizar la tarea n\'umero 3 ayudado solamente por el manual de la aplicaci\'on y le corresponde anunciar cuando haya finalizado. El tiempo l\'imite para la realizaci\'on de esta tarea es de 10 minutos.
    \item Retire la gu\'ia correspondiente a las tareas asistidas y el manual de la aplicaci\'on. Entregue la gu\'ia correspondiente a la \'ultima tarea. No existe tiempo l\'imite para esta tarea, aunque el sujeto puede decidir no continuar. Haga \'enfasis en este punto.
    \item El sujeto deber\'a realizar la \'ultima tarea  sin asistencia alguna, y le corresponde avisar cuando haya finalizado o se haya dado por vencido.
    \item Entregue el formulario para el sujeto del experimento, donde se esperan capturar las impresiones de \'este. Aguarde sin interferir mientras se completa el formulario.
    \item Agradezca la colaboraci\'on del sujeto del experimento.
\end{enumerate}


\subsection{Despu\'es del Experimento}

 Por cada tarea asistida realizada por el sujeto del experimento,deber\'a observar la grabaci\'on correspondiente y completar los siguientes apartados en el formulario del experimentador:

\subsubsection{Conteo de Resultados}

 Los comandos reconocidos por TamTam se clasifican en cuatro grupos:

\begin{itemize}
    \item Comandos Generales
    \begin{itemize}
        \item  Reproducir
        \item Pausar
        \item Parar
        \item Exportar
        \item Salir
        \item Reproducir Pista
        \item Habilitar/Silenciar Pista
    \end{itemize}
    \item Comandos de Configuraci\'on
    \begin{itemize}
        \item Control de Volumen
        \item Control de Tempo
        \item Selecci\'on de Instrumento
    \end{itemize}
    \item Comandos de Navegaci\'on
    \begin{itemize}
        \item Selecci\'on de Pista
        \item Selecci\'on de Comp\'as
        \item Selecci\'on de Tiempo
    \end{itemize}
    \item Comandos de Composici\'on
    \begin{itemize}
        \item Crear M\'usica
        \item Crear Nota
        \item Crear P\'agina
        \item Duplicar P\'agina
        \item Duplicar Nota
        \item Borrar Nota
        \item Editar Nota
    \end{itemize}
\end{itemize}

Por cada comando que haya pronunciado el sujeto, determine el grupo al que pertenece. Luego, en la tabla dispuesta para tal efecto, registre el resultado del comando, el cual puede ser uno de los siguientes:
\begin{itemize}
    \item Comando bien pronunciado y reconocido: comportamiento correcto del sistema.
    \item Comando bien pronunciado y no reconocido: se consideran como parte de este grupo los comandos v\'alidos que no hayan sido reconocidos o hayan sido reconocidos incorrectamente (como otro comando).
    \item Comando mal pronunciado y reconocido: palabras o frases fuera del lenguaje, comandos mal pronunciados, u otros errores del usuario que hayan sido reconocidos como un comando v\'alido (falso positivo).
    \item Comando mal pronunciado y no reconocido: comportamiento correcto del sistema. Error del usuario que se reconoce como tal.
\end{itemize}

Como resultado de este apartado se obtendr\'a la cantidad de ocurrencias para cada par <grupo, resultado>.
La precisi\'on debe calcularse sumando la cantidad de ocurrencias correspondientes a un comportamiento incorrecto del sistema y dividiendo sobre la cantidad total de ocurrencias.

Deben considerarse solamente los casos en que se pronunciaron correctamente los comandos, de modo a excluir el error humano del c\'alculo.


\subsubsection{An\'alisis del Error Humano}
En esta secci\'on, se clasifican los comandos mal pronunciados registrados en el apartado anterior en una de las siguientes categor\'ias:
\begin{itemize}
    \item El usuario no respet\'o el orden de las palabras: esto es, pronunci\'o las palabras de un comando en el orden incorrecto.
    \item El usuario se confundi\'o entre dos comandos: esto es, utiliz\'o palabras correspondientes a m\'as de un comando.
    \item El usuario dud\'o mientras pronunciaba el comando: esto es, titube\'o o repiti\'o una palabra del comando.
    \item El usuario utiliz\'o palabras que no forman parte del lenguaje.
\end{itemize}

Este conteo podr\'ia ayudar a identificar las principales dificultades del usuario con la interfaz.

\subsubsection{Observaciones}

En este apartado debe registrarse cualquier detalle adicional que pudiese, a criterio del experimentador, llevar a la obtenci\'on de conclusiones \'utiles.

Para la \'ultima tarea, adem\'as de los puntos ya mencionados, registre:
\begin{enumerate}
    \item Si el usuario termin\'o la tarea \\
    La tarea se considera terminada si el usuario sigui\'o una secuencia l\'ogica de comandos que llevar\'ian a su culminaci\'on (m\'as all\'a de los errores del reconocimiento).
    \item Grado de correctitud del resultado \\
    Para este punto son de utilidad las siguientes definiciones:
    \begin{itemize}
        \item Inserci\'on: operaci\'on de a\\\'unnadir una nota.
        \item Sustituci\'on: operaci\'on de reemplazar una nota por otra de distinto tono o duraci\'on.
        \item Eliminaci\'on: operaci\'on de eliminar una nota.
    \end{itemize}
\end{enumerate}

El grado se correctitud del resultado se estimar\'a mediante la Distancia de Levenshtein, es decir, la cantidad de inserciones + sustituciones + eliminaciones necesarias para llegar al resultado correcto (menos es mejor).
