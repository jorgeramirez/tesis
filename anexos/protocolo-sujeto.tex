\section{Protocolo del Sujeto de la Prueba}

Este documento representa una gu\'ia a ser utilizada para llevar a cabo la prueba de usabilidad. 
Describe los pasos a seguir para cada etapa de la misma.

\subsection{Presentaci\'on y Test de Memoria}

\begin{enumerate}
    \item En primer lugar, el responsable de la prueba le comentar\'a brevemente acerca de 
    \foreign{TamTam Listens}, la prueba de la cual formar\'a parte y sus objetivos.
    \item Luego de la introducci\'on, realizar\'a un breve test de memoria guiado por el responsable de la prueba. Tiempo Aproximado: 10 minutos
    \newcounter{enumTemp}
    \setcounter{enumTemp}{\theenumi}
\end{enumerate}

\subsection{Videotutorial y Primera Tarea Asistida}
\begin{enumerate}
    \setcounter{enumi}{\theenumTemp}
    \item A continuaci\'on, observar\'a un videotutorial relacionado a la aplicaci\'on. En el mismo, se
     explicar\'an los conceptos b\'asicos y algunos comandos disponibles para componer m\'usica utilizando 
    \foreign{TamTam Listens}. 
    Tiempo Aproximado: 6 minutos
    \item Una vez terminado el video, es momento de preguntas. Consulte cualquier punto que no le haya quedado claro al responsable de la prueba, quien le ayudar\'a a aclarar sus dudas.
    \item El facilitador le entregar\'a el manual de la aplicaci\'on. Este documento servir\'a de ayuda memoria para dar sus primeros con \foreign{TamTam Listens}. Lea el manual y consulte cualquier duda con el responsable de la prueba.
        Tiempo aproximado: 10 minutos
    \item Hora de poner a prueba lo aprendido. Se le entregar\'an las gu\'ias correspondientes a las tareas asistidas. Con la ayuda del manual, intente realizar la primera tarea. En caso de duda, consulte al facilitador. Como se trata de su primera vez usando TamTam, la tarea ser\'a muy simple.
    Tiempo Aproximado: 10 minutos
    \setcounter{enumTemp}{\theenumi}
\end{enumerate}

\subsection{Videotutorial y otras Tareas Asistidas}

\begin{enumerate}
    \setcounter{enumi}{\theenumTemp}
    \item Seguido, observar\'a otro videotutorial de \foreign{TamTam Listens}. Se introducir\'an algunos comandos adicionales para controlar la aplicaci\'on. Tiempo Aproximado: 5 minutos
    \item Momento de realizar la tarea n\'umero 2. Esta tarea le permitir\'a practicar los comandos adicionales. Utilice el manual para realizar la tarea. En caso de duda, consulte al facilitador.
    Tiempo Aproximado: 10 minutos
    \item A continuaci\'on deber\'a realizar la tarea n\'umero 3. Esta vez, deber\'a realizarla sin hacer preguntas al responsable de la prueba.
    Tiempo Aproximado: 10 minutos
    \item Al terminar la tarea 3, avise al responsable de la prueba. \'el retirar\'a la gu\'ia correspondiente. Tambi\'en deber\'a entregar el manual de la aplicaci\'on. La siguiente ser\'a la \'ultima tarea, por lo que deber\'a realizarla sin ayuda.
    \setcounter{enumTemp}{\theenumi}
\end{enumerate}

\subsection{Tarea Final sin Asistencia}

\begin{enumerate}
    \setcounter{enumi}{\theenumTemp}
    \item Ya casi termina. Para esta tarea, la n\'umero 4, no se preocupe del tiempo. Lo importante es terminar el trabajo de la mejor manera. 
    \item El resultado de la tarea 4, si lo hizo bien, ser\'a una sencilla m\'usica infantil bastante conocida. Reproduzca su obra. ¿Reconoce la melod\'ia?
    \item Al terminar, o si decide rendirse (no lo haga sin haberlo intentado mucho, por favor), avise al responsable de la prueba.
    \setcounter{enumTemp}{\theenumi}
\end{enumerate}

\subsection{Formulario y Fin de la Prueba}

\begin{enumerate}
    \setcounter{enumi}{\theenumTemp}
    \item Por \'ultimo, el facilitador le entregar\'a un formulario con unas cuantas preguntas acerca de su experiencia con \foreign{TamTam Listens}. T\'omese su tiempo, consulte de ser necesario, y por favor responda con sinceridad: su opini\'on es muy importante para la prueba.
    \item Lleg\'o al final. Muchas gracias por haber colaborado con el estudio de usabilidad! 
\end{enumerate}
