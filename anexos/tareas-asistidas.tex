\section{Tareas Asistidas}

Estas tareas tienen como objetivo ayudarle a aprender como utilizar TamTam Listens.
Intente seguir la secuencia de pasos. Si tiene dudas, consulte el manual de la aplicaci\'on.
Si contin\'ua con dificultades, puede hacer preguntas al responsable del experimento.

\subsection{Tarea 1}
\begin{enumerate}
    \item Crear una nueva m\'usica (debe obtenerse una sola partitura completamente vac\'ia)
    \item En la pista uno, seleccionar como instrumento la Guitarra El\'ectrica
    \item En la pista dos, seleccionar como instrumento el Clarinete
    \item En la pista cuatro, seleccionar como instrumento el Tri\'angulo
    \item En el comp\'as uno de la pista uno, dibujar tres notas de duraci\'on 4
    \item En el comp\'as cuatro de la pista uno, dibujar dos notas de duraci\'on 4
    \item Reproducir
    \item Salir
\end{enumerate}

\subsection{Tarea 2}
\begin{enumerate}
    \item Crear una nueva m\'usica
    \item En el comp\'as uno de la pista uno, dibujar dos notas de duraci\'on 6
    \item Cambiar el inicio de la \'ultima nota creada al tiempo 9.
    \item En el comp\'as tres de la pista cuatro, dibujar una nota de duraci\'on 4
    \item Seleccionar la primera nota del comp\'as uno
    \item Reemplazar la nota seleccionada por otra
    \item Duplicar la pista uno en la tres
    \item Duplicar la partitura
    \item Eliminar las notas de la pista tres
    \item Reproducir la pista uno
    \item Reproducir todo
    \item Exportar el trabajo realizado
    \item Salir
\end{enumerate}

A partir de esta tarea no se admiten preguntas sobre comandos al responsable del experimento.
Intente seguir la gu\'ia, usando el manual de la aplicaci\'on en caso de dudas.

\subsection{Tarea 3}

\begin{enumerate}
    \item Crear una nueva m\'usica
    \item En la pista uno, seleccionar como instrumento la Flauta
    \item En la pista dos, seleccionar como instrumento el Piano
    \item En la pista cuatro, seleccionar como instrumento la Guitarra El\'ectrica
    \item En la pista cinco, seleccionar como instrumento la Bater\'ia
    \item En el comp\'as uno de la pista uno, crear un nota de duraci\'on 4
    \item En el comp\'as cuatro de la pista uno, crear dos notas de duraci\'on 6
    \item Copiar el contenido de la pista uno en la pista dos
    \item En el comp\'as cuatro de la pista dos, seleccionar la segunda nota y disminuir su duraci\'on.
    \item Borrar la \'unica nota del comp\'as uno de la pista dos
    \item Crear una nueva partitura en blanco
    \item En el comp\'as uno de la pista cuatro, crear una nota de duraci\'on 10
    \item En el comp\'as uno de la pista cinco, crear dos notas
    \item En el comp\'as cuatro de la pista cinco, crear tres notas
    \item Reproducir todo
    \item Exportar el trabajo realizado
    \item Salir
\end{enumerate}
