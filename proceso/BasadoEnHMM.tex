%!TEX root = ../tesis.tex
%!TEX root = ../tesis.tex
\section{Modelos Ocultos de Markov}
\label{sec:hmm}

Para introducir el concepto de modelo oculto de Markov (\gls{hmm} por sus siglas en ingl\'es), 
en el cual se basa el proceso de reconocimiento del habla que se busca explicar, 
se utilizar\'a un ejemplo cl\'asico de la literatura relacionada con el tema.
El ejemplo se denomina \emph{El modelo de la urna y la pelota} \cite{Rabiner89atutorial}:

\begin{figure}[H] 
\centering
\includegraphics[width=0.8\textwidth]{./graphics/urnas.png}
\caption{Representaci\'on gr\'afica del modelo de la urna y la pelota \cite{LleidaModelado}.}
\label{figure:urnas}
\end{figure}

\begin{quote}
	Se asume que hay N urnas (grandes) de vidrio en una habitaci\'on. Dentro de cada urna hay un gran 
	n\'umero de pelotas de colores. Un genio est\'a en la habitaci\'on y, de acuerdo a un proceso aleatorio, elige una urna inicial.

	De esta urna extrae una pelota de manera aleatoria y su color se anota como la observaci\'on, 
	pues el observador desconoce la urna de donde sali\'o la pelota.
	La pelota se coloca de nuevo en la urna y se repiten la selecci\'on de la urna y la pelota respectivamente.
	
	Este proceso genera una secuencia aleatoria de colores.
\end{quote}
\vspace*{1\baselineskip}

En el caso mencionado, pueden distinguirse dos procesos estoc\'asticos:
\begin{itemize}
	\item Del primer proceso se obtiene como salida una secuencia de urnas. Sin embargo, el observador
	no puede visualizar esta secuencia, es decir, la misma permanece oculta.
	\item Del segundo proceso se obtiene como salida una secuencia de colores. La probabilidad de observar
	un color depende de la urna seleccionada previamente, debido a que la cantidad de pelotas de un determinado
	color var{\'\i}a en cada urna.
\end{itemize}

Este sencillo sistema puede modelarse como un modelo oculto de Markov definiendo los siguientes par\'ametros:
\begin{enumerate}
	\item El n\'umero de urnas, a las cuales se denomina estados del modelo.
	\item La cantidad de colores posibles de las pelotas en las urnas, a los cuales se 
	denomina s{\'\i}mbolos observables.
	\item La funci\'on que determina la transici\'on entre urnas.
	\item La funci\'on que determina la elecci\'on de una pelota de determinado color, dada una urna.
	\item La funci\'on que determina la elecci\'on de la urna inicial.
\end{enumerate}

Con la ayuda de este ejemplo, podemos definir formalmente un \gls{hmm}. Un modelo oculto de Markov es un 
aut\'omata finito estoc\'astico entrenable \cite{KouemouHistory2011}, que implica un doble proceso estoc\'astico:

\begin{itemize}
\item El primer proceso, que produce una secuencia de estados, no observable. 
La secuencia de estados que produce permanece oculta.
\item El segundo proceso produce una secuencia de observaciones, donde la probabilidad de 
una observaci\'on est\'a dada por una funci\'on definida para cada estado correspondiente al proceso anterior.
\end{itemize}

Un modelo oculto de Markov puede ser caracterizado mediante los siguientes elementos \cite{Rabiner89atutorial}:

\begin{enumerate}
	\item $N$, el n\'umero de estados del modelo. Se representan los estados individuales como: 
		\begin{align}
			S=\{S_1,S_2,\ldots,S_N\}\label{eq:hmmS}
	\end{align}

	\item $M$, el n\'umero de s{\'\i}mbolos observables por estado. Se representan los s{\'\i}mbolos individuales 
		como: 
		\begin{align}
			V=\{v_1,v_2,\ldots,v_M\}\label{eq:hmmV}
	\end{align}

	\item La distribuci\'on de probabilidad de transici\'on de estados $A = \left\{a_{ij}\right\}$.
		Siendo $q_t$ el estado del modelo generado en el tiempo $t$, puede definirse $a_{ij}$ como:

		\begin{align}
			a_{ij} = P[q_{t+1} = S_j \mid q_t = S_i], & & 1 \leq i,j & \leq N\label{eq:hmmA}
		\end{align}

	\item La distribuci\'on de probabilidad de los s{\'\i}mbolos observables en el estado $j$, $b_j(v_k)$. 
		Esta distribuci\'on es una funci\'on de la observaci\'on $v_k$, definida en cada estado.
		Siendo $q_t$ el estado del modelo generado en el tiempo $t$:

		\begin{align}
			b_j(v_k) = P[v_k \text{ en el momento } t \mid q_t = S_j], & & 1 \leq j & \leq N \label{eq:hmmB}
			\\& & 1 \leq k & \leq M \nonumber
		\end{align}

	\item La distribuci\'on de probabilidad del estado inicial $\pi=\left\{\pi_i\right\}$, donde:
		\begin{align}
			\pi_i = P[q_1=S_i], & & 1 \leq i \leq N \label{eq:hmmPI}
		\end{align}
\end{enumerate}

\begin{figure}[H] 
\centering
\includegraphics[width=0.7\textwidth]{./graphics/hmm.png}
\caption{Representaci\'on gr\'afica de un modelo oculto de Markov.}
\label{figure:hmm}
\end{figure}

\section[El Reconocimiento del Habla como Problema Estad{\'\i}stico]
{El Reconocimiento del Habla como Problema \\ Estad{\'\i}stico}

Para un lenguaje $L$ y una entrada ac\'ustica $X$, el problema del reconocimiento del habla puede definirse 
como \cite{Jurafsky}:

\begin{quote}
\emph{La b\'usqueda de la oraci\'on m\'as probable perteneciente al lenguaje L, dada la entrada ac\'ustica X.}
\end{quote}

La secuencia de observaciones ac\'usticas $O$ puede representarse como:

\begin{align}
O = o_1,o_2,o_3,\ldots,o_T\label{eq:asrO}
\end{align}

donde la se\~nal de voz fue dividida en $T$ muestras de igual duraci\'on.

La oraci\'on de salida, a su vez, puede representarse como:

\begin{align}
\hat{W}  = w_1,w_2,w_3,\ldots,w_M\label{eq:asrW}
\end{align}

donde la cadena est\'a compuesta por $M$ palabras.

De esta manera, la definici\'on introducida anteriormente puede expresarse matem\'aticamente como:

\begin{align}
\hat{W} = \argmax_{W \in L} P(W|O)
\end{align}

Usando la Regla de Bayes la expresi\'on anterior puede reescribirse como:

\begin{align}
\hat{W} = \argmax_{W \in L} \frac{P(O|W)P(W)}{P(O)}
\end{align}

Como se desea obtener la oraci\'on con mayor probabilidad dada una entrada ac\'ustica, la entrada es
la misma para todas las oraciones evaluadas y su probabilidad de ocurrencia $P(O)$ se mantiene constante.
Puesto de otra manera, el t\'ermino $P(O)$ es independiente de $W$, por lo cual puede despreciarse. 

Por tanto:

\begin{align}
\hat{W} = \argmax_{W \in L} P(O|W)P(W)
\end{align}

El primer t\'ermino representa la probabilidad de una entrada ac\'ustica dada una secuencia de palabras, tambi\'en
conocida como verosimilitud de observaci\'on o modelo ac\'ustico. El segundo t\'ermino es la probabilidad 
\foreign{a priori} de ocurrencia de una secuencia de palabras, tambi\'en conocida como probabilidad previa o 
modelo de lenguaje. Esto es:

\begin{align}
\hat{W} = \argmax_{W \in L} \overbrace{P(O|W)}^\text{M. ac\'ustico}\overbrace{P(W)}^\text{M. de lenguaje}
\label{eq:asrFundamental}
\end{align}

Esta ecuaci\'on es el fundamento del enfoque estad{\'\i}stico al problema del reconocimiento del habla, base de los
sistemas \mbox{modernos \cite{RabinerStatistical2006}}.

\begin{figure}[H] 
\centering
\includegraphics[width=0.8\textwidth]{./graphics/proceso.png}
\caption{Proceso b\'asico del reconocimiento del habla. Traducido a partir de \cite{VerenichASR}.}
\label{figure:proceso}
\end{figure}

La figura \ref{figure:proceso} ilustra de modo general la arquitectura de un sistema de reconocimiento del habla.
El proceso b\'asico del reconocimiento del habla, puede descomponerse en dos etapas o fases, cada una de las cuales
recibe una entrada (o varias) y produce una salida determinada.

%Figura Jurafsky
\begin{itemize}
\item La primera fase o \emph{extracci\'on de caracter{\'\i}sticas} tiene como objetivo caracterizar la se\~nal
de voz para obtener una representaci\'on adecuada para el decodificador. Para tal efecto, produce vectores de
caracter{\'\i}sticas espectrales a partir del sonido que recibe como entrada.
\item La segunda fase o \emph{decodificaci\'on} tiene como objectivo producir la secuencia de palabras m\'as probable
dados los vectores de caracter{\'\i}sticas resultantes de la fase anterior. Para ello se sirve un modelo ac\'ustico, un
modelo de lenguaje y un algoritmo de decodificaci\'on.
\end{itemize}

Las siguientes secciones explican de manera m\'as detallada los conceptos y algoritmos relacionados a cada fase.
Tanto el modelo ac\'ustico como el modelo de lenguaje pueden requerir una fase de entrenamiento, 
previa a la utilizaci\'on del sistema de reconocimiento del habla.
Por motivos de claridad, los detalles de esta fase se presentan en \'ultimo lugar.

\subsection{Fase 1: Feature Extraction}
\label{sec:featureExtraction}

%!TEX root = ../tesis.tex
\section{Fase 2: Decodificaci\'on}
\label{sec:decoding}

En la fase de decodificaci\'on se utiliza un algoritmo decodificador, que depende de un modelo de lenguaje 
y el modelo ac\'ustico, para obtener la secuencia de palabras m\'as probable en base a los vectores 
de caracter{\'\i}sticas que se calcularon en la etapa anterior. 
Esta secci\'on describe los elementos involucrados en esta transformaci\'on, la cual constituye el paso
final del proceso simplificado del reconocimiento del habla que se pretende presentar.

\subsection{Modelo de Lenguaje}
Un modelo de lenguaje busca predecir la probabilidad de una secuencia de palabras pertenecientes a un lenguaje.

Formalmente, sea $V$ el conjunto finito de palabras que componen un lenguaje, tambi\'en conocido como 
vocabulario, y $V^\dag$ el conjunto infinito de oraciones que pueden formarse con palabras pertenecientes 
al vocabulario.

Un modelo de lenguaje \cite{CollinsLanguage} consiste en un conjunto finito $V$ y una funci\'on 
de probabilidad $P(x_1,x_2,\ldots,x_n)$ tal que:
\begin{enumerate}

\item $\forall (x_1,x_2,\ldots,x_n) \in V^\dag, P(x_1,x_2,\ldots,x_n) \ge 0$

\item $\displaystyle \sum_{(x_1,x_2,\ldots,x_n) \in V^\dag} P(x_1,x_2,\ldots,x_n) = 1$
\end{enumerate}


En resumen, un modelo de lenguaje define la probabilidad de ocurrencia de una secuencia de palabras
$x_1,x_2,\ldots,x_n$ para un lenguaje dado.

La t\'ecnica basada en n-gramas es la predominante para construir modelos de lenguaje, 
debido a su simplicidad y efectividad \cite{GaoComparative2010}.
Sea $W^L_1$ una cadena de $L $ palabras pertenecientes a un vocabulario $V$ dado. 
Un modelo de lenguaje basado en n-gramas asigna la probabilidad a $W^L_1$ de acuerdo a:

\begin{align}
p(W^L_1) = \displaystyle \prod^L_{i = 1} w_i \mid w^{i - 1}_1 \approx \displaystyle \prod^L_{i = 1} w_i \mid w^{i - 1}_{i - n + 1}
\end{align}

Esta aproximaci\'on se basa en la suposici\'on de Markov de que cada palabra depende solo de las $n - 1$ palabras precedentes, lo cual disminuye significativamente la complejidad del c\'alculo de la probabilidad que se busca para
secuencias de gran longitud.

A modo de ejemplo, se propone la tarea de calcular la probabilidad de la frase \mbox{``El hombre corre''}
utilizando un modelo basado en bigramas ($n=2$).

Sean:
\begin{itemize}
 	\item $\text{\textless} s\text{\textgreater}$ un s{\'\i}mbolo especial utilizado para indicar el inicio 
 	de la secuencia de palabras.
  	\item $\text{\textless} /s\text{\textgreater}$ un s{\'\i}mbolo especial utilizado para indicar el fin 
  	de la secuencia de palabras.
\end{itemize} 

Aplicando la f\'ormula presentada anteriormente, se tiene que:
\begin{equation*}
p(\text{el, hombre, corre}) = p(el \mid \text{\textless} s\text{\textgreater}) \, 
p(\emph{\text{hombre}} \mid el) \, p(corre \mid \emph{\text{hombre}}) \, 
p(\text{\textless} /s\text{\textgreater} \mid corre)
\end{equation*}


Otro tipo de modelo de lenguaje es el basado en una gram\'atica. Las gram\'aticas tienen como ventaja que pueden
utilizarse sin entrenamiento previo. Su principal desventaja est\'a en la dificultad de definir gram\'aticas formales
para lenguajes complejos \cite{Wang2000}.

Un modelo de lenguaje basado en una gram\'atica tradicional asigna una probabilidad de 0 o 1 a cualquier secuencia,
dependiendo de si esta puede o no derivarse a partir de las reglas de la gram\'atica. Sin embargo, si se cuenta con
datos de entrenamiento, es posible asignar una probabilidad a cada regla de modo a mejorar 
la estimaci\'on \cite{huang-handbook10}.

A modo de ejemplo, la siguiente gram\'atica podr{\'\i}a utilizarse como modelo de lenguaje para un sistema simple
de reconocimiento del habla para el acceso a informaci\'on:

\begin{bnf*}
\bnfprod{pregunta}
{\bnfts{Cu\'al} \bnfsp \bnfts{es} \bnfsp \bnfts{la} \bnfpn{info} \bnfsp  \bnfts{en} \bnfsp \bnfpn{ciudad}} \\
\bnfprod{info}
{\bnfts{temperatura} \bnfor \bnfts{presi\'on atmosf\'erica} \bnfor \bnfts{hora}} \\
\bnfprod{ciudad}
{\bnfts{Par{\'\i}s} \bnfor \bnfts{Nueva York} \bnfor \bnfts{Roma}}
\end{bnf*}

\subsection{Modelo Ac\'ustico}
El modelo ac\'ustico permite estimar el t\'ermino $P(O|W)$ de la ecuaci\'on \ref{eq:asrFundamental}, es decir,
la probabilidad de una entrada ac\'ustica dada una secuencia de palabras.
El mismo representa un factor cr{\'\i}tico para la precisi\'on de los resultados obtenidos, y puede decirse que
se trata del componente central de cualquier sistema de reconocimiento del habla \cite{huang-handbook10}.

Es en este elemento del reconocedor en el cual se aplica el concepto de modelos ocultos de Markov que se present\'o
anteriormente. Un modelo ac\'ustico est\'a compuesto por varios modelos ocultos de Markov, cada uno con
los siguientes par\'ametros:


\begin{enumerate}[A)]
	\item \textbf{Estados}


	Un fonema es la unidad b\'asica de sonido capaz de alterar el significado de una palabra \cite{Armbruster2003}.
	Los fonemas a su vez pueden dividirse en unidades subfon\'eticas, las cuales resultan m\'as adecuadas para
	caracterizar las transiciones entre fonemas propias del habla.

	A menudo se elige dividir los fonemas en tres partes para los sistemas de reconocimiento del habla.
	De esta manera, la primera parte depende del fonema anterior, la segunda representa al fonema en cuesti\'on
	y la tercera depende del fonema siguiente \cite{CMUConcepts}.

	Estas unidades fon\'eticas, fonemas o subfonemas, corresponden a los estados del modelo oculto de Markov.
	El decodificador permite obtener la secuencia de estados m\'as probable dada una secuencia de observaciones.
	Esta secuencia de estados $Q$ puede representarse como:

	\begin{align}
		Q = q_1,q_2,q_3,\ldots,o_T\label{eq:hmmQ}
	\end{align}

	Donde:
	\begin{itemize}
		\item $T$ es el n\'umero de observaciones.
		\item $q_1,q_2,q_3,\ldots,q_T \in S$ es la secuencia de estados.
	\end{itemize}

	La salida del decodificador supone un problema, teniendo en cuenta que el resultado deseado es la secuencia 
	de palabras $W$ de la ecuaci\'on \ref{eq:asrW}, no la de fonemas $Q$.

	El inconveniente anterior se soluciona mediante el diccionario fon\'etico, el cual contiene correspondencias entre palabras y secuencias de fonemas. El mismo forma parte del modelo ac\'ustico \cite{huang-handbook10}.

	\item \textbf{Observaciones}


	Las caracter{\'\i}sticas espectrales de la onda sonora representan los s{\'\i}mbolos observables del habla.
	Los vectores de caracter{\'\i}sticas que se calcularon en la etapa anterior corresponden a las observaciones
	del modelo oculto de Markov.

	Esto es, siendo $O$ la secuencia de observaciones de la ecuaci\'on \ref{eq:asrO} 
	y $V$ el conjunto de s{\'\i}mbolos observables de la ecuaci\'on \ref{eq:hmmV}, $o_1,o_2,o_3,\ldots,o_T \in V$.

	\item \textbf{Distribuciones de Probabilidad}

	Sea $o_t$ el vector de caracter{\'\i}sticas que se observa en el tiempo $t$, es decir, uno de los elementos
	de la ecuaci\'on \ref{eq:asrO}.
	La probabilidad de una observaci\'on como consecuencia de la pronunciaci\'on de un determinado fonema 
	corresponde a la funci\'on $b_j(v_k)$ del modelo oculto de Markov, definida en la ecuaci\'on \ref{eq:hmmB}.
	Por tanto, la probabilidad de observaci\'on de $o_t$ est\'a dada por $b_j(o_t)$ 

	Como un vector de caracter{\'\i}sticas puede tomar un gran n\'umero de valores, los m\'etodos utilizados con
	mayor frecuencia en la actualidad coinciden en tratarlo como una variable continua.

	Una \gls{fdp} de una variable aleatoria continua describe la probabilidad relativa seg\'un la cual 
	dicha variable aleatoria tomar\'a determinado valor \cite{Evans2011}.

	El m\'etodo m\'as extendido para calcular $b_j(o_t)$ se basa en funciones Gaussianas para definir 
	una \gls{fdp} \cite{Jurafsky}.
	La versi\'{o}n simple del m\'etodo Gaussiano asume que los valores del vector de observaciones $o_t$ presentan una distribuci\'on normal. A continuaci\'on, se presenta la definici\'on formal 
	de $b_j(o_t)$ para este m\'etodo.

	Sean:

	\begin{itemize}
		\item $\mu_j$ el vector media de la curva Gaussiana.
		\item $\Sigma_j$ la matriz covarianza de la curva Gaussiana.
	\end{itemize}

	\begin{align}
    	b_j(o_t) = \frac{1}{\sqrt{(2\pi)|\Sigma_j|}}e^{[(o_t-\mu_j)'\Sigma_j^{-1}(o_t-\mu_j)]}\label{eq:hmmGaussian}
	\end{align}

	En la pr\'actica, la mayor{\'\i}a de los reconocedores utiliza m\'as de una funci\'on Gaussiana, combinando
	los valores mediante una t\'ecnica conocida como mezclas Gaussianas \cite{huang-handbook10}.

	Existen adem\'as otros m\'etodos para definir la probabilidad de observaci\'on, como la cuantizaci\'on
	vectorial \cite{Burton1983} y las redes neuronales \cite{KristineApplying1995}.

	En general, los par\'ametros probabil{\'\i}sticos del \gls{hmm} se calculan durante la fase de entrenamiento,
	antes de la utilizaci\'on del sistema de reconocimiento del habla. Estos valores incluyen:

	\begin{itemize}
		\item La probabilidad de observar un determinado fonema al inicio de la oraci\'on: $\pi_i$.
		\item La probabilidad de transici\'on entre fonemas: $a_{ij}$.
		\item Los par\'ametros relacionados a $b_j(v_k)$. 

		En el caso del m\'etodo Gaussiano simple: los par\'ametros $\mu_j$ y $\Sigma_j$ de la 
		ecuaci\'on \ref{eq:hmmGaussian}.
	\end{itemize}

\end{enumerate}

\begin{figure}[H] 
\centering
\includegraphics[width=0.5\textwidth]{./graphics/hmm_palabra.png}
\caption{Posible representaci\'on ac\'ustica de la palabra ``mes''. Basado en \cite{Jurafsky}.}
\label{figure:hmm-palabra}
\end{figure}

La cantidad de modelos ocultos de Markov depende del enfoque del modelo ac\'ustico, pudiendo darse 
que \cite{Livescu2012}:

\begin{itemize}
 	\item El modelo define un \gls{hmm} por palabra:


 		Este tipo de modelo requiere de ejemplos de pronunciaci\'on de cada palabra del lenguaje para 
 		su entrenamiento, por lo cual es raramente utilizado en la pr\'actica.
 	\item El modelo define un \gls{hmm} por fonema:


 		Este tipo de modelo permite el reconocimiento
 		de palabras para las cuales no se cuenta con ejemplos de pronunciaci\'on.
 	\item El modelo define un \gls{hmm} por fonema, dependiendo del contexto:


 		El sonido particular de un fonema puede variar de acuerdo al fonema anterior y al siguiente.
 		Este tipo de modelo define un \gls{hmm} para cada una de esas variaciones, y es utilizado
 		en la mayor{\'\i}a de los sistemas de reconocimiento del habla \cite{Odell95theuse}.
 \end{itemize}  

\subsection{Espacio de B\'usqueda}
El objetivo de la fase de decodificaci\'on puede ser considerado esencialmente como un problema de 
b\'usqueda \cite{huang-handbook10}.
El algoritmo decodificador intenta encontrar la secuencia de estados m\'as probable dada la secuencia de
vectores de caracter{\'\i}sticas que recibe como entrada, en un \'unico \gls{hmm} que constituye su 
espacio de b\'usqueda.

Tomando como ejemplo el modelo de la urna y la pelota de la secci\'on \ref{sec:hmm}, el algoritmo decodificador
busca la secuencia de urnas m\'as probable dada la secuencia de colores que se observ\'o.

Teniendo en cuenta que el modelo ac\'ustico est\'a constituido por varios modelos ocultos de Markov, se plantea
la necesidad de definir un \'unico \gls{hmm} para el algoritmo decodificador. Suponiendo que el modelo ac\'ustico
define un \gls{hmm} por fonema, esto puede hacerse de la siguiente manera:

\begin{enumerate}
	\item Se concatenan los modelos ocultos de Markov de los fonemas de manera a obtener un \gls{hmm} 
	por cada palabra. 
	La probabilidad de transici\'on entre fonemas est\'a dada por el diccionario fon\'etico.
	\item Se concatenan los modelos ocultos de Markov de las palabras de manera a obtener un \'unico \gls{hmm} que
	constituye el espacio de b\'usqueda. La probabilidad de transici\'on entre palabras est\'a dada por el modelo
	de lenguaje.
\end{enumerate}

\begin{figure}[H] 
\centering
\includegraphics[width=0.5\textwidth]{./graphics/espacio.png}
\caption{Espacio de b\'usqueda para un lenguaje simple de cuatro palabras. Traducido a partir de \cite{RenalsSearch}.}
\label{figure:espacio-busqueda}
\end{figure}



\subsection{Algoritmo de Viterbi}
Una vez calculados los vectores de caracter{\'\i}sticas, y habiendo construido el espacio de b\'usqueda, 
se cuenta con todos los elementos necesarios para el algoritmo decodificador. Un algoritmo frecuentemente 
utilizado para la decodificaci\'on se denomina algoritmo de Viterbi.

El algoritmo de Viterbi, tomando como entrada una secuencia de observaciones y un \'unico aut\'omata, 
encuentra el camino \'optimo a trav\'es del aut\'omata, esto es, la mejor secuencia de estados. 
La descripci\'on y el pseudoc\'odigo que se presentan en esta secci\'on
est\'an basados principalmente en \cite{Jurafsky, Rabiner89atutorial}.

M\'as formalmente, se busca la mejor secuencia de estados $Q = (q_1,q_2,\ldots,q_T)$ 
dados una secuencia de observaciones $O = (o_1,o_2,\ldots,o_T)$ y un modelo oculto de Markov, $\lambda$.

El algoritmo utiliza una matriz de probabilidades $viterbi$, donde cada celda $viterbi[i,t]$ 
contiene la probabilidad del mejor camino teniendo en cuenta las $t$ primeras observaciones y 
terminando en el estado $i$ del modelo. Esto es:

\begin{align}
	viterbi[i,t] = \displaystyle \max_{q_1,q_2,\ldots,q_{t - 1}} P(q1,q2,\ldots,q_{t - 1},
		q_t = i,o_1,o_2,\ldots,o_t \mid \lambda) 	
\end{align} 

Para calcular los valores de $viterbi[i,t]$, el algoritmo de Viterbi asume la invariante de la 
programaci\'on din\'amica. Esto es, se asume que si el mejor camino para una secuencia de observaciones 
pasa por un estado $q_i$, entonces este camino incluye el mejor camino hasta $q_i$ inclusive. 
Este supuesto, aunque no siempre sea correcto, permite descomponer el problema y simplificar su soluci\'on,
mediante la siguiente relaci\'on de recurrencia:

\begin{align}
	viterbi[i,t] = \displaystyle \max_i (viterbi[i,t-1]a_{i,j})b_j(o_t)
\end{align}

\begin{algorithm}[H]
\caption{Algoritmo de Viterbi} \label{viterbi}
\begin{algorithmic}[1]
\REQUIRE $observaciones$ de longitud $T$, $grafo\mbox{-}estados$.
\ENSURE $estados$, el mejor camino.
\STATE $num\mbox{-}estados \leftarrow$ CANTIDAD-DE-ESTADOS($grafo\mbox{-}estados$) 
\STATE Crear una matriz de probabilidades $viterbi[num\mbox{-}estados, T]$
\FOR{cada estado $s$ desde $0$ hasta $num\mbox{-}estados$}
	\STATE $viterbi[s,0] = \pi_s$
\ENDFOR
\FOR{cada paso $t$ desde $0$ hasta $T - 1$}
	\FOR{cada estado $s$ desde $0$ hasta $num\mbox{-}estados$}
		\FOR{cada transici\'on $s$ desde s especificada por el $grafo\mbox{-}estados$}
		\STATE $nuevo\mbox{-}puntaje \leftarrow viterbi[s,t] * a[s,s'] * b_{s'}[o_t]$
		\IF{$viterbi[s',t+1] = 0 \parallel nuevo\mbox{-}puntaje > viterbi[s',t+1]$}
			\STATE $viterbi[s',t+1] \leftarrow nuevo\mbox{-}puntaje$
			\STATE $puntero\mbox{-}retroceso[s',t+1] \leftarrow s$
		\ENDIF  
		\ENDFOR
	\ENDFOR
\ENDFOR
\STATE $estados \leftarrow$ retroceso desde la celda con mayor valor en la \'ultima columna de $viterbi[]$
\\ \COMMENT{Usando $puntero-retroceso$}.
\RETURN $estados$
\end{algorithmic}
\end{algorithm}

El algoritmo de Viterbi procesa por completo el tiempo $t$ antes de continuar con el tiempo $t + 1$, por
lo cual se dice que es un algoritmo s{\'\i}ncrono en el tiempo \cite{huang-handbook10}.

En la pr\'actica, para vocabularios extensos, el espacio de b\'usqueda resulta demasiado grande.
Como soluci\'on a este problema suelen podarse los caminos poco probables en cada tiempo $t$,
utilizando una t\'ecnica conocida como b\'usqueda en haz \cite{Jurafsky}.

\subsection{Algoritmo A*}

Otro algoritmo que puede utilizarse para la decodificaci\'on es el algoritmo A*. Este algoritmo
utiliza una funci\'on heur{\'\i}stica, que debe definirse, para elegir los caminos a expandir
durante el proceso de b\'usqueda \cite{Russell2003Solving}. 
El problema de definir una heur{\'\i}stica adecuada a\'un no ha sido resuelto, aunque existen soluciones propuestas.

El algoritmo A* puede calcular probabilidades correspondientes al tiempo $t + 1$ sin haber completado 
las del tiempo $t$, por lo cual es considerado un algoritmo as{\'\i}ncrono en el tiempo.

Con una heur{\'\i}stica adecuada, el algoritmo A* puede utilizarse para espacios de b\'usqueda 
muy grandes \cite{huang-handbook10}.
A\'un as{\'\i}, el algoritmo de Viterbi con b\'usqueda en haz es el m\'etodo utilizado con mayor frecuencia
en los sistemas de reconocimiento del habla \cite{huang-handbook10}. 
Esto se debe principalmente a la ventaja en cuanto a eficiencia en t\'erminos de tiempo, 
relacionada a la b\'usqueda s{\'\i}ncrona en el tiempo \cite{huang-handbook10}.


\subsection{Resumen}

\begin{figure}[H] 
\centering
\includegraphics[width=0.8\textwidth]{./graphics/decodificacion.png}
\caption{Fase de decodificaci\'on. Gr\'afico basado en \cite{VerenichASR}.}
\label{figure:hmm}
\end{figure}

La entrada de la fase de decodificaci\'on es el conjunto de vectores de caracter{\'\i}sticas espectrales,
los cuales se calcularon en base a la entrada ac\'ustica durante la fase de extracci\'on de caracter{\'\i}sticas.

El modelo de lenguaje busca predecir la probabilidad de una secuencia de palabras pertenecientes a un lenguaje.
Dos tipos de modelo de lenguaje com\'unmente utilizados en el reconocimiento del habla son los basados en
n-gramas y los basados en gram\'aticas.

El modelo ac\'ustico permite estimar la probabilidad de una entrada ac\'ustica dada una secuencia de palabras.
El modelo ac\'ustico est\'a formado por varios modelos ocultos de Markov y el diccionario fon\'etico.

El algoritmo decodificador busca la secuencia de fonemas o subfonemas m\'as probable dados una 
secuencia de observaciones y un modelo oculto de Markov. Este \'unico modelo oculto de Markov se construye
mediante los componentes del modelo ac\'ustico y el modelo de lenguaje. El algoritmo de Viterbi y el 
algoritmo A* son dos algoritmos decodificadores utilizados frecuentemente en el reconocimiento del habla.

La secuencia de fonemas o subfonemas puede convertirse a una secuencia de palabras utilizando el diccionario
fon\'etico que forma parte del modelo ac\'ustico.

La salida de la fase de decodificaci\'on es la secuencia de palabras m\'as probable dados los vectores
de caracter{\'\i}sticas espectrales, los cuales se calcularon en base a la entrada ac\'ustica durante
la fase de extracci\'on de caracter{\'\i}sticas. Con esto llega a su culminaci\'on el proceso b\'asico
del reconocimiento del habla.
%!TEX root = ../tesis.tex
\subsection{Fase Previa: Entrenamiento}
\label{sec:training}

Para que el proceso que se ha presentado en las secciones anteriores funcione, es necesaria una fase
previa de entrenamiento de los modelos probabil{\'\i}sticos que componen el reconocedor del habla. 

Los cuatro modelos probabil{\'\i}sticos que deben entrenarse son \cite{Jurafsky}:
\begin{itemize}
	\item Modelo oculto de Markov L\'exico: la estructura del grafo de estados
	\item Probabilidades del modelo de lenguaje: $P(W)$
	\item Verosimilitudes de observaci\'on: $b_j(o_t)$
	\item Probabilidades de transici\'on: $a_{ij}$
\end{itemize}

Para ello se cuenta con \cite{Jurafsky}:
\begin{itemize}
	\item  Un corpus de voz, compuesto por una colecci\'on de grabaciones de voz junto
	con sus transcripciones de texto.
	\item  Un corpus mucho mayor de texto para entrenar el modelo de lenguaje, compuesto por las transcripciones
	del corpus de voz junto con otros textos similares.
	\item A menudo, un corpus menor de voz etiquetado fon\'eticamente. Esto es, donde fragmentos
	de la se\~nal est\'an asociados a su fonema correspondiente.
\end{itemize}

\emph{Estructura del grafo de estados} 

Se construye en base a diccionarios de pronunciaci\'on como el diccionario \mbox{CMUDdict \cite{CMUdict}}. 
Los estados se definen de acuerdo a los fonemas, o en muchos casos de acuerdo a subdivisiones de estos, denominadas subfonemas.

\emph{Modelo de lenguaje} 

Un modelo basado en n-gramas se entrena contando las ocurrencias de cada n-grama en el corpus de texto, para luego 
suavizar y normalizar cada conteo. As{\'\i}, el contador suavizado y normalizado de un n-grama constituye 
su \mbox{probabilidad \cite{CollinsLanguage}}.

\emph{Par\'ametros del modelo oculto de Markov} 

Las probabilidades de estimaci\'on se estiman inicialmente asumiendo que, para cada estado, cualquier transici\'on posible 
a otro estado es igualmente probable. Las verosimilitudes de observaci\'on se estiman inicialmente mediante el peque\~no corpus 
de voz etiquetado fon\'eticamente.

Una vez realizada la estimaci\'on inicial, el siguiente paso para entrenar estos par\'ametros difiere dependiendo de la utilizaci\'on
de redes neuronales o funciones Gaussianas \cite{Jurafsky}. A continuaci\'on se presenta brevemente el enfoque para cada caso, 
incluyendo referencias a materiales que desarrollan los conceptos de manera m\'as extendida:

\begin{itemize}
	\item \emph{Redes Neuronales}: el entrenamiento de la red neuronal se realiza utilizando el algoritmo de Propagaci\'on hacia 
	Atr\'as \cite{Russell2003Learning}. Este algoritmo requiere conocer el fonema correspondiente a cada observación. 

	Para ello se utiliza el alineamiento Viterbi forzado \cite{JelinekStatistical1998}, el cual recibe los vectores de características y la secuencia de palabras correctas y produce la mejor secuencia de estados, donde cada estado puede alinearse a un vector. 

	Se le denomina \emph{forzado} debido a que puede verse como el algoritmo de Viterbi con una restricción más: debe encontrar el mejor camino 
	\emph{que pase por una secuencia de palabras dada}.

	\item \emph{Distribuci\'on Gaussiana}: para el caso de las funciones de distribución de probabilidad Gaussianas, se utiliza
	el algoritmo de \mbox{Avance-Retroceso \cite{Jurafsky}}.

	Este algoritmo estima dos probabilidades, llamadas \emph{de avance} y \emph{de retroceso} dadas las estimaciones iniciales de
	$a$ y $b$. Luego, en base a estas probabilidades se mejora la estimación. Este proceso se repite hasta que los valores convergen.
\end{itemize}





