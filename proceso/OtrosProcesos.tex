\section{Otros Modelos de Procesos}
\label{sec:otrosModelos}

Otros enfoques han sido desarrollados para el problema de reconocimiento del habla. En las secciones que siguen
se presentan algunos de estos modelos.

\subsection{Distorsi\'on Din\'amica Temporal}
\label{sec:dtw}

La Distorsi\'on Din\'amica Temporal (o DTW por sus siglas en ingl\'es) es una t\'ecnica t\'ipica del enfoque
basado en comparaci\'on de patrones \cite{GaikwadAReview2010}. Este m\'etodo permite encontrar una alineaci\'on \'optima 
entre dos secuencias que pueden variar en tiempo y velocidad, bajo ciertas restricciones \cite{MullerInformation2007}.
Las secuencias se distorsionan de manera no lineal con respecto a la dimensi\'on tiempo para determinar una medida
de similitud independiente a ciertas variaciones no lineales que ocurran sobre la dimensi\'on tiempo \cite{AnusuyaSpeech2009}.

Este algoritmo forma un espacio de bidimensional $C$, denominado matriz de costo, con las dos secuencias $X$ e $Y$. El objetivo es encontrar
una alineaci\'on entre $X$ e $Y$ de costo total (suma de distancias locales entre elementos) m\'inimo. 
Formalmente a esta alineaci\'on se la conoce como trayectoria de
deformaci\'on, consiste en una secuencia que satisface tres condiciones: restricci\'on de punto final, 
monoton\'ia y tama\~no de paso \cite{MullerInformation2007}.

DTW ha probado ser un m\'etodo eficiente para el reconocimiento de palabras pronunciadas pausadamente \cite{MyersALevel1981} y ha sido
adaptado para el reconocimiento de palabras enlazadas, con cierto nivel de 
\mbox{\'exito \cite{MyersALevel1981, SakoeTwoLevel1979, RabinerApplication1980}}.

\subsection{Redes Neuronales}
\label{sec:otrosModelosANN}

Como se mencion\'o anteriormente el concepto de Redes Neuronales Artificiales (RNA) aplicado a el reconocimiento
del habla se volvi\'o a introcudir en los a\~nos 80, 
y viene desarroll\'andose como m\'etodo alternativo desde entonces. Las RNA son capaces de resolver tareas de reconocimiento
m\'as complicadas, pero no escalan tan bien como los HMM cuando se trata de grandes vocabularios \cite{VimalaReview2012}.

Existen dos enfoques b\'asicos para la clasificaci\'on del habla utilizando redes neuronales: est\'atico y din\'amico. En la
clasificaci\'on est\'atica la red ve toda la entrada de voz a la vez, y toma una sola decisi\'on. Por otro lado, en el enfoque
din\'amico la red ve solo una peque\~na ventana de la entrada, y esta ventana se desliza sobre la entrada mientras la red
realiza una serie de decisiones locales, que deben ser integradas a una decisi\'on global posteriormente \cite{TebelskisSpeech1995}.

La clasificaci\'on de fonemas, el reconocimiento de palabras pueden llevarse a cabo con un alto
grado de precisi\'on utilizando m\'etodos est\'aticos o din\'amicos. Aunque el enfoque din\'amico, aplicado al reconocimiento de
palabras, se adapta mejor a la variabilidad de duraci\'on de una palabra \cite{TebelskisSpeech1995}.

Existen tambi\'en sistemas h\'ibridos RNA-HMM que utilizan las Redes Neuronales para el reconocimiento de fonemas y los Modelos
Ocultos de Markov para el modelo de lenguaje \cite{VimalaReview2012}.
