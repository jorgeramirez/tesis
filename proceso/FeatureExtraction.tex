%!TEX root = ../tesis.tex
\subsection{Fase 1: Extracci\'on de caracter{\'\i}sticas}
\label{sec:featureExtraction}

En la primera fase, la entrada es la se\~nal de voz correspondiente al emisor, la cual se divide en muestras de corta duraci\'on llamadas tramas. A trav\'es de un proceso de procesamiento de la onda sonora, se obtienen vectores de caracter{\'\i}sticas representativas para cada trama. Esta secci\'on describe los conceptos relacionados a esta transformaci\'on.

\subsubsection{Ondas Sonoras}

La entrada para un reconocedor del habla, y en particular para la fase 1, es la misma que la del o{\'\i}do humano: una serie de cambios en la presi\'on del aire \cite{YoungUniversity2007}. Estos cambios se originan en el emisor, y son causados por el aire que es expulsado de los pulmones, pasa por el tracto vocal y salen por la boca. As{\'\i}, para el proceso de s{\'\i}ntesis del habla en el ser humano, los pulmones son la fuente
del sonido y el trato vocal es el filtro que genera los distintos tipos de \mbox{sonido \cite{BradburyLineal2000}}.

Como mencionaba Semat ya en los a\~nos 50, existen dos aspectos del sonido: el f{\'\i}sico y el perceptual. De acuerdo al aspecto considerado, var{\'\i}an los t\'erminos con los cuales se describe al sonido; as{\'\i}, un f{\'\i}sico describe un sonido en t\'erminos de frecuencia, amplitud y n\'umero de sobretonos, mientras un m\'usico utiliza t\'erminos como tono, volumen o \mbox{timbre \cite{SematPhysics1958}}.

Aunque no existe una correspondencia directa entre las car\'acter{\'\i}sticas f{\'\i}sicas y las perceptuales \cite{SematPhysics1958}, existe una correlaci\'on en ciertos casos. As{\'\i}, para los sonidos con una frecuencia alta se percibe un tono m\'as agudo, aunque esta relaci\'on no es lineal. Igualmente, a mayor amplitud de una onda sonora esta se percibe con mayor \mbox{volumen \cite{YoungUniversity2007}}.

\subsubsection{Espectro}
El an\'alisis espectral est\'a basado en la conclusi\'on de Fourier que establece que una onda compleja puede ser representada como la sumatoria de varias ondas simples a diferentes frecuencias. El espectro es la representaci\'on de las distintas frecuencias que componen una onda, que puede visualizarse mediante una gr\'afica de amplitud en funci\'on de la \mbox{frecuencia \cite{Jurafsky}}. 

Los picos espectrales del espectro del sonido se conocen como formantes \cite{Fant1960acoustic}. Distinto fonemas poseen formantes a frecuencias 
determinadas, raz\'on por la cual el an\'alisis espectral tiene un rol determinante en la determinaci\'on de la identidad de vocales y otros 
\mbox{fonemas \cite{LadefogedCourse2006}}.

%Figura%

Un espectrograma, por su parte, es una matriz de celdas indexadas por frecuencia en el eje vertical y por tiempo en el eje horizontal, donde
el nivel de sombreado de una celda indica la amplitud de la componente a una frecuencia y un tiempo dados. Aunque rara vez se use directamente
para caracterizar la se\~nal, por resultar poco conveniente en t\'erminos del tama\~no de la representaci\'on, la gran mayor{\'\i}a de las
caracter{\'\i}sticas utilizadas para el reconocimiento del habla est\'an basadas en el \mbox{espectrograma \cite{Ellis08anintroduction}}.

%Figura%

\subsubsection{Proceso de Extracci\'on de Caracter{\'\i}sticas}
El proceso puede resumirse en los siguientes pasos \cite{Jurafsky}:

\begin{enumerate}
\item \emph{Muestreo}: consiste en medir la amplitud de la se\~nal en un momento dado; la frecuencia de muestreo es el n\'umero de muestras que 
se toma por segundo. 

De acuerdo a la f\'ormula de Nyquist, es necesario que la frecuencia de muestreo sea al menos el doble de la frecuencia 
correspondiente a la onda que se desea medir. Por ejemplo, para las conversaciones telef\'onicas, cuyas frecuencias no superan los 4000 Hz,
una frecuencia de muestreo de 8000 Hz resulta suficiente.

\item \emph{Cuantificaci\'on}: consiste en representar el n\'umero real correspondiente a la amplitud como un n\'umero entero, normalmente de 8
o 16 bits, para cada muestra. Con este paso finaliza la transformaci\'on de anal\'ogico a digital de la se\~nal de voz.

\item \emph{Conversi\'on}: una vez digitalizada, la se\~nal se convierte a un conjunto de caracter{\'\i}sticas espectrales. Los detalles de este 
paso dependen en gran medida del conjunto de caracter{\'\i}sticas que se selecciona para representar a la se\~nal.

\end{enumerate}

\subsubsection{Caracter{\'\i}sticas}
Un buen conjunto de caracter{\'\i}sticas re\'une las siguientes cualidades \cite{KesarkarFeature2003}:
\begin{itemize}
\item Las caracter{\'\i}sticas son perceptualmente significativas, es decir, an\'alogas a las utilizadas por el sistema auditivo humano.
\item Las caracter{\'\i}sticas son invariantes, es decir, robustas con respecto a las variaciones en el canal y el emisor.
\item Las caracter{\'\i}sticas capturan la din\'amica espectral, es decir, los cambios del espectro en el tiempo.
\end{itemize}

Algunos de los conjuntos de caracter{\'\i}sticas que se utilizan son:
\begin{itemize}
\item \emph{Codificaci\'on Predictiva Lineal (LPC)}: representaci\'on del espectro basada en la idea de que una muestra sonora puede ser 
aproximada mediante la combinaci\'on lineal de muestras \mbox{anteriores \cite{KesarkarFeature2003}}.
\item \emph{An\'alisis Cepstral en escala de Mel}: es el conjunto de caracter{\'\i}sticas m\'as com\'unmente utilizado. Basado en el concepto 
de ceptro, una representaci\'on que convierte los efectos del filtrado sobre una onda en una operaci\'on de adici\'on. Los coeficientes 
cesptrales se representan en la escala de Mel, una escala de frecuencia no lineal de frecuencia basada en la percepci\'on de 
\mbox{tonos \cite{Ellis08anintroduction}}.
\item \emph{An\'alisis Predictivo Lineal Perceptual}: se basa en las caracter{\'\i}sticas LPC y las modifica de manera consistente a la 
percepci\'on por parte del ser humano. Por ejemplo, toma en cuenta los problemas de las personas para percibir ondas de alta frecuencia y la 
relaci\'on entre volumen e intensidad como factores para la transformaci\'on de las \mbox{caracter{\'\i}sticas \cite{Jurafsky}}. 
\end{itemize}

La evaluaci\'on y comparaci\'on de estos conjuntos de caracter{\'\i}sticas espectrales, de manera a determinar el m\'as indicado para el 
reconocimiento del habla, es tema de numerosos trabajos de 
\mbox{investigaci\'on \cite{DorraComparative2006, SarosiComparison2011, ElminirEvaluation2012}}.