%!TEX root = ../../tesis.tex
\section{Videojuegos}
\label{sec:videojuegos}

La integraci\'on del reconocimiento del habla al \'area de los videojuegos es otra l{\'\i}nea de
investigaci\'on existente \cite{SporkaNonSpeech2006, JanickiAutomatic2011}. 
Una aplicaci\'on habitual se da en los juegos educativos, en especial los orientados a ni\~nos 
con dificultades de lenguaje, concentraci\'on o aprendizaje en general; un ejemplo es el juego
comercial Say-N-Play \cite{SayNPlay}.

Gracias a la inclusi\'on de tecnolog{\'\i}as de reconocimiento del habla en los tel\'efonos modernos,
estas pueden ser utilizadas en diferentes aspectos relacionados a los juegos: como mecanismo de entrada 
de comandos, para \foreign{chat} multijugador, o como medio acceso a contenido 
\mbox{exclusivo \cite{JoselliMobile2009}}.

\foreign{Lifeline} es un juego lanzado para la consola \emph{PlayStation 2} en el a\~no 2003 en Jap\'on (lanzado en Estados Unidos
el a\~no siguiente). Este
juego es considerado un cl\'asico entre los fan\'aticos principalmente por la innovaci\'on propuesta: el jugador controla 
el juego completo utilizando comandos de voz \cite{Lifeline}.

\foreign{Tazti} es un software de reconocimiento del habla que permite, entre otras cosas, operar juegos de computadora
utilizando comandos de voz \cite{tazti}. El concepto b\'asico utilizado por \foreign{Tazti} es el de crear un perfil para
cada juego y mapear comandos de voz con teclas de la computadora que realizan cierta acci\'on en el juego.

\foreign{Dragon Gaming Speech Pack}, es una propuesta de \foreign{Nuance Inc.} que permite utilizar comandos de
voz para controlar varios juegos populares de la industria de los videojuegos \cite{DragonGamingSpeech}. Este software
es un paquete de extensi\'on del software de reconocimiento \foreign{Dragon NaturallySpeaking}, de la misma compa\~n\'ia.

Cabe destacar tambi\'en el rumor reciente de una funcionalidad de control por voz en
la consola de juegos \emph{Xbox 720}, sucesora de la \mbox{Xbox 360 \cite{IgnXbox}}. Aunque actualmente la \emph{Xbox 360}
soporta ciertos comandos de voz b\'asicos \cite{KinectVoice}, su sucesora brindar\'ia un mayor soporte y as\'i buscar
una interacci\'on m\'as natural para el jugador.
