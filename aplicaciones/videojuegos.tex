%!TEX root = ../../tesis.tex
\section{Videojuegos}
\label{sec:videojuegos}

La integraci\'on del reconocimiento del habla al \'area de los videojuegos es otra l{\'\i}nea de
investigaci\'on existente \cite{SporkaNonSpeech2006, JanickiAutomatic2011}. 
Una aplicaci\'on habitual se da en los juegos educativos, en especial los orientados a ni\~nos 
con dificultades de lenguaje, concentraci\'on o aprendizaje en general; un ejemplo es el juego
comercial Say-N-Play \cite{SayNPlay}.

Gracias a la inclusi\'on de tecnolog{\'\i}as de reconocimiento del habla en los tel\'efonos modernos,
estas pueden ser utilizadas en diferentes aspectos relacionados a los juegos: como mecanismo de entrada 
de comandos, para \foreign{chat} multijugador, o como medio acceso a contenido 
\mbox{exclusivo \cite{JoselliMobile2009}}.

Cabe destacar tambi\'en el rumor reciente de una funcionalidad de control por voz en
la consola de juegos XBOX 720, sucesora de la \mbox{XBOX 360 \cite{IgnXbox}}.

