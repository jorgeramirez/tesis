%!TEX root = ../../tesis.tex
\section{Telefon\'ia}
\label{sec:telefonia}

El reconocimiento del habla fue introducido al \'area de las telecomunicaciones a inicios de la d\'ecada
de los 90 por dos motivos: para reducir costos y para generar ganancias en base a nuevos servicios \cite{RabinerApplications1997}.
A continuaci\'on se mencionan aplicaciones relacionadas a esta \'area, sistemas utilizados o sujetos de investigaci\'on, 
organizadas en estas dos categor{\'\i}as. 

Entre las aplicaciones que buscan la reducci\'on de costos pueden citarse:

\begin{itemize}
	\item  Automatizaci\'on de servicios de Operador: sistemas como el \gls{vrcp} de \gls{att} 
	y el \gls{aabs} de \gls{nortel} utilizan el reconocimiento del habla para manejar tareas tradicionalmente delegadas
	a un operador como consultas sobre facturaci\'on y llamadas \mbox{asistidas \cite{RabinerApplications1997}}.

	Para 1995, esta aplicaci\'on hac{\'\i}a posible un ahorro de m\'as de 100 millones de d\'olares 
	en \mbox{\gls{att} \cite{WilponApplications1994}}.

	\item Automatizaci\'on de Consulta de Directorio: sistemas que asisten a los operadores
	para responder una consulta al directorio telef\'onico, reduciendo el espacio de b\'usqueda mediante
	reconocimiento de ciudades o incluso nombres o \mbox{apellidos \cite{RabinerApplications1997}}.

	\item Enrutamiento de llamadas: sistemas que permiten dirigir una llamada a la persona adecuada,
	por ejemplo, en el contexto de atenci\'on al \mbox{cliente \cite{Sachs97howmay}}.
\end{itemize}

Entre las aplicaciones que buscan producir ganancias en base a nuevos modelos de servicio se encuentran:

\begin{itemize}
	\item Servicios bancarios: sistemas que ofrecen consulta de balances, transacciones, etc. mediante
	reconocimiento del \mbox{habla \cite{PreeEnhancing1999}}.

	\item Pron\'osticos del clima: sistemas que proveen informaci\'on relacionada al clima a nivel
	regional o mundial en base a una consulta realizada a trav\'es del tel\'efono. Como ejemplo puede
	citarse el proyecto \foreign{Jupiter} de \gls{att} \cite{ZueJupiter2000}, entre \mbox{otros \cite{ZibertBiliengual2003}}.

	\item Servicios de reserva: sistemas que permiten realizar reserva de pasajes de avi\'on o tren 
	mediante reconocimiento del habla. Como ejemplos pueden citarse el proyecto \foreign{Mercury} 
	de \gls{att} \cite{Seneff2000Dialogue} y el proyecto \foreign{Talk`n'Travel} de 
	\mbox{Verizon \cite{StallardEvaluation2001}}.

	\item Traducci\'on Autom\'atica del habla: sistemas de traducci\'on que permiten a dos personas 
	que hablan diferentes idiomas mantener una conversaci\'on a trav\'es del tel\'efono. 
	En este punto cabe destacar el proyecto \foreign{Spectra} de \mbox{\gls{att} \cite{Rangarajan2012}}.
\end{itemize}

Menci\'on especial merece el proyecto \foreign{Watson} de \gls{att} \cite{AttWatson}, el cual consiste en 
un motor de reconocimiento de prop\'osito general y una colecci\'on de \emph{plugins}, que incluye 
funcionalidades de reconocimiento y s{\'\i}ntesis del habla, reconocimiento de expresiones faciales, 
entre otras. 

El proyecto \foreign{Watson} sirve como base para muchos otros en \gls{att}. 
Adem\'as, todas estas funcionalidades est\'an disponibles para desarrolladores externos mediante una \gls{api}. 