%!TEX root = ../../tesis.tex
\section{Industria Automotriz}
\label{sec:automotriz}

La implementaci\'on de sistemas de reconocimiento del habla para su utilizaci\'on en los autom\'oviles representa
tambi\'en un \'area de investigaci\'on activa desde hace varios \mbox{a\~nos \cite{HanaiS94, HuaSpeech2010}}. 
El desarrollo se divide en las siguientes \'areas \cite{Kirriemuir2003Speech}:

\begin{itemize}
	\item Dispositivos manos libres para la utilizaci\'on de tel\'efonos celulares en el interior del veh{\'\i}culo.
	\item Instrucciones mediante voz a los dispositivos de navegaci\'on. Por ejemplo: ``{?`}Qu\'e tan lejos est\'a el siguiente desv{\'\i}o?''.
	\item Interacci\'on por voz con sistema de control del veh{\'\i}culo. Por ejemplo: ``Enciende la radio y sintoniza la emisora X''.
	\item Sistemas de manejo por voz.
\end{itemize}

Como es de esperarse, el \'ultimo punto es el menos desarrollado por cuestiones de seguridad. A\'un as{\'\i}, grandes compa\~n{\'\i}as
de la industria automotriz, como Toyota, financian investigaciones relacionadas al reconocimiento del \mbox{habla \cite{HoshinoSpeech2004}}.

Prueba de la factibilidad y los avances logrados en el \'area son los sistemas de control por voz ya inclu{\'\i}dos en los
veh{\'\i}culos de importantes marcas como el \foreign{Blue\&Me} de Fiat \cite{FiatBlue}, el \foreign{Sync} de Ford \cite{FordSync}
y \foreign{Entune} \cite{ToyotaEntune} de Toyota.
