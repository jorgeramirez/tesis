\section{Tel\'efonos Inteligentes e Interfaces Web}
\label{sec:smartphones}

Los avances en la computaci\'on y redes inal\'ambricas ha ocasionado
que los esfuerzos de los investigadores se enfoquen en la aplicaci\'on de tecnolog\'ias de
reconocimiento del habla en dispositivos m\'oviles como, por ejemplo, los tel\'efonos 
\mbox{inteligentes \cite{TanAutomatic2008}}.

La arquitectura de los sistemas de reconocimiento del habla de acuerdo al proceso visto en el 
cap\'itulo \ref{sec:proceso}, puede descomponerse en dos partes: el \foreign{front-end} ac\'ustico,
donde se lleva a cabo la extracci\'on de caracter\'isticas  y el \foreign{back-end} donde se 
lleva a cabo la decodificaci\'on. 
Zaykovskiy en \cite{ZaykovskiySurvey2006} analiza las tres arquitecturas que se han desarrollado
para aplicar \gls{asr} en dispositivos m\'oviles y las clasifica de acuerdo a la ubicaci\'on de las dos partes
mencionadas anteriormente:

\begin{itemize}
    \item Sistemas embebidos de Reconocimiento del habla: este sistema opera completamente en el cliente.
    \item \gls{nsr}: este sistema opera completamente en el servidor. 
    \item \gls{dsr}: en este sistema la extracci\'on de caracter\'isticas se realiza
        en el cliente y el proceso de decodificaci\'on en el servidor.
\end{itemize}

Independientemente de la arquitectura, el reconocimiento del habla en dispositivos m\'oviles dio lugar a una 
serie de aplicaciones. \foreign{Siri} \cite{AppleSiri, OneAccordSiri} y \foreign{Google Now} \cite{GoogleNow} 
son aplicaciones de asistente personal inteligente, permiten responder preguntas,
hacer recomendaciones, realizar acciones, y otras tareas a trav\'es de servicios web. Verbio ASR \cite{VerbioASR} y 
Pocketsphinx \cite{HugginsDainesPocketSphinx2006, PocketSphinxHomePage} son motores de \gls{asr} para dispositivos
m\'oviles, que permiten desarrollar distintos tipos de aplicaciones (interfaces multimodales, etc)

En lo que respecta a aplicaciones web, mejorar la accesibilidad es el principal objetivo para 
los sistemas de reconocimiento. Una tecnolog\'ia que se encuentra en desarrollo es la 
\foreign{Web Speech \gls{api}} \cite{GoogleWebSpeechAPI}, que permite a los desarrolladores integrar 
reconocimiento y s\'intesis del habla a sus aplicaciones web utilizando una \gls{api} que forma parte 
del navegador\footnote{La \foreign{Web Speech \gls{api}} es una propuesta de \foreign{Google} a la W3C para integrar
una \gls{api} de reconocimiento del habla al navegador. Actualmente solo es implementada por el navegador
\foreign{Google Chrome} utilizando los servidores de \foreign{Google} para el procesamiento.}.
\gls{wami} \cite{GruensteinWami2008, WamiHome} es otra \gls{api} JavaScript para desarrollar interfaces web 
accesibles mediante la voz. \gls{wami} no proporciona un reconocedor, pero provee las herramientas necesarias 
para dar una interfaz web a un motor de reconocimiento (Sphinx, por ejemplo).