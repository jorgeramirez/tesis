\section{Dom\'otica}
\label{sec:domotica}

La industria de la dom\'otica esta creciendo r\'apidamente dentro del \'area de las tecnolog\'ias de la
comunicaci\'on e informaci\'on, motivada por la necesidad de ayudar a mejorar la autonom\'ia y bienestar 
de las personas discapacitadas o de edad avanzada, especialmente aquellas que viven solas 
\cite{AlshuVoice2011}.

Este crecimiento desarroll\'o el concepto de casas inteligentes, equipadas con sensores, actuadores, 
dispositivos automatizados y software centralizado que se encarga de controlar distintos elementos del 
hogar: persianas, iluminaci\'on, sistemas de alarmas, computadoras, y 
m\'as \cite{LecouteuxSpeech2011, UshaWireless2012}.

El reconocimiento del habla ha recibido limitada atenci\'on en el dominio de las casas inteligentes. 
Lecouteux y sus colaborades presentan en \cite{LecouteuxSpeech2011} el proyecto \foreign{Sweet-Home} que 
tiene como objetivo permitir que personas con ciertos impedimentos f\'isicos controlen su ambiente 
dom\'estico, utilizando la voz como medio de interacci\'on. En este trabajo los investigadores evaluan 
distintas t\'ecnicas para el reconocimientos
de comandos de voz en un ambiente equipado con micr\'ofonos ubicados a la distancia. El mejor desempe\~no 
obtenido presenta una \gls{wer} igual a 11,4\%.

HAL \cite{HAL} de la compa\~n\'ia \foreign{Home Automated Living} es un producto comercial que permite
controlar diversas funciones del hogar mediante la voz o a trav\'es de una interfaz web: 
iluminaci\'on, reproducci\'on de audio y v\'ideo, configuraci\'on de la temperatura, entre otras.
Adem\'as posibilita que los usuarios pregunten por el estado de cotizaciones en la 
bolsa, reporte del clima, resultados deportivos, etc.
