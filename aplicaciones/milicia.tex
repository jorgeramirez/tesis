%!TEX root = ../tesis.tex
\section{Milicia}
\label{sec:milicia}

Las caracter{\'\i}sticas propias del reconocimiento del habla permiten a un usuario utilizar una computadora
en lugares o situaciones en los que ser{\'\i}a imposible o peligroso hacerlo en el modo convencional\cite{Rufiner2004}.
Esta y otras potenciales ventajas llevaron a la \gls{nato} a formar un grupo de estudio dedicado al estudio del
procesamiento del habla.

Este grupo de estudio ha llevado a cabo una serie de experimentos e investigaciones
enfocados a la aplicaci\'on de sistemas de reconocimiento del habla en aplicaciones militares.
Entre estos pueden citarse los trabajos de Beek \cite{BeekAn1977} y Weinstein \cite{WeinsteinOpportunities1991},
en los cuales se identifican potenciales aplicaciones de tecnolog\'ias del habla agrupadas en cinco categor{\'\i}as:
seguridad, mando y control, transmisi\'on de datos y comunicaci\'on, procesamiento de voz distorsionada
y aplicaciones para entrenamiento.

Espec\'ificamente, en \cite{PigeonUse2006} se mencionan los siguientes campos de aplicaci\'on del reconocimiento 
del habla en la milicia:

\begin{itemize}
    \item Mando y Control: consiste en la interacci\'on humana con computadoras, sistemas
	de armas y sensores, por voz (en aviones de combate o helic\'opteros, por ejemplo). Pero esto
	requiere un alto desempe\~no, en tiempo real, de las tecnolog\'ias de reconocimiento.
	\item Comunicaciones: se refiere a la combinaci\'on de reconocimiento y s{\'\i}ntesis del habla para
	garantizar la fidelidad de una transmisi\'on a\'un en condiciones de ruido.
    \item Acceso a computadoras e informaci\'on: es una parte crucial para las operaciones militares modernas. El
	reconocimiento del habla puede utilizarse para operar computadoras y consultar informaci\'on utilizando la voz.
	Esto es importante para personal que trabaja bajo mucha presi\'on y tienen la vista y las manos ocupadas.
    \item Inteligencia: implica el procesamiento de una gran variedad de tipos de informaci\'on (texto y audio). 
    El inter\'es militar radica en la utilizaci\'on de tecnolog\'ias del habla y lenguaje para el an\'alisis y
    procesamiento de la excesiva cantidad de informaci\'on disponible actualmente.
    \item Entrenamiento: consiste en utilizar el reconocimiento del habla en tareas de entrenamiento de las fuerzas
	militares. Permitiendo que el personal pueda interactuar, mendiante la voz, con sistemas avanzados de simulaci\'on.
	\item Operaciones multinacionales: se refiere la combinaci\'on del reconocimiento del habla y la traducci\'on
	autom\'atica para coordinar fuerzas militares que hablan distintos idiomas.
\end{itemize}

Cabe destacar que varias de las categor{\'\i}as mencionadas en \cite{PigeonUse2006} pueden aplicarse
a otras \'areas. Por ejemplo, los sistemas de reconocimientos del habla utilizados en autom\'oviles 
pueden considerarse como aplicaciones de mando y control.