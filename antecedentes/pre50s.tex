\section{Antes de 1950}
\label{sec:pre50s}

Aunque los primeros intentos por parte de la comunidad cient\'{i}fica de construir sistemas de reconocimiento autom\'{a}tico del habla no se dieron hasta la d\'{e}cada de los 50, la primera m\'{a}quina capaz de responder
a un comando de voz fue Radio Rex, un juguete fabricado en \mbox{1920 \cite{AnusuyaSpeech2009}}.

Radio Rex era un peque\~{n}o perro de pl\'{a}stico sobre una base de hierro que se manten\'{i}a dentro de una casita
mediante la fuerza de atracci\'{o}n de un im\'{a}n alimentado por un conductor sensible a una frecuencia de 500 Hz.
La presencia de una onda a esta frecuencia causaba vibraci\'{o}n en el conductor, interrumpiendo de
esta manera la corriente y causando que Rex saliese de su casa. 

El sonido de la ``e'' en Rex posee una componente correspondiente a 500 Hz, lo cual le permit\'{i}a al can responder a su nombre.

En el \'{a}mbito acad\'{e}mico, cabe destacar la investigaci\'{o}n conducida por Fletcher en los laboratorios Bell
durante los a\~{n}os 20 sobre la relaci'{o}n entre el espectro de una se\~{n}al de voz (distribuci\'{o}n de 
la amplitud a trav\'{e}s de la frecuencia) sobre el sonido percibido por un ser humano. Las conclusiones de 
este trabajo ser\'{i}a de gran influencia para el desarrollo del sintetizador de habla VODER por parte de 
Dudley en la siguiente \mbox{d\'{e}cada \cite{JuangAutomaticSpeech}}. 