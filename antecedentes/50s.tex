\section{1950 - 1960}
\label{sec:50s}

Los inicios del reconocimiento del habla fueron bastante limitados. Esto se debi\'o a que el procesamiento 
de se\~{n}ales y tecnolog\'{i}as computacionales eran a\'un bastante primitivas. 
La mayor\'{i}a de los sistemas desarrollados utilizaron resonancias espectrales prove\'{i}das por 
un banco de filtro anal\'{o}gico \cite{Furui50Years2004}.
 
En esta \'{e}poca varios investigadores intentaron
explotar las ideas fundamentales propuestas por la Fon\'{e}tica Ac\'{u}stica \cite{AnusuyaSpeech2009}, que
describe el habla en t\'{e}rminos de elementos fon\'{e}ticos (los sonidos b\'{a}sicos de un lenguaje) y trata de
explicar c\'{o}mo son generados ac\'{u}sticamente al hablar \cite{JuangAutomaticSpeech}. 
Este conjunto de elementos incluye a los fonemas y la posici\'on y modos de articulaci\'on de la boca, 
las cuerdas vocales y otros \'organos del aparato fonador utilizados para producir los sonidos
en varios contextos fon\'{e}ticos.

Los formantes son muy importantes en el proceso humano de reconocimiento del habla, estos son bandas de frecuencias en 
donde se concentra gran parte de la energ\'{i}a en el espectro de un sonido \cite{HawkinsAcoustic2009}. Los estudios
de la estructura de los formantes sirvieron como gu\'{i}a para que investigadores de los Laboratorios Bell crearan, en 1952, 
un Reconocedor Autom\'{a}tico de D\'{i}gitos, denominado Audrey \cite{DavisAutomatic1952}. Teniendo en cuenta la naturaleza
estad\'{i}stica del habla, Audrey utiliza un proceso de correspondencia de patrones para identificar los d\'{i}gitos. Esta
correspondencia de patrones involucra comparar caracter\'{i}sticas derivadas de una se\~{n}al desconocida con
otras caracter\'{i}sticas derivadas de una se\~{n}al conocida. La precisi\'{o}n de Audrey varia entre 97 y 99 por ciento si
los d\'{i}gitos son pronunciados con una pausa de 350 ms.

En otros sistemas de reconocimiento que datan de esta \'{e}poca, Olson y Belar de los Laboratorios RCA construyeron, en 1956,
un sistema para reconocer 10 s\'{i}labas de un \'{u}nico interlocutor \cite{OlsonPhonetic1956}. En 1959 Forgie y Forgie de 
los Laboratorios Lincoln del MIT desarrollaron un reconocedor, independiente del interlocutor, de 10 vocales del
\mbox{Ingl\'{e}s \cite{ForgieResults1959}}.
