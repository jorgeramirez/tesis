\section{1950 - 1960}
\label{sec:50s}

Los inicios del reconocimiento del habla fueron bastante limitados. Esto se debi\'o a que el procesamiento 
de se\~{n}ales y tecnolog\'{i}as computacionales eran a\'un bastante primitivas. 

La mayor\'{i}a de los sistemas desarrollados utilizaron las resonancias espectrales de las vocales,
extra{\'\i}das a partir de las se\~nales de salida de un banco de filtros anal\'{o}gicos y 
circuitos l\'ogicos \cite{Furui50Years2004}.
 
En esta \'{e}poca varios investigadores intentaron explotar las ideas fundamentales propuestas por 
la fon\'{e}tica ac\'{u}stica \cite{AnusuyaSpeech2009}, que describe el habla en t\'{e}rminos de 
elementos fon\'{e}ticos (los sonidos b\'{a}sicos de un lenguaje) y trata de
explicar c\'{o}mo son generados ac\'{u}sticamente al hablar \cite{JuangAutomaticSpeech}. 
Este conjunto de elementos incluye a los fonemas, la posici\'on y los modos de articulaci\'on 
de los \'organos del aparato fonador, como la boca y las cuerdas vocales, utilizados para 
producir los sonidos en varios contextos fon\'{e}ticos.

Los formantes son muy importantes en el proceso humano de reconocimiento del habla, estos son 
bandas de frecuencias en donde se concentra gran parte de la energ\'{i}a en el espectro de un 
sonido \cite{HawkinsAcoustic2009}. Los estudios de la estructura de los formantes sirvieron 
como gu\'{i}a para que investigadores de los Laboratorios Bell crearan, en 1952, un reconocedor
autom\'{a}tico de d\'{i}gitos, denominado \foreign{Audrey} \cite{DavisAutomatic1952}.

Teniendo en cuenta la naturaleza estad\'{i}stica del habla, Audrey utilizaba un proceso de 
correspondencia de patrones para identificar los d\'{i}gitos. Esta correspondencia de patrones 
involucraba comparar caracter\'{i}sticas derivadas de una se\~{n}al desconocida con otras 
caracter\'{i}sticas derivadas de una se\~{n}al conocida. La precisi\'{o}n de
Audrey variababa entre 97 y 99 por ciento, habi\'endose realizado un ajuste previo al hablante,
si los d\'{i}gitos eran pronunciados con una pausa de 350 ms\cite{DavisAutomatic1952}.

En otros sistemas de reconocimiento que datan de esta \'{e}poca, Olson y Belar de los Laboratorios 
de la \gls{rca} construyeron, en 1956, un sistema para reconocer 10 s\'{i}labas de un \'{u}nico 
interlocutor \cite{OlsonPhonetic1956}. En 1959 Forgie y Forgie de los Laboratorios Lincoln del \gls{mit} 
desarrollaron un reconocedor de vocales del ingl\'{e}s independiente del 
interlocutor\cite{ForgieResults1959}.
