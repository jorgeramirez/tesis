%!TEX root = ../tesis.tex
\chapter{Antecedentes Hist\'{o}ricos}
\label{sec:antecedentes}

El progreso del reconocimiento del habla, desde las primeras publicaciones cient\'{i}ficas sobre
an\'{a}lisis y s\'{i}ntesis del habla en la primera mitad del siglo XX hasta llegar a las respuestas 
r\'{a}pidas y cargadas de humor de asistentes virtuales inteligentes como Siri, es el producto de d\'{e}cadas 
de investigaci\'{o}n motivadas por el sue\~{n}o de construir una m\'{a}quina capaz de mantener una 
conversaci\'{o}n con un ser humano.

Este cap\'{i}tulo presenta una s\'{i}ntesis de acontecimientos significativos relacionados al
desarrollo del reconocimiento del habla desde sus inicios. Aunque resulte dif\'{i}cil abarcar casi un siglo
de avances en pocas p\'{a}ginas, este repaso hist\'{o}rico puede resultar de utilidad para comprender el
panorama actual y la perspectiva a futuro de esta importante \'{a}rea de investigaci\'{o}n.

\section{Antes de 1950}
\label{sec:pre50s}

Los primeros intentos por parte de la comunidad cient\'{i}fica de construir sistemas de 
reconocimiento autom\'{a}tico del habla no se dieron hasta la d\'{e}cada de los 50. 
Sin embargo, la primera m\'{a}quina capaz de respondera un comando de voz fue Radio Rex, 
un juguete fabricado en \mbox{1920 \cite{AnusuyaSpeech2009}}.

Radio Rex era un peque\~{n}o perro de pl\'{a}stico sobre una base de hierro que se manten\'{i}a dentro de una casita
mediante la fuerza de atracci\'{o}n de un im\'{a}n alimentado por un conductor sensible a una frecuencia de 500 Hz.
La presencia de una onda a esta frecuencia causaba vibraci\'{o}n en el conductor, interrumpiendo de
esta manera la corriente y causando que Rex saliese de su casa. 

El sonido de la ``e'' en Rex posee una componente correspondiente a 500 Hz, lo cual le permit\'{i}a al can responder a su nombre.

En el \'{a}mbito acad\'{e}mico, cabe destacar la investigaci\'{o}n conducida por Fletcher en los laboratorios Bell
durante los a\~{n}os 20 sobre la relaci'{o}n entre el espectro de una se\~{n}al de voz (distribuci\'{o}n de 
la amplitud a trav\'{e}s de la frecuencia) sobre el sonido percibido por un ser humano \cite{FletcherNature1922}. 
Las conclusiones de este trabajo ser\'{i}an de gran influencia para el desarrollo del sintetizador de habla VODER por parte de Dudley en la siguiente \mbox{d\'{e}cada \cite{JuangAutomaticSpeech}}. 

\section{1950 - 1960}
\label{sec:50s}

Los inicios del reconocimiento del habla fueron bastante limitados. Esto se debi\'o a que el procesamiento 
de se\~{n}ales y tecnolog\'{i}as computacionales eran a\'un bastante primitivas. 

La mayor\'{i}a de los sistemas desarrollados utilizaron las resonancias espectrales de las vocales,
extra{\'\i}das a partir de las se\~nales de salida de un banco de filtros anal\'{o}gicos y 
circuitos l\'ogicos \cite{Furui50Years2004}.
 
En esta \'{e}poca varios investigadores intentaron explotar las ideas fundamentales propuestas por 
la fon\'{e}tica ac\'{u}stica \cite{AnusuyaSpeech2009}, que describe el habla en t\'{e}rminos de 
elementos fon\'{e}ticos (los sonidos b\'{a}sicos de un lenguaje) y trata de
explicar c\'{o}mo son generados ac\'{u}sticamente al hablar \cite{JuangAutomaticSpeech}. 
Este conjunto de elementos incluye a los fonemas, la posici\'on y los modos de articulaci\'on 
de los \'organos del aparato fonador, como la boca y las cuerdas vocales, utilizados para 
producir los sonidos en varios contextos fon\'{e}ticos.

Los formantes son muy importantes en el proceso humano de reconocimiento del habla, estos son 
bandas de frecuencias en donde se concentra gran parte de la energ\'{i}a en el espectro de un 
sonido \cite{HawkinsAcoustic2009}. Los estudios de la estructura de los formantes sirvieron 
como gu\'{i}a para que investigadores de los Laboratorios Bell crearan, en 1952, un reconocedor
autom\'{a}tico de d\'{i}gitos, denominado \foreign{Audrey} \cite{DavisAutomatic1952}.

Teniendo en cuenta la naturaleza estad\'{i}stica del habla, Audrey utilizaba un proceso de 
correspondencia de patrones para identificar los d\'{i}gitos. Esta correspondencia de patrones 
involucraba comparar caracter\'{i}sticas derivadas de una se\~{n}al desconocida con otras 
caracter\'{i}sticas derivadas de una se\~{n}al conocida. La precisi\'{o}n de
Audrey variaba entre 97 y 99 por ciento, habi\'endose realizado un ajuste previo al hablante,
si los d\'{i}gitos eran pronunciados con una pausa de 350 ms\cite{DavisAutomatic1952}.

En otros sistemas de reconocimiento que datan de esta \'{e}poca, Olson y Belar de los Laboratorios 
de la \gls{rca} construyeron, en 1956, un sistema para reconocer 10 s\'{i}labas de un \'{u}nico 
interlocutor \cite{OlsonPhonetic1956}. En 1959 Forgie y Forgie de los Laboratorios Lincoln del \gls{mit} 
desarrollaron un reconocedor de vocales del ingl\'{e}s independiente del 
interlocutor\cite{ForgieResults1959}.

\section{1960 - 1970}
\label{sec:60s}

En esta \'{e}poca las computadoras no eran a\'un lo suficientemente r\'{a}pidas, los investigadores
construyeron hardware especializado para desempe\~{n}ar tareas de reconocimiento del habla \cite{Furui50Years2004}.
En esta d\'{e}cada varios laboratorios Japoneses empezaron a involucrarse en el \'{a}rea de reconocimiento. En 1961,
Suzuki y Nakata de los Laboratorios Radio Research de Tokio construyeron un hardware para reconocimiento de vocales
del Japon\'{e}s \cite{SuzukiRecognition1961}. Sakai y Doshita, de la Universidad de Kyoto construyeron, en 1962, un
hardware reconocedor de fonemas \cite{SakaiThePhonetic1962}. Otro trabajo realizado por investigadores Japoneses fue
el reconocedor de d\'{i}gitos construido por Negata y sus colegas en 1963 \cite{NagataSpoken1963}. El esfuerzo de
Sakai y Doshita implicaba el uso, por primera vez, de un segmentador de voz para el an\'{a}lisis y reconocimiento
del habla en diferentes regiones de la sentencia de entrada \cite{JaisalAReview2012}.

Uno de los problemas dif\'{i}ciles del reconocimiento del habla se encuentra en la no uniformidad de las escalas de
tiempos en eventos relacionados al habla \cite{Furui50Years2004}. Tom Martin y sus colegas de los 
Laboratorios \gls{rca} desarrollaron un conjunto de m\'{e}todos para la normalizaci\'{o}n del tiempo, 
basados en la habilidad de determinar de manera confiable el inicio y fin de una oraci\'{o}n \cite{MartinSpeech1964}.
Casi al mismo tiempo, en la Uni\'{o}n Sovi\'{e}tica, el profesor Vintsyuk propuso la utilizaci\'{o}n de
programaci\'{o}n din\'{a}mica para el alineamiento de un par de se\~{n}ales de voz 
(conocido como Dynamic Time Warping), incluyendo adem\'{a}s algoritmos para el reconocimiento de palabras 
enlazadas \cite{VintsyukSpeech1968}. De manera independiente en Jap\'{o}n, Sakoe y Chiba de los Laboratorios de la
\gls{nec} tambi\'{e}n empezaron a utilizar programaci\'{o}n din\'{a}mica para solucionar el problema de no uniformidad
\cite{SakoeDynamic1978}.

%!TEX root = ../tesis.tex
\section{1970 - 1980}
\label{sec:70s}

El cambio de paradigma m\'as importante para el progreso del reconocimiento del habla ha sido, sin duda alguna,
la introducci\'on de m\'etodos estad{\'\i}sticos, especialmente el procesamiento estoc\'astico con modelos ocultos de
Markov (HMM por sus siglas en ingl\'es). A\'un despu\'es de 30 a\~nos, esta metodolog{\'\i}a a\'un predomina en los
sistemas \mbox{actuales \cite{BakerResearch2009}}.

La teor{\'\i}a b\'asica de los modelos ocultos de Markov fue publicada en una serie de papers, hoy en d{\'\i}a cl\'asicos,
de Baum y sus colegas a finales de los a\~nos 60 e inicios de los a\~nos 70. Esta fue implementada por Baker en
el Carnegie Mellon University y por Jelinek y otros colegas en \mbox{IBM \cite{Rabiner89atutorial}}.

El programa de investigaci\'on sobre comprensi\'on del lenguaje, abierto con fondos de la Agencia de Proyectos de Investigaci\'on Avanzada (ARPA por sus siglas en ingl\'es) a inicios de la d\'ecada de los 70 tuvo como 
resultado varios nuevos sistemas y \mbox{tecnolog{\'\i}as \cite{Furui50Years2004}}.

Entre los sistemas construidos en el marco de este programa se destaca Harpy, de la Universidad de Carnegie Mellon,
capaz de reconocer el habla usando un vocabulario compuesto de 1011 palabras con precisi\'on razonable. Harpy fue el
primer sistema en un utilizar una red de estados finitos para reducir el c\'omputo y determinar eficientemente la
cadena de salida. Sin embargo, los m\'etodos para optimizar las redes de estado finito no surgieron hasta inicios de
la d\'ecada de los \mbox{90 \cite{JuangAutomaticSpeech}}.

Otros sistemas desarrollados bajo el mismo programa son Hersay(-II) tambi\'en de CMU y HWIM de BBN. A pesar proponer
nuevos enfoques sumamente interesantes, ninguno de estos proyectos lleg\'o a cumplir la meta en cuanto a rendimiento
del programa al momento de su conclusi\'on, en \mbox{1976 \cite{JuangAutomaticSpeech}}.
\section{1980 - 1990}
\label{sec:80s}

El problema de reconocer una secuencia de palabras enlazadas, pronunciadas fluidamente, era el foco
de investigaci\'{o}n de esta d\'{e}cada \cite{Furui50Years2004}. El m\'{e}todo conocido como 
Dynamic Time Warping, desarrollado en la d\'{e}cada de los 60, demostr\'{o} ser \'{u}til para solucionar el problema de no uniformidad
en palabras pronunciadas aisladamente. Variaciones de esta t\'{e}cnica han sido aplicadas para el 
reconocimiento de palabras enlazadas, como el algoritmo elaborado por Myers y Rabiner de los laboratorios Bell
en 1981 \cite{MyersALevel1981}.

La d\'{e}cada de los 80 fue caracterizada por un giro en la metodolog\'{i}a aplicada al proceso de
reconocimiento del habla, pas\'{o} de un paradigma intuitivo basado en reconocimiento de patrones 
a uno m\'{a}s riguroso basado en un modelo estad\'{i}stico. En la actualidad, el modelo estad\'{i}stico
desarrollado en los a\~{n}os 80 sigue sirviendo como base para los sistemas de reconomiento del habla.

Una tecnolog\'{i}a clave desarrollada en esta \'{e}poca fue el Modelo Oculto de M\'{a}rkov o 
HMM (por sus siglas del ingl\'{e}s Hidden Markov Model). A pesar de que el concepto de modelo oculto
de M\'{a}rkov ya era conocido en ese entonces, la metodolog\'{i}a se termin\'{o} de desarrollar reci\'{e}n a mediados
de los 80 \cite{JuangAutomaticSpeech}, y fue adoptada como m\'{e}todo preferido para el reconocimiento del habla luego de 
masivas publicaciones \cite{LevinsonAnIntroduction1983, FergusonHidden1980}.

En 1986, Furui propuso una nueva t\'{e}cnica para el reconocimiento de palabras aisladas basada en
una combinaci\'{o}n de par\'{a}metros din\'{a}micos e instant\'{a}neos 
extra\'{i}dos del espectro de voz \cite{FuruiSpeaker1986}.

En esta \'{e}poca los esfuerzos de IBM se centralizaban en la creaci\'{o}n de un modelo de lenguaje en t\'{e}rminos de
reglas estad\'{i}sticas que permitan describir cu\'{a}l era la probabilidad de que una 
secuencia de s\'{i}mbolos aparezca en la se\~{n}al de voz \cite{Furui50Years2004}. El resultado es el
modelo conocido como N-gram, el cual define la probabilidad de ocurrencia de una secuencia ordenada
de palabras \cite{JelinekTheDevelopment1986}.

Otra tecnolog\'{i}a que avanz\'{o} en esta \'{e}poca se conoce como Redes Neuronales Artificiales. La idea 
de utilizar Redes Neuronales se dio en los a\~{n}os 50, pero no tuvo
mucho \'{e}xito debido a que era poco pr\'{a}ctico en ese entonces. La llegada del modelo de Procesamiento Distribuido 
en Paralelo (PDP) y el m\'{e}todo de entrenamiento conocido como Retropropagaci\'{o}n
revivieron la idea de imitar el procesamiento neural humano. Esto ocasion\'{o} la reintroducci\'{o}n de
las Redes Neuronales en el proceso de reconocimiento del habla, aunque las aplicaciones se centraban en tareas simples como reconocer unos pocos 
fonemas o unas pocas vocales \cite{JuangAutomaticSpeech}.


%!TEX root = ../tesis.tex
\section{1990 - 2000}
\label{sec:90s}

La d\'ecada de los 90 se vio marcada por la combinaci\'on de los avances tecnol\'ogicos relacionados al reconocimiento del habla con el incremento de poder computacional y capacidad de almacenamiento, que result\'o en el desarrollo y
lanzamiento de aplicaciones reales basadas en esta \mbox{\'area \cite{JuangAutomaticSpeech, GauvainLarge2000}}.

Entre estas se pueden citar:
\begin{itemize}
\item \foreign{How May I Help You?}, un sistema de enrutamiento de llamadas basado en preguntas abiertas presentado por 
AT\&T en \mbox{1997 \cite{Sachs97howmay}}. 

\item \foreign{Jupiter} y \foreign{Mercury}, ambos desarrollados en el MIT a finales de los 90; sistemas de consulta de informaci\'on meteorol\'ogica \cite{ZueJupiter2000} y  de reserva de vuelos \cite{Seneff2000Dialogue} respectivamente.

\item \foreign{Dragon Dictate} y \foreign{Dragon Naturally Speaking}, productos de \foreign{Dragon Systems}, l{\'\i}der hasta la actualidad en el mercado de programas de reconocimiento del habla, lanzados en 1995 y 1997 
\mbox{respectivamente \cite{BarnettMultilingual1996, BlandingSpeechless2012}}.
\end{itemize}

En el \'ambito acad\'emico, se investigaron distintas tecnolog{\'\i}as a fin de mejorar la robustez de los sistemas ante el ruido, las diferencias en la voz, el micr\'ofono y los canales de transmis{\'\i}\'on. Los m\'etodos estudiados incluyen la regresi\'on lineal,
la descomposici\'on de modelos y el m\'aximo estructural \mbox{\foreign{a posteriori} \cite{AnusuyaSpeech2009}}.

Tambi\'en se prest\'o especial atenci\'on a la evaluaci\'on del rendimiento de los sistemas basados en reconocimiento del habla para distintas tareas, que iban desde maniobras militares hasta transcripci\'on de noticieros \cite{JuangAutomaticSpeech}.
\section{2000 - Actualidad}
\label{sec:post2000s}

% 47, 76, 25, 23, 38, 36, 90

En los inicios del segundo milenio, el \'{a}rea de reconocimiento del habla ya hab\'{i}a avanzado 
considerablemente. Resultados prometedores fueron obtenidos en el reconocimiento de palabras aisladas
dependientes del interlocutor, y los esfuerzos de los investigadores se centraban en el habla espont\'{a}nea,
independencia del interlocutor y robustez en condiciones ruidosas \cite{RonzhinSurvey2006}.

El programa \gls{ears} de \gls{darpa} se llev\'{o} a cabo para desarrollar tecnolog\'{i}as para la transcripci\'{o}n 
autom\'{a}tica con el objetivo de conseguir mejores resultados. 
Las tareas incluyen detecci\'{o}n de l\'{i}mites de oraciones, palabras de relleno 
y fonemas o palabras repetidas conocidos como disfluencias. 
La meta era permitir que las computadoras puedan tener un mejor desempe\~{n}o en la detecci\'{o}n, extracci\'{o}n, 
s\'{i}ntesis y traducci\'{o}n de informaci\'{o}n importante \cite{LiuStructural2005, SoltauThe\gls{ibm}2005}.

Como se mencion\'{o} anteriormente, los investigadores se enfocaban en el reconocimiento del habla espont\'{a}nea, 
ya que el nivel de precisi\'{o}n disminu\'{i}a bastante bajo esta situaci\'{o}n. Varios programas de 
investigaci\'{o}n se llevaron a cabo para incrementar el nivel de precisi\'{o}n. En el 2005 finaliz\'{o} un 
programa de 5 a\~{n}os que se llev\'{o} a cabo en Jap\'{o}n, los resultados obtenidos incluyen un \gls{csj} 
y nuevas t\'{e}cnicas como: modelado ac\'{u}stico flexible, detecci\'{o}n de l'{i}mites de una oraci\'on,
modelado de pronunciaci\'{o}n, adaptaciones al modelo ac\'{u}stico y de lenguaje, 
y resumen autom\'{a}tico \cite{FuruiRecent2005}.

Con el objetivo de mejorar la robustez de los resultados prove\'{i}dos por los sistemas de reconocimiento del habla, 
especialmente en el habla espont\'{a}nea, importantes esfuerzos se enfocaron en desarrollar t\'{e}cnicas para medir 
la confiabilidad de lossistemas. Estas tareas de investigaci\'{o}n se agrupan bajo el nombre de 
Reconocimiento Robusto del Habla. Esta \'{a}rea propuso la utilizaci'{o}n de \gls{cm} para indicar 
la confiabilidad de los resultados de un sistema de reconocimiento \cite{JiangConfidence2005}.

A partir de la segunda mitad de los 2000 el reconocimiento del habla lleg\'{o} a los dispositivos m\'{o}viles. 
En el 2008 fue lanzado Google Voice Search para iPhone permitiendo que usuarios puedan hacer consultas al 
buscador de Google con la voz, actualmente el software forma parte del la aplicaci\'{o}n Google Search y 
se encuentra disponible para Android y otras plataformas \cite{GoogleSearch}. 
Siri\footnote{Siri fue inicialmente desarrollada por Siri Inc. en el 2007, la empresa fue posteriormente 
adquirida por Apple Inc. en el 2010.}es una aplicaci\'{o}n de asistente personal inteligente lanzada como parte
integral de iOS\footnote{iOS es un sistema operativo m\'{o}vil desarrollado por Apple Inc.}desde el 2011, 
la aplicaci\'{o}n utiliza procesamiento del lenguaje natural para responder preguntas, hacer recomendaciones y
realizar acciones a trav\'{e}s de servicios web \cite{AppleSiri}.

En lo que respecta a aplicaciones web, en el 2012 Google introdujo la \gls{api} 
\foreign{Web Speech \gls{api}}, la cual permite a los desarrolladores incorporar 
reconocimiento y s\'{i}ntesis del habla a sus aplicaciones o sitios web \cite{GoogleWebSpeechAPI}.
