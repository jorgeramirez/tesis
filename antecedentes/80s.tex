\section{1980 - 1990}
\label{sec:80s}

El problema de reconocer una secuencia de palabras enlazadas, pronunciadas fluidamente, fue el foco
de investigaci\'{o}n de esta d\'{e}cada \cite{Furui50Years2004}. El m\'{e}todo conocido como 
Dynamic Time Warping, desarrollado en la d\'{e}cada de los 60, demostr\'{o} ser \'{u}til para solucionar el problema de no uniformidad
en palabras pronunciadas aisladamente. Variaciones de esta t\'{e}cnica han sido aplicadas para el 
reconocimiento de palabras enlazadas, como el algoritmo elaborado por Myers y Rabiner de los laboratorios Bell
en 1981 \cite{MyersALevel1981}.

La d\'{e}cada de los 80 fue caracterizada por un giro en la metodolog\'{i}a aplicada al proceso de
reconocimiento del habla, pasando de un paradigma intuitivo basado en reconocimiento de patrones 
a uno m\'{a}s riguroso basado en un modelo estad\'{i}stico. En la actualidad, el modelo estad\'{i}stico
desarrollado en los a\~{n}os 80 sigue sirviendo de base para los sistemas de reconomiento del habla.

Aunque el concepto de modelo oculto de M\'{a}rkov ya era conocido y aplicado al reconocimiento del habla 
en algunos laboratorios, la metodolog\'{i}a termin\'{o} de desarrollarse reci\'{e}n a mediados 
de los 80 \cite{JuangAutomaticSpeech}, y fue adoptada como m\'{e}todo preferido para el reconocimiento 
del habla luego de numerosas publicaciones \cite{LevinsonAnIntroduction1983, FergusonHidden1980}.

En 1986, Furui propuso una nueva t\'{e}cnica para el reconocimiento de palabras aisladas basada en
una combinaci\'{o}n de par\'{a}metros din\'{a}micos e instant\'{a}neos 
extra\'{i}dos del espectro de voz \cite{FuruiSpeaker1986}.

En esta \'{e}poca los esfuerzos de \gls{ibm} se centraban en la creaci\'{o}n de un modelo de lenguaje 
en t\'{e}rminos de reglas estad\'{i}sticas que permitan describir cu\'{a}l era la probabilidad de que una 
secuencia de s\'{i}mbolos aparezca en la se\~{n}al de voz \cite{Furui50Years2004}. El resultado es el
modelo conocido como N-grama, el cual define la probabilidad de ocurrencia de una secuencia ordenada
de palabras \cite{JelinekTheDevelopment1986}.

Otra tecnolog\'{i}a que avanz\'{o} en esta \'{e}poca se conoce como Redes Neuronales Artificiales. Las 
redes neuronales surgieron en los a\~{n}os 50, pero no produjeron resultados notables 
inicialmente \cite{JuangAutomaticSpeech}. La llegada del modelo de \gls{pdp} y el m\'{e}todo de entrenamiento conocido como Retropropagaci\'{o}n
revivieron la idea de imitar el procesamiento neural humano. Esto ocasion\'{o} la reintroducci\'{o}n de
las Redes Neuronales en el proceso de reconocimiento del habla, aunque las aplicaciones se centraban en tareas simples
como reconocer unos pocos fonemas o unas pocas vocales \cite{JuangAutomaticSpeech}.

