%!TEX root = ../tesis.tex
\section{1990 - 2000}
\label{sec:90s}

La d\'ecada de los 90 se vio marcada por la combinaci\'on de los avances tecnol\'ogicos relacionados al reconocimiento del habla con el incremento de poder computacional y capacidad de almacenamiento, que result\'o en el desarrollo y
lanzamiento de aplicaciones reales basadas en esta \mbox{\'area \cite{JuangAutomaticSpeech, GauvainLarge2000}}.

Entre estas se pueden citar:
\begin{itemize}
\item \foreign{How May I Help You?}, un sistema de enrutamiento de llamadas basado en preguntas abiertas presentado por 
\gls{att} en \mbox{1997 \cite{Sachs97howmay}}. 

\item \foreign{Jupiter} y \foreign{Mercury}, ambos desarrollados en el \gls{mit} a finales de los 90; sistemas de consulta de informaci\'on meteorol\'ogica \cite{ZueJupiter2000} y  de reserva de vuelos \cite{Seneff2000Dialogue} respectivamente.

\item \foreign{Dragon Dictate} y \foreign{Dragon Naturally Speaking}, productos de \foreign{Dragon Systems}, l{\'\i}der hasta la actualidad en el mercado de programas de reconocimiento del habla, lanzados en 1995 y 1997 
\mbox{respectivamente \cite{BarnettMultilingual1996, BlandingSpeechless2012}}.
\end{itemize}

En el \'ambito acad\'emico, se investigaron distintas tecnolog{\'\i}as a fin de mejorar la robustez de los sistemas ante el ruido, las diferencias en la voz, el micr\'ofono y los canales de transmis{\'\i}\'on. Los m\'etodos estudiados incluyen la regresi\'on lineal,
la descomposici\'on de modelos y el m\'aximo estructural \mbox{\foreign{a posteriori} \cite{AnusuyaSpeech2009}}.

Tambi\'en se prest\'o especial atenci\'on a la evaluaci\'on del rendimiento de los sistemas basados en reconocimiento del habla para distintas tareas, que iban desde maniobras militares hasta transcripci\'on de noticieros \cite{JuangAutomaticSpeech}.