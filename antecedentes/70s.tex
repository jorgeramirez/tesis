%!TEX root = ../tesis.tex
\section{1970 - 1980}
\label{sec:70s}

El cambio de paradigma m\'as importante para el progreso del reconocimiento del habla ha sido, sin duda alguna,
la introducci\'on de m\'etodos estad{\'\i}sticos, especialmente el procesamiento estoc\'astico basado en el
concepto de \gls{hmm}. A\'un despu\'es de 30 a\~nos, esta metodolog{\'\i}a a\'un predomina en los
sistemas \mbox{actuales \cite{BakerResearch2009}}.

La teor{\'\i}a b\'asica de los modelos ocultos de Markov fue publicada en una serie de \emph{papers}, 
hoy en d{\'\i}a cl\'asicos, de Baum y sus colegas a finales de los a\~nos 60 e inicios de los a\~nos 70. 
Esta fue implementada por Baker en el Carnegie Mellon University y por Jelinek y otros colegas en 
\mbox{\gls{ibm} \cite{Rabiner89atutorial}}.

El programa de investigaci\'on sobre comprensi\'on del lenguaje, abierto con fondos de la \gls{arpa} 
a inicios de la d\'ecada de los 70, tuvo como resultado varios nuevos sistemas y 
\mbox{tecnolog{\'\i}as \cite{Furui50Years2004}}.

Entre los sistemas construidos en el marco de este programa se destaca Harpy, de la Universidad de Carnegie Mellon,
capaz de reconocer el habla usando un vocabulario compuesto de 1011 palabras con una tasa de 
precisi\'on de \mbox{90\% \cite{Newell1978}.} Harpy fue el primer sistema en un utilizar una red de estados finitos
para reducir el c\'omputo y determinar eficientemente la cadena de salida. 
Sin embargo, los m\'etodos para optimizar las redes de estado finito no surgieron hasta inicios de
la d\'ecada de los \mbox{90 \cite{JuangAutomaticSpeech}}.

Otros sistemas desarrollados bajo el mismo programa son Hersay(-II) tambi\'en de \gls{cmu} y \gls{hwim} 
de \gls{bbn} \cite{JuangAutomaticSpeech}. A pesar proponer nuevos enfoques sumamente interesantes, 
ninguno de estos proyectos lleg\'o a cumplir la meta en cuanto a rendimiento
del programa al momento de su conclusi\'on, en \mbox{1976 \cite{JuangAutomaticSpeech}}.