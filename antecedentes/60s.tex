\section{1960 - 1970}
\label{sec:60s}

En esta \'{e}poca las computadoras no eran a\'un lo suficientemente r\'{a}pidas, los investigadores
construyeron hardware especializado para desempe\~{n}ar tareas de reconocimiento del habla \cite{Furui50Years2004}.
En esta d\'{e}cada varios laboratorios Japoneses empezaron a involucrarse en el \'{a}rea de reconocimiento. En 1961,
Suzuki y Nakata de los Laboratorios Radio Research de Tokio construyeron un hardware para reconocimiento de vocales
del Japon\'{e}s \cite{SuzukiRecognition1961}. Sakai y Doshita, de la Universidad de Kyoto construyeron, en 1962, un
hardware reconocedor de fonemas \cite{SakaiThePhonetic1962}. Otro trabajo realizado por investigadores Japoneses fue
el reconocedor de d\'{i}gitos construido por Negata y sus colegas en 1963 \cite{NagataSpoken1963}. El esfuerzo de
Sakai y Doshita implicaba el uso, por primera vez, de un segmentador de voz para el an\'{a}lisis y reconocimiento
del habla en diferentes regiones de la sentencia de entrada \cite{JaisalAReview2012}.

Uno de los problemas dif\'{i}ciles del reconocimiento del habla se encuentra en la no uniformidad de las escalas de
tiempos en eventos relacionados al habla \cite{Furui50Years2004}. Tom Martin y sus colegas de los 
Laboratorios \gls{rca} desarrollaron un conjunto de m\'{e}todos para la normalizaci\'{o}n del tiempo, 
basados en la habilidad de determinar de manera confiable el inicio y fin de una oraci\'{o}n \cite{MartinSpeech1964}.
Casi al mismo tiempo, en la Uni\'{o}n Sovi\'{e}tica, el profesor Vintsyuk propuso la utilizaci\'{o}n de
programaci\'{o}n din\'{a}mica para el alineamiento de un par de se\~{n}ales de voz 
(conocido como \foreign{Dynamic Time Warping}), incluyendo adem\'{a}s algoritmos para el reconocimiento de palabras 
enlazadas \cite{VintsyukSpeech1968}. De manera independiente en Jap\'{o}n, Sakoe y Chiba de los Laboratorios de la
\gls{nec} tambi\'{e}n empezaron a utilizar programaci\'{o}n din\'{a}mica para solucionar el problema de no uniformidad
\cite{SakoeDynamic1978}.
