%!TEX root = ../tesis.tex
\chapter{Trabajos Futuros}
\label{sec:trabajos-futuros}

El \'area de reconocimiento del habla fue avanzando con el correr de las d\'ecadas. El tiempo invertido
en investigaci\'on y desarrollo de la misma permiti\'o la creaci\'on de algoritmos que permiten
reconocer la voz humana con una precisi\'on aceptable, \'utiles para poder plantear nuevas alternativas
de interacc\'on entre computadoras y usuarios. 

Este trabajo de grado presenta el \'area de reconocimiento del habla explorando: su historia, el proceso
b\'asico y los algoritmos implicados, estado del arte, aplicaciones, tecnolog\'ias y herramientas disponibles.
Adem\'as, para contrastar teor\'ia y practica se presenta el programa \emph{TamTam Listens} que permite componer
m\'usica utilizando comandos dos voz. Finalmente, se presenta una prueba de usabilidad utilizada para evaluar la
aplicaci\'on desarrollada.

A continuaci\'on se presentan posibles trabajos futuros que podr\'ian realizarse tomando como punto de partida
el trabajo de grado presentado, y as\'i profundizar m\'as en el \'area de reconocimiento del habla.

\section{Modelo Ac\'ustico Paraguayo}

Uno de los componentes utilizados para el reconocimiento de comandos es el modelo ac\'ustico. La interfaz implementada
utiliza el prove\'ido por el proyecto \emph{VoxForge}. Se podr\'ia buscar utilizar un modelo ac\'ustico propio con el
acento caracter\'istico de nuestra regi\'on y as\'i, se espera, tener mejores resultados en lo que respecta a precisi\'on.

\section{Reducci\'on de ruido}

El ruido ambiente es un problema que afecta al reconocimiento del habla. La precisi\'on de la soluci\'on desarrollada en este trabajo es 
considerablemente sensible al ruido ambiente, es por eso que la prueba de usabilidad fue realizada en un ambiente aislado
y libre de ruido. Por lo tanto, se podr\'ian investigar los algoritmos de reducci\'on de ruido disponibles y ver su aplicabilidad
al software desarrollado. Adem\'as, cabe resaltar la importancia de la reducc\'on de ruido para que las interfaces de reconocimiento
de voz puedan utilizarse en distintos tipos de ambientes (el interior de un autom\'ovil, por ejemplo).