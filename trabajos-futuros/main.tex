%!TEX root = ../tesis.tex
\chapter{Trabajos Futuros}
\label{sec:trabajos-futuros}

El \'area de reconocimiento del habla fue avanzando con el correr de las d\'ecadas. El tiempo invertido
en investigaci\'on y desarrollo de la misma permiti\'o la creaci\'on de algoritmos que permiten
reconocer la voz humana con una precisi\'on aceptable, \'utiles para poder plantear nuevas alternativas
de interacc\'on entre computadoras y usuarios. 

Este trabajo de grado presenta el \'area de reconocimiento del habla explorando: su historia, el proceso
b\'asico y los algoritmos implicados, estado del arte, aplicaciones, tecnolog\'ias y herramientas disponibles.
Adem\'as, para contrastar teor\'ia y practica se presenta el programa \emph{TamTam Listens} que permite componer
m\'usica utilizando comandos dos voz. Finalmente, se presenta una prueba de usabilidad utilizada para evaluar la
aplicaci\'on desarrollada.

A continuaci\'on se presentan posibles trabajos futuros que podr\'ian realizarse tomando como punto de partida
el trabajo de grado presentado, y as\'i profundizar m\'as en el \'area de reconocimiento del habla.

\section{Probar TamTam Listens con personas con discapacidad}

Uno de los beneficios de las aplicaciones basadas en reconocimiento del habla, es que permiten a personas con alg\'un
tipo de discapacidad poder interactuar con la computadora. Como se mencion\'o en el secci\'on~\ref{sec:tamtam-edit}
una persona con problemas motrices, por ejemplo, podr\'ia tener dificultades para operar el mouse o el teclado. 
Por lo tanto, utilizar comandoz de voz para interactuar con la computadora, podr\'ia resultar de mayor utilidad
en estos casos.

Habiendo obtenido resultados satisfactorios en la prueba con usuarios llevada a cabo como parte de este trabajo, 
se podr\'ia plantear a continuación la realización de pruebas de usabilidad de la interfaz \emph{TamTam Listens} 
con personas que tienen alg\'un tipo de discapacidad de modo a poder evaluar el desempe\~no de la aplicaci\'on. 
Esto permitir\'a identificar puntos a modificar en la aplicaci\'on y as\'i poder aumentar la experiencia de usuario.

\section{Modelo Ac\'ustico Paraguayo}  

Uno de los componentes utilizados para el reconocimiento de comandos es el modelo ac\'ustico. La interfaz implementada
utiliza el prove\'ido por el proyecto \emph{VoxForge}. Se podr\'ia buscar utilizar un modelo ac\'ustico propio con el
acento caracter\'istico de nuestra regi\'on y as\'i, se espera, tener mejores resultados en lo que respecta a precisi\'on.

\section{Reducci\'on de ruido}

El ruido ambiente es un problema que afecta al reconocimiento del habla. La precisi\'on de la soluci\'on desarrollada en este trabajo es 
considerablemente sensible al ruido ambiente, es por eso que la prueba de usabilidad fue realizada en un ambiente aislado
y libre de ruido. Por lo tanto, se podr\'ian investigar los algoritmos de reducci\'on de ruido disponibles y ver su aplicabilidad
al software desarrollado. Adem\'as, cabe resaltar la importancia de la reducc\'on de ruido para que las interfaces de reconocimiento
de voz puedan utilizarse en distintos tipos de ambientes (el interior de un autom\'ovil, por ejemplo).

\section{Optimizar generaci\'on de Modelo Ac\'ustico}

Como ya se mencion\'o anteriormente, el modelo ac\'ustico es uno de los componentes necesarios para implementar un sistema de 
reconocimiento del habla. Para poder generar un modelo se necesita realizar una cantidad considerable de procesamiento sobre
un conjunto de datos, cuanto mayor sea el tama\~no de datos para entrenamiento, se lograr\'a obtener un mejor modelo ac\'ustico. Es por
esto que la generaci\'on del modelo ac\'ustico puede llegar a ser un proceso lento si se quiere generar uno en base a un conjunto de datos
muy grande.

\emph{Grid Computing} es una colecci\'on de computadoras dispersas geogr\'aficamente que colaboran entre si para alcanzar objetivos comunes y resolver
tareas espec\'ificas. Este sistema distribuido opera distribuyendo el trabajo que se debe realizar entre las distintas m\'aquinas disponibles en la red
y as\'i poder procesar grandes cantidades de datos y de este modo, poder resolver tareas complejas.

En base a lo expuesto, se podr\'ia plantear la utilizaci\'on de \emph{Grid Computing} para realizar el proceso de entrenamiento y as\'i
poder optimizar el proceso de generaci\'on del modelo ac\'ustico para grandes cantidades de datos.

\section{Exploración de otros dominios de aplicación}
La exploración de otros posibles dominios de aplicación, mediante la integraci\'on del reconocimiento del habla a proyectos existentes, puede dar lugar a interesentes trabajos.

Buti\'a es un proyecto uruguayo de rob\'otica educativa \cite{RoboticaEducativa} que plantea como objetivo 
crear una plataforma simple y econ\'omica que permita a los alumnos de colegios p\'ublicos,
en coordinaci\'on con profesores, interiorizarse con la programaci\'on del comportamiento de robots\cite{Butia}.
La plataforma utilizada por los ni\~nos para poder controlar las acciones b\'asicas del robot es el programa
\emph{Tortugarte}. Este es un programa, que al igual que \emph{Tamtam Edit} corre en la computadora \emph{XO} del
proyecto \gls{olpc}.

Un interesante trabajo ser\'ia la integraci\'on de comandos de voz a la lista de acciones disponibles 
en \emph{Tortugarte} y as\'i brindar la posibilidad de operar el robot utilizando comandos de voz.

El proyecto Aguar\'a busca aprovechar el gran potencial que tiene el Paraguay con respecto a la producci\'on de
energ\'ia el\'ectrica. Consiste en el primer auto el\'ectrico fabricado en Paraguay con el apoyo del \gls{pti}. 

En la sección~\ref{sec:automotriz} se mencion\'o que los avances en el \'area de reconocimiento
permitieron incorporar \'esta tecnolog\'ia dentro de los autom\'oviles, para operar componentes b\'asicos del panel.
Agregar soporte para comandos de voz al veh\'iculo el\'ectrico denominado Aguar\'a podría plantearse como objetivo
para otro trabajo futuro.
