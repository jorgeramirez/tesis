%!TEX root = ../tesis.tex
\section{Dise\~no de la Interfaz}
\label{sec:disenho-interfaz}

El principal objetivo de la fase de dise\~no de la interfaz por voz del usuario consisti\'o
en definir el lenguaje que se utilizar{\'\i}a para la misma y el protocolo de interacci\'on entre
el usuario y la aplicaci\'on.

Es decir, se definieron los comandos que tendr{\'\i}an significado para el sistema y la manera
en que el usuario utilizar{\'\i}a los mismos para acceder a las funcionalidades implementadas.

En primer punto que se consider\'o importante fue la naturalidad de los comandos
de acuerdo al dominio de la aplicaci\'on. Esto teniendo en cuenta que el usuario deb{\'\i}a
recordarlos y utilizarlos para interactuar con el sistema.

En el caso particular de \emph{TamTam Listens} esto signific\'o buscar t\'erminos musicales
que pudiesen asociarse con los elementos que formaban parte de la aplicaci\'on: partitura,
pista, comp\'as, nota, etc.

Un error que puede cometerse al intentar dise\~nar una interfaz por voz del usuario,
el cual estaba presente en el primer prototipo de \emph{TamTam Listens}, es el intento
de asociar comandos de voz a acciones que se realizar{\'\i}an normalmente con la interfaz
gr\'afica.

El reemplazar directamente clicks y pulsaciones de teclas por comandos de voz da
como resultado una interfaz poco natural y dif{\'\i}cil de utilizar. Por tanto, fue
necesario replantear por completo el modo de acceso a las funcionalidades de manera
independiente a la interfaz gr\'afica.

En base a la experiencia, resulta recomendable asociar un comando a una funcionalidad
y no a una \'unica acci\'on en la interfaz tradicional. A modo de ejemplo, cambiar el
volumen en \emph{TamTam Edit} requer{\'\i}a presionar un bot\'on para desplegar un submen\'u 
y utilizar un {slider}, mientras que en \emph{TamTam Listens} era necesario un
\'unico comando ``Aumentar/Disminuir Volumen''.  