%!TEX root = ../tesis.tex
\section{Dise\~no de la Interfaz}
\label{sec:disenho-interfaz}

El principal objetivo de la fase de dise\~no de la interfaz por voz del usuario consisti\'o
en definir el lenguaje que se utilizar{\'\i}a para la misma y el protocolo de interacci\'on entre
el usuario y la aplicaci\'on.

Es decir, se definieron los comandos que tendr{\'\i}an significado para el sistema y la manera
en que el usuario utilizar{\'\i}a los mismos para acceder a las funcionalidades implementadas.

Las conclusiones relacionadas a esta etapa se mencionan y describen a continuaci\'on.

\subsection{La naturalidad del lenguaje es de gran importancia para la interfaz}
Al utilizar una interfaz basada en reconocimiento del habla el usuario debe recordar los
comandos disponibles, de modo a utilizarlos para interactuar con el sistema.

La utilizaci\'on de comandos poco apropiados puede llevar a una interacci\'on innecesariamente
complicada entre la persona y la aplicaci\'on, dificultando el aprendizaje
de los comandos por parte del usuario.

Por el contrario, la utilizaci\'on de comandos naturales (palabras intuitivas) para el usuario puede colaborar
con la fluidez de la interacci\'on, facilitando el aprendizaje de los comandos por parte del usuario. 

Por este motivo, el primer punto que se consider\'o importante fue la naturalidad de los comandos
de acuerdo al dominio de la aplicaci\'on.

En el caso particular de \emph{TamTam Listens} esto signific\'o buscar t\'erminos musicales
que pudiesen asociarse con los elementos que formaban parte de la aplicaci\'on: partitura,
pista, comp\'as, nota, etc.

Las recomendaciones realizadas por los usuarios, en su mayor{\'\i}a cambios en las palabras o comandos utilizados,
confirman la importancia de este elemento para una interfaz mediante voz.   


\subsection{Interactuar con la aplicaci\'on, no con la interfaz gr\'afica}
Un error que puede cometerse al intentar dise\~nar una interfaz por voz del usuario,
el cual estaba presente en el primer prototipo de \emph{TamTam Listens}, es el intento
de asociar comandos de voz a acciones que se realizar{\'\i}an normalmente con la interfaz
gr\'afica.

El reemplazar directamente clics y pulsaciones de teclas por comandos de voz da
como resultado una interfaz poco natural y dif{\'\i}cil de utilizar. Por tanto, fue
necesario replantear por completo el modo de acceso a las funcionalidades de manera
independiente a la interfaz gr\'afica.

En base a la experiencia, resulta recomendable asociar un comando a una funcionalidad
y no a una \'unica acci\'on en la interfaz tradicional. A modo de ejemplo, para cambiar el
volumen en \emph{TamTam Edit} se requiere presionar un bot\'on para desplegar un submen\'u 
y utilizar un \emph{slider}, mientras que en \emph{TamTam Listens} es necesario un
\'unico comando ``Aumentar/Disminuir Volumen''.

\subsection{Utilizar el sonido como medio de retroalimentaci\'on}
El protocolo de interacci\'on entre el usuario y la aplicaci\'on puede pensarse de forma similar a una 
conversaci\'on entre personas. El usuario accede a las funcionalidades a trav\'es de comandos de voz.
A su vez, la aplicaci\'on puede utilizar la voz, o el sonido en general, para comunicar un mensaje al usuario.

Esta \'ultima posibilidad puede ser \'util como medio de retroalimentaci\'on entre la aplicaci\'on y el usuario.
Por ejemplo, se pueden utilizar notificaciones de audio para confirmar la realizaci\'on de una determinada
operaci\'on. Esto resulta especialmente importante teniendo en cuenta el error propio de todo sistema
de reconocimiento del habla.

Adem\'as, utilizando s{\'\i}ntesis del habla, se puede implementar un recordatorio de los comandos
disponibles. Esta es una herramienta v\'alida para facilitar el aprendizaje del lenguaje por parte del
usuario.

En el caso particular de \emph{TamTam Listens}, se utilizaron notificaciones de audio para confirmar
ciertas operaciones. Por ejemplo, al crear o editar una nota, se reproduce la misma a modo de
confirmaci\'on.

