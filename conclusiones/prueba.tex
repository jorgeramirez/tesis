%!TEX root = ../tesis.tex
\section{Prueba con Usuarios: Correlaci\'on}
A\'un teniendo en cuenta que un estudio de correlaci\'on no permite determinar causalidad
entre variables, los resultados expuestos en la tabla \ref{sec:tabla-correlacion} 
permiten identificar relaciones cuando menos interesantes entre las mismas que podr{{\'\i}}an
merecer su estudio en investigaciones futuras.

El an\'alisis de estas relaciones corresponde a uno de los objetivos del estudio de usabilidad
realizado con \foreign{TamTam Listens}, mencionados en la secci\'on \ref{sec:objetivos-estudio}.

\subsection[Errar es humano: Los usuarios prefieren probar (y equivocarse) a leer]
{Errar es humano: Los usuarios prefieren probar \\ (y equivocarse) a leer}
A primera vista, La correlaci\'on positiva de la tasa de acierto del sistema con la
tasa de error humano y con la cantidad de errores por usuario podr{{\'\i}}a parecer
poco intuitiva.

En la pr\'actica, este valor podr{{\'\i}}a explicarse bas\'andose en las observaciones registradas 
de las pruebas. En las mismas, se pudo observar que varios usuarios para quienes la tasa de
aciertos del sistema era elevada prefer{{\'\i}}an descubrir y aprender los comandos a trav\'es
de la prueba y el error a leer m\'as detalladamente el manual de la aplicaci\'on o tratar de
recordar detenidamente el comando correcto.

As{{\'\i}} tambi\'en, la cantidad total de comandos que utiliz\'o el usuario result\'o relacionada
positivamente con ambas medidas del error humano y con el acierto del sistema. Mediante las 
observaciones, pudo notarse que los usuarios para para quienes la tasa de
aciertos del sistema era elevada se ve{{\'\i}}an inclinados a intentar descubrir y utilizar nuevos
comandos, incluso si estos no eran estrictamente necesarios para la resoluci\'on de las tareas.

Estas tendencias en la conducta de los participantes sirven como argumento a favor de la importancia 
de la naturalidad del lenguaje, de modo a facilitar el aprendizaje de los comandos por parte del usuario.

Las relaciones anteriormente mencionadas parecen sugerir que el error humano no es necesariamente
un indicador negativo, afirmaci\'on que parece reforzarse con la correlaci\'on positiva, aunque moderada
a d\'ebil, entre las medidas del error humano y la correctitud de la \'ultima tarea. Este valor
indica que, en general, los usuarios que se equivocaron m\'as completaron mejor la tarea cuatro.

\subsection[La memoria del usuario facilita la utilizaci\'on de interfaces \\ mediante voz]
{La memoria del usuario facilita la utilizaci\'on de \\ interfaces mediante voz}
Los objetivos de la prueba con usuarios, mencionados en la secci\'on \ref{sec:objetivos-estudio},
inclu{\'\i}an el estudio de los factores externos a la aplicaci\'on. Uno de estos factores
es la memoria del usuario.

Teniendo en cuenta la la invisibilidad y transitoriedad de las interfaces mediante
voz, se esperaba que la memoria del usuario influyera en los resultados de
las pruebas.

Efectivamente, la memoria del usuario result\'o relacionada negativamente con las medidas 
de error humanoy la duraci\'on de las tareas, lo cual puede considerarse como un indicador de 
la importancia de este factor.

Esto es, en general, los usuarios con mayor puntaje en el test de memoria
tardaron menos en realizar las tareas y cometieron menos errores.

\subsection[Ciertas correlaciones permiten evaluar una interfaz mediante voz del usuario]
{Ciertas correlaciones permiten evaluar una interfaz \\ mediante voz del usuario}

Los resultados para algunas correlaciones son de esperarse, por tratarse de cuestiones
previsibles, de cierta manera. Si los resultados obtenidos no coinciden con las expectativas,
puede tomarse esto como un indicador de alg\'un problema con la interfaz. 

Por ejemplo, se esperaba que los usuarios para quienes la tasa de aciertos del sistema fuese superior
obtuviesen mejores resultados en las tareas, lo cual pudo verificarse mediante la correlaci\'on 
entre la tasa de acierto del sistema y la correctitud de la \'ultima tarea.

Otra relaci\'on previsible que pudo comprobarse a trav\'es del estudio de correlaci\'on fue la que
se dio entre el error humano y el tiempo que tom\'o terminar las tareas. As{{\'\i}}, los usuarios que
m\'as se equivocaron tardaron m\'as en finalizar las tareas.

Algunas correlaciones d\'ebiles pueden considerarse como indicadores positivos. Teniendo en cuenta
el alto grado de correctitud promedio de la \'ultima tarea (87,5\%), la aparente ausencia de relaci\'on 
de este factor con la memoria del usuario y la duraci\'on de las tareas resulta alentador para
\emph{TamTam Listens}. Esto es, m\'as all\'a de su memoria y cuanto tardaron en finalizar las tareas,
los usuarios lograron resultados, en general, satisfactorios.

Estos criterios de evaluaci\'on propuestos responden a otro de los objetivos mencionados
en la secci\'on \ref{sec:objetivos-estudio}.

\section[Prueba con Usuarios: An\'alisis del Error Humano]
{Prueba con Usuarios: An\'alisis del Error \\ Humano}
Los resultados del an\'alisis del error humano, expuestos en la secci\'on \ref{sec:resultados-error-humano},
aportaron datos importantes. En base a los mismos, se presentan las siguientes recomendaciones.

\subsection{Evitar comandos de m\'as de 4 palabras}
Mediante an\'alisis del error humano fue posible estudiar la relaci\'on entre la longitud de los comandos
y la precisi\'on de los usuarios. Los resultados obtenidos muestran que la tasa de error
var{{\'\i}}a poco para los comandos de 2, 3, 4 palabras y aumenta considerablemente para los comandos
de 5 y 6 palabras.

Por este motivo, se sugiere que los comandos de una interfaz por voz del usuario 
no superen la longitud de 4 palabras, de ser posible.

Sin embargo, cabe destacar que esto representa un desaf{{\'\i}}o adicional a tener en cuenta.
Durante la etapa de implementaci\'on, pudo constatarse que los comandos de poca longitud presentaban
a menudo tasas de errores superiores a la media del sistema.
Mantener el equilibrio entre error humano y error de la aplicaci\'on, de acuerdo a la longitud
de los comandos, podr{{\'\i}}a suponer un problema no trivial. 

\subsection{Dar la debida importancia al entrenamiento}
El an\'alisid del error humano con respecto al tiempo transcurrido arroj\'o tambi\'en
resultados interesantes. La tasa de error humano se presenta m\'as elevada desde el inicio hasta
el 40\% del tiempo transcurrido, lo cual podr{{\'\i}}a atribuirse a un periodo de aprendizaje
de la interfaz.

La recomendaci\'on en este caso es otorgar la debida importancia al entrenamiento previo,
el cual se considera como el principal medio para disminuir la tasa de error humano inicial.
La duraci\'on y el dise\~no de las actividades de entrenamiento son puntos a tener en cuenta
al presentar una interfaz mediante voz del usuario.

\subsection{Considerar factores humanos como la fatiga}
Analizando los resultados, pudo apreciarse un leve aumento en la tasa de error hacia el final del tiempo
transcurrido, lo que podr{{\'\i}}a un efecto del agotamiento sobre el usuario utilizando la
interfaz mediante voz.

Esta informaci\'on representa una llamada de atenci\'on hacia los factores humanos de la interacci\'on.
Los efectos influencia del cansancio, la frustraci\'on ante el error e incluso el estado de \'animo
del usuario no deben menospreciarse, e incluso podr{\'\i}an considerarse como tema para estudios
posteriores.

\subsection{Reconsiderar los comandos con tasa de error humano elevada}
Por \'ultimo, el error humano por comando puede servir para identificar comandos 
problem\'aticos que pueden modificarse en una versi\'on siguiente de la aplicaci\'on.

Los comandos con mayor tasa de error humano se exponen en la tabla \ref{sec:tabla-lista-comandos-error}.
Estos comandos son candidatos a ser modificados de modo a mejorar la usabilidad de la
aplicaci\'on.

La identificaci\'on de problemas potenciales al utilizar una interfaz mediante voz, as{\'\i} como las sugerencias
ofrecidas en base al an\'alisis realizado, se encontraban entre los objetivos planteados en la secci\'on 
\ref{sec:objetivos-estudio}.

\section{Prueba con Usuarios: Encuesta}
Los resultados de la encuesta realizada a los usuarios, expuestos en la secci\'on \ref{sec:resultados-encuesta},
permitieron obtener informaci\'on importante con respecto a su grado de satisfacci\'on y su opini\'on 
posterior a la utilizaci\'on de la aplicaci\'on.

\subsection[Opini\'on positiva de los usuarios hacia \emph{TamTam Listens} y el \\ reconocimiento del habla]
{Opini\'on positiva de los usuarios hacia \emph{TamTam Listens} y el reconocimiento del habla}
En primer lugar, los valores obtenidos en las preguntas de estilo Likert que se formularon
llevan a concluir un elevado grado satisfacci\'on con respecto a las palabras y los comandos
utilizados y el tiempo de entrenamiento previo a las pruebas.
Adem\'as, se midi\'o un grado de predisposici\'on considerable por parte de los usuarios de utilizar
las interfaces mediante voz con frecuencia en la vida diaria.

\subsection[Los puntos a mejorar est\'an en las respuestas de los usuarios]
{Los puntos a mejorar est\'an en las respuestas de los \\ usuarios}
De todas formas, habiendo reescalado los resultados de cada usuario, el grado de satisfacci\'on
con las palabras utilizadas recibi\'o el puntaje m\'as bajo en comparaci\'on con los dem\'as puntos.
Debe destacarse que esto no significa que el grado de satisfacci\'on sea bajo (la califaci\'on promedio
fue de 6,17 de 7), sino que es el punto que m\'as puede mejorarse.

Las sugerencias que brindaron los usuarios permitieron identificar palabras y comandos que 
resultaban poco naturales e inadecuados a su criterio. En ese sentido, representan una herramienta
importante al momento de definir las mejoras que pueden incluirse en versiones posteriores de la
aplicaci\'on. Las recomendaciones m\'as comunes fueron:

\begin{itemize}
	\item Reemplazar la palabra ``Duplicar'' por ``Copiar'': con esto, el comando 
	``Duplicar Pista Uno en Pista Dos'' quedar{{\'\i}}a como ``Copiar Pista Uno en Pista Dos''. 
	\item Reemplazar la palabra ``Exportar'' por ``Guardar'': con esto, el comando 
	``Exportar M\'usica'' quedar{{\'\i}}a como ``Guardar M\'usica''.
	\item Eliminar la palabra ``Nueva'' de los comandos de creaci\'on: con esto, el comando 
	``Crear Nueva M\'usica'' quedar{{\'\i}}a como ``Crear M\'usica''.
\end{itemize}

\subsection{Preferir aplicaciones poco interactivas para una interfaz por voz}
Finalmente, la mayor{{\'\i}}a de los usuarios expres\'o que utilizar{{\'\i}}a una interfaz mediante voz
preferentemente en casos con un bajo grado de interactividad y un lenguaje sencillo.
Por ejemplo, los usuarios expresaron que les gustar{{\'\i}}a manejar el televisor, la radio, 
el horno microondas, entre otros, mediante la voz.

Este punto es de gran importancia, y deber{\'\i}a ser tenido en cuenta para la selecci\'on de las aplicaciones
que formen parte de trabajos futuros.
