%!TEX root = ../tesis.tex
\section{Implementaci\'on de la Interfaz}
\label{sec:implementacion-interfaz}

Una vez que se tuvo en claro lo que se buscaba implementar, el siguiente paso
fue la selecci\'on de las herramientas que se utilizar{\'\i}an para hacerlo. 

La herramienta elegida tiene gran influencia sobre el posterior proceso de
desarrollo, por lo cual esta decisi\'on debe tomarse analizando las caracter{\'\i}sticas
propias del proyecto en cuesti\'on.

A modo de ejemplo, algunas cuestiones que pueden tomarse en consideraci\'on son:

\begin{itemize}
	\item El tiempo del que se dispone.
	\item El dinero del que se dispone.
	\item El conocimiento t\'ecnico del equipo de desarrolladores.
	\item La plataforma sobre la cual debe ejecutarse el sistema.
	\item La necesidad de que el sistema funcione sin conexi\'on a internet.
	\item El soporte existente para el idioma que se busca reconocer.
\end{itemize}

La evaluaci\'on de varias opciones disponibles para la implementaci\'on de un sistema
basado en reconocimiento del habla, cuyos resultados se incluyen como parte de este
trabajo, permiti\'o realizar la selecci\'on de manera debidamente informada y justificada.

Habiendo seleccionado Pocketsphinx y el modelo ac\'ustico Voxforge para
la implementaci\'on, el mayor problema durante la fase de desarrollo estuvo relacionado
al plugin para \emph{Gstreamer} que se planeaba utilizar para integrar \emph{TamTam Edit}
con el motor de reconocimiento.

Luego de numerosos intentos fallidos, finalmente se opt\'o por desechar el mencionado plugin
e implementar una soluci\'on utilizando \emph{DBus} para la integraci\'on. Al exponerse los
resultados del reconocimiento del habla mediante un demonio, este enfoque permite integrar
Pocketsphinx con cualquier aplicaci\'on que utilice \emph{DBus} como m\'etodo de comunicaci\'on
entre procesos.

Otro inconveniente que puede mencionarse es la elevada tasa de error que se obtuvo en las
pruebas preliminares para ciertos comandos. Este problema se presentaba especialmente para
los comandos de una o dos palabras y oblig\'o a realizar modificaciones sobre el lenguaje
inicialmente planteado.