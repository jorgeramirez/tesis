%!TEX root = ../tesis.tex
\section{Implementaci\'on de la Interfaz}
\label{sec:implementacion-interfaz}

Luego de definir el alcance y la experiencia de usuario, el siguiente paso
fue la selecci\'on de las herramientas que se utilizar{\'\i}an para hacerlo. 

\subsection{Seleccionar las herramientas de acuerdo a las \mbox{caracter{\'\i}sticas} del proyecto}
La herramienta elegida tiene gran influencia sobre el posterior proceso de
desarrollo, por lo cual esta decisi\'on debe tomarse analizando las caracter{\'\i}sticas
propias del proyecto en cuesti\'on.

Algunas cuestiones que pueden tomarse en consideraci\'on son:

\begin{itemize}
    \item El tiempo del que se dispone.
    \item El dinero del que se dispone.
    \item El conocimiento t\'ecnico del equipo de desarrolladores.
    \item La plataforma sobre la cual debe ejecutarse el sistema.
    \item La necesidad de que el sistema funcione sin conexi\'on a internet.
    \item El soporte existente para el idioma que se busca reconocer.
\end{itemize}

La evaluaci\'on de varias opciones disponibles para la implementaci\'on de un sistema
basado en reconocimiento del habla, cuyos resultados se incluyen como parte de este
trabajo, permiti\'o realizar la selecci\'on de manera debidamente informada y justificada.

\subsection{Considerar la posibilidad de errores en el \emph{software}}
Habiendo seleccionado Pocketsphinx y el modelo ac\'ustico Voxforge para
la implementaci\'on, el mayor problema durante la fase de desarrollo estuvo relacionado
al componente para \foreign{GStreamer} \footnote{\foreign{GStreamer es una plataforma multimedia
de c\'odigo abierto que permite integrar componentes de reproducci\'on y procesamiento de audio y v{\'\i}deo
para el desarrollo de aplicaciones audiovisuales.}}
\cite{GstreamerPocketsphinx} que se planeaba utilizar 
para integrar \foreign{TamTam Edit} con el motor de reconocimiento.

Las primeras pruebas de \foreign{TamTam Listens} arrojaron tasas de error y tiempos de
respuesta muy elevados. Identificar al componente para \foreign{GStreamer} como la causa del problema
y buscar una soluci\'on conllev\'o un retraso significativo en la etapa de desarrollo.

Luego de numerosos intentos fallidos, finalmente se opt\'o por desechar el mencionado componente
e implementar una soluci\'on utilizando \emph{D-Bus} para la integraci\'on. Al exponerse los
resultados del reconocimiento del habla mediante un demonio, este enfoque permite integrar
Pocketsphinx con cualquier aplicaci\'on que utilice \emph{D-Bus} como m\'etodo de comunicaci\'on
entre procesos.

\subsection{Realizar pruebas y modificaciones tempranas}
Una vez que se cuenta con un prototipo m{\'\i}nimo de la aplicaci\'on, la realizaci\'on de pruebas preliminares
puede ofrecer resultados interesantes para el mejoramiento de la 
interfaz por voz del usuario.

En las caso de \foreign{TamTam Listens}, las pruebas preliminares hicieron notoria la elevada tasa
de error que se obten{\'\i}a para determinados comandos. Este problema se presentaba especialmente para
los comandos de una o dos palabras.

Adem\'as, las pruebas preliminares hicieron evidente el problema inicial de \foreign{TamTam Listens}
al registrarse la pronunciaci\'on de palabras fuera del vocabulario de la aplicaci\'on por parte del usuario.
Este inconveniente, y el mecanismo implementado para minimizar su efecto, fueron presentados en la 
secci\'on \ref{sec:oov}. 

Estos resultados obligaron a realizar modificaciones sobre el lenguaje inicialmente planteado,
permitiendo disminuir la tasa de error del sistema previamente a las pruebas con usuarios
finales.