%!TEX root = ../tesis.tex
\chapter{Conclusiones}
\label{sec:conclusiones}

La realizaci\'on de este trabajo final de grado supuso primeramente la necesidad de estudiar
el \'area de reconocimiento del habla: su historia, el estado del arte y los fundamentos
te\'oricos involucrados en el proceso b\'asico de reconocimiento.

Se realiz\'o posteriormente una evaluaci\'on de varias herramientas disponibles para la
implementaci\'on de un sistema de reconocimiento del habla, con respecto a un conjunto
criterios propuestos, de modo a facilitar la selecci\'on de la herramienta adecuada de
acuerdo al caso.

Para llevar a la pr\'actica lo aprendido, se dise\~n\'o e implement\'o una interfaz por voz
del usuario para una aplicaci\'on simple de composici\'on musical y se llevaron a cabo
pruebas de la misma con usuarios.

Este cap{\'\i}tulo presenta las conclusiones que se obtuvieron durante el transcurso
del trabajo realizado, acerca del reconocimiento del habla y su aplicaci\'on a las
interfaces de usuarios.

%!TEX root = ../tesis.tex
\section{Dise\~no de la Interfaz}
\label{sec:disenho-interfaz}

El principal objetivo de la fase de dise\~no de la interfaz por voz del usuario consisti\'o
en definir el lenguaje que se utilizar{\'\i}a para la misma y el protocolo de interacci\'on entre
el usuario y la aplicaci\'on.

Es decir, se definieron los comandos que tendr{\'\i}an significado para el sistema y la manera
en que el usuario utilizar{\'\i}a los mismos para acceder a las funcionalidades implementadas.

Las conclusiones relacionadas a esta etapa se mencionan y describen a continuaci\'on.

\subsection{La naturalidad del lenguaje es de gran importancia para la interfaz}
Al utilizar una interfaz basada en reconocimiento del habla el usuario debe recordar los
comandos disponibles, de modo a utilizarlos para interactuar con el sistema.

La utilizaci\'on de comandos poco apropiados puede llevar a una interacci\'on innecesariamente
complicada entre la persona y la aplicaci\'on, dificultando el aprendizaje
de los comandos por parte del usuario.

Por el contrario, la utilizaci\'on de comandos naturales (palabras intuitivas) para el usuario puede colaborar
con la fluidez de la interacci\'on, facilitando el aprendizaje de los comandos por parte del usuario. 

Por este motivo, el primer punto que se consider\'o importante fue la naturalidad de los comandos
de acuerdo al dominio de la aplicaci\'on.

En el caso particular de \emph{TamTam Listens} esto signific\'o buscar t\'erminos musicales
que pudiesen asociarse con los elementos que formaban parte de la aplicaci\'on: partitura,
pista, comp\'as, nota, etc.

\subsection{Interactuar con la aplicaci\'on, no con la interfaz gr\'afica}
Un error que puede cometerse al intentar dise\~nar una interfaz por voz del usuario,
el cual estaba presente en el primer prototipo de \emph{TamTam Listens}, es el intento
de asociar comandos de voz a acciones que se realizar{\'\i}an normalmente con la interfaz
gr\'afica.

El reemplazar directamente clics y pulsaciones de teclas por comandos de voz da
como resultado una interfaz poco natural y dif{\'\i}cil de utilizar. Por tanto, fue
necesario replantear por completo el modo de acceso a las funcionalidades de manera
independiente a la interfaz gr\'afica.

En base a la experiencia, resulta recomendable asociar un comando a una funcionalidad
y no a una \'unica acci\'on en la interfaz tradicional. A modo de ejemplo, cambiar el
volumen en \emph{TamTam Edit} requer{\'\i}a presionar un bot\'on para desplegar un submen\'u 
y utilizar un {slider}, mientras que en \emph{TamTam Listens} era necesario un
\'unico comando ``Aumentar/Disminuir Volumen''.

\subsection{Utilizar el sonido como medio de retroalimentaci\'on}
El protocolo de interacci\'on entre el usuario y la aplicaci\'on puede pensarse de forma similar a una 
conversaci\'on entre personas. El usuario accede a las funcionalidades a trav\'es de comandos de voz.
A su vez, la aplicaci\'on puede utilizar la voz, o el sonido en general, para comunicar un mensaje al usuario.

Esta \'ultima posibilidad puede ser \'util como medio de retroalimentaci\'on entre la aplicaci\'on y el usuario.
Por ejemplo, se pueden utilizar notificaciones de audio para confirmar la realizaci\'on de una determinada
operaci\'on. Esto resulta especialmente importante teniendo en cuenta el error propio de todo sistema
de reconocimiento del habla.

Adem\'as, utilizando s{\'\i}ntesis del habla, se puede implementar un recordatorio de los comandos
disponibles. Esta es una herramienta v\'alida para facilitar el aprendizaje del lenguaje por parte del
usuario.

En el caso particular de \emph{TamTam Listens}, se utilizaron notificaciones de audio para confirmar
ciertas operaciones. Por ejemplo, al crear o editar una nota, se reproduc{\'\i}a la misma a modo de
confirmaci\'on.


%!TEX root = ../tesis.tex
\section{Implementaci\'on de la Interfaz}
\label{sec:implementacion-interfaz}

Una vez que se tuvo en claro lo que se buscaba implementar, el siguiente paso
fue la selecci\'on de las herramientas que se utilizar{\'\i}an para hacerlo. 

La herramienta elegida tiene gran influencia sobre el posterior proceso de
desarrollo, por lo cual esta decisi\'on debe tomarse analizando las caracter{\'\i}sticas
propias del proyecto en cuesti\'on.

A modo de ejemplo, algunas cuestiones que pueden tomarse en consideraci\'on son:

\begin{itemize}
	\item El tiempo del que se dispone.
	\item El dinero del que se dispone.
	\item El conocimiento t\'ecnico del equipo de desarrolladores.
	\item La plataforma sobre la cual debe ejecutarse el sistema.
	\item La necesidad de que el sistema funcione sin conexi\'on a internet.
	\item El soporte existente para el idioma que se busca reconocer.
\end{itemize}

La evaluaci\'on de varias opciones disponibles para la implementaci\'on de un sistema
basado en reconocimiento del habla, cuyos resultados se incluyen como parte de este
trabajo, permiti\'o realizar la selecci\'on de manera debidamente informada y justificada.

Habiendo seleccionado Pocketsphinx y el modelo ac\'ustico Voxforge para
la implementaci\'on, el mayor problema durante la fase de desarrollo estuvo relacionado
al plugin para \emph{Gstreamer} que se planeaba utilizar para integrar \emph{TamTam Edit}
con el motor de reconocimiento.

Luego de numerosos intentos fallidos, finalmente se opt\'o por desechar el mencionado plugin
e implementar una soluci\'on utilizando \emph{DBus} para la integraci\'on. Al exponerse los
resultados del reconocimiento del habla mediante un demonio, este enfoque permite integrar
Pocketsphinx con cualquier aplicaci\'on que utilice \emph{DBus} como m\'etodo de comunicaci\'on
entre procesos.

Otro inconveniente que puede mencionarse es la elevada tasa de error que se obtuvo en las
pruebas preliminares para ciertos comandos. Este problema se presentaba especialmente para
los comandos de una o dos palabras y oblig\'o a realizar modificaciones sobre el lenguaje
inicialmente planteado.
%!TEX root = ../tesis.tex
\section{Prueba con Usuarios}
\label{sec:prueba}

En esta sección se presentan las conclusiones que pudieron obtenerse a partir de los
resultados de las pruebas de \emph{TamTam Listens} realizadas con usuarios.

\subsection{Correlación}
\section{An\'alisis del Error}
\section{Encuesta}




