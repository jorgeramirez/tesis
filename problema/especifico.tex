%!TEX root = ../tesis.tex
\section{Alcance}
\label{sec:problema-especifico}

De modo a contrastar la pr\'actica con el conocimiento te\'orico adquirido en
el transcurso de este trabajo de investigaci\'on, se incluye el dise\~no e implementaci\'on de
una interfaz mediante voz del usuario para una aplicaci\'on como parte del mismo.

Para obtener conclusiones del proceso de dise\~no e implementaci\'on y de la posterior evaluaci\'on
de la interfaz, se consider\'o que la misma deb{\'\i}a ser de cierta complejidad. Esto es, la
aplicaci\'on implementada deb{\'\i}a cumplir con los siguientes requisitos:

\begin{itemize}
     \item Nivel de interacci\'on medio o alto entre el usuario y la aplicaci\'on.
     \item Lenguaje de tama\~no considerable.
     \item Comandos reconocidos de varias longitudes.
     \item Interacci\'on normalmente prolongada entre el usuario y la aplicaci\'on.
 \end{itemize} 

Cumpliendo con las condiciones anteriormente mencionadas, se decidi\'o dise\~nar e implementar 
una interfaz para un programa simple de composici\'on musical.
Se opt\'o por utilizar una interfaz multimodal, 
haciendo uso de soporte visual para minimizar el efecto de las limitaciones
relacionadas a la voz que se mencionaron previamente.

Las funcionalidades m{\'\i}nimas requeridas para el sistema se citan a continuaci\'on:
%Explicar mejor
\begin{itemize}
    \item El usuario puede crear m\'usica con la aplicaci\'on: esta es la funcionalidad de
    mayor importancia y el objetivo del sistema.
    \item El usuario puede seleccionar distintos instrumentos musicales: esto
    permite la combinaci\'on de los distintos instrumentos para la composici\'on de la m\'usica.
    \item El usuario puede configurar el volumen de cada instrumento musical: de
    modo a poder combinar los distintos instrumentos con distinta intensidad en el
    sonido.
    \item El usuario puede crear sonidos: un sonido es la unidad b\'asica para la
    creaci\'on de una m\'usica. B\'asicamente, se compone m\'usica mediante la combinaci\'on
    de sonidos.
    \item El usuario puede configurar el tono y la duraci\'on de los sonidos: de modo
    a permitir un gran n\'umero de combinaciones posibles para la composici\'on musical.
    \item El usuario puede reproducir, pausar y detener la m\'usica: de modo a poder
    escuchar y apreciar su obra.
    \item El usuario puede guardar la m\'usica que crea: de modo a poder escuchar
    su obra en reproductores de audio y compartirla.
\end{itemize}

Una vez definida la tem\'atica de la aplicaci\'on a implementar, llevar a la pr\'actica
los conocimientos adquiridos requiere definir ciertas cuestiones:

\begin{itemize}
    \item Definici\'on de Comandos: se refiere al proceso de definir el comando, o la secuencia de
    comandos, que se utilizar\'a para exponer cada funcionalidad prove{\'\i}da por el sistema.

    El lenguaje aceptado define la interacci\'on entre el usuario y la aplicaci\'on, y es de esperarse
    que resulte importante para la naturalidad de la misma y para la usabilidad del sistema en
    general.

    \item Herramientas a utilizar: existe un gran n\'umero de opciones disponibles
    para la implementaci\'on de una interfaz basada en voz del usuario.
    La selecci\'on de la herramienta tiene un gran impacto sobre el proceso de implementaci\'on
    y debe realizarse de acuerdo a las prioridades del caso.

    La clasificaci\'on y evaluaci\'on inclu{\'\i}da en el cap{\'\i}tulo anterior puede resultar de
    utilidad para tomar esta decisi\'on de manera informada y en base a criterios bien establecidos.

    \item Evaluaci\'on de la Aplicaci\'on: una evaluaci\'on de la aplicaci\'on implementada puede ayudar
    a identificar problemas y sugerir mejoras o nuevas funcionalidades. Adem\'as, en el caso particular
    de este trabajo, se espera obtener conclusiones acerca de las interfaces basadas en el reconocimiento
    del habla.

    Para realizar la evaluaci\'on deben establecerse varios puntos como: los objetivos, la metodolog{\'\i}a,
    los factores a tener en cuenta, las m\'etricas que se utilizar\'an, etc.

 \end{itemize} 