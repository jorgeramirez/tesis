%!TEX root = ../tesis.tex
\section{Alcance}
\label{sec:problema-especifico}

Este trabajo de investigaci\'on incluye el dise\~no e implementaci\'on de
una interfaz mediante voz del usuario para una aplicaci\'on de composici\'on musical,
como una forma de contrastar la pr\'actica con el conocimiento te\'orico adquirido en
el transcurso del mismo.

Para interactuar con la aplicaci\'on desarrollada se decidi\'o utilizar una interfaz
multimodal, haciendo uso de soporte visual para minimizar el efecto de las limitaciones
relacionadas a la voz que se mencionaron previamente.

Las funcionalidades m{\'\i}nimas requeridas para el sistema se citan a continuación:
%Explicar mejor
\begin{itemize}
	\item El usuario puede crear música con la aplicación: esta es la funcionalidad de
    mayor importancia y el objetivo del sistema.
	\item El usuario puede seleccionar distintos instrumentos musicales: esto
    permite la combinación de los distintos instrumentos para la composición de la música.
	\item El usuario puede configurar el volumen de cada instrumento musical: de
    modo a poder combinar los distintos instrumentos con distinta intensidad en el
    sonido.
	\item El usuario puede crear sonidos: un sonido es la unidad básica para la
    creación de una música. Básicamente, se compone música mediante la combinación
    de sonidos.
	\item El usuario puede configurar el tono y la duración de los sonidos: de modo
    a permitir un gran número de combinaciones posibles para la composición musical.
	\item El usuario puede reproducir, pausar y detener la música: de modo a poder
    escuchar y apreciar su obra.
	\item El usuario puede guardar la música que crea: de modo a poder escuchar
    su obra en reproductores de audio y compartirla.
\end{itemize}