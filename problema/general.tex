%!TEX root = ../tesis.tex
\section{Descripci\'on General}
\label{sec:problema-general}

El reconocimiento del habla (tambi\'en conocido como reconocimiento autom\'atico del habla) es el proceso
de convertir una se\~nal de voz en una secuencia de palabras, mediante un algoritmo implementado
como programa \mbox{computacional \cite{JaisalAReview2012}}.

Las interfaces mediante voz del usuario son sistemas computaciones especializados que permiten la
interacci\'on entre seres humanos y computadoras (otros sistemas computacionales) a trav\'es de
s{\'\i}ntesis y reconocimiento del habla. Estas interfaces presentan ciertas caracter{\'\i}sticas que las
distinguen de las interfaces visuales \cite{GabrielVoice2007}:

\begin{itemize}
	\item Transitoriedad: la voz desaparece tan pronto como se termina de pronunciar una oraci\'on,
	lo cual obliga a recordar lo que se dijo. Las interfaces visuales, por otro lado, son persistentes.
	\item Invisibilidad: la voz no es visible, lo cual hace dif{\'\i}cil indicar al usuario las opciones
	disponibles y los comandos necesarios para ejecutarlas. En las interfaces visuales los men\'ues
	cumplen esta funci\'on.
	\item Asimetr{\'\i}a: la voz puede producirse r\'apidamente, pero comprender lo que se escucha requiere
	m\'as tiempo. As{\'\i}, un usuario puede hablar m\'as r\'apido de lo que escribe con un teclado; sin embargo,
	al usuario le toma m\'as tiempo comprender lo que escucha que lo que lee.
\end{itemize}

Estas caracter{\'\i}sticas suponen un desaf{\'\i}o adicional al momento de dise\~nar interfaces mediante voz del
usuario. A\'un as{\'\i}, estas interfaces poseen un gran potencial en situaciones en las cuales la
combinaci\'on tradicional de teclado, rat\'on y monitor resulta problem\'atica \cite{NielsenVoice2003}:

\begin{itemize}
	\item Usuarios con discapacidades, las cuales les impiden manejar apropiadamente el rat\'on y/o
	el teclado o visualizar la informaci\'on en el monitor.
	\item Usuarios en situaciones de manos y vista ocupadas: como la conducci\'on de un veh{\'\i}culo o
	la reparaci\'on de equipamiento complejo.
	\item Usuarios sin acceso a un teclado o monitor: en este caso los usuarios podr{\'\i}an acceder
	a un sistema a trav\'es de un tel\'efono convencional.
\end{itemize}

