%!TEX root = ../tesis.tex
\section{Descripci\'on General}
\label{sec:problema-general}

El reconocimiento del habla (tambi\'en conocido como reconocimiento autom\'atico del habla) es el proceso
de convertir una se\~nal de voz en una secuencia de palabras, mediante un algoritmo implementado
como programa \mbox{computacional \cite{JaisalAReview2012}}.

Las interfaces mediante voz del usuario son sistemas computaciones especializados que permiten la
interacci\'on entre seres humanos y computadoras (otros sistemas computacionales) a trav\'es de
s{\'\i}ntesis y reconocimiento del habla. Estas interfaces presentan ciertas caracter{\'\i}sticas que las
distinguen de las interfaces visuales \cite{GabrielVoice2007}:

\begin{itemize}
	\item Transitoriedad: la voz desaparece tan pronto como se termina de pronunciar una oraci\'on,
	lo cual obliga a recordar lo que se dijo. Las interfaces visuales, por otro lado, son persistentes.
	\item Invisibilidad: la voz no es visible, lo cual hace dif{\'\i}cil indicar al usuario las opciones
	disponibles y los comandos necesarios para ejecutarlas. En las interfaces visuales los men\'ues
	cumplen esta funci\'on.
	\item Asimetr{\'\i}a: la voz puede producirse r\'apidamente, pero comprender lo que se escucha requiere
	m\'as tiempo. As{\'\i}, un usuario puede hablar m\'as r\'apido de lo que escribe con un teclado; sin embargo,
	al usuario le toma m\'as tiempo comprender lo que escucha que lo que lee.
\end{itemize}

Estas caracter{\'\i}sticas suponen un desaf{\'\i}o adicional al momento de dise\~nar interfaces mediante voz del
usuario. A\'un as{\'\i}, estas interfaces poseen un gran potencial en situaciones en las cuales la
combinaci\'on tradicional de teclado, rat\'on y monitor resulta problem\'atica \cite{NielsenVoice2003}.
Pueden mencionarse como ejemplos los siguientes casos:

\begin{itemize}
	\item Usuarios con discapacidades, las cuales les impiden manejar apropiadamente el rat\'on y/o
	el teclado o visualizar la informaci\'on en el monitor.
	\item Usuarios en situaciones de manos y vista ocupadas: como la conducci\'on de un veh{\'\i}culo o
	la reparaci\'on de equipamiento complejo.
	\item Usuarios sin acceso a un teclado o monitor: en este caso los usuarios podr{\'\i}an acceder
	a un sistema a trav\'es de un tel\'efono convencional.
\end{itemize}

El dise\~no de una interfaz por voz del usuario plantea varias cuestiones interesantes para
su estudio, entre las cuales pueden mencionarse:

\begin{itemize}
	\item Dominio de la aplicaci\'on: cabe preguntarse si existen dominios m\'as o menos adecuados para su
	integraci\'on con el reconocimiento del habla. Y, de ser as{\'\i}, {?`}c\'omo pueden identificarse?
	
	\item Nivel de interactividad de la aplicaci\'on: el grado de interacci\'on entre el usuario y
	la aplicaci\'on podr{\'\i}a ser un elemento que merece consideraci\'on.
	
	Las aplicaciones que requieren que el usuario pronuncie uno o pocos comandos para cumplir con su
	prop\'osito podr\'ian parecer m\'as apropiadas para la implementaci\'on de una interfaz por voz del usuario.
	Sin embargo, {?`}es posible utilizar interfaces por voz de usuario para aplicaciones altamente
	interactivas sin que la productividad del usuario se vea perjudicada?

	\item Tama\~no del Lenguaje: relacionado a la usabilidad de la aplicaci\'on.

	Se puede argumentar a favor de un lenguaje extenso con gran cantidad de comandos reconocidos, 
	teniendo en cuenta la naturalidad de la interacci\'on con el usuario que esto permitir{\'\i}a.

	Sin embargo, un gran n\'umero de comandos reconocidos podr{\'\i}a resultar dif{\'\i}cil de recordar para
	el usuario. Por tanto, {?`}c\'omo puede medirse el efecto del tama\~no del lenguaje sobre la interfaz? 
	Adem\'as, {?`}es posible estimar un tama\~no ideal del lenguaje utilizado?

	\item Longitud de los comandos: con respecto a la usabilidad de la aplicaci\'on.

	El debate con respecto a la cantidad de palabras que forman un comando es bastante similar
	al punto anterior. De igual manera es v\'alido preguntarse {?`}c\'omo medir el efecto de la longitud
	de los comandos sobre la interfaz? y, de ser posible, {?`}c\'omo estimar una longitud recomendada?

	\item Duraci\'on de la interacci\'on: se refiere al tiempo que el usuario interact\'ua con la aplicaci\'on
	a trav\'es de la interfaz por voz del usuario.

	Las caracter{\'\i}sticas propias de este tipo de interfaces, anteriormente mencionadas, podr{\'\i}an sugerir
	que son m\'as adecuadas para interacciones breves. Sin embargo, {?`}es posible utilizarlas para 
	interacciones m\'as prolongadas? y adem\'as, {?`}puede medirse el efecto de la duraci\'on sobre la
	productividad del usuario? 

\end{itemize}


