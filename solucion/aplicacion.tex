%!TEX root = ../tesis.tex

\section{Aplicaci\'on Desarrollada}
\label{sec:aplicacion-desarrollada}

Como se expone en la secci\'on~\ref{sec:problema-especifico}, se propone el dise\~no
de una interfaz para componer m\'usica utilizando la voz. Adem\'as, este trabajo 
opta por la metodolog\'ia de trabajo de c\'odigo abierto, lo cual implica: 

\begin{itemize}
    \item Utilizar tecnolog\'ias de c\'odigo abierto para el desarrollo de la interfaz.
    \item Establecer un proceso de desarrollo transparente y abierto.
\end{itemize}

La metodolog\'ia adoptada para implementar la interfaz permite utilizar un proyecto
existente como punto de partida, y as\'i dise\~nar e implementar una interfaz alternativa que permite
controlar la aplicaci\'on por comandos de voz. Adem\'as,
el hecho de incorporar una nueva interfaz, permite analizar
una de las motivaciones de la secci\'on~\ref{sec:motivacion}: un programa de composici\'on
musical, que recibe comandos sonoros y emite tambi\'en un resultado sonoro, podr\'ia ser
m\'as natural para el usuario. 

La interfaz desarrollada se denomina \emph{TamTam Listens} y toma como punto de partida la 
aplicaci\'on \emph{TamTam Edit} de la plataforma educativa \foreign{Sugar}.

\subsection{TamTam Edit}
\label{sec:tamtam-edit}

La m\'usica es a menudo descrita como la forma m\'as pura de representaci\'on matem\'atica, es m\'as,
te\'oricos de la m\'usica han utilizado las matem\'aticas para resolver problemas musicales
\cite{TheSoundOfNumbers}. Esta fue la inspiraci\'on para la creaci\'on del compendio de 
actividades\footnote{Una Actividad, es una aplicaci\'on en el entorno de escritorio \emph{Sugar}.}
conocido como \emph{TamTam} desarrollado para la computadora XO\footnote{La XO, es una computadora 
port\'atil de bajo costo y consumo desarrollada por el proyecto \gls{olpc}.},
con los siguientes objetivos:

\begin{itemize}
    \item Proveer a los ni\~nos un ambiente de informaci\'on cultural construyendo m\'usica y sonidos.
    \item Brindar una experiencia sonora/musical divertida para usuarios sin conocimientos musicales.
    \item Promover un camino hacia experiencias musicales m\'as sofisticadas.
    \item Promover un instrumento musical con su propio ``sonido''.
    \item Desarrollar un ambiente din\'amico y mutable que propone la simpleza y permite la complejidad.
    \item Favorecer la creaci\'on de m\'usica grupalmente.
    \item Introducir los conceptos musicales y otros como: programaci\'on y audio.
\end{itemize}

\emph{TamTam Edit} es una aplicaci\'on, parte del conjunto de actividades musicales 
\emph{TamTam}, que proporciona una interfaz intuitiva para crear, modificar y organizar notas ubicadas 
en pistas virtuales.
Adem\'as incluye una paleta de casi cien tipos de sonidos y modelos de construcci\'on musical que permite 
crear distintos tipos de variaciones en estilos musicales \cite{TamTamWiki}.


Las secciones principales del programa se pueden observar en la figura~\ref{figure:ui-tamtam} 

\begin{figure}[H]
\centering
\includegraphics[width=0.8\textwidth]{./graphics/ui-tamtam-edit.png}
\caption{Interfaz de \emph{Tamtam Edit} y sus secciones principales.}
\label{figure:ui-tamtam}
\end{figure}

Como se puede apreciar en la figura~\ref{figure:ui-tamtam}, la interfaz se encuentra organizada en 
cinco pistas y cada pista (1) tiene asociada un instrumento (la quinta pista esta reservada para
instrumentos de tipo bater{\'\i}a). Cada pista se divide en 4 
compases (del comp\'as uno al cuatro) y cada comp\'as (2) se divide en 12 tiempos (del uno al doce). 
Las notas (3) se dibujan en los compases, como se puede ver, la longitud de la nota indica su 
duraci\'on y su altura el tipo de nota. Adem\'as la aplicaci\'on esta organizada en partituras (4), que 
son como hojas de cuaderno, y son \'utiles para componer m\'usicas largas.

En general, la interacci\'on entre el usuario y \foreign{TamTam Edit} que se presenta en
la figura~\ref{figure:ui-tamtam} se realiza de modo tradicional.
El usuario utiliza el teclado y el rat\'on para interactuar con la aplicaci\'on a trav\'es de elementos
gr\'aficos como botones y men\'us. 

Este trabajo de grado busca incorporar un medio de interacci\'on alternativo al propuesto 
por \emph{TamTam Edit}. A continuaci\'on se presentan los motivos que determinaron la elecci\'on 
de esta aplicaci\'on como punto de partida:

\begin{itemize}
    \item Impacto social: \emph{TamTam Edit} es una aplicaci\'on que forma parte de la plataforma establecida
    por el proyecto \gls{olpc}, el cual busca potenciar el proceso educativo \cite{OLPC}. 
    Los ni\~nos con alg\'un tipo de discapacidad tienen a menudo problemas para utilizar una computadora mediante
    dispositivos perif\'ericos tradicionales como el mouse o el teclado.

    En estos casos, la posibilidad de operar esta actividad mediante la voz ser{\'\i}a de gran beneficio 
    para los ni\~nos, mejorando la accesibilidad de la plataforma, y generando por tanto un impacto social
    positivo importante.
    \item Naturalidad de interacci\'on: utilizar la voz para interactuar con una aplicaci\'on podr{\'\i}a
    ofrecer una mayor naturalidad con respecto al enfoque tradicional de interacci\'on, teniendo en cuenta
    que es el medio de interacci\'on entre las personas.
    \item No reinventar la rueda: como el objetivo es incorporar una interfaz basada en reconocimiento del
    habla a una aplicaci\'on, se elige \emph{TamTam Edit} para no construir la aplicaci\'on desde cero,
    sino extender las capacidades de una ya existente.
    \item C\'odigo abierto: el motivo anterior es posible gracias a que \emph{TamTam Edit} es un 
    proyecto de c\'odigo abierto, lo cual brinda a los desarrolladores la libertad de extender las 
    funcionalidades de la aplicaci\'on.
    \item Lenguaje de programaci\'on: la actividad se encuentra implementada en el lenguaje de 
    programaci\'on Python. La librer{\'\i}a de reconocimiento del habla a ser utilizada proporciona 
    \emph{bindings} 
    \footnote{Un \emph{binding} es un componente \emph{software}
   que permite hacer uso de las funcionalidades prove{\'\i}das por una librer{\'\i}a, implementada
    en un determinado lenguaje de programaci\'on, utilizando un lenguaje de programaci\'on diferente. 
    En este caso particular, \emph{Pocketsphinx} est\'a implementado en C y C++.} para 
    este lenguaje, lo cual supone una ventaja importante para la implementaci\'on de la interfaz.
\end{itemize}

\subsection{Tamtam Listens}
\label{sec:tamtam-listens}

La interfaz de interacci\'on alternativa para la aplicaci\'on \emph{Tamtam Edit} se denomina  
\emph{Tamtam Listens}. \emph{Tamtam Listens} permite al usuario componer m\'usica utilizando comandos
de voz para acceder a las diferentes funcionalidades ofrecidas por \mbox{\emph{Tamtam Edit}.}

Al ofrecer al usuario final un medio de interacci\'on humano-computadora diferente, \emph{TamTam Listens}
debe ser intuitivo y f\'acil de usar. De modo a lograr esto, la soluci\'on a implementar no debe 
ofrecer al usuario la posibilidad de controlar el \foreign{mouse} o los componentes de la interfaz 
gr\'afica de \emph{TamTam Edit} utilizando la voz. 

Una correspondencia directa entre los comandos de voz y la interfaz gr\'afica puede resultar en 
un flujo de interacci\'on poco natural e inadecuado, motivado \'unicamente por la intenci\'on err\'onea 
de imitar el flujo de interacci\'on de las interfaces de escritorio tradicionales. 
La necesidad de ofrecer un flujo de interacci\'on diferente y apropiado para una interfaz mediante voz
como parte de \emph{TamTam Listens} se hizo clara durante la fase de dise\~no.

Para comprender la diferencia, puede considerarse el siguiente ejemplo. 
Crear una nota exige una secuencia de operaciones con el \foreign{mouse} al utilizarse la
interfaz tradicional: presionar el bot\'on de la herramienta correspondiente, seleccionar la pista
donde se desea crear la nota, utilizar el \foreign{mouse} para definir la 
duraci\'on de la nota, etc. Al utilizarse una interfaz mediante voz del usuario, la misma operaci\'on
puede realizarse pronunciando un comando como ``crear nota do''.

\subsection{Ejemplo: Componiendo una escala simple}
\label{sec:ejemplo-escala}

Como presentaci\'on del modelo de interacci\'on de la interfaz propuesta, se transcribe a 
continuaci\'on un tutorial de uso incluido en el manual de uso de \emph{Tamtam Listens}. 
Esto, de modo a ejemplificar una breve interacci\'on entre el usuario y la aplicaci\'on.

El tutorial sirve de gu{\'\i}a al usuario para realizar una sencilla composici\'on musical.
Para componer una escala simple con \foreign{TamTam Listens}, pueden seguirse los siguientes pasos:

\begin{enumerate}
  \item Para empezar, debemos obtener una partitura en blanco. Lo conseguimos pronunciando el comando: 
  ``Crear Nueva M\'usica''.

  \item Seleccionamos los instrumentos que queremos utilizar, diciendo:
    \begin{itemize}
      \item ``Piano en Pista Uno''
      \item ``Guitarra El\'ectrica en Pista Dos''
      \item ``Teclado en Pista Tres''
      \item ``Flauta en Pista Cuatro''
    \end{itemize}

  \item Antes de crear las notas, debemos ubicarnos en el punto donde queremos empezar.
  Para ubicarnos en el tiempo uno, del comp\'as uno de la pista uno:
  \begin{itemize}
    \item ``Pista Uno''
  \end{itemize}
        Si quisi\'esemos empezar en el tiempo uno del comp\'as dos, bastar{\'\i}a con decir:
  \begin{itemize}
    \item ``Comp\'as Dos''
  \end{itemize}
        En caso de querer empezar en el tiempo siete, decimos:
  \begin{itemize}
    \item ``Tiempo Siete''
  \end{itemize}

  \item Ya seleccionado el punto inicial, estamos listos para crear las notas:
  \begin{itemize}
    \item ``Crear Nota Do''
    \item ``Crear Nota Re''
    \item ``Crear Nota Mi''
    \item ``Crear Nota Fa''
    \item ``Crear Nota Sol''
    \item ``Crear Nota La''
    \item ``Crear Nota Si''
  \end{itemize}


  \item Como no queremos trabajar de m\'as, duplicamos las pistas para escuchar los dem\'as instrumentos:
  \begin{itemize}
    \item ``Duplicar Pista Uno en Pista Dos''
    \item ``Duplicar Pista Dos en Pista Tres''
    \item ``Duplicar Pista Tres en Pista Cuatro''
  \end{itemize}

  \item Para escuchar nuestra m\'usica: ``Reproducir M\'usica''.

\end{enumerate}


\subsection{Comandos V\'alidos de la Aplicaci\'on}
\label{sec:comandos-validos}

Como puede apreciarse en el ejemplo anterior, la interacci\'on entre el usuario y la aplicaci\'on
ocurre {\'\i}ntegramente a trav\'es de comandos de voz. Estos comandos son pronunciados por el usuario
e interpretados por \emph{TamTam Listens}, haciendo posible de este modo la composici\'on musical. 

Los comandos de voz que el usuario puede utilizar con \emph{TamTam Listens} se clasifican en:

\begin{itemize}
    \item Comandos Generales (G): independientes del contexto de la aplicaci\'on, es decir, no dependen
    de la pista o el comp\'as seleccionados. Por ejemplo: ``Crear Nueva M\'usica''.
    \item Comandos de Pista (P): dependientes de la pista seleccionada. Por ejemplo, al pronunciar 
    ``Comp\'as Dos'', el comp\'as espec{\'\i}fico seleccionado depende de la pista 
    previamente seleccionada.
    \item Comandos de Comp\'as (C): dependientes de la pista y el comp\'as seleccionados. Por ejemplo, 
    al pronunciar ``Crear Nota Do'', la ubicaci\'on espec{\'\i}fica de la nota creada depende de la
    pista y el comp\'as seleccionados.
\end{itemize}


A continuaci\'on se presentan los distintos comandos soportados, utilizando grafos para representarlos y 
as\'i facilitar su comprensi\'on. El nodo coloreado hace referencia a la \'ultima palabra de un comando 
v\'alido para la aplicaci\'on.

\subsubsection{Comandos Generales}

Los comandos generales son independientes con respecto a otros comandos de la aplicaci\'on, a continuaci\'on se presentan los distintos
comandos disponibles en esta categor\'ia. 
\begin{figure}[H] 
\centering
\includegraphics[width=0.5\textwidth]{./graphics/cmd-musica.png}
\caption{Comandos para reproducir, pausar, parar, exportar y crear una m\'usica}
\label{figure:cmd-crear-musica}
\end{figure}

En la figura~\ref{figure:cmd-crear-musica} se pueden observar los comandos m\'as b\'asicos
de la aplicaci\'on, explicados a continuaci\'on:

\begin{itemize}
\item \emph{reproducir m\'usica}: permite reproducir la m\'usica creada.
\item \emph{pausar m\'usica}: permite pausar la reproducci\'on actual dejando la l{\'\i}nea de reproducci\'on en el
punto de pausa.
\item \emph{parar m\'usica}: permite parar la reproducci\'on actual y ubica la l{\'\i}nea de reproducci\'on al inicio de
la m\'usica.
\item \emph{crear nueva m\'usica}:  permite crear una nueva composici\'on, dejando como resultado una
partitura en blanco.
\item \emph{exportar m\'usica}: permite guardar la m\'usica creada en un archivo para que reproducirse en un
reproductor multimedia.
\end{itemize}

\begin{figure}[H]
\begin{minipage}[b]{0.5\linewidth}
\centering
\includegraphics[width=0.6\linewidth]{./graphics/salir.png}
\caption{Comando para salir de la aplicaci\'on}
\label{figure:cmd-salir}
\end{minipage}
\quad
\begin{minipage}[b]{0.5\linewidth}
\centering
\includegraphics[width=0.6\linewidth]{./graphics/cmd-vol.png}
\caption{Comandos para aumentar/disminuir el volumen general}
\label{figure:cmd-vol}
\end{minipage}
\end{figure}

El comando de la figura~\ref{figure:cmd-salir} permite salir de \emph{TamTam Listens}, basta con decir
``salir de tamtam''. Por otro lado, los comandos de la figura~\ref{figure:cmd-vol}
y~\ref{figure:cmd-tempo} permiten controlar, respectivamente, el volumen y tempo general de la aplicaci\'on. Por ejemplo: ``aumentar volumen'', ``disminuir tempo''.

\begin{figure}[H]
\begin{minipage}[b]{0.5\linewidth}
\centering
\includegraphics[width=0.6\linewidth]{./graphics/cmd-tempo.png}
\caption{Comandos para aumentar/disminuir el tempo general de la aplicaci\'on}
\label{figure:cmd-tempo}
\end{minipage}
\quad
\begin{minipage}[b]{0.5\linewidth}
\centering
\includegraphics[width=0.6\linewidth]{./graphics/partitura-1.png}
\caption{Comandos para crear, limpiar y duplicar la partitura actual}
\label{figure:cmd-partitura-1}
\end{minipage}
\end{figure}

Los comandos de las figuras~\ref{figure:cmd-partitura-1} y~\ref{figure:cmd-partitura-2} afectan a
la partitura actual, como se explican a continuaci\'on:

\begin{itemize}
    \item \emph{crear nueva  partitura}:  permite crear una nueva partitura en blanco. Utilizaci\'on, 
    ``crear nueva partitura''.
    \item \emph{limpiar  partitura}: permite limpiar el contenido de la partitura actual, es decir, 
    borrar todas las notas. Utilizaci\'on, ``limpiar partitura''.
    \item \emph{duplicar partitura}: crea una nueva partitura con el mismo contenido que la partitura 
    actual. Utilizaci\'on, ``duplicar partitura''.
\end{itemize}

\begin{figure}[H]
\begin{minipage}[b]{0.5\linewidth}
\centering
\includegraphics[width=0.6\linewidth]{./graphics/partitura-2.png}
\caption{Comandos para navegar entre partituras}
\label{figure:cmd-partitura-2}
\end{minipage}
\quad
\begin{minipage}[b]{0.5\linewidth}
\centering
\includegraphics[width=0.6\linewidth]{./graphics/cmd-pista-1.png}
\caption{Comando para ubicarse en una pista}
\label{figure:cmd-pista-1}
\end{minipage}
\end{figure} 

En la figura~\ref{figure:cmd-pista-1} puede apreciarse el comando que permite al usuario ubicarse 
en una pista en particular. Por ejemplo, para ubicarse en la pista tres debe decir “pista tres”. 
Adem\'as de controlar el volumen general de la aplicaci\'on, en 
la figura~\ref{figure:cmd-vol-pista} se pueden ver los comandos para controlar el volumen de una pista en 
particular. Para aumentar el volumen de la pista tres, el usuario debe 
decir ``aumentar volumen de pista tres''.

\begin{figure}[H] 
\begin{minipage}[b]{0.5\linewidth}
\centering
\includegraphics[width=0.8\linewidth]{./graphics/vol-pista.png}
\caption{Comandos para aumentar/disminuir el volumen de una pista en particular}
\label{figure:cmd-vol-pista}
\end{minipage}
\quad
\begin{minipage}[b]{0.5\linewidth}
\centering
\includegraphics[width=0.9\linewidth]{./graphics/rep-pista.png}
\caption{Comandos para reproducir, silenciar, habilitar y limpiar una pista en particular}
\label{figure:cmd-rep-pista}
\end{minipage}
\end{figure}

Los comandos de la figura~\ref{figure:cmd-rep-pista} permiten: reproducir, silenciar, habilitar y limpiar el contenido de una
pista en particular. Por ejemplo, para reproducir las notas de la pista uno, el usuario debe decir ``reproducir pista uno''. 
Para poder generar distintos tipos de sonidos con \emph{TamTam Listens}, los usuarios de pueden asignar
instrumentos a cada una de las pistas de
la aplicaci\'on, esto se puede realizar con los comandos de la figura~\ref{figure:cmd-inst-p1-4} y~\ref{figure:cmd-inst-p5}. Para
asignar el piano a la pista dos, basta con decir ``piano en pista dos''.


\begin{figure}[H]
\centering
\includegraphics[width=0.8\textwidth]{./graphics/inst-p1-4.png}
\caption{Selecci\'on de instrumento para pistas del uno al cuatro}
\label{figure:cmd-inst-p1-4}
\end{figure}

\begin{figure}[H] 
\begin{minipage}[b]{0.5\linewidth}
\centering
\includegraphics[width=1\linewidth]{./graphics/inst-p5.png}
\caption{Selecci\'on de instrumento para la pista cinco}
\label{figure:cmd-inst-p5}
\end{minipage}
\quad
\begin{minipage}[b]{0.5\linewidth}
\centering
\includegraphics[width=1\linewidth]{./graphics/dup-pista.png}
\caption{Comando para duplicar las notas de una pista en otra}
\label{figure:cmd-dup-pista}
\end{minipage}
\end{figure}

Generalmente las composiciones presentan cierta secuencia de notas que se repiten para varios 
instrumentos. El comando de la figura \ref{figure:cmd-dup-pista} permite duplicar el contenido, 
es decir las notas, de una pista en otra. Por ejemplo, 
``duplicar pista uno en pista dos'' permite duplicar las notas de la pista uno en la pista dos.

\subsubsection{Comandos de Pista} 

Este comando ubica al usuario dentro de una pista en particular. Por lo tanto, se debe
seleccionar una pista para poder utilizar este comando.

\begin{figure}[H] 
\centering
\includegraphics[width=0.4\linewidth]{./graphics/cmd-compas.png}
\caption{Comando para ubicarse en un comp\'as}
\label{figure:cmd-compas}
\quad
\end{figure}

\subsubsection{Comandos de Comp\'as}

Estos comandos son muy importantes para la aplicaci\'on ya que permiten crear, modificar, 
eliminar las notas musicales. El comando de la figura~\ref{figure:cmd-crear-nota} permite crear notas en el  
comp\'as actual, por ejemplo ``crear nota do'' crea la nota do en el comp\'as previamente seleccionado. Por otro lado, en la figura~\ref{figure:cmd-tiempo-compas} 
se muestra el comando que permite al usuario ubicarse en un  
tiempo en espec\'ifico dentro del comp\'as actual. Esto es \'util para crear una nota a partir de ese punto o 
para seleccionar una nota que se encuentre en ese tiempo.

\begin{figure}[H]
\begin{minipage}[b]{0.5\linewidth}
\centering
\includegraphics[width=1\linewidth]{./graphics/cmd-crear-nota.png}
\caption{Comando para crear una nota}
\label{figure:cmd-crear-nota}
\end{minipage}
\quad
\begin{minipage}[b]{0.5\linewidth}
\centering
\includegraphics[width=1.1\linewidth]{./graphics/cmd-tiempo-compas.png}
\caption{Comando para ubicarse en un tiempo dado, dentro de un comp\'as}
\label{figure:cmd-tiempo-compas}
\end{minipage}
\end{figure}

As\'i como puede duplicarse notas de una pista a otra, tambi\'en puede duplicarse una nota de un comp\'as a otro utilizando el comando
de la figura~\ref{figure:cmd-dup-nota}, por ejemplo ``duplicar en pista uno compas dos'' permite duplicar una nota en el segundo comp\'as de la pista 
uno. Para poder eliminar una nota, previamente seleccionada, el usuario debe utilizar el comando
de la figura~\ref{figure:cmd-del-nota}.

\begin{figure}[H]
\begin{minipage}[b]{0.5\linewidth}
\centering
\includegraphics[width=1.2\linewidth]{./graphics/cmd-dup-nota.png}
\caption{Comando para duplicar una nota previamente seleccionada}
\label{figure:cmd-dup-nota}
\end{minipage}
\quad
\begin{minipage}[b]{0.5\linewidth}
\centering
\includegraphics[width=0.5\linewidth]{./graphics/del-note.png}
\caption{Comando para eliminar un nota previamente seleccionada}
\label{figure:cmd-del-nota}
\end{minipage}
\end{figure}

En la figura~\ref{figure:cmd-dur} se pude observar el comando que permite modificar la duraci\'on de 
una nota inmediatamente despu\'es de haberla creado o una nota previamente seleccionada. Finalmente, el comando
presentado en la figura~\ref{figure:cmd-note-tiempo} permite modificar el tiempo en el que inicia la 
nota inmediatamente despu\'es de haberla creado o una nota previamente seleccionada.

\begin{figure}[H]
\begin{minipage}[b]{0.5\linewidth}
\centering
\includegraphics[width=0.9\linewidth]{./graphics/cmd-dur.png}
\caption{Comando que permite configurar la duraci\'on de una nota}
\label{figure:cmd-dur}
\end{minipage}
\quad
\begin{minipage}[b]{0.5\linewidth}
\centering
\includegraphics[width=1.1\linewidth]{./graphics/cmd-note-tiempo.png}
\caption{Comando que permite configurar el inicio de una nota dentro del comp\'as}
\label{figure:cmd-note-tiempo}
\end{minipage}
\end{figure}