\section{M\'etricas de Evaluaci\'on}
\label{sec:metricas}

Para poder juzgar a la interfaz dise\~nada se fijan un conjunto de m\'etricas
de evaluaci\'on orientadas a la usabilidad del software.

Existen muchos factores que determinan si una interfaz es buena o mala. En \cite{AhmedUsability2006} 
se presenta un modelo unificado de m\'etricas de usabilidad, el cual establece este conjunto de factores.
A continuaci\'ion se listan algunos:

\begin{itemize}
    \item \emph{Eficiencia}, es decir, la cantidad necesaria de recursos que el usuario utiliza
	en relaci\'on a la efectividad alcanzada.
    \item \emph{Efectividad}, o la capacidad del software de permitir al usuario completar las tareas
	de manera precisa.
    \item \emph{Productividad}, relaci\'on entre el nivel de efectividad y los recursos utilizados por los
	usuarios (por ejemplo, tiempo para completar la tarea) y el sistema. En contraste con la eficiencia,
	la productividad se centra en la cantidad de salida \'util que se obtiene de la interacci\'on del
	usuario y el software.
     \item \emph{Satisfacci\'on}, que hace referencia a la respuesta subjetiva del usuario respecto al uso de la
	aplicaci\'on.
    \item \emph{Facilidad de Aprendizaje}, o la facilidad con que se pueden aprender los pasos necesarios
	para completar una tarea en particular.
\end{itemize}

\subsection{Speech Usability Metric}

Las interfaces gr\'aficas de usuario tienen un conjunto bien definido de m\'etodos de evaluaci\'on que permiten mejorar
la usabilidad de la interfaz. Sin embargo, \'estos m\'etodos no pueden ser aplicados a la evaluaci\'on de interfaces
basadas en reconocimiento del habla, por este motivo se desarrollaron nuevas t\'ecnicas para evaluar la usabilidad de 
los sistemas basados en reconocimiento del habla.

El m\'etodo utilizado para evaluar la interfaz dise\~nada e implementada en este trabajo se denomina \foreign{Speech Usability Metric}
o SUM \cite{GuptaUsability}. SUM fue desarrollado para evaluar sistemas basados en reconocimiento del habla,
permitiendo al dise\~nador definir m\'etricas que son relevantes para la interfaz y especificar m\'etricas 
objetivo que determinan si la interfaz es lo suficientemente buena. SUM se
define de la siguiente manera:

\begin{equation}
    SUM = X * (\text{Satisfacci\'on del usuario}) + Y * (\text{Precisi\'on}) + Z * (\text{Tiempo de finalizaci\'on de tarea})
\end{equation}

Donde $X + Y + Z = 1$ y $X, Y, Z > 0$.

Como se puede observar SUM utiliza tres m\'etricas ponderadas: Satisfacci\'on del usuario, Precisi\'on y Tiempo de finalizaci\'on de tarea. Los
pesos son definidos por el dise\~nador de acuerdo a la importancia de la m\'etrica. Sin embargo, SUM es lo suficientemente flexible para permitir
la incorporaci\'on de nuevas m\'etricas. La Satisfacci\'on del usuario se mide realizando encuestas y/o entrevistas. La Precisi\'on mide la
efectividad del motor de reconocimiento del habla. El Tiempo de finalizaci\'on de tarea indica que el tiempo que le toma a un usuario
completar una tarea determinada.
