\section{M\'etricas de Evaluaci\'on}
\label{sec:metricas}

Para poder juzgar a la interfaz dise\~nada se fijan un conjunto de m\'etricas
de evaluaci\'on orientadas a la usabilidad del software.

En \cite{AhmedUsability2006} se presenta un modelo unificado de mediciones de usabilidad, el cual
establece un conjunto de factores de usabilidad cada uno de los cuales corresponde a una
faceta de la usabilidad. A continuaci\'on se citan los factores que se
consideran acordes para la interfaz desarrollada:

\begin{itemize}
    \item Eficiencia, es decir, la cantidad necesaria de recursos que el usuario utiliza
	en relaci\'on a la efectividad alcanzada.
    \item Efectividad, o la capacidad del software de permitir al usuario completar las tareas
	de manera precisa.
    \item Productividad, relaci\'on entre el nivel de efectividad y los recursos utilizados por los
	usuarios (por ejemplo, tiempo para completar la tarea) y el sistema. En contraste con la eficiencia,
	la productividad se centra en la cantidad de salida \'util que se obtiene de la interacci\'on del
	usuario y el software.
    \item Satisfacci\'on, que hace referencia a la respuesta subjetiva del usuario respecto al uso de la
	aplicaci\'on.
    \item Facilidad de Aprendizaje, o la facilidad con que se pueden aprender los pasos necesarios
	para completar una tarea en particular.
\end{itemize}

A continuaci\'on se listan las m\'etricas de evaluaci\'on a utilizar:

% agregar mas

\begin{itemize}
    \item Tasa de finalizaci\'on, es una simple medici\'on de usabilidad. Generalmente se
	considera una m\'etrica binaria (1 = Tarea Completa y 0 = Tarea Incompleta).
    \item Tiempo de finalizaci\'on de tarea: mide cu\'anto tiempo es utilizado para
	completar la tarea.
    \item Nivel de satisfacci\'on de tarea: luego de que los usuarios realicen una tarea, hacer
	que respondan un cuestionario sobre la dificultad de realizar dicha tarea.
\end{itemize}
