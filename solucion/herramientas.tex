\section{Tecnolog\'ia a Utilizar}
\label{sec:tecnologia-utilizada}
Para implementar el reconocimiento  de comandos de voz necesario para \foreign{TamTam Listens}, 
se utilizan los proyectos de c\'odigo abierto \emph{Voxforge} \cite{Voxforge} y 
\emph{PocketSphinx} \cite{PocketSphinxHomePage}. 

\subsection{Voxforge}
\label{sec:voxforge-solucion}

\foreign{Voxforge} es un proyecto que busca recopilar grabaciones de voz de modo a crear 
y ofrecer varios corpus de habla bajo una licencia que permita su libre utilizaci\'on. 

A partir de estos corpus, es posible construir modelos ac\'usticos para su uso con motores de 
reconocimiento del habla de c\'odigo abierto, como \foreign{Pocketsphinx}.

\foreign{Voxforge} cuenta con grabaciones en diferentes idiomas, entre ellos ingĺ\'es, franc\'es, 
alem\'an y espa\~nol.


\subsection{PocketSphinx}
\label{sec:pocketsphinx-solucion}

El cap\'itulo~\ref{sec:tecnologias} clasifica a las herramientas que permiten implementar soluciones
basadas en reconocimiento del habla en tres categor\'ias: Aplicaciones, \gls{api}s y Librer\'ias. 
Dada la naturaleza de este trabajo, se opta por elegir una librer\'ia para implementar la interfaz
alternativa a la ofrecida por \emph{TamTam Edit}.

Para la implementaci\'on de la interfaz operable a trav\'es de la voz se elige la librer\'ia 
\emph{PocketSphinx} que, como se menciona en la secci\'on~\ref{sec:pocketsphinx}, es un motor de 
reconocimiento del habla orientado a la optimizaci\'on del rendimiento y la portabilidad.

Uno de los objetivos espec\'ificos de este trabajo es aplicar y contrastar en la pr\'actica
los conocimientos te\'oricos adquiridos. Como indica la secci\'on~\ref{sec:librerias}, una librer\'ia es
el tipo de herramienta que requiere un conocimiento t\'ecnico espec\'ifico del \'area, adem\'as de
brindar una alta flexibilidad permitiendo al programador manipular los distintos componentes del
proceso de reconocimiento del habla. Por este motivo, una librer\'ia se considera la herramienta 
m\'as adecuada para cumplir el objetivo mencionado anteriormente.

A continuaci\'on se presentan los motivos t\'ecnicos que determinaron la elecci\'on de esta librer\'ia:

\begin{itemize}
    \item \emph{PocketSphinx} est\'a orientada a la optimizaci\'on del rendimiento, resultando adecuada 
    para sistemas con recursos limitados, como la computadora \emph{XO}.
    \item La librer\'ia es un proyecto de c\'odigo abierto, por lo tanto se cumple la metodolog\'ia 
    de trabajo adoptada.
    \item \emph{PocketSphinx} ofrece soporte \emph{offline}, lo cual permite su utilizaci\'on en ambientes
    sin conexi\'on a internet, como ocurre frecuentemente con las \emph{XO}.
    \item Existen \foreign{bindings} para el lenguaje de programaci\'on Python, lo cual hace muy 
    sencilla la tarea de utilizar la librer\'ia desde el c\'odigo Python.
\end{itemize}
