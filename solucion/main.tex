%!TEX root = ../tesis.tex
\chapter{Soluci\'on Propuesta}
\label{sec:solucion}

Numerosas aplicaciones en el \'area del reconocimiento del habla han sido desarrolladas: \emph{Siri} \cite{AppleSiri}, \emph{Google Now} \cite{GoogleNow}, 
\emph{Dragon Naturally Speaking} \cite{DragonNaturallySpeaking}, etc. Cada una de estas
aplicaciones satisface distintos tipos de requerimientos establecidos por los usuarios finales. En el
cap\'itulo anterior se marc\'o la diferencia en el proceso de dise\~nar interfaces basadas en reconocimiento
del habla y el proceso de implementar interfaces, para una aplicaci\'on determinada, basadas en reconocimiento del habla.

Este cap\'itulo presenta, justifica y eval\'ua la soluci\'on espec\'ifica al problema de implementar una interfaz
basada en reconomiento del habla para una aplicaci\'on determinada.


\section{Tecnolog\'ia a Utilizar}
\label{sec:tecnologia-utilizada}

\subsection{TamTam Edit}

La aplicaci\'on que se ha elegido para este trabajo de grado consiste en una
Actividad\footnote{Una Actividad, es una aplicaci\'on en el entorno de escritorio \emph{Sugar}}
musical para la
computadora
XO,\footnote{La XO, es una computadora port\'atil de bajo costo y consumo desarrollada por el proyecto \emph{One Laptop Per Child}}denominada \emph{TamTam Edit}.

\emph{TamTam Edit} es una aplicaci\'on, que forma parte del conjunto de actividades musicales \emph{TamTam}, que proporciona
una interfaz intuitiva para crear, modificar y organizar notas ubicadas en pistas virtuales. Adem\'as incluye una paleta de
casi cien tipos de sonidos y modelos de construcci\'on musical que permite crear distintos tipos de variaciones en estilos
musicales \cite{TamTamWiki}.

Las secciones principales del programa se pueden observar en la figura~\ref{figure:ui-tamtam} 

\begin{figure}[H]
\centering
\includegraphics[width=\textwidth]{./graphics/ui-tamtam-edit.png}
\caption{Interfaz de \emph{Tamtam Edit} y sus secciones principales}
\label{figure:ui-tamtam}
\end{figure}

Como se puede apreciar, la interfaz se encuentra organizada en cinco pistas y cada pista tiene asociada un 
instrumento (la quinta pista esta reservada para instrumento de tipo batería). Cada pista se divide en 4 
compases (del compás uno al cuatro) y cada compás se divide en 12 tiempos (del uno al doce). 
Las notas se dibujan en los compases, como se puede ver, la longitud de la nota indica su duración y su 
altura el tipo de nota. Además la aplicación esta organizada en partituras, que son como hojas de 
cuaderno, y son útiles para componer músicas largas.

A continuaci\'on se presentan los motivos que determinaron la elecci\'on de esta aplicaci\'on:

\begin{itemize}
    \item \emph{Impacto social:} \emph{TamTam Edit} es una aplicaci\'on que forma parte de una plataforma establecida
    por el proyecto \gls{olpc}. Este proyecto busca proveer
	a cada ni\~no de pa\'ises en desarrollo de una computadora de bajo costo con el
	fin de potenciar el proceso educativo \cite{OLPC}. Ni\~nos con alg\'un tipo de discapacidad podr\'ian tener
	problemas para utilizar la computadora. Por ejemplo, un ni\~no con problemas motrices podr\'ia tener dificultades para operar el mouse
	o el teclado. Para estos casos, operar esta actividad utilizando la voz beneficiar\'ia a los ni\~nos con discapacidades
	y por lo tanto la aplicaci\'on tendr\'ia un impacto social positivo, debido a la mejora en la accesibilidad.
    \item \emph{Naturalidad de interacci\'on:} utilizar la voz para interactuar con una aplicaci\'on presenta una mayor
	naturalidad con respecto al enfoque tradicional de interacci\'on, debido a que es el medio de
	interacci\'on entre las personas.
    \item \emph{No reinventar la rueda:} como el objetivo es incorporar una interfaz basada en reconocimiento del habla a una aplicaci\'on,
	se elige \emph{TamTam Edit} para no construir la aplicaci\'on desde cero, sino extender las capacidades de una ya existente.
    \item \emph{C\'odigo abierto:} el motivo anterior es posible gracias a que \emph{TamTam Edit} es un proyecto de c\'odigo abierto, lo cual
	brinda a los desarrolladores la libertad de extender las funcionalidades de la aplicaci\'on.
    \item \emph{Lenguaje de programaci\'on:} la actividad se encuentra implementada en el lenguaje de programaci\'on Python. Este es el
	lenguaje elegido para realizar la interfaz, adem\'as la librer'ia de reconocimiento del habla a ser utilizada proporciona
	\emph{bindings} para este lenguaje.
\end{itemize}



\subsection{PocketSphinx}

Para la implementaci\'on de la interfaz operable a trav\'es de la voz se ha elegido la librer\'ia \emph{PocketSphinx} que,
como se menciona en la secci\'on \ref{sec:pocketsphinx}, es un motor de reconocimiento del habla orientado a la optimaci\'on
del rendimiento y la portabilidad.

Uno de los objetivos espec\'ificos de este trabajo de grado es aplicar y contrastar en la pr\'actica
los conocimientos te\'oricos adquiridos. Como indica la secci\'on \ref{sec:librerias}, una librer\'ia es
el tipo de herramienta que requiere un conocimiento t\'ecnico espec\'ifico del \'area, adem\'as de
brindar una alta flexibilidad permitiendo al programador manipular los distintos componentes del
proceso de reconocimiento del habla. Por lo tanto una librer\'ia es la herramienta m\'as adecuada
para cumplir el objetivo mencionado anteriormente.

A continuaci\'on se presentan los motivos t\'ecnicos que determinaron la elecci\'on de esta librer\'ia:

\begin{itemize}
    \item Existen \foreign{bindings} para el lenguaje de programaci\'on Python, lo cual hace muy sencilla la tarea de utilizar
	la librer\'ia desde el c\'odigo Python.
    \item \emph{PocketSphinx} esta orientada a la optimizaci\'on del rendimiento, resultando adecuada para sistemas con
	recursos limitados, como es el caso de la computadora \emph{XO}.
    \item La librer\'ia es un proyecto de c\'odigo abierto, por lo tanto se cumple la metodolog\'ia de trabajo que se ha
	establecido para este trabajo de grado: utilizar herramientas de c\'odigo abierto para la implementaci\'on de la
	interfaz.
\end{itemize}

%!TEX root = ../tesis.tex
\section{Tamtam Listens}
\label{sec:tamtam-listens}
La interfaz de interacci\'on alternativa para la aplicaci\'on \emph{Tamtam Edit} se denomina  
\emph{Tamtam Listens}. \emph{Tamtam Listens} es el resultado de la combinaci\'on
de las tecnolog\'ias mencionadas anteriormente.


Al ofrecer al usuario final un medio de interacci\'on humano-computadora diferente, \emph{TamTam Listens} debe
ser intuitivo y f\'acil de usar. De modo a lograr esto, la soluci\'on a implementar no debe ofrecer al usuario
la posibilidad de controlar el \foreign{mouse} o los componentes de la interfaz gr\'afica de \emph{TamTam Edit}
utilizando la voz. Esto se debe a que controlar el \foreign{mouse} con la voz o, mejor dicho, hacer una
correspondencia directa entre los comandos de voz y la interfaz gr\'afica puede resultar en un flujo de interacci\'on
poco conveniente, ya que estar\'ia usando un flujo de interacci\'on bastante parecido al utilizado por la interfaces
de escritorio tradicionales. Por lo tanto, \emph{TamTam Listens} debe ofrecer un nuevo flujo de interacci\'on basado
en el habla. Esta importante lecci\'on se hizo clara durante la fase de dise\~no.

Para comprender la diferencia, se explica el siguiente ejemplo. 
Crear una nota exige una secuencia de operaciones con el \foreign{mouse} al utilizarse la
interfaz tradicional: hacer clic en el bot\'on de la herramienta correspondiente, hacer clic en la pista
donde se desea crear la nota, arrastrar el mouse con el clic izquierdo presionado para definir la duraci\'on
de la nota, etc.

Al utilizarse una interfaz mediante voz del usuario, la misma operaci\'on puede realizarse pronunciando un
comando como ``crear nota do''.


La figura~\ref{figure:tamtam-listens-arq} muestra la arquitectura b\'asica de \emph{TamTam Listens}. Se pueden
observar los componentes que forman parte de la soluci\'on desarrollada.

\begin{figure}[H] 
\centering
\includegraphics[width=0.7\textwidth]{./graphics/tamtam-listens-arq.png}
\caption{Arquitectura b\'asica de \emph{TamTam Listens}}
\label{figure:tamtam-listens-arq}
\end{figure}

Como sugiere la figura~\ref{figure:tamtam-listens-arq}, \emph{TamTam Listens} a\~nade soporte de reconocimiento
del habla a \emph{TamTam Edit} utilizando la librer\'ia \emph{PocketSphinx}.A favor de una soluci\'on modular,
el reconocimiento de comandos de voz de \emph{PocketSphinx} es expuesto como 
como un  servicio del sistema, a trav\'es de llamadas a \emph{D-bus}\cite{Dbus2013}, el cual es
un mecanismo para comunicaci\'on entre procesos de Linux\footnote{Linux es un sistema operativo cuyo \foreign{kernel} y paquetes
    de software son desarrollados bajo el modelo de c\'odigo abierto\cite{LinuxGuideCert}}.

\emph{TamTam Listens} consume como un servicio el reconocimiento del habla expuesto a trav\'es de \emph{D-Bus}. Al estar implementado de esta manera, el
servicio de reconocimiento puede ser utilizado no solo por \emph{Tamtam Listens} sino por cualquier aplicaci\'on 
de Linux, utilizando un protocolo de paso de mensajes. 

Para hacer posible el reconocimiento del habla debe definirse el vocabulario que forma parte del dominio de 
la aplicaci\'on, su pronunciaci\'on y el conjunto de oraciones (en este caso, comandos v\'alidos) soportados por la aplicaci\'on.
En otras palabras, \emph{Pocketsphinx} requiere un modelo ac\'ustico y un modelo de lenguaje para funcionar.

\subsection{Modelo Ac\'ustico y Modelo de Lenguaje}

El modelo ac\'ustico utilizado tiene como base las grabaciones de \foreign{Voxforge} en idioma espa\~nol,
y se encuentra disponible entre los recursos del proyecto \foreign{Pocketsphinx} para su utilizaci\'on
con el motor de reconocimiento del habla.

Una gram\'atica en formato JSGF \cite{JSGF2000} fue utilizada como modelo de lenguaje, lo cual permiti\'o
definir los comandos soportados por la aplicaci\'on de una manera sencilla.

\subsection{Diccionario Fon\'etico}

El diccionario fon\'etico define la secuencia de fonemas de cada palabra del lenguaje.
Este diccionario es utilizado por \emph{PocketSphinx} para asociar los fonemas reconocidos por
el modelo ac\'ustico a palabras v\'alidas del dominio de la aplicaci\'on.
En la figura~\ref{figure:fragmento-dic} se puede observar un fragmento del diccionario fon\'etico
utilizado por \emph{TamTam Listens}.

\lstset{
  basicstyle=\scriptsize,        % the size of the fonts that are used for the code
  breakatwhitespace=false,         % sets if automatic breaks should only happen at whitespace
  frame=single,                    % adds a frame around the code
  language=Octave,                 % the language of the code
  numbersep=5pt,                   % how far the line-numbers are from the code
  showstringspaces=false,          % underline spaces within strings only
  stepnumber=2,                    % the step between two line-numbers. If it's 1, each line will be numbered
  tabsize=2                       % sets default tabsize to 2 spaces
}

\begin{figure}[H]
\begin{lstlisting}
REPRODUCIR RR E P R O D U S I R
PAUSAR P A U S A R
PARAR  P A R A R
GENERAR  J E N E R A R
PARTITURA P A R T I T U R A
SIGUIENTE S I G I E N T E
ANTERIOR A N T E R I O R
\end{lstlisting}
\caption{Fragmento del diccionario fon\'etico utilizado en \emph{Tamtam Listens}.}
\label{figure:fragmento-dic}
\end{figure}

Como se puede observar, el formato del diccionario es bastante simple. Primero se define la palabra y 
a continuaci\'on los fonemas que la componen separados por espacios.
Una vez definidas las palabras de la aplicaci\'on, lo siguiente es determinar si una secuencia de palabras es o no un 
comando v\'alido.

\subsection{Gram\'atica}

La secuencia de comandos v\'alidos que soporta la aplicaci\'on se define en el modelo de lenguaje. Para \'esto
\emph{TamTam Listens} define una gram\'atica \gls{jsgf}. En la figura~\ref{figure:fragmento-gram} 
puede observarse un fragmento de la gram\'atica utilizada por la aplicaci\'on.

\begin{figure}[H]
\begin{lstlisting}
#JSGF V1.0;
grammar tamtam;

public <tamtam-listens> = <comando> | <pagina> | <pista-a>     | <pista-b>  | 
                          <seleccionar-compas> | <crear-nota>  | <seleccionar-nota> | 
                          <duplicar-nota>      | <borrar-nota> | <volumen> | <tempo> | 
                          <configurar-nota>    | <loop>;

<comando>  = REPRODUCIR MUSICA  | PAUSAR MUSICA   | PARAR MUSICA | GENERAR MUSICA | 
             CREAR NUEVA MUSICA | EXPORTAR MUSICA | SALIR DE TAMTAM;
<pagina>   = ( CREAR NUEVA | DUPLICAR | LIMPIAR ) PARTITURA | PARTITURA <orden>;
<orden>    = ( ANTERIOR | SIGUIENTE );
<loop>   COMODINASURA)+;
\end{lstlisting}
\caption{Fragmento de la gram\'atica utilizada en \emph{Tamtam Listens}.}
\label{figure:fragmento-gram}
\end{figure} 

Los comandos de \emph{TamTam Listens} pueden clasificarse en: Generales (G), de Pista (P) y de Comp\'as (C). La clasificaci\'on
se basa en la dependencia que existe entre comandos de la aplicaci\'on. Los comandos generales no dependen de ning\'un otro comando.
Los comandos de Pista dependen de los comandos Generales y, a su vez, los comandos de comp\'as dependen de los comandos de Pista.
A continuaci\'on se presentan los distintos comandos soportados, utilizando grafos para representarlos y as\'i facilitar
su comprensi\'on. El nodo coloreado hace referencia a la \'ultima palabra de un comando v\'alido para la aplicaci\'on

\subsubsection{Comandos Generales}

Los comandos generales son independientes con respecto a otros comandos de la aplicaci\'on, a continuaci\'on se presentan los distintos
comandos disponibles en esta categor\'ia. 
\begin{figure}[H] 
\centering
\includegraphics[width=0.5\textwidth]{./graphics/cmd-musica.png}
\caption{Comandos para reproducir, pausar, parar, exportar y crear una m\'usica}
\label{figure:cmd-crear-musica}
\end{figure}

En la figura~\ref{figure:cmd-crear-musica} se pueden observar los comandos m\'as b\'asicos
de la aplicaci\'on, explicados a continuaci\'on:

\begin{itemize}
\item \emph{reproducir m\'usica}: permite reproducir la m\'usica creada.
\item \emph{pausar m\'usica}: permite pausar la reproducci\'on actual dejando la l{\'\i}nea de reproducci\'on en el
punto de pausa.
\item \emph{parar m\'usica}: permite parar la reproducci\'on actual y ubica la l{\'\i}nea de reproducci\'on al inicio de
la m\'usica.
\item \emph{crear nueva m\'usica}:  permite crear una nueva composici\'on, dejando como resultado una
partitura en blanco.
\item \emph{exportar m\'usica}: permite guardar la m\'usica creada en un archivo para que reproducirse en un
reproductor multimedia.
\end{itemize}

\begin{figure}[H]
\begin{minipage}[b]{0.5\linewidth}
\centering
\includegraphics[width=0.6\linewidth]{./graphics/salir.png}
\caption{Comando para salir de la aplicaci\'on}
\label{figure:cmd-salir}
\end{minipage}
\quad
\begin{minipage}[b]{0.5\linewidth}
\centering
\includegraphics[width=0.6\linewidth]{./graphics/cmd-vol.png}
\caption{Comandos para aumentar/disminuir el volumen general}
\label{figure:cmd-vol}
\end{minipage}
\end{figure}

El comando de la figura~\ref{figure:cmd-salir} permite salir de \emph{TamTam Listens}, simplemente hay que decir ``salir de tamtam''. Por otro lado, los comandos de la figura~\ref{figure:cmd-vol}
y~\ref{figure:cmd-tempo} permiten controlar, respectivamente, el volumen y tempo general de la aplicaci\'on. Por ejemplo: ``aumentar volumen'', ``disminuir tempo''.

\begin{figure}[H]
\begin{minipage}[b]{0.5\linewidth}
\centering
\includegraphics[width=0.6\linewidth]{./graphics/cmd-tempo.png}
\caption{Comandos para aumentar/disminuir el tempo general de al aplicaci\'on}
\label{figure:cmd-tempo}
\end{minipage}
\quad
\begin{minipage}[b]{0.5\linewidth}
\centering
\includegraphics[width=0.6\linewidth]{./graphics/partitura-1.png}
\caption{Comandos para crear, limpiar y duplicar la partitura actual}
\label{figure:cmd-partitura-1}
\end{minipage}
\end{figure}

Los comandos de las figuras~\ref{figure:cmd-partitura-1} y~\ref{figure:cmd-partitura-2} afectan a
la partitura actual, como se explican a continuaci\'on:

\begin{itemize}
    \item \emph{crear nueva  partitura}:  permite crear una nueva partitura en blanco. Utilizaci\'on, ``crear nueva partitura''
    \item \emph{limpiar  partitura}: permite limpiar el contenido de la partitura actual, es decir, borrar todas las notas. Utilizaci\'on, ``limpiar partitura''
    \item \emph{duplicar partitura}: crea una nueva partitura con el mismo contenido que la partitura actual. Utilizaci\'on, ``duplicar partitura''.
\end{itemize}

\begin{figure}[H]
\begin{minipage}[b]{0.5\linewidth}
\centering
\includegraphics[width=0.6\linewidth]{./graphics/partitura-2.png}
\caption{Comandos para navegar entre partituras}
\label{figure:cmd-partitura-2}
\end{minipage}
\quad
\begin{minipage}[b]{0.5\linewidth}
\centering
\includegraphics[width=0.6\linewidth]{./graphics/cmd-pista-1.png}
\caption{Comando para ubicarse en una pista}
\label{figure:cmd-pista-1}
\end{minipage}
\end{figure} 

En la figura~\ref{figure:cmd-pista-1} puede apreciarse el comando que permite al usuario ubicarse en una pista en particular. Por  
ejemplo, para ubicarse en la pista tres debe decir “pista tres”. Adem\'as de controlar el volumen general de la aplicaci\'on, en 
la figura~\ref{figure:cmd-vol-pista} se pueden ver los comandos para controlar el volumen de una pista en particular. Para aumentar
el volumen de la pista tres, el usuario debe decir ``aumentar volumen de pista tres''.

\begin{figure}[H] 
\begin{minipage}[b]{0.5\linewidth}
\centering
\includegraphics[width=0.8\linewidth]{./graphics/vol-pista.png}
\caption{Comandos para aumentar/disminuir el volumen de una pista en particular}
\label{figure:cmd-vol-pista}
\end{minipage}
\quad
\begin{minipage}[b]{0.5\linewidth}
\centering
\includegraphics[width=0.9\linewidth]{./graphics/rep-pista.png}
\caption{Comandos para reproducir, silenciar, habilitar y limpiar una pista en particular}
\label{figure:cmd-rep-pista}
\end{minipage}
\end{figure}

Los comandos de la figura~\ref{figure:cmd-rep-pista} permiten: reproducir, silenciar, habilitar y limpiar el contenido de una
pista en particular. Por ejemplo, para reproducir las notas de la pista uno, el usuario debe decir ``reproducir pista uno''. 
Para poder generar distintos tipos de sonidos con \emph{TamTam Listens}, los usuarios de pueden asignar
instrumentos a cada una de las pistas de
la aplicaci\'on, \'esto se puede realizar con los comandos de la figura~\ref{figure:cmd-inst-p1-4} y~\ref{figure:cmd-inst-p5}. Para
asignar el piano a la pista dos, basta con decir ``piano en pista dos''.


\begin{figure}[H]
\centering
\includegraphics[width=0.8\textwidth]{./graphics/inst-p1-4.png}
\caption{Selecci\'on de instrumento para pistas del uno al cuatro}
\label{figure:cmd-inst-p1-4}
\end{figure}

\begin{figure}[H] 
\begin{minipage}[b]{0.5\linewidth}
\centering
\includegraphics[width=1\linewidth]{./graphics/inst-p5.png}
\caption{Selecci\'on de instrumento para la pista cinco}
\label{figure:cmd-inst-p5}
\end{minipage}
\quad
\begin{minipage}[b]{0.5\linewidth}
\centering
\includegraphics[width=1\linewidth]{./graphics/dup-pista.png}
\caption{Comando para duplicar las notas de una pista en otra}
\label{figure:cmd-dup-pista}
\end{minipage}
\end{figure}

Generalmente las composiciones tienen cierta secuencia de notas que se repiten para varios instrumentos. El comando de la 
figure~\ref{figure:cmd-dup-pista} permite duplicar el contenido, es decir las notas, de una pista en otra. Por ejemplo, 
``duplicar pista uno en pista dos'' permite duplicar las notas de la pista uno en la pista dos.

\subsubsection{Comandos de Pista} 

Este comando ubica al usuario dentro de una pista en particular. Por lo tanto, se debe
seleccionar una pista para poder utilizar este comando.

\begin{figure}[H] 
\centering
\includegraphics[width=0.4\linewidth]{./graphics/cmd-compas.png}
\caption{Comando para ubicarse en un comp\'as}
\label{figure:cmd-compas}
\quad
\end{figure}

\subsubsection{Comandos de Comp\'as}

Estos comandos son muy importantes para la aplicaci\'on ya que permiten crear, modificar, 
eliminar las notas musicales. El comando de la figura~\ref{figure:cmd-crear-nota} permite crear notas en el  
comp\'as actual, por ejemplo ``crear nota do'' crea la nota do en el comp\'as previamente seleccionado. Por otro lado, en la figura~\ref{figure:cmd-tiempo-compas} 
se muestra el comando que permite al usuario ubicarse en un  
tiempo en espec\'ifico dentro del comp\'as actual. Esto es \'util para crear una nota a partir de ese punto o 
para seleccionar una nota que se encuentre en ese tiempo.

\begin{figure}[H]
\begin{minipage}[b]{0.5\linewidth}
\centering
\includegraphics[width=1\linewidth]{./graphics/cmd-crear-nota.png}
\caption{Comando para crear una nota}
\label{figure:cmd-crear-nota}
\end{minipage}
\quad
\begin{minipage}[b]{0.5\linewidth}
\centering
\includegraphics[width=1.1\linewidth]{./graphics/cmd-tiempo-compas.png}
\caption{Comando para ubicarse en un tiempo dado, dentro de un comp\'as}
\label{figure:cmd-tiempo-compas}
\end{minipage}
\end{figure}

As\'i como puede duplicarse notas de una pista a otra, tambi\'en puede duplicarse una nota de un comp\'as a otro utilizando el comando
de la figura~\ref{figure:cmd-dup-nota}, por ejemplo ``duplicar en pista uno compas dos'' permite duplicar una nota en el segundo comp\'as de la pista 
uno. Para poder eliminar una nota, previamente seleccionada, el usuario debe utilizar el comando
de la figura~\ref{figure:cmd-del-nota}.

\begin{figure}[H]
\begin{minipage}[b]{0.5\linewidth}
\centering
\includegraphics[width=1.2\linewidth]{./graphics/cmd-dup-nota.png}
\caption{Comando para duplicar una nota previamente seleccionada}
\label{figure:cmd-dup-nota}
\end{minipage}
\quad
\begin{minipage}[b]{0.5\linewidth}
\centering
\includegraphics[width=0.5\linewidth]{./graphics/del-note.png}
\caption{Comando para eliminar un nota previamente seleccionada}
\label{figure:cmd-del-nota}
\end{minipage}
\end{figure}

En la figura~\ref{figure:cmd-dur} se pude observar el comando que permite modificar la duraci\'on de 
una nota inmediatamente despu\'es de haberla creado o una nota previamente seleccionada. Finalmente, el comando
presentado en la figura~\ref{figure:cmd-note-tiempo} permite modificar el tiempo en el que inicia la 
nota inmediatamente despu\'es de haberla creado o una nota previamente seleccionada.

\begin{figure}[H]
\begin{minipage}[b]{0.5\linewidth}
\centering
\includegraphics[width=0.9\linewidth]{./graphics/cmd-dur.png}
\caption{Comando que permite configurar la duraci\'on de una nota}
\label{figure:cmd-dur}
\end{minipage}
\quad
\begin{minipage}[b]{0.5\linewidth}
\centering
\includegraphics[width=1.1\linewidth]{./graphics/cmd-note-tiempo.png}
\caption{Comando que permite configurar el inicio de una nota dentro del comp\'as}
\label{figure:cmd-note-tiempo}
\end{minipage}
\end{figure}

\subsection{Palabras fuera del Vocabulario}
\label{sec:oov}
La pronunciaci\'on de palabras fuera del vocabulario ocurre frecuentemente en numerosas aplicaciones
del reconocimiento del habla, y es una fuente conocida de errores en el reconocimiento \cite{Bazzi00Modeling}.

De modo a minimizar la cantidad de errores como consecuencia de las palabras fuera del vocabulario,
\foreign{TamTam Listens} integra a su gram\'atica un modelo de palabra gen\'erica. Este componente se
define mediante el no terminal $<loop>$ de la figura \ref{figure:fragmento-gram}.

Un modelo de palabra gen\'erica admite cualquier palabra que pueda pronunciarse, es decir, permite
cualquier secuencia arbitraria de fonemas \cite{Bazzi00Modeling}. 
En el caso de \foreign{TamTam Listens}, el no terminal $<loop>$ corresponde a una o m\'as repeticiones del
terminal $COMODIN$.

A su vez, el terminal $COMODIN$ est\'a asociado a cada uno de los fonemas en el diccionario fon\'etico de
\foreign{TamTam Listens}, como se observa en la figura \ref{figure:fragmento-comodin}.

\begin{figure}[H]
\begin{lstlisting}
COMODIN   A
COMODIN(2)  B
COMODIN(3)  C
COMODIN(4)  CH
COMODIN(5)  D
\end{lstlisting}
\caption{Definici\'on parcial de COMODIN en el diccionario de \emph{Tamtam Listens}.}
\label{figure:fragmento-comodin}
\end{figure}


\subsection{Ejemplo: Componiendo una escala simple}
Para finalizar este cap{\'\i}tulo, se transcribe a continuaci\'on un tutorial de uso incluido en el manual
de uso de \foreign{TamTam Listens}. Esto, de modo a ejemplificar una breve interacci\'on entre el usuario
y la aplicaci\'on.

El tutorial sirve de gu{\'\i}a al usuario para realizar una sencilla composici\'on musical.
Para componer una escala simple con \foreign{TamTam Listens}, pueden seguirse los siguientes pasos:

\begin{enumerate}
  \item Para empezar, debemos obtener una partitura en blanco. Lo conseguimos pronunciando el comando: 
  ``Crear Nueva M\'usica''.

  \item Seleccionamos los instrumentos que queremos utilizar, diciendo:
    \begin{itemize}
      \item ``Piano en Pista Uno''
      \item ``Guitarra El\'ectrica en Pista Dos''
      \item ``Teclado en Pista Tres''
      \item ``Flauta en Pista Cuatro''
    \end{itemize}

  \item Antes de crear las notas, debemos ubicarnos en el punto donde queremos empezar.
  Para ubicarnos en el tiempo uno, del comp\'as uno de la pista uno:
  \begin{itemize}
    \item ``Pista Uno''
  \end{itemize}
        Si quisi\'esemos empezar en el tiempo uno del comp\'as dos, bastar{\'\i}a con decir:
  \begin{itemize}
    \item ``Comp\'as Dos''
  \end{itemize}
        En caso de querer empezar en el tiempo siete, decimos:
  \begin{itemize}
    \item ``Tiempo Siete''
  \end{itemize}

  \item Ya seleccionado el punto inicial, estamos listos para crear las notas :
  \begin{itemize}
    \item ``Crear Nota Do''
    \item ``Crear Nota Re''
    \item ``Crear Nota Mi''
    \item ``Crear Nota Fa''
    \item ``Crear Nota Sol''
    \item ``Crear Nota La''
    \item ``Crear Nota Si''
  \end{itemize}


  \item Como no queremos trabajar de m\'as, duplicamos las pistas para escuchar los dem\'as instrumentos:
  \begin{itemize}
    \item ``Duplicar Pista Uno en Pista Dos''
    \item ``Duplicar Pista Dos en Pista Tres''
    \item ``Duplicar Pista Tres en Pista Cuatro''
  \end{itemize}

  \item Bastante sencillo. Para escuchar nuestra m\'usica: ``Reproducir M\'usica''.

\end{enumerate}

La uni\'on de los componentes presentados en este cap\'itulo hizo posible implementar el motor 
de reconocimiento del habla como servicio de \emph{D-Bus} y as\'i ofrecer \emph{TamTam Listens} como una interfaz 
alternativa a la propuesta por \emph{TamTam Edit}.

En el siguiente cap\'itulo se presenta la evaluaci\'on de la interfaz implementada, teniendo en cuenta 
aspectos t\'ecnicos y funcionales.

